\section{Diskussionen}

\diskussion{Nextbike}{1. Lesung}{Referat für Verkehr}
    {
    Der StuRa berät über den aktuellen Stand der Verhandlungen zu Nextbike und bespricht folgende Fragen:\\
    \begin{enumerate}
    \item Wie ermitteln wir eine möglichst repräsentative Interessenlage der Studierenden?
    \item Welche Ziele verfolgen wir in den Verhandlungen?
    \item Welche Argumente sollten wir in den Verhandlungen verwenden?
    \item Wie entscheidet sich die VS, ob sie den Vertrag annimmt oder ablehnt?
    \end{enumerate}
    Das Verkehrsreferat wird, soweit die Fragen abschließend geklärt werden können, sich an den Ergebnissen der Diskussion orientieren.
    }{
        Der Antrag dient einerseits dazu, das weitere Vorgehen von Refkonf und Verkehrsreferat zu bestimmen. Andererseits soll damit auch das Plenum auf den aktuellen Stand zu den Verhandlungen und der Nextbike-Nutzung gebracht werden.
    }{
        Wir würden 32000€ sparen, wenn wir die Fahrräder direkt leihen würden.\\
        habe die Zahlen für letzes Jahr noch nicht bekommen, verstösst auch gegen vertrag\\
        Bezahlen wir aufs ganze Jahr geshen zu viel, ist nur dieses Jahr ein ausreißer?\\
            Daten werden veröffentlicht\\
        ander VSen haben tolle Grafiken\\
        ist denn eine Urabstimmung geplant?\\
            StuRa soll diese Frage klären, fahrplan für die Zukunft\\
        Wie teure ist Urabstimmung, wieviel vorlauf\\
            Die Urabstimmung nur als Umfrage, ist billiger und im Endeffekt dieselbe Wirkung\\
        Diskussion in so grosser Runde nicht zielführend, Arbeitskreis zu Thema, weniger Aufwand, gleiches Ergebnis\\
        In neuverhandlungen werden Dinge selten Günstiger, gute Verhandlungen ausschlagebend\\
    }
\antrag
{
Nicht-Einsehbarkeit der Teilnehmendenlisten auf Moodle
}{
    1. Lesung
}{
    GHG Heidelberg
}{
    Der StuRa fordert, die Teilnehmendenlisten von den Moodle-Kursen für Studierende nicht-einsehbar zu
    machen.\\
    Alternativ kann den Studierenden auch die Möglichkeit gegeben werden, ihren Namen in einer
    Teilnehmendenliste nur einsehbar zu machen, wenn sie das ausdrücklich gestatten.
}{
    Da viele Studierende es als unangenehm empfinden, dass öffentlich einsehbar ist, welche Kurse sie
    besuchen, sollten die Teilnehmendenlisten nicht für Studierende öffentlich sein. Zudem sind wir der
    Meinung, dass gerade Zweitnamen, von denen einige eine ganze Menge haben, die Öffentlichkeit nichts
    angehen und privat sein sollten.\\
    An der Pädagogischen Hochschule Heidelberg ist es beispielsweise bereits möglich, dass der eigene
    Name nicht in der Teilnehmendenliste zu sehen ist. Daran sollte sich die Uni ein Beispiel nehmen.\\
    Ein ähnliches Recht auf Anonymität hat der StuRa bereits in der 5. Legislatur für alle Studierenden
    eingefordert:\\
    \url{https://www.stura.uni-heidelberg.de/fileadmin/Intern/Protokolle_und_Beschluesse/5/Beschluesse/Beschluesse_des_StuRa_5_Legislatur.pdf}
}{
    Manchmal ganz praktisch, wenn man Fragen an Kommilitonen hat kann man einfach E-Mails schreiben.\\
        Opt-In besser, E-Mails können schon nicht mehr eingesehen werden.\\
    Meist mehr als 200 Leute könne Daten einsehen. In Gesellschaft wird sonst viel Wert auf Datenschutz gelegt.\\
    Moolde fragt immer nach weiteren Informationen zum Profil, für wen sollen die Infos sein?\\
        ja wird weitergeleitet\\
    Toller Antrag zum  Studienalltag! Vielleicht schon zu spezifisch, nicht nur aufs Moodle begrenzt. Mehr Opt-in auf allen Ebenen der Uni. Frage nach Datensparsamkeit, Datenschutzschulungen des EDV-Referats passend zum Thema\\
        nich nicht auf andere Themn gestoßen, gerne zusammenarbeiten\\
    Nutzen der Namen zweifehalft, wenn eh keine E-Mails abgerufen werden können\\
    Wird das Freiwillige freigeben von Daten in Antrag aufgenommen?\\
        Bereits im Antrag\\
    Bei einer Vorlesung konnte man einsehen, wer welche Klasuur nachschreiben muss, es werden Listen von Namen und dazugehörigen Matrikelnummern verschickt\\
        bitte meldet euch bei der GHG, dann kann das noch mit aufgenommen werden.\\
}{tba}{tba}{tba}
\antrag{Corona-Vollversammlung}{1. Lesung}{Leonard Späth (SDS Heidelberg)}
    {
        Der StuRa beschließt die Organisation einer (satzungstechnisch inoffiziellen) Vollversammlung
        für alle Studierenden zur Lage der Krise an den Hochschulen und in der Gesellschaft.
        Diese wird mit finanzieller und ideeller Unterstützung der Referatekonferenz von Interessierten
        Studierenden organisiert und durchgeführt. 
    }{
        Die Krise hat schon vorher bestehende Missstände weiter verschärft. Soziale Ungleichheiten haben
        massiv zugenommen, die Ungerechtigkeiten im Bildungssystem treten verstärkt offen zu Tage, viele
        kleinere Kulturbetriebe haben Existenznöte.\\
        Die gesellschaftlichen Problematiken zeigen sich auch an den Hochschulen. Während es Konzernhilfen
        in Milliardenhöhe gibt, werden Tausende Studierende mit ihren finanziellen Problemen alleine
        gelassen. Während Produktion weiter anläuft, wird noch nicht einmal die Möglichkeit gegeben,
        kleinere Seminare an der Uni in Präsenz stattfinden zu lassen. Während Erstsemester in ein
        teilweise miserabel organisiertes Digitalsemester eingeladen werden, wird weiter an der
        Prüfungsfixierung festgehalten, ohne Studierenden mit an die Pandemie angepassten Prüfungsbedingungen
        entgegen zu kommen.\\
        Aus diesem Grund braucht es ein Informations, - Diskussions, und Austauschangebot an alle 
        Studierende, in der wir gemeinsam mit ebenfalls von Pandemie besonders betroffenen, wie etwa
        Künstler*innen oder überlasteten Beschäftigten im Gesundheitssystem diskutieren wollen, wie man eine
        soziale, emanzipatorische, und gesundheitlich verantwortungsbewusste Lösung für die Krise finden
        kann.\\
        Eine Vollversammlung bietet dafür die Möglich. Diese sieht unsere Satzung offiziell nicht vor.
        Allerdings ist es trotzdem möglich, als Studierendenvertretung und Fachschaften zusammen einzuladen.
        Diese könnte (je nach Situation) in Präsenz, online oder auch als Hybridformat stattfinden.\\
        Ein möglicher Tagungsablauf für diese wäre:\\
        \begin{itemize}
            \item Inputs von Studierendenseite zur aktuellen sozialen Situation, der Arbeit an der Hochschule
                für die Bewältigung der Pandemie usw.
            \item Eindrücke und Grußworte von Gästen aus Kulturbetrieben, dem Gesundheitswesen, der
                Wissenschaft (zum Beispiel Sozialforschung) oder Schulen und Kindergärten.
            \item Je nach Teilnehmerzahl Diskussion und Erarbeitung von Handlungsperspektiven als
                Studierendenschaft im Plenum oder in Kleingruppen.
            \item Gemütlicher Ausklang (Socialising)
        \end{itemize}
        Die Studierendenschaft sollte dies unterstützen durch:
        \begin{itemize}
            \item Finanzielle Unterstützung durch Social-Media und Printwerbung
            \item Verbreitung über ihre Kanäle
            \item Sonstige ideelle Beratung und Unterstützung (Raumanfragen, digitale Infrastruktur)
        \end{itemize}
        Das ist nur ein erster Vorschlag. Zentral ist aber, dass wir Analyse, Aktion und das Soziale
        Miteinander zusammen bringen.\\
        Dieser grobe Tagesordnungsvorschlag wird die Möglichkeit gegen ein Angebot an alle Studierenden
        zu machen. Zugleich könnten wir unsere studentischen Kämpfe für bessere Ausfinanzierung und
        Prüfungsbedingungen mit den aktuellen Herausforderungen in der Pandemiezeit finden. Und nicht
        zuletzt geben wir auch ein Angebot zum sozialen Austausch, der doch gerade jetzt, wichtiger denn je ist.
    }{
        Momentan nur Digital möglich, Organe der VS müssen zusammenarbeiten, AkLele, konkreter Fahrplan von Nöten\\
        Präsenz. Hybrid oder Online? Schon im Januar, Durchführung in Präsenz nicht absehbar\\
        Was sollen wir aus der Vollversammlung mitnehmen, bis jetzt nur politische Ziele, nichts für uns an der Uni\\
            Relativ wenig Auseinandersetzung mit Coona und den Hintergründen, \\
            Ziele: Solidarsemesterkampange, Demokratieabbau wegen wegfallenden Senatssitzungen, mehr Einsicht in Handlungen des Rektorats\\
            im Januar digital\\
            Studis zum mitarbeiten erreichen, Selbstermächtigungsgedanke\\
        Antrag enthält auch Finanzposten, über welchen Rahmen reden wir hier?\\
            mehr als 100€ facebook , 100€ flyer, 100€ essen wäre es nicht\\
        Wer setzt das denn um, die Referate?\\
            Zusammenarbeit mit Referaten und Gruppen, mit PoBi-referat oder aklele Mail über die Liste um Interessierte zu finden\\
        Wäre es möglich, eine Sondersitzung des Stura als Vollversammlung zu vermarkten\\
            Ist nicht das gleiche wie eine StuRa sitzung, etikett des StuRa soll wegewischt werden, unabhängig von den Strukturen der Vs, informelles Format bessere Wahl\\
            Extra Sitzung zu möglichen Corona-Hilfsangeboten, müsste gut vorbereitet sein\\
        Anzahl Teilnehende?\\
            50 - 200 Leute (Vermutung)\\
        was sollen die 200 Leute machen? Diskutieren oder Dinge beschließen? was soll das Bringen? Diskussiojne über ausgearbeitete Anträge, ansonsten Chaos, kein roter Faden\\
        Man kann nicht zu Vollversammlung einladen, müssen wir mit mehr als 200 Leuten rechnen (28000 Studierende), man könnte es auch einfach nicht Vollversammlung sonder Workshop oder Austauschtreffen nennen, Begriff der Vollversammlung nicht ganz trivial\\
        Frage scheint eher um Format zu gehen, Ausserordentliche Stur-Sitzung mit einem Austausch-TOP, explizit auch nicht-reguläre Mitglieder einladen.\\
        Meinungsbild anstatt beschluss an dieser Stelle, bei Mehrheit Mail-Verteiler / Telegramgruppe, unter Leitung des Vorsitz, 5-10 Leute notwendig für Orga\\
            Bewerbung der StuRa-Sitzung wichtig, offenes Format,Diskussion anhand von ausgearbeiteten Anträgen sinnvoll, vorformulierter Diskussionsgegenstand\\
        
    }{tba}{tba}{tba}
\GOantrag{Meinungsbild zu StuRa-Sondersitzung im Januar}{}{}{}{keine Gegenrede}{}{}{}
\GOantrag{Zurückspringen zu Queer-Referat-Kandidatur}{}{}{}{}{}{}{}
\antrag{
    Positionierung zu Verankerung von öffentlichem Tagen des Senats in der Verfahrensordnung der
    Ruprecht-Karls-Universität
    }{
        1. Lesung
    }{
        Juso Hochschulgruppe
    }{
        Die Verfasste Studierendenschaft fordert eine Änderung der Verfahrensordnung der
        Ruprecht-Karls-Universität Heidelberg, dass diese öffentliches Tagen des Senats,
        ähnlich wie in §3 GeschO-RefKonf und §6 GeschO-StuRa, festschreibt.
    }{
        Die Aufgaben des Senates sind im Landeshochschulgesetz festgelegt: "Der Senat entscheidet in
        Angelegenheiten von Forschung, Kunstausübung, künstlerischen Entwicklungsvorhaben, Lehre, Studium und
        Weiterbildung, soweit diese nicht durch Gesetz einem anderen zentralen Organ oder den Fakultäten
        zugewiesen sind." (§ 19 LHG). Die sehr weitreichenden Entscheidungen, die der Senat trifft,
        beeinflussen die Studierenden direkt. Deshalb sollten sie wenigstens eine Stimme haben können wenn
        diese getroffen werden.\\
        Auch braucht es Transparenz im Senat. Studierende sollten mitbekommen können, was gerade an ihrer
        Universität passiert, unter anderem auch damit sie bei der Senatswahl  gute Entscheidungen treffen
        können. Es scheint ein bisschen absurd, dass Studierende zwar ihre Vertrteter:innen wählen, aber dann
        aber nicht sehen, was diese in ihrem Namen abstimmen.\\
        Öffentliche Senatssitzungen sorgen dafür, dass sich Studierende mehr einbringen können, da sie die
        Funktionsweise des Senats im Detail kennen lernen können und sich so einbringen können, was zu
        besseren Entscheidungen führt. Dies nutzt auch dem Senat selbst.\\
        Wenn es natürlich Themen gibt, die sensibler sind,  dann würden Regelungen, wie sie z. B. der
        StuRa jetzt schon hat einen Ausschluss der Öffentlichkeit ermöglichen. Es spricht wenig gegen das
        öffentliche Tagen des Senats aber eine Menge dafür.
    }{
        \textbf{1. Lesung:}\\
        LHG regelt die Nichtöffentlichkeit des Senats, legiglich Ausnahmen sind vorgesehen, BW eines der Wenigen BL die das so vorsehen, Konkreter Vorschlag zur Änderung der Verfahrensordnung\\
            hier fehlt was\\
    }{tba}{tba}{tba}


   %\hyperref[https://www.stura.uni-heidelberg.de/fileadmin/Intern/Protokolle_und_Beschluesse/5/Beschluesse/Beschluesse_des_StuRa_5_Legislatur.pdf]{link}