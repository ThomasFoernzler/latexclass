\newcommand{\finanzantrag}[5]{%1title2antragssteller3antragstext4begründung5diskussion
    \subsection{#1}
    Antragssteller: #2
    \paragraph{Antragstext:}\phantom{spacer}\\
    #3
    \paragraph{Antragsbeschreibung:}\phantom{spacer}\\
    #4
    \paragraph{Diskussion:}\phantom{spacer}\\
    #5
}
\section{Finanzanträge}

\finanzantrag{Unterstützung der Campus Debatte Heidelberg}{Rederei e.V.}
{
    Der StuRa unterstützt Die Rederei Heidelberg e.V. bei der Ausrichtung der Campus Debatte Heidelberg vom 26.03 bis 28.03 2021.
}{
    \textbf{Infos zum Antragssteller:}\\
    Die Rederei e.V. ist ein in Heidelberg ansässiger Debattierclub. Seit 2001 vermitteln wir argumentative und rhetorische Fähigkeiten an Studierende aller Fachrichtungen. Wir glauben, dass Debattenkultur allen Menschen helfen kann, einen sachlichen und ergebnisoffenen Diskurs zu führen. Unsere Veranstaltungen sind nicht auf Studierende begrenzt. Auch andere junge Erwachsene wie SchülerInnen, Azubis oder Berufstätige sind bei uns willkommen. Wir treffen uns zwei Mal wöchentlich (aktuell über Zoom) für Debatten- und Trainingsabende und bieten auch darüber hinaus Seminare und Trainingseinheiten zur Verbesserung debattierrelevanter Fähigkeiten an.\\[1em]
    \textbf{Projektbeschreibung:}\\
    Die Campus Debatte Heidelberg ist Teil der Campus-Debatten-Turnierserie.  Diese besteht aus vier jährlich stattfinden Turnieren, welche nach der deutschsprachigen Debattiermeisterschaft die größten deutschsprachigen Debattierturniere sind. Hier messen sich die besten Debattierenden aus Deutschland, Österreich und der Schweiz im argumentativen Wettstreit über verschiedenste Themen. Von Politik und internationale Beziehungen über gesellschaftliche Fragen bis hin zu philosophischen Dilemmata ist das Themenfeld sehr weit. Über drei Tage finden fünf Vorrunden sowie die Halbfinals und das Finale statt.  Das Finale der Campus Debatte wird öffentlichkeitswirksam beworben und steht allen Interessierten offen. Die Veranstaltung ist kostenfrei und soll einen Einblick darin geben, wie ein geordneter, respektvoller und argumentativ hochwertiger Diskurs aussehen kann.\\
    Als Kooperationspartner haben wir den Dachverband VDCH, Verband der Debattierclubs an Hochschulen, an unserer Seite. Hierüber erhalten wir einen großen Teil der nötigen Fördergelder. Wir haben außerdem bereits die SRH als Partner gewinnen können, welche uns ihre Räumlichkeiten für die Vorrunden des Turniers zur Verfügung stellt. Im Gegenzug werden wir dort einzelne Trainings abhalten, um Studierende der SRH mit dem Debattieren vertraut zu machen.\\[1em]
    \textbf{Wer kann teilnehmen:}\\
    Teilnehmen kann, wer einem der über 70 Debattierclubs (davon zwei in Heidelberg und einer in Mannheim) angehört, die Mitglieder des Verbands der Debattierclubs an Hochschulen sind. Diese sind in Deutschland, Österreich und der Schweiz ansässig.\\
    Insgesamt werden etwa 100 Studierende an der Campus Debatte teilnehmen, etwa weitere 150 nicht debattierende Interessierte erwarten wir nach bisherigen Erfahrungen zum öffentlichen Finale. Wir werden selbst neben der Organisation hoffentlich noch einigen eigenen Teams aus jeweils drei Studierenden der Universität Heidelberg die Chance geben können, an dem Turnier teilzunehmen.\\
    Außerdem werden Teams des anderen Heidelberger Debattierklubs sowie des Mannheimer Debattierclubs antreten.\\[1em]
    \textbf{Erwartete Ergebnisse:}\\
    Wie auch in den letzten Jahren ist es unser Ziel, die besten Debattierenden Deutschlands zu finden und zu küren. Im Vordergrund steht uns aber auch, in Heidelberg durch die öffentliche Finalveranstaltung ein Bewusstsein für Debattenkultur zu schaffen. Gerne würden wir auch über die Region hinaus Debattierclubs stärken, indem wir auf diese aufmerksam machen. Wir glauben, dass eine Debattenkultur, in der das beste Argument Gehör findet, einen wichtigen Gegenpol zum aktuellen politischen Klima darstellt. Durch eine weitere Verbreitung des Debattiersports glauben wir, dass junge Erwachsene im Privaten wie im Beruflichen respektvoller miteinander zu diskutieren lernen.\\[1em]
    \textbf{Antragsbegründung:}\\
    Die vier Turniere der Campus Debatten-Serie sind nach der deutschen Debattiermeisterschaft die wichtigsten Veranstaltungen der studentischen Debattierszene im deutschsprachigen Raum. Sie bieten neben dem kompetitiven Turnier einen Ort zur Vernetzung und zum Treffen wichtiger Entscheidungen für das kommende Jahr.\\
    Neben einem kompetitiven Charakter hat ein solches Turnier aber auch einen höchst integrativen Charakter, da die Teilnahme nicht an irgendwelche Qualifikationen gebunden ist, sondern jedem Mitglied eines der vielen Debattierclubs offensteht. Um die Teilnahme nun tatsächlich allen zu ermöglichen, sind wir auf zahlreiche Sponsoren angewiesen. Diese ermöglichen es Jahr für Jahr, den Teilnahmebeitrag in einem angemessenen Rahmen zu halten. Aktuell planen wir hier mit 30 € pro Person.\\[1em]
    \emph{Zur Begründung der Unterstützung nicht-Heidelberger Studis:}\\
    Wir denken, diese Unterstützung beruht auf einem Geben-und-Nehmen-Prinzip. So war es auch in den letzten Jahren der Fall, dass die StuRas, StuPas, Astas (und was es sonst noch so gibt) der Ausrichteruniversitäten diese Turniere stets unterstützt haben. Von dieser Unterstützung profitieren jährlich viele Heidelberger Studierenden, sodass wir glauben, dass es legitim ist, dass in diesem Jahr die Verfasste Studierendenschaft Heidelbergs die Studierenden anderen Universitäten bei ihrem Aufenthalt in Heidelberg unterstützt.\\[1em]
    \textbf{Finanzvolumen:}\\
    Wir beantragen beim StuRa Unterstützung in Höhe von 2000 €.\\[1em]
    Ansonsten beantragen wir keine Mittel bei der verfassten Studierendenschaft. Der Teilnahmebeitrag wird am Ende so gewählt werden, dass er die restlichen noch nicht gedeckten Kosten abdeckt. Zurzeit planen wir mit ungefähr 30 €. Es laufen aktuell noch Anfragen bei der Leonie-Wild-Stiftung, der Manfred-Lautenschläger-Stiftung und der Stadt-Heidelberg-Stiftung sowie beim Lions Club Heidelberg – etwa 30 weitere Anfragen wurden inzwischen leider abgelehnt. Sollten wir weitere Unterstützer finden können, werden wir am Ende auch weniger Geld vom StuRa nutzen. Dies war bereits vor 2 Jahren bei der von uns ausgerichteten deutschsprachigen Debattiermeisterschaft der Fall, als wir vom StuRa 5000€ bewilligt bekamen und am Ende nur etwa 1500€ in Anspruch genommen haben.\\[1em]
    \begin{tabular}{l l}
        Wieviel beantragt ihr beim Studierendenrat?                             & 2000 €                     \\
        Wieviel wird bei der Verfassten Studierendenschaft insgesamt beantragt? & Keine weiteren Anträge    \\
        Wieviel wird über Mittel weiterer Stellen finanziert?                   & Bisher 7176,39 €           \\
        Habt ihr Einnahmen bei der Veranstaltung?                               & Nein                      \\
        Wie hoch ist das Gesamtvolumen des Projekts                             & Aktuelle Planung: 12.128 € \\
    \end{tabular}
    \newline
    \vspace*{2em}
    \newline
    \begin{tabular}{c c p{10cm}}
        \textbf{Verwendungszweck} & \textbf{Kosten} & \textbf{Begründung} \\
        Verpflegung & 2550€ & \\
        Unterkunft & 6900€ & Hier würden wir die StuRa-Unterstützung anrechnen.\\
        Finale & 430€ & Ein öffentliches Finale (inklusive der Ehrenjury, für welche Reisekosten unter dem Punkt Transport anfallen) ist eine der Bedingungen der Hauptförderer unseres Dachverbandes (dies sind die Zeit-Stiftung Ebelin und Gerd Bucerius sowie die Karl-Schlecht-Stiftung). \\
        Transport & 700€ & \\
        Socials & 400€ & Werden coronabedingt ggf ausfallen und sind noch nicht geplant. Das Geld ist für eventuelle Raummieten eingeplant. Kosten vor Ort (z.B. Getränke) müssten die Teilnehmenden hier selbst zahlen. \\
        Sonstiges & 648€ & Hierunter fallen Druckkosten, eine Veranstaltungsversicherung und ein Sicherheitspuffer. Sollte der Sicherheitspuffer nicht gebraucht werden, werden wir entsprechend weniger Förderung in Anspruch nehmen. \\ 
        \textbf{Gesamt} & \textbf{12.128€} & \\       
    \end{tabular}
}{
    \textbf{1. Lesung:}\\
    schon vor 2 Jahren nur ca 1500€ von 5000€ abgerufen, weil ncoh weiter Sponsoren gefunden wurden

}
\finanzantrag{Studieren ohne Grenzen}{Studieren Ohne Grenzen Heidelberg e.V.}
{
    Der StuRa unterstützt die Arbeit der Hochschulgruppe Studieren Ohne Grenzen Heidelberg e.V. und stellt 575 € für Werbe- und Eventausgaben im ersten Halbjahr 2021 zur Verfügung. Die Veranstaltungen umfassen Informationsabende sowie eine Lesung und/oder Podiumsdiskussion mit dem Schwerpunkt Internationale Zusammenarbeit/Postkoloniale Perspektiven.
}{
    \textbf{Infos zum Antragssteller:}\\
    Wir sind der studentische, gemeinnützige Zweigverein „Studieren Ohne Grenzen Heidelberg e.V.“ Unsere Mitglieder sind Studierende der Universität Heidelberg.\\
    Homepage unseres Vereins: \url{https://www.studieren-ohne-grenzen.org/lokalgruppen/heidelberg/}\\[1em]
    \textbf{Projektbeschreibung und Antragsbegründung:}\\
    Wir veranstalten jedes Semester mehrere Informationsabende, um Studierende der Universität Heidelberg über unsere Arbeit zu informieren. Dies beinhaltet insbesondere Informationen über unsere Projektarbeit mit konkreten Länderbeispielen, wie z.B. die Bildungssituation im Norden Sri Lankas, sowie die Möglichkeit sich mit unserer Arbeit und Zielen auseinanderzusetzen. Teil unseres Vereinszweckes ist es über die Bildungssituation in verschiedenen Teilen der Welt zu informieren und ein Bewusstsein über die damit zusammenhängenden Probleme und Weiterentwicklungsmöglichkeiten zu schaffen. Wir ermutigen dabei alle Studierende, sich und ihre Ideen entweder in Form eines Engagements in unserem Verein oder aber in Form der Diskussion bei einer unserer Veranstaltungen einzubringen.\\
    Regelmäßig veranstalten wir öffentlichkeitswirksame Vorträge, Lesungen und Filmabende zu bildungspolitischen und projektbezogenen Themen in Kooperation mit anderen Initiativen und Hochschulgruppen. Leider mussten wir aufgrund der derzeit geltenden Kontaktbeschränkungen unseren für 2020 geplanten Workshop zu postkolonialer Reflexion absagen. (Für diesen hatte uns der StuRa bereits Unterstützung zugesagt). Da auch für die erste Jahreshälfte 2021 schwer absehbar ist, ob Veranstaltungen mit Präsenz stattfinden können, planen wir ein Format, das entweder virtuell oder mit Anwesenheit umgesetzt werden kann. Hier haben wir bereits in Zusammenarbeit mit dem Hauptverein Studieren Ohne Grenzen Deutschland e.V. Erfahrungen sowohl mit einer Präsenz-Lesung im Jahr 2019 sowie mit einer Online-Lesung mit Ronja Wurm-Seibel (Autorin von „Ausgerechnet Kabul“; Facebook-Veranstaltung: \url{https://www.facebook.com/events/2716619911988037/)} sammeln können. Zusätzlich können wir auf die Erfahrungen von anderen lokalen Hochschulgruppen zurückgreifen (z.B. \url{https://www.facebook.com/events/711780249441520}). Auf Basis dieser Erfahrungen rechnen wir mit einem Honorarbetrag von bis zu 500 €. Darüber hinaus möchten wir diese Veranstaltung sowie unsere Informationsabende, die zu Anfang des Semesters stattfinden, gerne über Facebook bewerben.\\
    Die finanzielle Unterstützung für die Veranstaltung sowie für die Werbung, würden es uns ermöglichen Vortragende einzuladen wie z.B. noch einmal Ronja Wurm-Seibel oder andere Persönlichkeiten, deren Bericht oder Lesung für viele Teilnehmende interessant ist, die aber auch ein angemessenes Honorar gezahlt bekommen möchten.\\
    Unsere Events richten sich an alle Heidelberger Bürger*innen, insbesondere jedoch an Studierende. Wir rechnen damit, dass an unseren Informationsabenden, der Lesung/Podiumsdiskussion sowie weiteren Aktionen im ersten Halbjahr 2021 insgesamt bis zu ca. 150 Personen teilnehmen. Unsere Werbemaßnahmen erreichen die meisten potentiellen Teilnehmenden und informieren diese. Um unsere kulturellen und bildungspolitischen Veranstaltungen möglichst interessant gestalten und können und möglichst vielen Studierenden die Teilnahme zu ermöglichen, beantragen wir eine Unterstützung in der Höhe der oben genannten Summe.\\[1em]
    \textbf{Finanzvolumen des Antrags:}\\
    \newline
    \begin{tabular}{l l}
        Wieviel beantragt ihr beim Studierendenrat?                             & 575€                     \\
        Wieviel wird bei der Verfassten Studierendenschaft insgesamt beantragt? & 575€    \\
        Wieviel wird über Mittel weiterer Stellen finanziert?                   &           \\
        Habt ihr Einnahmen bei der Veranstaltung?                               & Nicht geplant                      \\
        Wie hoch ist das Gesamtvolumen des Projekts                             & 575€ \\
    \end{tabular}
    \newline
    \vspace*{2em}
    \newline
    \begin{tabular}{p{4cm} p{2cm} p{9cm}}
        \textbf{Verwendungszweck} & \textbf{Kosten} & \textbf{Begründung} \\
        Online-Werbung (Facebook) & 50€ & Aufgrund der derzeitigen Pandemie-Lage planen wir das Bewerben unserer Veranstaltungen auf Facebook-Werbung zu beschränken. Hiermit haben wir im vergangenen Jahr gute Erfahrungen gemacht. \\
        Honorar für Lesung bzw. Podiumsdiskussion \newline (Schwerpunkt: Internationale Zusammenarbeit/Postkoloniale Perspektive) & 500€ & Wir planen eine Lesung ähnlich zu dem diesjährig stattfindenden Format mit Ronja Wurm-Seibel und/oder eine Podiumsdiskussion mit dem Schwerpunkt Internationale Zusammenarbeit. Dies findet je nach der Entwicklung der Corona-Pandemie in Präsenz oder virtuell statt.\\
        Büromaterial (u.a. Briefmarken, Papier, Folien) & 25€ & \\
        \textbf{Gesamt} & \textbf{575€} & \\      
    \end{tabular}
    \newline
    \emph{Weitere Informationen}\\
    Wir werden den Rahmen unseres Finanzierungsantrags vom 09.Juni 2020 bei Weitem nicht ausschöpfen. Wir mussten unseren geplanten Workshop zu postkolonialer Reflexion absagen bzw. verschieben und haben aufgrund der derzeitigen Lage keinerlei „physische“ Werbung (d.h. Plakate, Flyer) gedruckt.
}{
    \textbf{1. Lesung:}\\
    keine Fragen
}
\finanzantrag{Club für Wissenschaft und Kultur}{Heidelberger Club für Wirtschaft und Kultur e.V.}
{
    Der StuRa unterstützt die Durchführung des 3-tägigen Heidelberger Symposiums 2021 unter dem Motto Unruhe bewahren. Das Symposium umfasst Vorträge, Diskussionen, Kolloquien sowie ein kulturelles Rahmenprogramm. Aufgrund der Pandemielage wird das Symposium 2021 voraussichtlich als Digitalveranstaltung stattfinden – mit Hybrid-Option aus Präsenz und Digital, falls es die Pandemie-Lage zulässt.
}{
    \textbf{Infos zum Antragssteller:}\\
    \url{www.hcwk.de}\\
    Der Heidelberger Club für Wirtschaft und Kultur e.V. (HCWK) ist eine unabhängige, überparteiliche und fachübergreifende Studierendeninitiative. Er wurde 1988 mit dem Ziel gegründet, die Ausbildung an den Universitäten durch Praxisbezug und interdisziplinären Austausch zu ergänzen. Zu diesem Zweck organisiert der Club jährlich ein mehrtägiges Forum zu einem aktuellen Thema von gesellschaftlicher Relevanz. Der Heidelberger Club ist als gemeinnütziger Verein anerkannt. Seine Tätigkeit wird durch Spenden, den Verzicht der Referierenden auf Honorare sowie das ehrenamtliche Engagement der Organisierenden getragen. Förderer aus der Wirtschaft und dem Stiftungswesen sowie ein hochkarätig besetztes Kuratorium unterstützen den Club finanziell und ideell.\\[1em]
    \textbf{Projektbeschreibung und Antragsbegründung:}\\
    Das 32. Heidelberger Symposium widmet sich dem Thema Unruhe bewahren und findet vom 20. bis 22. Mai 2021 unter der Schirmherrschaft von MEP Nico Semsrott statt. Wir erwarten erneut mehr als 1000 Teilnehmende und rund 40 bedeutsame Persönlichkeiten als Referierende. Gemeinsam mit allen Teilnehmenden möchten wir uns mit einer Welt befassen, die sich fortwährend in Unruhe befindet und uns auf die Dualität von Bewegung und Stillstand, von Aktivität und Passivität, von Fortschritt und Konservatismus einlassen. Das Symposium verspricht einen angeregten Diskurs mit einer Vielzahl an politischen, gesellschaftlichen, kulturellen und naturwissenschaftlichen Themen – auch als Digitalveranstaltung.\\
    Das Symposium richtet sich an Studierende aller Studienfächer und Fakultäten. Wie auch in den letzten Jahren erwarten wir etwa 1000 Studierende aus Heidelberg – sei es als reine Digitalveranstaltung oder als Hybrid-Veranstaltung aus Präsenzvorträgen und Streaming. Das Heidelberger Symposium ist mit das größte studentische Symposium Deutschlands und somit einzigartig für den interdisziplinären und interfakultären Austausch von Studierenden. Es fördert die Weiterbildung von Studierenden über das eigentliche Studium hinaus.\\[1em]
    \textbf{Finanzvolumen des Antrags:}\\
    4.500,00 €\\
    Insgesamt rechnen wir mit Kosten von rund 37.000,00 €, die hauptsächlich über Fördermittel und Mitgliedsbeiträge gedeckt werden müssen. Je höher die Fördermittel ausfallen, desto günstiger werden die Ticketpreise für Studierende. Falls das Symposium ausschließlich digital stattfinden sollte, möchten wir die Veranstaltung kostenlos für Studierende anbieten, soweit es unsere Finanzen zulassen. Eine Förderung durch den StuRa kommt daher direkt den Studierenden zu Gute und erlaubt uns auch unabhängiger von Sponsoren zu agieren.\\[1em]
    \textbf{Verwendungszweck der Mittel:}\\
    Die beantragten Fördermittel durch den StuRa sollen für Werbeausgaben (Flyer, Plakate) und für die vegane, nachhaltige und ökologische Verpflegung (Getränke, Lunchpakete) der Teilnehmenden und Referierenden im Falle einer Hybridveranstaltung sowie evtl. anfallende Fahrtkosten der Referierenden eingesetzt werden.\\
    \begin{longtable}{p{12cm} p{3cm}}
        \endfirsthead
        \endhead
        \endfoot
        \endlastfoot
        \textbf{Ausgaben} & \\
        \textbf{Vereins- und Bürobetrieb} & \\
        Miete der Vereins- und Büroräume (12 Monate) & 1.800,00 € \\
        Bürobedarf & 380,00 € \\
        Telekommunikation, Serverkosten, Onlineauftritt & 1.300,00 € \\
        Laufende Amtskosten, Bankgebühren & 1.260,00 € \\
        \textbf{Mittel- und langfristige Vorbereitung des Symposiums} & \\
        Öffentlichkeitsarbeit und Kuratorenbetreuung & 200,00 € \\
        Zwei Strategiewochenenden inkl. Verpflegung und Unterkunft & 2.200,00 €\\
        Corporate Design: Ausschreibung des Heidelberger Kunst- und Kulturpreis 2021 & 500,00 €\\
        Langfristige Werbeausgaben im Vorfeld inkl. frühzeitiger Werbemaßnahmen - Druckerzeugnisse ausgenommen & 4.000,00 €\\
        \textbf{Durchführung eines Digital-Symposiums} & \\
        Versicherung für das Symposium inkl. laufender Versicherungen & 1.420,00 €\\
        Genehmigungen und Gebühren & 400,00 € \\
        Stromversorgung & 250,00 € \\
        Simultanverdolmetschung von Vorträgen inkl. Miete der Konferenztechnik & 500,00 €\\
        Öffentlichkeitsarbeit und Pressearbeit während und nach dem Symposium & 200,00 €\\
        Technik inkl. Gebühren für Onlineplattform & 5.000,00 €\\
        Kulturelles Rahmenprogramm (u.a. Filmvorführung, Science Slam) & 1.000,00 €\\
        Werbeausgaben für Druckerzeugnisse (u.a. Flyer, Plakate, Dreifaltblätter) & 4.500,00 €\\
        \textbf{Extrakosten für Hybrid-Veranstaltung}&\\
        Miete der Veranstaltungsräume und Zelte inkl. Transport, Auf- und Abbau, Genehmigungen, Technik, Stromversorgung, Mülltonnen, Feuerlöscher & 3.500,00 €\\
        Miete Kühlschränke & 500,00 €\\
        Dekoration für Zelt und Veranstaltungsräume & 250,00 €\\
        evtl. Fahrtkosten für Referierende & 2.500,00 €\\
        evtl. Übernachtungskosten für Referierende & 800,00 €\\
        Teilnehmendenhandbücher und Willkommenstaschen & 1.000,00 €\\
        Verpflegung der Teilnehmenden und Referierenden während der drei Veranstaltungstage (Lunchpakete, Getränke) & 1.500,00 €\\
        Sicherung des Geländes bei Nacht durch einen Sicherheitsdienst & 560,00 €\\
        Hygienekonzept und Hygieneeinhaltung & 1.500,00 €\\
        \textbf{Ausgaben Gesamt} & 37.020,00 €\\
        \\
        \textbf{Einnahmen}&\\
        Mitgliedsbeiträge & 4.850,00 €\\
        Fördermittel durch Sponsoring & 13.550,00 €\\
        Fördermittel durch Stiftungen und Spenden & 14.050,00 €\\
        \textbf{Einnahmen Gesamt} & 32.450,00 €\\
        \\
        Ausgaben & 37.020,00 €\\
        Einnahmen & -32.450,00 €\\
        \textbf{Noch benötigte Fördersumme} & 4.570,00 €\\
    \end{longtable}
    \vspace*{2em}
    \emph{Weitere Informationen}\\
    Die beantragte Fördersumme soll hauptsächlich für Mehrkosten bei der Durchführung einer Hybrid-Veranstaltung (Verpflegung, Fahrtkosten) ausgegeben werden. Falls es bei einer reinen Digitalveranstaltung bleibt, sollen die Fördermittel für Werbekosten genutzt werden, um die Veranstaltung kostenlos anbieten zu können.
}{
    Diskussion
}
\subsection{Collegium Musicum Heidelberg (zurückgezogen)}
Der Antrag wurde vom Antragssteller zurückgezogen.
\finanzantrag{Konfliktbarometer}{Heidelberger Institut für Internationale Konfliktforschung e.V.}
{
    Der StuRa unterstützt den Druck des Konfliktbarometers 2020 des Heidelberger Instituts für Internationale Konfliktforschung (HIIK) mit 2.500 Euro.\\
    Das Konfliktbarometer enthält Übersichtsgrafiken, Konfliktkarten, regionale Einführungstexte, Kurzberichte zu ausgewählten Konflikten und Daten aller im Beobachtungsjahr 2020 bearbeiteten Konflikte.
}{
    \textbf{Infos zum Antragssteller:}\\
    \url{https://hiik.de}\\[1em]
    \textbf{Projektbeschreibung und Antragsbegründung:}\\
    Das Heidelberger Institut für Internationale Konfliktforschung (HIIK) ist ein unabhängiger, gemeinnütziger und interdisziplinärer Verein. Seit nunmehr 30 Jahren erforscht und dokumentiert das HIIK politische Konflikte weltweit. Die dabei erzielten Ergebnisse veröffentlichen wir jährlich im Conflict Barometer (CoBa), das wir kostenlos auf unserer Homepage zum Download bereitstellen und in Form von Druckexemplaren unseren Mitarbeitern, dem Advisory Board, sowie Kooperationspartnern und anderen nationalen und internationalen Institutionen zur Verfügung stellen. Es enthält Übersichtsgrafiken, Konfliktkarten und regionale Einführungstexte, sowie Kurzberichte zu ausgewählten Konflikten. Das CoBa wird zu Beginn des Folgejahres veröffentlicht und illustriert neben der aktuellen internationalen Konfliktlage auch die Entwicklung der Konflikte und Regionen im Zeitverlauf. Die Daten des HIIKs werden unter anderem von staatlichen und internationalen Organisationen, Nichtregierungsorganisationen, im wissenschaftlichen Bereich als auch in der Schulbildung genutzt. Neben der Publikation des Konfliktbarometers aktualisiert und pflegt das HIIK fortlaufend seine Datenbank CONTRA.\\
    Neben der praktischen Implementation unserer Ergebnisse in Zusammenarbeit mit dem Auswärtigen Amt, dem Bundeskriminalamt und internationalen Organisationen wie der UN, EU und Weltbank, ist das HIIK der Förderung der Wissenschaft sowie der Weitergabe seines Wissens an zivilgesellschaftliche Gruppen, SchülerInnen, StudentInnen und die interessierte Öffentlichkeit verpflichtet. Auch mit Hilfsorganisationen, wie beispielsweise der “Action Contre la Faim”, arbeitet das HIIK zusammen.\\
    Das HIIK trägt das Bekenntnis zu seiner Herkunft und seinem Sitz im Namen. Es ist ein studentischer und eigenständiger Verein, dem der Anspruch und Auftrag, Wissen über das Aufkommen, den Austrag und die Resolution politischer Konflikte weltweit zu verbreiten, am Herzen liegt. Wir möchten in Studierenden und Interessierten Begeisterung und Verständnis für die Konfliktforschung wecken und fördern. Auch soll Sensibilität für die Bedeutung politischer Konflikte für globale politische Zusammenhänge gefördert werden. Durch unseren Sitz in Heidelberg besteht eine besondere Bindung zu weiteren studentischen Initiativen wie Ruperto Carola, Galileo Consult, und FiS, der Stadt Heidelberg und der Metropolregion im Allgemeinen.\\
    Für Seminargruppen oder an Gymnasien der Region sind unsere ExpertInnen gefragte Vortragende und leiten Workshops, die sich wahlweise mit aktuellen Konflikten oder globalen Konflikt-Trends beschäftigen. Für das Regierungspräsidium Freiburg bot das HIIK Lehrerfortbildungen an, um die Unterrichtsgestaltung des neuen Moduls “Konflikt und Frieden” in Gemeinschaftskunde der Kursstufe mit Daten und Material des HIIK anzureichern. International geben wir unser Wissen in Form von Gastvorlesungen, Vorträgen, Interviews und Zeitungsberichten weiter. Im letzten Jahr haben wir MitarbeiterInnen eines Think-Tanks aus Addis Abeba, Äthiopien, empfangen, die unsere Daten zur Erforschung der Friedenseinsätze der Afrikanischen Union verwenden.\\[1em]
    \textbf{Warum den Druck unterstützen?}\\
    Ein Verein für Studierende\\
    Mit der Unterstützung des CoBa-Drucks wird nicht nur unser Verein unterstützt, sondern vor allem Studierende aus Heidelberg, Deutschland und der ganzen Welt. Das HIIK bietet für Studierende verschiedenster Fachrichtungen eine einzigartige Gelegenheit, neben ihrem Studium wissenschaftliche und praktische Erfahrungen zu sammeln. Das Konfliktbarometer ist das Ergebnis von intensiver Arbeit und Zusammenarbeit dieser Studierenden und trägt weiterführend ebenfalls zur Bildung von Studierenden im Allgemeinen bei. Auch wenn ein Großteil der MitarbeiterInnen aus den Sozial- und Geisteswissenschaften kommen, schätzt das HIIK als interdisziplinäre Forschungseinrichtung den Beitrag aus anderen Studiengängen sehr. Damit ermöglicht das HIIK Studierenden eine anwendungsbezogene Spezialisierung. Unser Ziel ist es dabei, unsere MitarbeiterInnen methodisch, geographisch und konflikttheoretisch zu ExpertInnen auszubilden, was zudem oftmals mit dem Erlernen einer oder mehrerer Fremdsprachen einhergeht. Gemäß dem Motto “Dem lebendigen Geist” der Universität Heidelberg ermöglichen wir ihren Studierenden eine breit angelegte Zusatzqualifikation, die sie noch während des Studiums weit über den Tellerrand blicken lässt und sie en passant zu SpezialistInnen für die „eigenen“ Konflikte sowie die bearbeiteten Länder macht. Dies bedeutet für unsere Mitglieder, dass sie bereits während des Studiums in einer international anerkannten Fachzeitschrift publizieren können. Durch unser mittlerweile großes Netzwerk können unsere Mitglieder Praktikumsplätze, etwa bei Botschaften oder internationalen Organisationen, leichter erhalten. Das HIIK fördert dabei den internationalen Wissenstransfer und ermöglicht es den MitarbeiterInnen “ihre” Konfliktregion kennenzulernen.\\
    Zu den wissenschaftlichen und forschungsbezogenen Vorteilen, hat uns die finanzielle Unterstützung durch den StuRa in vergangenen Jahren ermöglicht, einzelne Veranstaltungen auszurichten, die der Studierendenschaft als Ganzem zugutekamen. Beispielsweise wurde 2019 ein Workshop zum Thema „Counting the Dead“ eines Professoren aus Paris finanziert, sowie eine Konferenz mit Wissenschaftlern aus Addis Ababa ermöglicht. Mit der finanziellen Unterstützung würde der Studierendenrat dementsprechend einerseits die Arbeit eines zunehmend renommierten Vereins maßgeblich unterstützen und andererseits Studierenden die Möglichkeit bieten zusammenzukommen, sich auszubilden und zu engagieren.\\
    Finanzielle Unabhängigkeit und Planungssicherheit\\
    Als gemeinnütziger Verein finanzieren wir uns nahezu ausschließlich aus Mitgliedsbeiträgen und Spenden, zu einem kleinen Teil auch aus Vortragshonoraren. Unsere aktuell etwa 200 MitarbeiterInnen, die auf ehrenamtlicher Basis arbeiten, finanzieren somit maßgeblich die Projekte des Vereins. Als unabhängiger und gemeinnütziger Verein sind die Finanzierungsformen, die für uns in Frage kommen, eingeschränkt, weshalb wir finanziell und organisatorisch zunehmend an unsere Grenzen stoßen. Die finanziellen Mittel sind dementsprechend notwendig, um anschließend unsere Ergebnisse (in Form des CoBas) der Wissenschaft und Gesellschaft kostenfrei und breitest-möglich zur Verfügung zu stellen.\\
    Der Druck des Konfliktbarometers wurde in den vergangenen Jahren durch den Studierendenrat bezuschusst: Im Jahr 2017 mit 3.597,02 €, 2018 mit 2.979,95 €, 2019 mit 500,00 € und 2020 mit 2.500€. Das Logo des Studierendenrates wurde in diesen Jahren im Konfliktbarometer abgedruckt. Die Finanzierung hat uns ermöglicht ein zentrales Charakteristikum des HIIK, die Unabhängigkeit von Finanziers, zu wahren, was für den Wert unserer Arbeit von herausragender Bedeutung ist.\\
    Obwohl das HIIK versucht nach Möglichkeit ebenfalls andere Projekte und Veranstaltungen zu organisieren, ist der CoBa-Druck unser Hauptausgabeposten. Die verbleibenden Mittel werden anschließend für sonstige Projekte ausgegeben, die wiederum ebenfalls der Studierendenschaft zugutekommen. Die Unterstützung durch den StuRa erhöht unsere Planungssicherheit maßgeblich und ermöglicht die Organisation von mehr Projekten und Veranstaltungen für alle Studierenden. Da diese Veranstaltungen von uns erwünscht sind, allerdings abhängig sind von den uns zur Verfügung stehenden Ressourcen, wäre deshalb die Unterstützung des CoBa-Drucks in hohem Maße hilfreich (und womöglich für den Budgetplan des StuRa vorteilhafter).\\[1em]
    \textbf{Der Druck des CoBa}\\
    Das gedruckte Konfliktbarometer erfüllt verschiedene Zwecke, die für unseren Verein und auch unsere Mitglieder von großer Bedeutung sind. Erstens wird ein Teil der gedruckten Exemplare unseren ehrenamtlichen Mitarbeitern zur Verfügung gestellt. Neben dem Vorteil, damit händisch und unabhängig auf vergangene Ergebnisse zurückgreifen zu können, ist dies für uns eine wichtige Möglichkeit, um die Arbeit unserer Mitarbeiter zu honorieren. Die Auflagenhöhe wird dabei stets der Nachfrage unserer Mitarbeiter angepasst, weshalb immer nur so viele Druckexemplare in Auftrag gegeben werden, wie Interesse besteht. Hierzu gehören ebenfalls die Exemplare, die an unser Advisory Board geschickt werden, die ebenfalls ehrenamtlich maßgeblich zu der Veröffentlichung des CoBas beitragen und dessen wissenschaftlichen Wert bedeutend erhöhen. Ein gedrucktes Exemplar des CoBa ist unserer Ansicht nach daher eine Selbstverständlichkeit und bietet zudem die Möglichkeit, unseren Dank und unsere Wertschätzung zum Ausdruck zu bringen und somit wertvolle Kontakte und Kooperationen aufrechtzuerhalten. Zuletzt sind die gedruckten Exemplare auch für ein breiteres Publikum gedacht, das der Verein sich in den vergangenen 30 Jahren hat aufbauen können. Nicht nur werden die gedruckten Konfliktbarometer von nationalen und internationalen Institutionen angefragt, sie dienen auch dem Zweck unsere öffentliche Wahrnehmung und damit einhergehend die der Heidelberger Studierendenschaft zu erweitern (durch deren Verteilung an bspw. Bibliotheken und bei Konferenzen). Ein rückläufiger Austausch mit ExpertInnen und internationalen Organisationen würde im Umkehrschluss auch auf die Studierenden zurückfallen und die Möglichkeiten des Vereins einschränken.\\[1em]
    \textbf{Finanzvolumen des Antrags:}\\
    \newline
    \begin{tabular}{l l}
        Wieviel beantragt ihr beim Studierendenrat?                             & 2.500€       \\
        Wieviel wird bei der Verfassten Studierendenschaft insgesamt beantragt? & 2.500€    \\
        Wieviel wird über Mittel weiterer Stellen finanziert?                   & 1.500€ (Mitgliedbeiträge)      \\
        Habt ihr Einnahmen bei der Veranstaltung?                               & -\\
        Wie hoch ist das Gesamtvolumen des Projekts                             &  4000€ (letztes Jahr: 3995,95€)\\
    \end{tabular}
    \newline
    \vspace*{2em}
    \newline
    \begin{tabular}{p{4cm} p{2cm} p{9cm}}
        \textbf{Verwendungszweck} & \textbf{Kosten} & \textbf{Begründung} \\
        Druckkosten des Konfliktbarometers 2020 & 4000€ & (siehe oben)\\
        \textbf{Gesamt} & \textbf{4000€} & - \\  
    \end{tabular}
}
{
    \textbf{1. Lesung:}
    \ul{\li{Stura hat schon öfters den Druck finanziert, bekommt edr Stura eine Ausgabe davon?}
        \ul{\lii{Können gerne auch eine an Stura schicken, Stura hat Druck die letzen jahre teilfinanziert, stura wird auch im koba erwähnt}}
    }
}
\subsection{Weihnachtsgesteck}
Der Antrag wurde vom Antragssteller zurückgezogen.

\finanzantrag{Law NMUN}{Heidelberg Law NMUN e.V.}
{
    Der StuRa unterstützt die Teilnahme einer Heidelberger Delegation an der C’MUN 2021, sowie das damit zusammenhängende Vorbereitungsprogramm. Die Organisator*innen der C’MUN planen die Konferenz derzeit vom 22.-25. April 2021. Die Zeitangabe erfolgte auf Rückfrage und ist daher noch nicht offiziell, je nach Situation kann sich der Zeitraum noch verschieben. Sollte die C’MUN entgegen den Erwartungen überhaupt nicht oder nur online stattfinden, so werden wir alles daransetzen, die Teilnahme an einer vergleichbaren Konferenz im europäischen Ausland zu ermöglichen.

}{
    \textbf{Infos zum Antragssteller:}\\
    Wir stellen den Antrag im Namen des studentischen Vereins Heidelberg Law NMUN e.V. Dieser hat zum Ziel, Studierenden realitätsnahe Eindrücke von der Arbeitsweise der Vereinten Nationen zu vermitteln. Der Verein nimmt seit nunmehr neun Jahren an der jährlichen National Model United Nations Konferenz (NMUN) am Hauptsitz der Vereinten Nationen in New York City teil. Um das pandemiebedingte Kostenrisiko einzugrenzen wird im kommenden Sommersemester jedoch eine Konferenz in Europa besucht werden.\\
    Das Projekt ist in Form eines gemeinnützigen Vereins organisiert. Der Verein veranstaltet neben einer intensiven Vorbereitung der Konferenz auch regelmäßig Vorträge und sonstige Veranstaltungen zu verschiedenen weltpolitischen, rechtlichen und kulturellen Themen. Hierzu zählten in den vergangenen Jahren beispielsweise ein Vortrag und Rundgang im Patrick-Henry-Village zum Thema „Koordination der Aufnahme von Geflüchteten“, sowie eine Diskussionsrunde mit Professor Syed Imad-ud-Din Asad, LL.M. (Harvard) von zum Thema „Europe and Islamic Law“. Für das Wintersemester 2020/21 ist unter anderem ein Workshop mit der Europäischen Zentralbank zum Thema Finanzpolitik in Planung.\\
    \url{https://www.heidelberg-law-nmun.org/}\\[1em]
    \textbf{Projektbeschreibung und Antragsbegründung:}\\
    \begin{enumerate}
        \item \textbf{Die Konferenz}\\ Die C’MUN wird jährlich an der Universitat Pompeu Fabra in Barcelona ausgetragen. Sie ist das älteste und größte Model United Nations in Südeuropa und zählt international zu den renommiertesten MUN-Konferenzen. Etwa 250 Teilnehmerinnen und Teilnehmer von bis zu 60 Universitäten aus der ganzen Welt werden in 6 Komitees debattieren und Resolutionen erarbeiten. MUN Simulationen wie die C’MUN geben den Teilnehmenden einen authentischen Einblick in die Arbeitsweise der Vereinten Nationen und die hiermit verbundenen Sachzwänge der internationalen Diplomatie. Neben einem professionellen Gebrauch der englischen Sprache entwickeln die „Delegates“ in besonderem Maße Verhandlungsgeschick, Teamfähigkeit und ein souveränes Auftreten. Die Teilnehmerinnen und Teilnehmer wachsen persönlich und fachlich an den Erfahrungen und haben dabei auch die Möglichkeit, Auszeichnungen zu erhalten. Typischerweise ergeben sich bei den Konferenzen innerhalb kürzester Zeit Begegnungen mit einer Vielzahl anderer Studierender aus verschiedensten Ländern und kulturellen Hintergründen in Kontakt zu kommen. Oftmals entstehen dabei Freundschaften, die noch weit über die Studienzeit hinaus Bestand haben.
        \item \textbf{Vorbereitung}\\Über ein halbes Jahr hinweg bereitet sich die Delegation seit Semesterbeginn im November 2020 durch wöchentliche Trainingssessions auf die Teilnahme an der Konferenz vor. Im Zuge dessen werden vor allem die formalen Gepflogenheiten, die sog. Rules of Procedure, eingeübt und inhaltliche Stellungnahmen für die jeweils vertretenen Länder und Komitees, die sog. Position Papers, ausgearbeitet. Hinzu kommen Rhetorik- und Verhandlungstrainings sowie interne Probesimulationen. Parallel dazu arbeiten sich die Delegates thematisch in die politischen Standpunkte der von ihnen repräsentierten Länder ein. Hierzu bereiten die Delegates einerseits 10-minütige Kurzpräsentationen zu einem bestimmten landeskundlichen Thema vor, die sie dann den übrigen Delegates präsentieren. Andererseits arbeiten sie sich in die Themenbereiche ein, die in ihrem Komitee auf der Konferenz bearbeitet werden. Besonderen Stellenwert nimmt hierbei die Erarbeitung von schriftlichen Stellungnahmen ein, den sog. Position Papers, in denen die Standpunkte des repräsentierten Landes vorbereitend dargelegt werden.\\Um erste Praxiserfahrung zu sammeln soll Ende Februar die GerMUN in Berlin besucht werden. Sie findet vom 25. bis zum 28. Februar 2021 in Berlin-Wannsee statt. Gearbeitet wird in 2 bis 3 Komitees, die die UN-Generalversammlung sowie den UN-Sicherheitsrat simulieren. Der Veranstalter hat ein Hygienekonzept entwickelt, welches sicherstellen soll, dass die Konferenz trotz anhaltendem Infektionsgeschehen abgehalten werden kann. Am 23. Januar ist außerdem eine Teilnahme am Hohenheim Castle MUN (HCMUN) vorgesehen, einer eintägigen Konferenz mit Teilnehmer*innenn von der Universität Tübingen und der Universität Hohenheim. HCMUN wird dieses Jahr vermutlich online stattfinden und ist als erster Einstieg für die Delegates in den Ablauf einer Konferenz im kleineren Rahmen angedacht.\\
        \item \textbf{Die Studierendengruppe}\\Die Gruppe setzt sich aus insgesamt 14 Studierenden der juristischen Fakultät semesterübergreifend zusammen. Geleitet wird die Gruppe von Jonas Schmelzle als 1. Vorsitzenden und Johannes Lehmann als 2. Vorsitzenden. Gemeinsam treten sie als Faculty Advisors der 12-köpfigen Delegation auf und werden diese während der Konferenzreise betreuen. Unter den Delegates gibt es zwei Senior Delegates, welche bereits im vergangenen Jahr als Delegates teilgenommen haben und dem Vorstand bei administrativen Aufgaben im Rahmen der Vorbereitung zur Seite stehen. Die weiteren 10 Delegates wurden nach einer öffentlichen Ausschreibung an der Fakultät im Rahmen eines intensiven Auswahlverfahrens, das eine schriftliche Bewerbung samt inhaltlichem Aufsatz sowie ein persönliches Einzelinterview umfasste, ausgewählt. Sie zeichnen sich neben hervorragenden Schul- bzw. Studienleistungen durch ihr außeruniversitäres Engagement, ihr Interesse an tages- und weltpolitischen Fragestellungen sowie ihre Teamfähigkeit und Einsatzbereitschaft aus.\\
        \item \textbf{Kosten}\\Die Konferenzreise sowie die intensive Vorbereitung der Delegation sind mit hohen Kosten verbunden. Um die Eigenbeträge der Teilnehmer*innen so gering wie möglich zu halten, ist der Verein auf finanzielle Förderung angewiesen. Eine Unterstützung durch den Studierendenrat würde Studierenden, die sich durch hohe Eigenmotivation und überdurchschnittliches Engagement auszeichnen, die Teilnahme unabhängig vom finanziellen Hintergrund möglich machen.\\
        \item \textbf{Sonstige Projekte des Vereins}\\In den vergangenen Jahren nahm Heidelberg Law NMUN an der jährlichen National Model United Nations Konferenz (NMUN) am Hauptsitz der Vereinten Nationen in New York City teil. Seit 2013 gehört die Heidelberger NMUN-Delegation in New York zu einer der erfolgreichsten und wurde bereits als „outstanding delegation“ ausgezeichnet. Im Frühjahr 2021 ist zwar wieder eine Konferenz in New York geplant. Jedoch bestünde bei einer Anmeldung das hohe Risiko einer erneuten Absage, was hohe z.T. nicht stornierbare Kosten zur Folge hätte. Um den Kostenrahmen insgesamt einzugrenzen hat der Verein daher beschlossen, die US-Reise für ein weiteres Jahr auszusetzen.\\
        \item \textbf{Zeitplan}\\
        \begin{tabular}{c p{7cm}}
            Seit November 2020 & Wöchentliche Trainingssessions\\
            23. Januar & HCMUN Hohenheim\\
            9. Februar & Online-Workshop mit der Europäischen Zentralbank (genaues Datum noch nicht sicher)\\
            25. – 28. Februar & GerMUN Berlin\\
            1. – 4. März & Study Program Berlin\\
            17. – 21. April & Study Program in Barcelona\\
            22. -25. April & C’MUN Barcelona\\
        \end{tabular}
    \end{enumerate}
    \textbf{Finanzvolumen des Antrags:}\\
    \newline
    \begin{tabularx}{\textwidth}{X X}
        Wieviel beantragt ihr beim Studierendenrat?                             & Max 4000€       \\
        Wieviel wird bei der Verfassten Studierendenschaft insgesamt beantragt? & Max 4000€    \\
        Wieviel wird über Mittel weiterer Stellen finanziert?                   & 
        Zusage einer Spende von Rittershaus Rechtsanwälte \newline
        Antrag bei der PROMOS-Stipendium 2021 (DAAD)\newline
        Spendenanfragen wurden gestellt bei:
        \begin{itemize}
            \item Hogan Lovells LLP
            \item PWC Legal (450€)
            \item Büsing, Müffelmann \& Theye Rechtsanwälte (4290 € (?))
            \item KPMG
            \item Pöllath \& Partners
            \item Shearman \& Stearling
        \end{itemize}\\
        Habt ihr Einnahmen bei der Veranstaltung?                               & Eigenbeiträge iHv zunächst 400 €/Teilnehmer*in\\
        Wie hoch ist das Gesamtvolumen des Projekts                             & 15544,6 €\\
    \end{tabularx}
    \newline
    \vspace*{2em}
    \newline
    \begin{tabular}{p{4cm} p{1cm} p{1cm} p{9cm}}
        \textbf{Position} & \textbf{Kosten p.P.} & \textbf{Kosten insges.} & \textbf{Anmerkung} \\
        \textbf{Konferenz} & & & 6 Tage inkl An- \& Abreise\\
        1 Anmeldegebühr C'MUN Barcelona & 90€ & 1080€ & 12 Delegates\\
        2 Unterkunft während C'Mun & 100€ & 1400€ & 5 Nächte, 20€ p.P./Nacht Hostel/AirBnB\\
        3 Hin- \& Rückreise von Frankfurt nach Barcelona inkl. Züge & 250€ & 3500€ & Flüge: je 100€ p.P.p. Flug, Züge je 25€ p.P.p. Fahrt\\
        4 Transportpauschale Barcelona & 22,70€ & 317,80€ & 2x T-Casual (10 Fahrten 11,35€ p.P.)\\
        5 Verpflegungspauschale Barcelona & 60€ & 840€ & 10€ p.P. pro Tag\\
        6 Gesamtkosten der C'MUN & 522,70€ & 7317,80€ &\\
        7 Study Program & & & 5 Tage\\
        8 Verpflegungspauschale & 50€ & 700€ & 10€ p.P. pro Tag\\
        9 Unterkunft & 100€ & 1400€ & 5 Nächte, 20 € p.P./Nacht, Hostel/AirBnB\\
        10 Transportpauschale & 22,70€ & 317,80€ & 2x T-Casual (10 Fahrten für 11,35 € p.P.)\\
        11 Gesamtkosten Study Program & 172,70€ & 2410,80€ & \\
        12 Gesamtkosten C'MUN + Study Program & 695,40€ & 9728,60€ &\\
        13 Anmeldegebühr GerMUN & 190€ & 2660€ & 14 x inkl. Unterkunft \& Verpflegung\\
        14 Reisekosten hin/zurück Heidelberg - Berlin & 120€ & 1680€ & Je ca. 60€ pro Zugreise\\
        15 Transportpauschale Berlin & 34€ & 476€ & BVG 7 Tage Karte\\
        16 Kosten GerMUN insgesamt & 344€ & 4816€ &\\
        17 Kosten gesamte Projekt C'MUN + Study Program + GerMUN & 1093,40€ & 15544,60€ &\\
    \end{tabular}
    \emph{Weitere Informationen}\\
    Sollte eine Finanzierung einzelner Kostenpunkte nicht umsetzbar sein, wären wir dennoch sehr dankbar für eine Teilfinanzierung geringeren Umfangs. Als Antragsgegenstände eignen sich vor Allem die Kostenpunkte 6 (C’MUN), 12 (C’MUN + Study Program), 16 (GerMUN) und 17 (gesamtes Projekt).
}{
    TODO Weiterleiten
    niemand da
    im antrag ist nicht wirklich klar mit welchen kosten das verbunden ist, höhe der ksoten wäre interessant.
    wird mit den einnahmen von 400€ pro T. plus gemacht?
}
\finanzantrag{Kritjur Zoom}{Kritischen Jurist*innen Heidelberg}
{
    Der StuRa finanziert den Kritischen Jurist*innen Heidelberg in den Monaten Januar bis Juni 2021 eine Zoom-meetings-pro-Lizenz. Mithilfe dieser Lizenz wird den Kritischen Jurist*innen die Abhaltung der Plena und weiterer Sitzungen der Untergruppen ermöglicht.
}{
    \textbf{Infos zum Antragssteller:}\\
    Wir, die Kritischen Jurist*innen Heidelberg, sind eine (dauerhafte) linke kritische Initiative im juristischen Ausbildungsbereich. Wir wollen soziale und politische Bezüge von Recht reflektieren und einen kritischen sowie verantwortungsbewussten Umgang mit Recht fördern. Dazu gehört auch eine interdisziplinäre Perspektive auf gesellschaftliche Frage- und Problemstellungen und das Herausarbeiten deren Bedeutung für die Rechtswissenschaft.\\
    Homepage: \url{https://kritjurhd.jimdofree.com/}\\[1em]
    \textbf{Projektbeschreibung und Antragsbegründung:}\\
    Als Hochschulgruppe müssen wir uns regelmäßig in größeren Gruppen auszutauschen. Aufgrund der Corona-bedingten Einschränkungen sind wir zur Entwicklung und Weiterführung von Ideen, Projekten und Aktionen derzeit auf eine technisch verlässliche Online-Plattform angewiesen. Mithilfe der Zoom-Lizenz möchten wir nicht nur unsere in einem zwei-wöchentlichen Rhythmus stattfindenden Plena, sondern auch Besprechungen unserer Untergruppen ermöglichen. Eine Plattform, die uns ermöglicht, uns zu sehen, ermöglicht auch neu zur Gruppe hinzukommenden Studierenden, schnell Anschluss zu finden und sich zu vernetzen. Pro Meeting werden 20 – 40 Studierende von der Lizenz profitieren.\\[1em]
    \textbf{Finanzvolumen des Antrags:}\\
    \begin{tabular}{l l}
        Wieviel beantragt ihr beim Studierendenrat?                             & 83,94€       \\
        Wieviel wird bei der Verfassten Studierendenschaft insgesamt beantragt? & 233,94€    \\
        Wieviel wird über Mittel weiterer Stellen finanziert?                   & Bisher nicht geplant      \\
        Habt ihr Einnahmen bei der Veranstaltung?                               & nein \\
        Wie hoch ist das Gesamtvolumen des Projekts                             & 83,94€\\
    \end{tabular}
    \newline
    \textbf{Verwendungszweck:}\\
    (Monatlich abgerechnete) Zoom-meetings-pro-Lizenz 83,94€ (für 6 Monate)
    \emph{Weitere Informationen:}\\
    Möglich ist auch die Finanzierung nur eines Teils der monatlichen Lizenzen zu je 13,99€/Monat.
}{
    \textbf{1. Lesung:}\\
    warum braucht ihr eine Zoom lizenz, warum nicht bbb?
    TODO
}
\finanzantrag{Kritjur Vortrag Völkerstrafrecht}{Kritischen Jurist*innen Heidelberg}
{
    Der StuRa finanziert die Durchführung eines Vortrags zum Thema „Völkersstrafrechtliche Verfahren vor deutschen Gerichten“ durch Bereitstellung von Honoraren für einen der beiden Referent*innen von Amnesty International im Januar 2021.
}{
    \textbf{Infos zum Antragssteller:}\\
    Wir, die Kritischen Jurist*innen Heidelberg, sind eine (dauerhafte) linke kritische Initiative im juristischen Ausbildungsbereich. Wir wollen soziale und politische Bezüge von Recht reflektieren und einen kritischen sowie verantwortungsbewussten Umgang mit Recht fördern. Dazu gehört auch eine interdisziplinäre Perspektive auf gesellschaftliche Frage- und Problemstellungen und das Herausarbeiten deren Bedeutung für die Rechtswissenschaft.\\
    Homepage: \url{https://kritjurhd.jimdofree.com/}\\[1em]
    \textbf{Projektbeschreibung und Antragsbegründung:}\\
    In der universitären juristischen Ausbildung ist nur ein kleiner Teil der Studierenden mit völkerrechtlichen Inhalten befasst, da diese nicht zum Pflichtfachstoff der Ersten Juristischen Prüfung gehören. Völkerstrafrechtliche Fragestellungen bleiben meist gänzlich außen vor. Vor dem Hintergrund, dass jüngst verschiedene völkerstrafrechtliche Verfahren gegen mutmaßliche Mitglieder der Terrorgruppe Islamischer Staat wegen Völkermordes, Kriegsverbrechen und Verbrechen gegen die Menschlichkeit vor deutschen Oberlandesgerichteten begonnen haben, weitere Verfahren zu erwarten sind, die Thematik in der Juristischen Ausbildung hingegen keinen Raum hat, möchten wir den Studierenden einen Einblick in eines der laufenden Verfahren und grundsätzliche Fragestellungen völkerstrafrechtlicher Prozesse vor deutschen Gerichten ermöglichen. Die Referent*innen – Johanna Groß und Dr. Alexander Schwarz – sind seit vielen Jahren in der Koordinationsgruppe Völkerstrafrecht von Amnesty International aktiv und beobachten völkerstrafrechtliche Prozesse. Zurzeit beobachten Sie u.a. das Staatsschutzverfahren 5-3 StE 1/20-4-1/20 gegen Taha Al J. wegen Völkermordes an den Jesid*innen. Neben einem abstrakten völkerstrafrechtlichen Teil sollen die Erkenntnisse aus der Prozessbeobachtung dieses Verfahrens Gegenstand des Vortrags sein.\\
    Vorträge zu diesem Thema gab es an der Juristischen Fakultät in den letzten Semestern nicht. Für den Vortrag ist ein Zeitfenster von zwei Stunden vorgesehen. Erfahrungsgemäß profitieren von dem Vortrag 50-80 Studierende. \\
    Die Beantragung erfolgt allein für den Referenten Dr. Alexander Schwarz, da Frau Groß Mitglied der Universität Heidelberg ist.\\[1em]
    \textbf{Finanzvolumen des Antrags:}\\
    \begin{tabular}{l l}
        Wieviel beantragt ihr beim Studierendenrat?                             & 150€       \\
        Wieviel wird bei der Verfassten Studierendenschaft insgesamt beantragt? & 233,94€    \\
        Wieviel wird über Mittel weiterer Stellen finanziert?                   & Bisher nicht geplant      \\
        Habt ihr Einnahmen bei der Veranstaltung?                               & nein \\
        Wie hoch ist das Gesamtvolumen des Projekts                             & 150€\\
    \end{tabular}
    \newline
    \vspace*{2em}
    \newline
    \begin{tabular}{p{8cm} p{2cm} p{5cm}}
        \textbf{Verwendungszweck} & \textbf{Kosten} & \textbf{Begründung} \\
        Honorare für den Referenten des Vortrags „Völkerstrafrechtliche Verfahren vor deutschen Gerichten“
        \newline Angaben zu dem Referent:\newline
        Dr. Alexander Schwarz, Akademischer Assistent am Lehrstuhl für Europarecht, Völkerrecht und Öffentliches Recht an der Uni Leipzig und Mitglied der Koordinationsgruppe Völkerstrafrecht bei Amnesty International
        & 150€ & Der Referent soll für die mit der Vorbereitung und der Durchführung des Vortrags verbundenen Arbeit entschädigt werden.\\  
    \end{tabular}
    \emph{Weiter Informationen:}\\
    Auch eine anteilige Finanzierung wäre möglich.
}{
    \textbf{1. Lesung:}\\
    keine Fragen

}
\finanzantrag{Klimagerechte Wege aus dem Kapitalismus}{Klimakollektiv Heidelberg, Referat für Politische Bildung der Universität Heidelberg}
{
    Der StuRa/die RefKonf unterstützt die Durchführung einer dreitägigen Konferenz zum Thema „Klimagerechte Wege aus dem Kapitalismus“ mit finanziellen Mittel, die pandemiebedingt nicht wie geplant im Mai 2020 stattfinden konnte und um ein Jahr verschoben wird. Die Veranstaltung umfasst acht Vorträge, welche Alternativen in den Bereichen Ökonomie, Politik, Ernährung und Wohnen zum aktuellen kapitalistischen System aufzeigen. Die finanziellen Mittel fließen in Honorare von drei Referent*innen, die Erstattung ihrer Fahrtkosten sowie in den Druck von Werbematerial.
}{
    \textbf{Infos zum Antragssteller:}\\
    Klimakollektiv Heidelberg, Referat für Politische Bildung der Universität Heidelberg\\[1em]
    \textbf{Projektbeschreibung und Antragsbegründung:}\\
    Das Referat Politische Bildung organisiert in Kooperation mit dem Klimakollektiv Heidelberg und der Anarchistischen Gruppe Mannheim, sofern die Pandemie-Lage dies zulässt, vom 21. Bis 23.5.2021 eine Vortragsreihe, die sich mit Alternativen zur kapitalistischen Gesellschaft befasst und damit Wege in eine klima-, aber auch sozial gerechtere Gesellschaft aufzeigt. Die Konferenz war für den 22-25.Mai 2020 geplant, musste aber pandemiebedingt abgesagt werden. Nun möchten wir die Veranstaltung um ein Jahr verschieben und beantragen dafür eine zweckgebundene Rücklage.\\
    An drei Tagen finden acht verschiedene Vorträge von Referent*innen aus dem Raum Heidelberg und Umgebung sowie aus anderen Orten statt. Jedem Vortrag folgt Raum für Diskussion, zwischen den Vorträgen finden Pausen statt und am Samstagmittag gibt es ein selbstorganisiertes Mittagessen für alle Teilnehmenden, das auf Spendenbasis beruht, damit auch Teilnehmende mit geringem Einkommen nicht ausgeschlossen werden. Da nicht absehbar ist, wie viel Spenden dabei zusammenkommen, müssen wir das Geld für die Lebensmittel vorstrecken.\\
    Die seit zwei Jahren andauernden Proteste der Schüler*innen und Studierenden von „Fridays for Future“ in Heidelberg und weltweit setzen ein klares Zeichen: Die Forderung nach konsequenten Maßnahmen angesichts der Klimakrise. Als Teil der Universität sieht sich das Referat Politische Bildung in der Verantwortung, auch für ökologische und gesellschaftskritische Themen zu sensibilisieren und setzt sich damit für eine nachhaltige, zukunftsfähige Gesellschaft ein.\\
    Die Veranstaltung „Klimagerechte Wege aus dem Kapitalismus“ zielt darauf ab, mit Vorträgen und Diskussionsrunden einerseits zu informieren und dadurch politische und ökonomische Alternativen zur kapitalistischen Gesellschaft zu thematisieren. Andererseits werden Handlungsoptionen, die auch auf individueller und lokaler Ebene verfolgt werden können, aufgezeigt. Wir sind der Überzeugung, dass es eminent wichtig ist, auch die ökonomische Dimension von Klimapolitik mitzudenken, damit nicht durch eine sozial ungerechte Klimaschutzpolitik neue Ungleichheiten in der Gesellschaft entstehen. Die Vortragsreihe bietet einen Rahmen, verschiedene alternative Formen des Wirtschaftens, Zusammenlebens, der Selbstorganisation sowie des Bereichs Ernährung vorzustellen, und versteht sich als Raum, in dem kontroverse Diskussionen geführt werden sollen. Damit regt die Konferenz Studierende der Universität Heidelberg und die interessierte Öffentlichkeit zum weiteren Nachdenken an und ermutigt zu lokalem Handeln.\\
    Insgesamt sind acht Vorträge geplant, wobei die Hälfte der eingeladenen Referent*innen aus Heidelberg und der Umgebung kommt. Da die auswärtigen Referenten, allesamt ausgewiesene Experten für ihr jeweiliges Fachgebiet, ihren Lebensunterhalt unter anderem durch ihre Vortragstätigkeit bestreiten, bedarf es für Honorare und Fahrtkosten der folgenden Referent*innen, die aus anderen Orten anreisen, finanzieller Mittel:
    \begin{itemize}
        \item Elisabeth Voß | Solidarische Ökonomie. Die Berliner Publizistin und Betriebswirtin Elisabeth Voß bietet seit mehreren Jahrzehnten Vorträge, Workshops und Fortbildungen zum Thema „Alternatives Wirtschaften“ an. Sie hat nach ihrem BWL-Studium in einer Reihe von Alternativbetrieben in verschiedenen Ländern gearbeitet und vermittelt die dabei entstandenen Erfahrungen in Veranstaltungen, mit denen sie aktuell ihren Lebensunterhalt bestreitet.
        \item Anselm Schindler | Rojava. Der Münchner Referent, der als freier Journalist unter anderem für die taz, analyse \& kritik, die Junge Welt und das Neue Deutschland schreibt, hält einen Vortrag über die Selbstverwaltung und alternative ökonomische Modelle in den nordsyrischen Selbstverwaltungsgebieten. Er ist selbst für Recherchen schon viele Male in die Region gereist und engagiert sich bei der Initiative „Make Rojava Green again“, die sich für eine ökologisch und soziale gesellschaftliche Transformation einsetzt. Er ist daher ein ausgewiesener Experte für die Entwicklungen in Rojava.
        \item Rudolf Mühland: Libertäre Gesellschaftskritik. Rudolf Mühland ist Gewerkschafter in der Basisgewerkschaft „Freie Arbeiter*innen-Union“ in Düsseldorf und organisiert seit vielen Jahren Beschäftigte in Bereichen, in denen die DGB-Gewerkschaften wenig aktiv sind. Daneben ist er publizistisch aktiv und schreibt über eine freiheitliche und libertäre Kritik an der kapitalistischen Gesellschaft. Damit verbindet er eine profunde ökonomische Kenntnis aus der Gewerkschaftspraxis mit grundlegenden gesellschaftstheoretischen Reflexionen. Gerade diese Verbindung aus Theorie und Praxis ist für ein überwiegend studentisches Publikum besonders interessant.
        \item Hanna Poddig | Klimagerechtigkeit als Ausweg? Die bundesweit bekannte Hamburger Klimaaktivistin Hannah Poddig ist seit vielen Jahren in der Alternativ- und Umweltbewegung aktiv und beispielsweise bei der Aktionsgruppe „Robin Wood“ aktiv. Sie setzt sich allerdings auch seit langer Zeit mit wirtschaftspolitischen Fragen innerhalb dieser heterogenen Bewegung auseinander und wird aus nächster Nähe über die Diskussionen berichten, die aktuell über kapitalismuskritische Ansätze innerhalb der Klimabewegung geführt werden.
    \end{itemize}
    Die Veranstaltung richtet sich primär an Studierende aller Fachbereiche, darüber hinaus ist sie öffentlich für alle interessierten Menschen. Etwa 150-200 Studierende werden bei den unterschiedlichen Vorträgen insgesamt erwartet. Eine Vortragsreihe zu explizit den genannten Themen gibt es bislang nicht. Da die kooperierenden Gruppen über keine eigenen finanziellen Mittel für eine solche Veranstaltung verfügen, bedarf es einer Unterstützung durch weitere Instanzen.\\[1em]
    \textbf{Finanzvolumen des Antrags:}\\
    Für die Durchführung der Veranstaltung beantragen wir bei der Verfassten Studierendenschaft 2000 €. Anträge bei anderen Stellen wurden nicht gestellt. Förderungen durch andere Unterstützer*innen gibt es keine.\\
    \begin{longtable}{p{3cm} p{1cm} p{11cm}}
        \endfirsthead
        \endhead
        \endfoot
        \endlastfoot
        \textbf{Verwendungszweck} & \textbf{Kosten} & \textbf{Begründung} \\
        Honorar für Vortrag „Solidarische Ökonomie“ & 350€ & Die Berliner Publizistin und Betriebswirtin Elisabeth Voß bietet seit mehreren Jahrzehnten Vorträge, Workshops und Fortbildungen zum Thema „Alternatives Wirtschaften“ an. Sie hat nach ihrem BWL-Studium in einer Reihe von Alternativbetrieben in verschiedenen Ländern gearbeitet und vermittelt die dabei entstandenen Erfahrungen in Veranstaltungen, mit denen sie aktuell ihren Lebensunterhalt bestreitet.\\
        Fahrtkosten für Referentin Elisabeth Voß & 120€ & Referentin reist aus Berlin an.\\
        Honorar für Vortrag „Rojava“, Referent: Anselm Schindler & 350€ & Der Münchner Referent, der als freier Journalist unter anderem für die taz, analyse \& kritik, die Junge Welt und das Neue Deutschland schreibt, hält einen Vortrag über die Selbstverwaltung und alternative ökonomische Modelle in den nordsyrischen Selbstverwaltungsgebieten. Er ist selbst für Recherchen schon viele Male in die Region gereist und engagiert sich bei der Initiative „Make Rojava Green again“, die sich für eine ökologisch und soziale gesellschaftliche Transformation einsetzt. Er ist daher ein ausgewiesener Experte für die Entwicklungen in Rojava.\\
        Fahrtkosten für Referent Anselm Schindler & 90€ & Referent reist aus München an.\\
        Honorar für Vortrag „Libertäre Gesellschaftskritik“, Referent: Rudolf Mühland & 350€ & Rudolf Mühland ist Gewerkschafter in der Basisgewerkschaft „Freie Arbeiter*innen-Union“ in Düsseldorf und organisiert seit vielen Jahren Beschäftigte in Bereichen, in denen die DGB-Gewerkschaften wenig aktiv sind. Daneben ist er publizistisch aktiv und schreibt über eine freiheitliche und libertäre Kritik an der kapitalistischen Gesellschaft. Damit verbindet er eine profunde ökonomische Kenntnis aus der Gewerkschaftspraxis mit grundlegenden gesellschaftstheoretischen Reflexionen. Gerade diese Verbindung aus Theorie und Praxis ist für ein überwiegend studentisches Publikum besonders interessant.\\
        Fahrtkosten für Referent Rudolf Mühland & 80€ & Referent reist aus Düsseldorf an.\\
        Honorar für Vortrag „Klimagerechtigkeit als Ausweg?“, Referentin: Hanna Poddig & 350€ & Die bundesweit bekannte Hamburger Klimaaktivistin Hannah Poddig ist seit vielen Jahren in der Alternativ- und Umweltbewegung aktiv und beispielsweise bei der Aktionsgruppe „Robin Wood“ aktiv. Sie setzt sich allerdings auch seit langer Zeit mit wirtschaftspolitischen Fragen innerhalb dieser heterogenen Bewegung auseinander und wird aus nächster Nähe über die Diskussionen berichten, die aktuell über kapitalismuskritische Ansätze innerhalb der Klimabewegung geführt werden.\\
        Fahrtkosten für Referentin Hanna Poddig & 110€ & Referentin reist aus Hamburg an.\\
        Kosten für Verpflegung (Mittagessen auf Spendenbasis) & 100€ & Wir organisieren eine Mittagsmahlzeit für 50-100 Personen auf Spendenbasis, und rechnen mit einem Defizit\\
        Druckkosten Werbematerial & 100€ & Neben der Bewerbung per Website und Social Media werden Printmaterialien zur Bekanntmachung der Veranstaltung im öffentlichen Raum benötigt.\\
        \textbf{Gesamt} & \textbf{2000€} & - \\  
    \end{longtable}
    \emph{Weitere Informationen:}\\
    Der Antrag kann ggf. auch geteilt werden. Da die Vortragsreihe von ehrenamtlichen Initiativen organisiert wird und keine weiteren finanziellen Mittel zur Verfügung stehen, würden wir uns eine Finanzierung des Gesamtbetrags sehr wünschen.\\
}{
    \textbf{1. Lesung:}\\
    Bereits im Frühjahr wurde sich mit dem PoBiReferat ausgetauscht, PoBi-referat empfiehlt dem Stura den Antrag anzunehmen\\
    wo führen die wege hin?\\
        das soll von den Refernten geklärt werden, im allgemeinen Kapitalismuskritisch\\
    warum weg vom kapitalismus, nur Kommunismus als alternative\\
        kein ostblock-Kommunismus, freiheitliche formen der kapitalismuskritik, gibt auch andere sozialistische Richtungen, die sich kritisch gegenüber dem ostblockkommunismus äussert\\
    wie stellt sich den die Zusammenarbeit mit der anarchistischen Gruppe mannheim daransetzen\\
        sie organisiert in mannheim eine buchmesse, das knowhow soll bei dieser Veranstaltung helfen, ist in mannheim auch bei linken themen aktiv\\
    der stura finanziert nur, was nicht von den spenden gedeckt wird?\\
        ja\\
    wie wollt ihr die frage klären ob kapitalismus alternativlos ist wenn ihr nur kapitalismuskritiker einladet\\
        wir kommen aus fff und klima bewegung und finde, dass thema nicht genug behandelt wird\\
}
