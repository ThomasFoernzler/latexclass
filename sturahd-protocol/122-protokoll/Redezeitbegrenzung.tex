\section{Redezeitbegrenzung}
\GOantrag{
    Redezeitbegrenzung
}{
    Niklas Jargon
}{
    Der StuRa beschließt für den Verlauf der Sitzung:
    \begin{enumerate}
        \item Eine Begrenzung der Redezeit bei der Vorstellung eines Antrags durch den/die Antragsteller*in auf fünf Minuten.
        \item Eine Begrenzung der Redezeit bei einzelnen Diskussionsbeiträgen und Antworten auf Diskussionsbeiträge auf drei Minuten.
        \item Hievon unbeschadet bleibt die Möglichkeit, die Begrenzung der Redezeit durch GO-Antrag aufzuheben oder zu ändern.
    \end{enumerate}
}{
    Der Antragssteller möchte eine relative Verkürzung der StuRa-Sitzungen erreichen.\\
    Die beiden bisherigen Sitzungen der Legislatur dauerten jeweils deutlich über drei Stunden.
    Dabei ließ die Aufmerksamkeit und Diskussionsbereitschaft der Teilnehmenden gegen Ende der
    Sitzung merklich nach. Besonders lange Diskussionen zu einzelnen Tagesordnungspunkten zogen
    die Sitzungen in die Länge.\\
    Um den Zeitaufwand für einzelne TOPs zu verringern, ohne dabei die Möglichkeit einer
    demokratischen Debatte einzuschränken, bietet sich eine Begrenzung der Redezeit für einzelne
    Diskussionsbeiträge an. Hierdurch sollen besonders die Antragssteller*innen ermuntert werden,
    die Vorstellung ihrer Anträge auf die wesentlichen Punkte zu reduzieren. Detailfragen sollen
    in den Sitzungsunterlagen erläutert sein.\\
    Anmerkung des Antragsstellers: Die vorgeschlagenen Zeiten sind lediglich meine Einschätzung,
    wie lange ein Beitrag dauern muss und legitimerweise dauern darf. Sie sind daher diskutabel.
}{
    Inhaltliche Nachfrage:\\
    habt ihr den Erfahrungswerte wie lange die Sachen immer dauern?\\
    \ul{\lii{Vorstellung der Anträge dauert zu lange, deshalb Begrenzung.}}
    \ul{\lii{Ohne Gegenrede angenommen}}
}{-}{-}{-}