\section{Infos, Termine, Berichte}

\subsection{Bericht des Vorsitzes}
\ul{
    \li{wir haben uns mit Herrn Eitel, unserem Rektor, Frau Fuhrmann-Koch (Abteilung Kommunikation und Marketing, = KuM) und Herrn Punstein (Referent des Rektors) getroffen:
        \begin{itemize}
            \item weil ein Teil des Vorsitzes WLAN-Probleme hatte, konnten wir nur 30 min wirklich sprechen
            \item Herr Eitel will im SoSe v.a. den Fokus auf Hochschulwechsler legen
            \item wir haben über 4EU+ gesprochen
            \item die Gesprächsatmosphäre war gut
            \item wir treffen uns im Juli nochmal und wollen schwerpunktmäßig darüber sprechen, was wir aus den digitalen Corona-Semester mitnehmen können
        \end{itemize}
    }
    \li{wir haben uns auch in den letzten Wochen mit vielen Fachschaften getroffen
        \begin{itemize}
            \item mehr Austausch unter den Fachschaften wird begrüßt; wir werden dazu vermutlich im April einen Antrag einbringen, weil sich der StuRa (v.a. die Fachschaftsvertreter*innen) auf eine Plattform/Messengerdienst einigen sollten, wo es eine Fachschaften-Gruppe geben soll
            \item {
                auch ein Fachschaftsvernetzungstreffen findet Anklang, es wird im SoSe stattfinden. Es geht dabei vor allem darum, dass Fachschaften von anderen Fachschaften lernen und von deren Erfahrungen profitieren können. Themen, die sich mittlerweile herauskristallisiert haben, sind:
                \begin{itemize}
                    \item Wie kann man Studierende für die Fachschaft rekrutieren?
                    \item Wie laufen Ersti-Einführungen ab, welche Best-Practice-Beispiele gibt es?
                    \item Wie kann man als Fachschaft erreichen, was ihr wichtig ist?
                    \item Vernetzungsmöglichkeit mit anderen Fachschaften der Fakultät
                    \lii{ \textbf{falls ihr bzw. eure Fachschaft weitere Themen besprechen möchtet, gebt uns Bescheid!}}           
                \end{itemize}
            }
            \item zwei Fachschaften möchten all ihre Studierenden per Mail kontaktieren -> wir haben das mit KuM besprochen und die Fachschaften können das tun, wenn die*der Dekan*in dem zustimmt (\textbf{-> falls ihr als FS das irgendwann einmal machen möchtet,  meldet euch bei uns! Wir erklären euch dann, wie es funktioniert)}
        \end{itemize}
    }
    \li{Auskunft von Herrn Treiber zu Podiumsdiskussion:
        \begin{itemize}
            \item wir haben mit Herrn Treiber, der unser Ansprechpartner an der Uni für diverse Belange ist, gemailt und wir können keine Podiumsdiskussion zur Landtagswahl veranstalten
            \item insbesondere acht Wochen vor Wahlen gilt eine „gesteigerte Neutralitätspflicht“, d.h., dass man als staatliche Einrichtung (o.ä.) keine Wahl-bezogenen Veranstaltungen durchführen oder auch nur Räumlichkeiten zur Verfügung zu stellen soll; Grund hierfür sind das Demokratieprinzip und das Recht der Parteien auf Chancengleichheit (eine Veranstaltung, die die Wahlentscheidung beeinflussen könnte, steht dem entgegen)
            \item Auszug aus Herr Treibers Mail: „Deshalb sollen „in Einrichtungen im Geschäftsbereich des Wissenschaftsministeriums“ - und dazu zählt auch die Universität und die VS - "im Zeitraum von acht Wochen vor dem Wahltermin weder Besuche noch Veranstaltungen von oder mit Abgeordneten oder Wahlbewerbern (...) stattfinden“. 
            \item deshalb planen wir jetzt eine derartige Podiumsdiskussion für die Bundestagswahl
        \end{itemize}
    }
    \li{diese oder nächste Woche verschicken wir eine Mail an alle Studierenden}
    \li{Moderationsworkshop:
        \begin{itemize}
            \item es geht um Moderation von Gruppen(diskussionen), mit Fokus auf digitalen Sitzungen, weil diese besonders herausfordernd sind
            \item gedacht ist der Workshop für Studierende, die Sitzungen in der Fachschaft oder Hochschulgruppe leiten/moderieren
            \item wir führen ihn in Kooperation mit der Hochschuldidaktik/Schlüsselkompetenzen durch
            \item Anfang April wird eine Mail dazu verschickt mit einem „Reinschnupper-Video“
            \item am 23.4. zwischen 16 und 18 Uhr findet der Workshop digital statt
        \end{itemize}   
    }
    \li{Falk (FS Zahnmedizin) hat sich bereiterklärt, sich um die Belange von Studierenden mit Kind zu kümmern; sie haben derzeit auch wegen der schwierigen Betreuungssituation größere Probleme als andere Studierende; wir möchten Probleme sammeln und lösen! \textbf{-> wenn ihr Studis mit Kind kennt oder in eurem Fach einige Studis mit Kind sind, schreibt sie an und bittet sie, an f.busch@stura.uni-heidelberg.de zu schreiben}}
    \li{Gespräch mit Gleichstellungsbüro (Falk und Henrike haben teilgenommen):
        \begin{itemize}
            \item es gibt diverse Angebote für Studierende mit Kind
            \item teils werden sie nachgefragt, teils aber auch nicht
            \item unseres Erachtens gibt es ein Informationsdefizit, weswegen wir in der Mail an alle Studis diese oder nächste Woche auf das Gleichstellungsbüro und seine Angebote hinweisen wollen
            \item Es gibt eine Ausschreibung zu Diversity in der Lehre: „Wir finanzieren seit dem WS2020/21 Lehraufträge zu „Diversity Innovations“, also Seminare etc. die sich inhaltlich mit Vielfalt, unterschiedlichen Diversity-Dimensionen oder intersektionalen Themen beschäftigen. Auch im Sommersemester 2021 wollen wir hier wieder Institute unterstützen. Thematisch also sehr breit gespannt. Wir würden quasi nur finanzieren, die Vorbereitung und organisatorische Abwicklung läuft über die Institute, auch um den innerfachliche Austausch zum Thema zu fördern. Als Ideengeberin stehen wir natürlich gerne zur Seite.“ \href{https://www.uni-heidelberg.de/diversity/forschungundlehre/lehre.html}{Hier findet ihr alle Veranstaltungen}, die im WiSe 20/21 im Bereich Diversity angeboten werden. 
            \item \textbf{-> ihr als Fachschaft oder HSG könnt überlegen, ob ihr Ideen habt, was im SoSe in dem Bereich in eurem Fach angeboten werden kann!}
        \end{itemize}
    }
}

\myparagraph{Nachfragen:}
\ul{
	\li{Bezüglich Neutralitätsgebot: würde das nicht die Aktion des PoBi Referats betreffen?}
		\noli{\ul{
		\lii{Das mit der Diskussion geht dann leider nicht.
		}}}
	%\li{AfD nicht einladen?}
	%	\noli{\ul{
	%	\lii{Offiziell dürfen wir das nicht. Nächste Woche kommt Jonathan vom fzs und da können wir auf einer besseren Faktenbasis reden.
	%	}}}
}

\subsection{Bericht der Härtefallkommission}
Erfolgte mündlich.
\myparagraph{Nachfragen:}
\ul{
	\li{Die Satzung verbietet Förderung für mehr als 3 Monate. Sollte amn da nicht die Satzung ändern?}
		\noli{\ul{
		\lii{Das Problem ist, dass der Stura eine Zwangskörperschaft des öffentlichen Rechts ist. Deswegen braucht es gute Begründungen das zu machen. Wenn das politisch gewollt ist, ist das eine ganz andere Sache.
		}}}
	\li{Könnte man das temporär ändern?}
		\noli{\ul{
		\lii{Darauf müsste das Gremienreferat antworten. Die höchste Antragslast war nach dem ersten Lockdown. Denen geht es nicht anders. Dass es dieses Notlagenstipendium gibt ist nur ein Pflaster und das sind Aufgaben für andere Institutionen. 
		}}}
		\noli{\ul{
		\lii{Ergänzend: Wie viele Studierende haben ihr Studium abgebrochen, weil sie nicht wussten, dass es dieses Stipendium gibt. Wir sind alle Studierende und die HFK ist auch ein Zeitaufwand. Deswegen wären neue Mitglieder ganz gut.
		}}}
	\li{Würden die Referierenden empfehlen die Satzung zu verändern.}
		\noli{\ul{
		\lii{Wir als Stura können das noch mehr bewerben, um Studierende auf diese Option aufmerksam zu machen. Es wäre vielleicht besser es publik zu machen anstatt die Satzung zu ändern.
		}}}
		\noli{\ul{
		\lii{Was kann eine VS mit dem Geld am besten anfangen. Ob man das für ein Notlagenstipendium ausgeben will ist fraglich. Aber es wären auch zusätzliche Mitglieder für die HFK nötig, um das zu managen.
		}}}
	\li{Könnten wir ein Statement zur Abschaffung der Studiengebühren machen?}
		\noli{\ul{
		\lii{Das allgemeinpolitische Mandat wurde der VS entzogen. Aber das ist sehr hochschulpolitisch. Deswegen könnten wir das machen.
		}}}
	\li{Es stehen Wahlen bevor und da kann man das gut ansprechen. Auf Bundesebene ist das sehr traurig.}
	\li{Es ist noch ein Anliegen die HFK und das Notlagenstipendium bekannt zu machen. Muss man dafür einen Antrag stellen?}
		\noli{\ul{
		\lii{Wir brauchen hier mehr Personen die das ganze Prozedere in der HFK regeln. Und die HFK kann kaum die ganze Arbeit stemmen.
		}}}
	\li{Könnte man mehr Leute dafür begeistern durch Kompensation?}
		\noli{\ul{
		\lii{An wen war die Frage gerichtet?
		}}}
		\noli{\ul{
		\lii{Was die Aufwandsentschädigung angeht, wird das auch noch hier besprochen werden in der nächsten Sitzung.
		}}}
}
\subsubsection{GO-Antrag: Stellung des Antrags des SDS zu Burschenschaften}
\myparagraph{Antragstext:}
Der Antrag des SDS soll ohne den 2 Satz auf die Tagesordnung aufgenommen werden.
\myparagraph{Begründung:}
Er wurde von der Sitzungsleitung nicht aufgenommen. Als Begründung wurde der 2. Satz angegeben, welcher rechtlich nicht Ordnung sein soll. Der SDS würde ihn jetzt ohne den 2. Satz behandeln lassen.
\paragraph{Diskussion:}
\ul{
	\li{Der Antrag zu Burschenschaften soll gestellt werden ohne den zweiten Satz}
		\noli{\ul{
		\lii{Der zweite Satz will die Mitglieder von Burschenschaften von Ämtern der VS ausschließen. Das ist nicht nur rechtswidrig, sondern auch nicht wirklich mit der Demokratie vereinbar.
		}}}
	\li{Es ist nicht wirklich mit der Demokratie vereinbar, wenn die Sitzungsleitung einfach entscheidet, dass ein Antrag nicht diskutiert wird.}
		\noli{\ul{
		\lii{Das ist wehrhafte Demokratie.
		}}}
}
\ul{\lii{Die Sitzungsleitung hatte dem SDS geschrieben, dass sie bereit wäre, den Antrag ohne den 2. Satz in die TO aufzunehmen, aber keine Antwort erhalten. Der Antrag wird jetzt nachträglich von der Sitzungsleitung auf die TO aufgenommen.}}
\subsubsection{GO-Antrag: Fortsetzung der Tagesordnung}
\myparagraph{Antragstext:}
Die Diskussion soll beendet und die Tagesordnung fortgesetzt werden.
\myparagraph{Begründung:}
Die Diskussion sollte nicht jetzt geführt werden, später bietet sich auch noch die Gelgenheit.
\myparagraph{Gegenrede:}
Die Diskussion ist wichtig und es sollte Klarheit über die TO herrschen.\\
(Es wird angemerkt, dass der TOP aufgenommen ist, daraufhin wird die Gegenrede zurückgezogen)
\ul{\lii{Angenommen, Sitzung geht weiter}}

\subsection{Bericht des Referats für Hochschulpolitische Vernetzung}
Dieser Bericht umfasst für die für uns relevanten landesweiten Geschehnisse im Zeitraum Oktober 2020 - Januar 2021, insbesondere die Arbeit der LaStuVe.\\
In dieser Zeit fanden drei Landes-Asten-Konferenzen (LAKs) statt: Am 25.10., 29.11.2020 und 10.01.2021. Dabei waren, neben anderen, wichtige Beschlüsse:
\begin{itemize}
    \item nach einer Briefwahl im November/Dezember wurde das Ergebnis am 13.12.2020 festgestellt. Das neue Präsidium besteht aus: Rachel Acosta, Marc Baltrun, Andreas Bauer, Johanna Ehlers, Konstantin Schmidt
    \item Befürwortung eines optionalen Landesweiten Semestertickets solange Preis und Konditionen eines teil- oder vollsolidarischen Tickets nicht bestimmt werden können. Auch mögliche Härtefälle wurden bereits vorgeschlagen.
    \item Unterzeichnung des Offenen Briefs für eine transparente und nachhaltige Versorgungsanstalt der Bundes und der Länder (VBL): \href{https://lastuve-bawue.de/unterzeichnung-des-offenen-briefs-fuer-eine-nachhaltige-vbl/}{https://lastuve-bawue.de/unterzeichnung-des-offenen-briefs-fuer-eine-nachhaltige-vbl/}
    \item Forderungskatalog "Klima und Umwelt" der Landesstudierendenvertretung Baden-Württemberg, der sich an die Regierung BWs richtet.
\end{itemize}
Was hat die LaStuVe sonst noch gemacht:\\
Schon im Vorfeld des Wintersemesters hat der AK Corona einen Forderungskatalog (\href{https://lastuve-bawue.de/forderungskatalog-wintersemester-2020-2021/}{https://lastuve-bawue.de/forderungskatalog-wintersemester-2020-2021/}) zum Wintersemester entworfen und verschickt. Auch während dieses Semesters hat sich die LaStuVe in Gesprächen mit Akteur*innen für Freiversuchsregelungen (\href{https://lastuve-bawue.de/brief-an-landesrektorinnenkonferenz-lrk-landesweite-umsetzung-von-freiversuchen/}{https://lastuve-bawue.de/brief-an-landesrektorinnenkonferenz-lrk-landesweite-umsetzung-von-freiversuchen/}) und flexible Rücktrittsmöglichkeiten eingesetzt, außerdem stets auf generelle wie individuelle Missstände aufmerksam zu machen.\\
Aktuelle "Großprojekte" der Zeit:
\begin{itemize}
    \item Zu den Landtagswahlen im März 2021 wurde ein Studi-O-Mat entworfen, diverse hochschulpolitischen Thesen liegen den Parteien aktuell zur Stellungnahme vor und Mitte Februar soll beides veröffentlicht werden. Es wurde ein breites Themenspektrum gewählt, die Thesen sind mitunter bewusst diskursiv formuliert.
    \item Zuletzt hat die Konstituierung der Landesstudierendenvertretung etwas an Fahrt verloren. Es gab einige strittige Punkte, die bald einer Mehrheitsentscheidung der Studierendenschaften benötigen. Etwas komplizierter verhält es sich bei der Finanzierung der LaStuVe, dessen Varianten umfassender sind als das den Studierendenschaften einzelne Sätze zur Abstimmung gegeben werden.
\end{itemize}
4. HRÄG trat zum 01.01. in Kraft.
\myparagraph{Nachfragen:}
\ul{
	\li{Der Studiomat zur Landtagswahl: Wer hat das erstellt und wie sollen die anderen Referate dabei helfen?}
		\noli{\ul{
		\lii{Der Studiomat ist bei der Lak entstanden. Die Thesen wurden dann auch von der LAK bestätigt und die Parteien hatten 4 Wochen Zeit die Fragen zu beantworten. Viele Studierende kommen auch für das Studium nach BW und die haben nicht notwendigerweise den Wunsch wählen zu gehen.
		}}}
	\li{Es ist gut auf die StuVe zuzugehen. Aber die Stimmen werden nicht in der Sitzung gesammelt sondern davor. Dafür braucht es eine größere Strategie.}
		\noli{\ul{
		\lii{Das Studierendenwerksgesetz hat sich geändert. Bisher stand drin wieviel studentische Vertrende es gab und jetzt nicht mehr. Deswegen gibt es jetzt eine neue Satzung des Studierendenwerks. 
		}}}
	\li{Ist es nicht eher besser wenn man mit einem anderen Vorschlag hingeht?}
		\noli{\ul{
		\lii{Das ist richtig. Aber wir wollen zuerst an das Studierendenwerk herantreten und da schon im ersten Vorschlag eine höhere Anzahl an studentischen Vertretenden reinbringen.
		}}}
}

\subsection{Bericht zum Studierendenwerk \label{bericht:4}}
\GOantrag{Ausschluss der Öffentlichkeit}{Den Ausschluss der Öffentlichkeit für TOP 4.4.}{Efolgt mündlich.}{Keine Gegenrede.}{
    \abstimmungsergebnis{
        \fullref{bericht:4}
    }{
        tba%Ja
    }{
        tba%Nein
    }{
        tba%Enth
    }{
        \ul{\lii{Ohne Gegnrede angenommen.}}
    }
}