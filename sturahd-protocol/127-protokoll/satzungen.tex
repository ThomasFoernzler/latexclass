\section{Satzungen und Ordnungen}

\antrag{Neufassung der Satzung der Studienfachschaft UFG/VA}{4. Lesung}{Fachschaft UFG/VA}
{   
    Der Antragstext entspricht der Satzung im Anhang \fullref{sec:appendix1}.\\
    Hier wird lediglich noch einmal die Synopse wiederholt.\\[1.5em]

    \myparagraph{ Neue Präambel (es gibt keine ehemalige)}
    In dem Bestreben, der Fachschaftsarbeit an der Ruprecht-Karls Universität Heidelberg
    eine dauerhafte und bestimmte Grundlage zu geben, haben sich die Studierenden der
    Fächer Geoarchäologie, Ur- und Frühgeschichte sowie Vorderasiatische Archäologie als
    Fachschaft Ur- und Frühgeschichte und Vorderasiatische Archäologie (UFG/VA) folgende
    Satzung gegeben.\newline Die Fachschaft steht für ein Studium ein, in dem sich alle Studierenden individuell
    entfalten und das eigene Recht auf Selbstbestimmung – im Rahmen der Gesetze –
    ausleben kann. In unserem Einsatz für ein solches Studium sehen wir uns als politisch
    neutral und respektieren die Religionsfreiheit unserer Studierenden. Wir fühlen uns
    in unserem Engagement – im Rahmen der Gesetze – ausschließlich durch den freien
    Willen und die unverletzliche Würde des Menschen bestärkt und verpflichtet. Damit
    sich dieser Gedanke in seiner Lebendigkeit entfalten und unermüdlich, aufrichtig und
    frei innerhalb von Universität und Studierendenschaft wirken kann, geben wir uns
    folgende Satzung und nehmen im Rahmen der Erfüllung unserer Aufgaben nach § 65 LHG
    unser – begrenztes – politisches Mandat wahr. Zudem ist die Fachschaft darum bemüht,
    für ein besseres Miteinander von Studierenden und Institut und einen besseren
    Zusammenhalt der Studierenden zu sorgen. Begründung: Dies ist von der VS als
    Kernaufgabe der Fachschaften vorgegeben und hatte in der bisherigen Arbeit unserer
    Fachschaft auch eine wichtige Bedeutung.
    \begin{longtable}{|p{7.5cm}|p{7.5cm}|}
        \hline
        \multicolumn{2}{|c|}{Synopse}\\\hline
        Bisheriger Text & Neuer Text \\\hline
        \endfirsthead
        \hline
        Bisheriger Text & Neuer Text \\
        \hline
        \endhead
        \hline
        \multicolumn{2}{|r|}{Weiter auf der nächsten Seite...}\\
        \hline
        \endfoot
        \hline
        \multicolumn{2}{c}{Ende der Synopse} \\
        \endlastfoot
        \multicolumn{2}{|c|}{§1 Allgemeines}\\\hline
        (1)  Die Studienfachschaft vertritt die Studierenden des Fachbereichs „ Ur-
        und  Frühgeschichte und Vorderasiatische Archäologie“ und entscheidet insbesondere
        über fachspezifische Fragen und Anträge.&
        (1)  Die Studienfachschaft (im Folgenden „Fachschaft”) vertritt die Studierenden des
        Fachbereichs „Ur-und Frühgeschichte und Vorderasiatische Archäologie“ sowie „Geoarchä
        ologie” und entscheidet insbesondere über fachspezifische Fragen und Anträge.\\
        (2)  Die Zugehörigkeit zur Studienfachschaft ergibt sich aus der Liste in
        Anhang B.&
        (2)  Die Zugehörigkeit zur Studienfachschaft ergibt sich aus der Liste in Anhang B.\\
        (3)  Die Studienfachschaft stellt die studentischen Mitglieder der in ihrem
        Bereich arbeitenden\newline (4) Gremien oder beteiligt sich zumindest an einem gemeinsamen
        Wahlvorschlag für ebendiese. &
        (3)  Die Studienfachschaft stellt die studentischen Mitglieder der in ihrem Bereich
        arbeitenden Gremien oder beteiligt sich zumindest an einem gemeinsamen Wahlvorschlag
        für ebendiese.\\
        (5) Organe der Studienfachschaft sind die Fachschaftsvollversammlung und der
        Fachschaftsrat. &
        (4)  Organe der Studienfachschaft sind die Fachschaftsvollversammlung und der
        Fachschaftsrat.\\
        \hline
        \multicolumn{2}{|c|}{§2 Fachschaftsvollversammlung}\\\hline
        (1)  Die Fachschaftsvollversammlung ist die Versammlung der Mitglieder der
        Studienfachschaft. Sie tagt öffentlich, soweit gesetzliche Bestimmungen nicht
        entgegenstehen.&
        (1)  Die Fachschaftsvollversammlung ist die Versammlung der Mitglieder der
        Fachschaft.  Sie tagt öffentlich, soweit gesetzliche Bestimmungen nicht
        entgegenstehen.\\
        (2) Rede-, antrags- und stimmberechtigt sind alle anwesenden Mitglieder der
        Studienfachschaft.&
        (2)  Rede-, antrags- und stimmberechtigt sind alle anwesenden Mitglieder der
        Fachschaft. \\
        (3)  Beschlüsse werden mit einfacher Mehrheit gefasst.&
        (3)  Beschlüsse werden mit einfacher Mehrheit gefasst.\\
        (4)  Die gefassten Beschlüsse sind bindend für den Fachschaftsrat.&
        (4)  Die gefassten Beschlüsse sind bindend für den Fachschaftsrat.\\
        &
        (5) Die Fachschaftsvollversammlung bestimmt im Einvernehmen des Fachschaftsrats bis
        zu zwei Finanzverantwortliche der Fachschaft. Die Finanzverantwortlichen müssen
        eingeschriebene Studierende sein. Die Amtszeit beträgt in der Regel ein Jahr. \\
        &
        (6) Zum Ende der Amtszeit der Finanzverantwortlichen prüft der Fachschaftsrat deren
        Arbeit und beantragt anschließend die Entlastung der Finanzverantwortlichen in der
        Fachschaftsvollversammlung. Diese beschließt die Entlastung der
        Finanzverantwortlichen mit einfacher Mehrheit.\\
        &(7) Die Fachschaftsvollversammlung kann Abstimmungsempfehlungen für das StuRa-
        Mitglied beschließen. Diese sind nicht bindend.\\
        &(8) Die Fachschaftsvollversammlung bestimmt jeden November aus ihrer Mitte bis zu
        drei Personen, welche die Anträge für die Qualitätssicherungsnachfolgemittel (QSM)
        der Fachschaft vorbereiten (QSM-Kommission der Fachschaft). Näheres regelt § 5 dieser
        Satzung.\\
        (5)  Fachschaftsvollversammlungen müssen unverzüglich vom Fachschaftsrat
        einberufen werden:
        \begin{itemize}
        \item[5a]auf Antrag eines Drittels der Mitglieder des Fachschaftsrates oder
        \item[5b]auf schriftlichen Antrag von 1\% der Mitglieder der Studienfachschaft. 
        \end{itemize}&(9)  Fachschaftsvollversammlungen müssen unverzüglich vom Fachschaftsrat einberufen
        werden:
        \begin{itemize}
        \item[9a] auf Antrag eines Drittels der Mitglieder des Fachschaftsrates oder
        \item[9b] auf schriftlichen Antrag von 1\% der Mitglieder der Fachschaft. 
        \end{itemize}\\
        (6) Die Einberufung einer Fachschaftsvollversammlung muss mindestens fünf Tage
        vorher öffentlich und in geeigneter Weise bekannt gemacht werden.&
        (10) Die Einberufung einer Fachschaftsvollversammlung muss mindestens fünf Tage
        vorher öffentlich und in geeigneter Weise bekannt gemacht werden.\\
        &(11) Eine Fachschaftsvollversammlung ist beschlussfähig, wenn sie ordnungsgemäß
        einberufen wurde, mindestens die Hälfte der Fachschaftsräte und insgesamt mindestens
        2 Mitglieder der Fachschaft anwesend sind.\\\hline
        \multicolumn{2}{|c|}{§3 Fachschaftsrat}\\\hline
        (1)  Der Fachschaftsrat wird in gleicher, direkter, freier und geheimer Wahl gewählt.
        Es findet Personenwahl statt.&(1)  Der Fachschaftsrat wird in gleicher, direkter, freier und geheimer Wahl gewählt.
        Es findet Personenwahl statt.\\
        (2)  Alle Mitglieder der Studienfachschaft haben das aktive und passive Wahlrecht.&(2)  Alle Mitglieder der Studienfachschaft haben das aktive und passive Wahlrecht.\\
        (3)  Der Fachschaftsrat umfasst mindestens zwei Mitglieder.&(3)  Der Fachschaftsrat umfasst mindestens zwei und maximal acht Mitglieder.\\
        (4)  Der Fachschaftsrat nimmt die Interessen der Mitglieder der Studienfachschaft wahr. & (4)  Der Fachschaftsrat nimmt die Interessen der Mitglieder der Fachschaft wahr.\\
        (5)  Zu den Aufgaben des Fachschaftsrats gehören:
        \begin{itemize}
        \item[5a]Einberufung und Leitung der Fachschaftsvollversammlung.
        \item[5b]Ausführung der Beschlüsse der Fachschaftsvollversammlung.
        \item[5c]Führung der Finanzen.
        \item[5d]Beratung und Information der Studienfachschaftsmitglieder.
        \item[5e]Mitwirkung an der Lehrplangestaltung.
        \item[5f]Austausch und Zusammenarbeit mit den Mitgliedern des Lehrkörpers des Fachbereichs Ur- und Frühgeschichte und Vorderasiatische Archäologie.    
        \end{itemize}&
        (5)  Zu den Aufgaben des Fachschaftsrats gehören:
        \begin{itemize}
        \item[5a]Einberufung und Leitung der Fachschaftsvollversammlung.
        \item[5b]Ausführung der Beschlüsse der Fachschaftsvollversammlung.
        \item[5c]Führung de Finanzen sowie Prüfung der Arbeit der Finanzverantwortlichen sowie
        Beantragung der Entlastung dieser
        \item[5d]Beratung und Information der Studienfachschaftsmitglieder.
        \item[5e]Mitwirkung an der Lehrplangestaltung.
        \item[5f]Austausch und Zusammenarbeit mit den Mitgliedern des Lehrkörpers des Fachbereichs Ur- und Frühgeschichte und Vorderasiatische Archäologie.    
        \item[5g]Unterstützung der QSM-Kommission der Fachschaft bei ihrer Arbeit. 
        \end{itemize}\\
        (6)  Die Amtszeit der Mitglieder des Fachschaftsrats beträgt ein Jahr.&(6)  Die Amtszeit der Mitglieder des Fachschaftsrats beträgt ein Jahr. Die Amtszeit
        beginnt zum 01. April eines jeden Jahres.*\\
        (7)  Für das vorzeitige Ausscheiden aus dem Fachschaftsrat gilt § 35 OS. Außerdem
        scheidet eine Person aus dem Fachschaftsrat aus, wenn sie nicht mehr für einen der
        Studiengänge, welche die Studienfachschaft vertritt, immatrikuliert ist.&(7)  Für das vorzeitige Ausscheiden aus dem Fachschaftsrat gilt die
        Organisationssatzung des StuRa.\\
        (8) Im Falle des Ausscheidens eines Mitglieds des Fachschaftsrats rückt die Person
        mit der nachfolgenden Stimmenzahl für die verbleibende Amtszeit des ausscheidenden
        Mitglieds in den Fachschaftsrat nach.&(8) Im Falle des Ausscheidens eines Mitglieds des Fachschaftsrats rückt die Person
        mit der nachfolgenden Stimmenzahl für die verbleibende Amtszeit des ausscheidenden
        Mitglieds in den Fachschaftsrat nach.\\\hline
        \multicolumn{2}{|c|}{§4 Kooperation und Stimmführung im Studierendenrat}\\\hline
        (1)  Der Fachschaftsrat entsendet Vertreter/innen der Fachschaft in den
        Studierendenrat.&(1)  Der Fachschaftsrat entsendet ein Mitglied der Fachschaft in den Studierendenrat
        (StuRa).\\
        &(2) Der Fachschaftsrat entsendet zudem Stellvertreter*innen in den StuRa. \\
        (2)  Die Amtszeit der Vertreter/innen im StuRa beträgt ein Jahr.& (3)  Die Amtszeit der Entsandten im StuRa beträgt ein Jahr.\\
        (3)  Für das vorzeitige Ausscheiden aus dem StuRa gilt § 35 OS. & (4)  Für das vorzeitige Ausscheiden aus dem Studierendenrat gilt die
        Organisationssatzung des StuRa. \\
        &(5) Das StuRa-Mitglied und dessen Stellvertreter*innen können per Beschluss mit 2/3-
        Mehrheit in der Fachschaftsvollversammlung abberufen werden. \\
        &(6) Das StuRa-Mitglied und dessen Stellvertreter*innen stimmen nach bestem Wissen und
        Gewissen im Studierendenrat ab. \\
        &(7) Das StuRa-Mitglied und dessen Stellverterer*innen orientieren sich an den
        Abstimmungsempfehlungen der Fachschaftsvollversammlung. \\
        (4)  Die Studienfachschaft kann sich nach § 14 der Organisationssatzung der
        Studierendenschaft mit anderen Studienfachschaften zu einer Kooperation
        zusammenschließen.&
        (8)  Die Fachschaft kann sich nach § 14 der Organisationssatzung der
        Studierendenschaft mit anderen Fachschaften zu einer Kooperation zusammenschließen.\\\hline
        \multicolumn{2}{|c|}{§5 Qualitätssicherungsnachfolgemittel}\\\hline
        &(1) Die Fachschaftsvollversammlung bestimmt jeden November aus ihrer Mitte bis zu
        drei Personen, welche die Anträge für die QSM vorbereiten. Diese bilden die
        QSM-Kommission der Fachschaft.\\
        &(2) Nach Bildung der QSM-Kommission wird das QSM-Referat über dessen Mitglieder
        informiert. \\
        &(3) Vorschläge für die Verwendung der QSM müssen bis spätestens zwei Wochen vor
        Antragsfrist bei der QSM-Kommission der Fachschaft eingereicht werden.\\
        &(4) Bei der Vergabe sind die Mittel auf UFG und VA getrennt, der Anzahl der
        Studierenden entsprechend, zu veranschlagen. Die Mittel der Geoarchäologie werden
        denen der UFG zugerechnet. \\
        &(5) Per Beschluss der QSM-Kommission der Fachschaft können die Mittel auch gemeinsam
        veranschlagt werden. Sollte die Kommission nur aus einer Person, oder nur Personen
        einer der Fächer bestehen, so muss dieser Beschluss vom Fachschaftsrat getroffen
        werden. \\
        &(6) Aufgaben der QSM-Kommission der Fachschaft sind: 
        \begin{itemize}
        \item[6a] Die vorzeitige Information über den zur Verfügung stehenden Betrag für die QSM;
        \item[6b]Die Vorbereitung der Anträge für die QSM in Rücksprache mit der Fachschaft;
        \item[6c]   Die Fristgerechte Einreichung der QSM-Anträge.
        \end{itemize}\\
        & Die Änderung dieser Satzung tritt zum 01. Januar 2021 in Kraft.\\
    \end{longtable}
}{
    Einige der Änderungen sind zur Lesbarkeit, andere wie die Einführung einer Fachschaftseigenen QSM-Kommission entspringen der Notwendigkeit. Ebenso haben wir die Geoarchäologie, die wir ja auch vertreten, endlich mitaufgenommen.
}{
    \textbf{1. Lesung}
    \ul{
	\li{keine Fragen}
    }
    \textbf{2. Lesung}
    \ul{
	\li{keine Fragen}
    }
    \textbf{3. Lesung}
    \ul{
	\li{keine Fragen}
    }
    \textbf{4. Lesung}
    \ul{
	\li{keine Fragen}
    }
}{

}

\antrag{Fusion der Fachschaften Klassische Archäologie und Byzantinische Archäologie und Kunstgeschichte}{4. Lesung}{Fachschaft Byzantinische Archäologie und Kunstgeschichte, Fachschaft Klassische Archäologie}
{       Der Antragstext entspricht der Satzung im Anhang \fullref{sec:appendix2}.\\
        Hier wird lediglich noch einmal die Synopse wiederholt.\\[1.5em]
    \begin{longtable}{|p{7.5cm}|p{7.5cm}|}
        \hline
        \multicolumn{2}{|c|}{Synopse}\\\hline
        Bisheriger Text & Neuer Text \\\hline
        \endfirsthead
        \hline
        Bisheriger Text & Neuer Text \\
        \hline
        \endhead
        \hline
        \multicolumn{2}{|r|}{Weiter auf der nächsten Seite...}\\
        \hline
        \endfoot
        \hline
        \multicolumn{2}{c}{Ende der Synopse} \\
        \endlastfoot
        \multicolumn{2}{|c|}{Anhang D}\\\hline
        1. Ägyptologie                                                   & 1. Ägyptologie                                                   \\
        2. Alte Geschichte                                               & 2. Alte Geschichte                                               \\
        3. American Studies                                              & 3. American Studies                                              \\
        4. Anglistik                                                     & 4. Anglistik                                                     \\
        5. Assyriologie                                                  & 5. Assyriologie                                                  \\
        6. Byzantinische Archäologie und Kunstgeschichte                 &                                                                  \\
        7. Biologie                                                      & 6. Biologie                                                      \\
        8. Chemie und Biochemie                                          & 7. Chemie und Biochemie                                          \\
        9. Computerlinguistik                                            & 8. Computerlinguistik                                            \\
        10. Deutsch als Fremdsprache                                     & 9. Deutsch als Fremdsprache                                      \\
        11. Erziehung und Bildung                                        & 10. Erziehung und Bildung                                        \\
        12. Ethnologie                                                   & 11. Ethnologie                                                   \\
        13. Geographie                                                   & 12. Geographie                                                   \\
        14. Geowissenschaften                                            & 13. Geowissenschaften                                            \\
        15. Germanistik                                                  & 14. Germanistik                                                  \\
        16. Gerontologie \& Care                                         & 15. Gerontologie \& Care                                         \\
        17. Geschichte                                                   & 16. Geschichte                                                   \\
        18. Informatik                                                   & 17. Informatik                                                   \\
        19. Islamwissenschaft                                            & 18. Islamwissenschaft                                            \\
        20. Japanologie                                                  & 19. Japanologie                                                  \\
        21. Jura                                                         & 20. Jura                                                         \\
        22. Klassische Archäologie                                       & 21. Klassische und Byzantinische Archäologie                     \\
        23. Klassische Philologie                                        & 20. Klassische Philologie                                        \\
        24. Kunstgeschichte (Europäische)                                & 21. Kunstgeschichte (Europäische)                                \\
        25. Mathematik                                                   & 24. Mathematik                                                   \\
        26. Medizin Heidelberg                                           & 25. Medizin Heidelberg                                           \\
        27. Medizin Mannheim                                             & 26. Medizin Mannheim                                             \\
        28. Mittellatein/Mittelalterstudien                              & 27. Mittellatein/Mittelalterstudien                              \\
        29. Molekulare Biotechnologie                                    & 28. Molekulare Biotechnologie                                    \\
        30. Musikwissenschaft                                            & 29. Musikwissenschaft                                            \\
        31. Ostasiatische Kunstgeschichte                                & 30. Ostasiatische Kunstgeschichte                                \\
        32. Pharmazie                                                    & 31. Pharmazie                                                    \\
        33. Philosophie                                                  & 32. Philosophie                                                  \\
        34. Physik                                                       & 33. Physik                                                       \\
        35. Politikwissenschaft                                          & 34. Politikwissenschaft                                          \\
        36. Psychologie                                                  & 35. Psychologie                                                  \\
        37. Religionswissenschaft                                        & 36. Religionswissenschaft                                        \\
        38. Romanistik                                                   & 37. Romanistik                                                   \\
        39. Semitistik                                                   & 38. Semitistik                                                   \\
        40. Sinologie                                                    & 39. Sinologie                                                    \\
        41. Slavistik/Osteuropastudien                                   & 40. Slavistik/Osteuropastudien                                   \\
        42. Soziologie                                                   & 43. Soziologie                                                   \\
        43. Sport                                                        & 42. Sport                                                        \\
        44. Südasieninwissenschaften (Fachschaft am SAI)                 & 43. Südasieninwissenschaften (Fachschaft am SAI)                 \\
        45. Theologie (Evangelische)                                     & 44. Theologie (Evangelische)                                     \\
        46. Transcultural Studies (891)                                  & 45. Transcultural Studies (891)                                  \\
        47. Ur- und Frühgeschichte/Vorderasiatische Archäologie (UFG/VA) & 46. Ur- und Frühgeschichte/Vorderasiatische Archäologie (UFG/VA) \\
        48. Übersetzen und Dolmetschen (Fachschaft am IÜD)               & 47. Übersetzen und Dolmetschen (Fachschaft am IÜD)               \\
        49. Volkswirtschaftslehre (VWL)                                  & 48. Volkswirtschaftslehre (VWL)                                  \\
        \multicolumn{2}{|c|}{Anhang B}\\\hline
        (6) Byzantinische Archäologie und Kunstgeschichte (830, 8302, 8305, 8304)
        (Byzantinische Archäologie und Kunstgeschichte)\newline (22) Klassische Archäologie (831, 8317, 8312, 8315, 8314, 8347, 12N, 849) (Klassische
        Archäologie) & (21) Klassische und Byzantinische Archäologie (831, 8317, 8312, 8315, 8314, 8347,
        12N, 849) (Klassische Archäologie) und (830, 8302, 8305, 8304) (Byzantinische
        Archäologie und Kunstgeschichte)\\
    \end{longtable}
}{
    Nach der Fusion der Institute haben die beiden Fachschaften beschlossen, dass es für die Wahrnehmung der Vertretung der Studierenden der beiden Fächer leichter ist, sich zu einer FS zusammenzuschließen.
}{
    \textbf{1. Lesung}
    \ul{
	\li{keine Fragen}
    }
    \textbf{2. Lesung}
    \ul{
	\li{keine Fragen}
    }
    \textbf{3. Lesung}
    \ul{
	\li{keine Fragen}
    }
    \textbf{4. Lesung}
    \ul{
	\li{keine Fragen}
    }
}{

}

\antrag{Satzung der neuen Fachschaft Klassische und Byzantinische Archäologie}{4. Lesung}{Fachschaft Byzantinische Archäologie und Kunstgeschichte, Fachschaft Klassische Archäologie}
{
    \textbf{Satzung  der  Studienfachschaft  Klassische  und  byzantinische  Archäologie  der Universität Heidelberg}
    \paragraph{Präambel}
    Aufgrund von § 65 a Abs. 1 Landeshochschulgesetz vom 1. Januar 2005 in der Fassung des Artikels 1 des Gesetzes vom 1. April 2014 (GBl. S. 99) und § 17 Abs.4 Organisationssatzung der Verfassten Studierendenschaft  (Satzung)  vom  31.  Mai  2013  (Mitteilungsblatt  des  Rektors  S.  517  ff.)  zuletzt geändert  durch  Satzung  vom  17.  August  2015  (Mitteilungsblatt  des  Rektors  S.  1437  ff.)  hat  der Studierendenrat  (StuRa)  der  Universität  Heidelberg  am [Datum]die  nachfolgende  Satzung beschlossen.
    \paragraph{§1 Allgemeines}
    \begin{enumerate}
        \item[(1)] Die Studienfachschaft    vertritt    die    Studierenden    des    Fachbereichs    „Klassische Archäologie“  und  „ByzantinischeArchäologieund  Kunstgeschichte“  und  entscheidet insbesondere über fachspezifische Fragen und Anträge.
        \item[(2)] Die Zugehörigkeit zur Studienfachschaft ergibt sich aus der Liste in Anhang B.
        \item[(3)] Die  Studienfachschaft  stellt  die  studentischen  Mitglieder  der  in  ihrem  Bereich  arbeitenden Gremien  oder  beteiligt  sich  zumindest  an  einem  gemeinsamen  Wahlvorschlag  für  eben diese.
        \item[(4)] Organe der Studienfachschaft sind die Fachschaftsvollversammlung und der Fachschaftsrat. Weitere Organe sind möglich (nach §3 Abs. 2 OrgS und §11 Abs. 5 OrgS).
    \end{enumerate}

    \paragraph{§2 Fachschaftsvollversammlung}
    \begin{enumerate}
        \item[(1)] Die Fachschaftsvollversammlung ist die Versammlung der Mitglieder der Studienfachschaft.Sie tagt öffentlich, soweit gesetzliche Bestimmungen nicht entgegenstehen.
        \item[(2)] Rede-, antrags-und stimmberechtigt sind alle anwesenden Mitglieder der Studienfachschaft.
        \item[(3)] Von jeder Sitzung ist ein Protokoll anzufertigen und öffentlich zugänglich zu machen.
        \item[(4)] Beschlüsse werden mit einfacher Mehrheit gefasst.
        \item[(5)] Die gefassten Beschlüsse sind bindend für den Fachschaftsrat.
        \item[(6)] {Die  Fachschaftsvollversammlung  müssen  unverzüglich  vom  Fachschaftsrat  einberufen werden: 
            \begin{enumerate}
                \item[6a.] auf Antrag eines Drittels der Mitglieder des Fachschaftsrates
                \item[] ODER 
                \item[6b.]  auf schriftlichen Antrag von 1\% der Mitglieder der Studienfachschaft.
            \end{enumerate}}
        \item[(7)] Die Einberufung der Fachschaftsvollversammlung muss mindestens 5 Tage vorher öffentlich und in geeigneter Weise sowieortsüblich bekannt gemacht werden.   
    \end{enumerate}
    \paragraph{§3 Fachschaftsrat}
    \begin{enumerate}
        \item[(1)] Der  Fachschaftsrat  wird  in  gleicher,  direkter,  freier  und  geheimer  Wahl  gewählt.  Es  findet Personenwahl statt.
        \item[(2)] Alle Mitglieder der Studienfachschaft haben das aktive und passive Wahlrecht.
        \item[(3)] Der Fachschaftsrat umfasst mindestens zwei Mitglieder.Der Fachschaftsrat setzt sich durch einen  Vertreter  der  „Klassischen  Archäologie“  und  der  „Byzantinischen  Archäologie  und Kunstgeschichte“  zusammen,  um  (4)  optimal  gewährleisten  zu  können,  sofern  sich  aus beiden Fächern jeweils einen Vertreter finden lassen.
        \item[(4)] Der Fachschaftsrat nimmt die Interessen der Mitglieder der Studienfachschaft wahr.
        \item[(5)] {Zu den Aufgaben des Fachschaftsrats gehören:
            \begin{enumerate}
                \item[5a] Einberufung und Leitung der Fachschaftsvollversammlung.
                \item[5b] Ausführung der Beschlüsse der Fachschaftsvollversammlung.
                \item[5c] Führung der Finanzen.
                \item[5d] Beratung und Information der Studienfachschaftsmitglieder.
                \item[5e] Mitwirkung an der Lehrplangestaltung.
                \item[5f] Austausch und Zusammenarbeit mit den Mitgliedern des Lehrkörpers derFachbereiche„Klassische Archäologie“ und „Byzantinische Archäologieund Kunstgeschichte“.
            \end{enumerate}}
        \item[(6)] Die Amtszeit der Mitglieder des Fachschaftsrats beträgt ein Jahr.
        \item[(7)] Für das vorzeitige Ausscheiden aus dem Fachschaftsrat gilt § 47 OrgS. 
    \end{enumerate}
    \paragraph{§4 Fachschaftsvollversammlung}
    \begin{enumerate}
        \item[(1)] Der Fachschaftsrat   entsendet   Vertreter*innender   Fachschaft   in   den   StuRa.   Eine Stellvertretung ist möglich. 
    \end{enumerate}
    \paragraph{§5 Übergangsbestimmung}
    \begin{enumerate}
        \item[(1)] Diese  Satzung  tritt  zum  01.04.2021  in  Kraft.  Die  Wahl  der  neuen  FSR-Mitglieder  nach dieser Satzung wird im Wintersemester 2020/21 durchgeführt.
        \item[(2)] Übergangsregelung  für  die  Finanzen:  die  Budgets  der  beiden  bisherigen  Fachschaften werden zum 01.05.21 zusammengelegt und von der neuen Fachschaft bewirtschaftet.
        \item[(3)] Übergangsregelgung für die Entsendung in den StuRa: die bisherigen Vertreter*innen beider bisherigen Fachschaften bleiben bis 30.09.21 im Amt. Danach wird nach der neuen Satzung enstsandt
        \item[(4)] Übergangsregelung  für  die  QSM:  das  Vorschlagsrecht  für  die  QSM  2020  der  bisherigen Fachschaften Klassische Archäologie und Byzantinischen Archäologie und Kunstgeschichte werden von den bisherigen Fachschaftsräten der Klassische Archäologie und Byzantinische Archäologie und Kunstgeschichte wahrgenommen. Für QSM, für die nach dem 1.4.21 kein Vorschlag vorliegt oder die zurückfließen nimmt der neue FSR das Vorschlagsrecht wahr.  
    \end{enumerate}
}{
    Nach der Fusion der Institute haben die beiden Fachschaften beschlossen, dass es für die Wahrnehmung der Vertretung der Studierenden der beiden Fächer leichter ist, sich zu einer FS zusammenzuschließen.
}{
    \textbf{1. Lesung}
    \ul{
	\li{keine Fragen}
    }
    \textbf{2. Lesung}
    \ul{
	\li{keine Fragen}
    }
    \textbf{3. Lesung}
    \ul{
	\li{keine Fragen}
    }
    \textbf{4. Lesung}
    \ul{
	\li{keine Fragen}
    }
}{

}
\iffalse
\antrag{Antrag zur Festschreibung von Digitalen Wahlen in der regulären Wahlzeit \label{satzungen:4}}{2. Lesung}{Liste Juso-HSG}
{
    \begin{longtable}{|p{7.5cm}|p{7.5cm}|}
        \hline
        \multicolumn{2}{|c|}{Synopse}\\\hline
        Bisheriger Text & Neuer Text \\\hline
        \endfirsthead
        \hline
        Bisheriger Text & Neuer Text \\
        \hline
        \endhead
        \hline
        \multicolumn{2}{|r|}{Weiter auf der nächsten Seite...}\\
        \hline
        \endfoot
        \hline
        \multicolumn{2}{c}{Ende der Synopse} \\
        \endlastfoot
        \multicolumn{2}{|c|}{§36 wird gestrichen und als neuer §22 übernommen}\\\hline
        § 36 Digitalisiertes Wählerverzeichnis \newline Das Wählerverzeichnis für die nach Abschnitt II durchzuführenden Wahlen und Abstimmungen kann digital geführt werden. Die gemäß den Bestimmungen dieser Ordnung vorzunehmenden Bestätigungen, Berichtigungen, Eintragungen etc. können entsprechend elektronisch kenntlich gemacht oder eingetragen werden. Ist dies nicht möglich, ist über den Vorgang ein Vermerk auf Papier anzufertigen.
        &
        § 22 Digitalisiertes Wählerverzeichnis \newline Das Wählerverzeichnis für die nach Abschnitt II durchzuführenden Wahlen und Abstimmungen kann digital geführt werden. Die gemäß den Bestimmungen dieser Ordnung vorzunehmenden Bestätigungen, Berichtigungen, Eintragungen etc. können entsprechend elektronisch kenntlich gemacht oder eingetragen werden. Ist dies nicht möglich, ist über den Vorgang ein Vermerk auf Papier anzufertigen.
        \\\hline
        \multicolumn{2}{|c|}{§36a und §36b werden gestrichen}\\\hline
        § 36a Digitale Stimmabgaben bei Wahlen nach Abschnitt II \newline
        (1) Die Stimmabgabe bei nach Abschnitt II durchzuführenden Wahlen kann abweichend von der dort vorgesehenen Urnenwahl als digital (online) Wahl durchgeführt werden, wenn der Wahlausschuss dies mit Zustimmung der Referatekonferenz beschließt. Dieser Beschluss darf nur gefasst werden, wenn die Einhaltung der Wahlgrundsätze (§ 65a Absatz 2 Satz 1 LHG und § 44 Absatz 1 Satz 1 OrgS) einschließlich der Öffentlichkeit der Wahl gewährleistet werden kann. Er soll nur gefasst werden, wenn rechtliche Vorgaben oder tatsächlichen Ereignisse (bspw. Versammlungsverbote, Ausgangssperren, Naturkatastrophen, Einstellung oder Beschränkung der Präsenzlehre, etc.) die Durchführung der Wahlen als Urnenwahl nicht möglich machen oder diese zumindest als nicht zweckmäßig erscheinen.\newline
        (2) Für die Durchführung von digitalen (online) Wahlen werden ergänzende Satzungsbestimmungen erlassen, die insbesondere Näheres bestimmen zu: - Wahrung der Öffentlichkeit der Wahl und des Wahlgeheimnisses - Technische Anforderungen an das System (Schutz vor Manipulationen) - Wahlzeitraum und Form der Stimmabgabe - Feststellung des Wahlergebnisses - Vorgehen bei Störung der Wahl, Verlängerung des Wahlzeitraumes - Besonderheiten der Bekanntmachung - gegebenenfalls weitere notwendige Modifikationen zu dieser Wahlordnung\newline
    &
        §23 Regelung der digitalen Stimmabgabe
        \begin{enumerate}
            \item Eine Stimmabgabe digital (online) ist grundlegend gestattet.
            \item {Die Wahrung von
                \begin{itemize}
                    \item der Öffentlichkeit der Wahl und des Wahlgeheimnisses
                    \item den technische Anforderungen an das System (Schutz vor Manipulationen)
                    \item dem Wahlzeitraum und Form der Stimmabgabe
                    \item den Feststellung des Wahlergebnisses
                    \item Vorgehen bei Störung der Wahl, Verlängerung des Wahlzeitraumes
                    \item Besonderheiten der Bekanntmachung
                \end{itemize}
                obliegt den Wahlveranstaltern, die durch den StuRa eingesetzt werden.}
            \item Digitale (Online) Wahlen ersetzen nicht die ursprünglichen Wahlarten, sondern ergänzen sie.
            \item Erfolgt die Wahl per Brief so findet § 13 ausgenommen der Absätze 1, 5 und 9 Satz 2 entsprechende Anwendung. Wird die Wahl digital (online) durchgeführt, so ist sie über ein (online) Wahl-oder Versammlungs-Tool durchzuführen. Im Rahmen der hierfür zumutbaren technischen, personellen und finanziellen Möglichkeiten ist sicherzustellen, dass die Wahl ohne eine Möglichkeit zur Manipulation und unter Wahrung des Wahlgeheimnisses erfolgen kann.
        \end{enumerate} \\
        § 36b Digitale oder briefliche Stimmabgaben bei Wahlen nach Abschnitt III\newline
        (1) Die Stimmabgabe bei nach Abschnitt III durchzuführenden Wahlen kann abweichend von § 28 Absatz 4 digital (online) oder per Brief erfolgen, wenn es dem Studierendenrat aufgrund von rechtlichen Vorgaben oder tatsächlichen Ereignissen, die außerhalb seiner Verantwortung liegen (bspw. Versammlungsverbote, Ausgangssperren, Naturkatastrophen, etc.), unmöglich ist, sich zu versammeln und die Sitzungsleitung sodann seine Entscheidungen im Wege von in seiner Geschäftsordnung vorgesehenen Alternativen (Umlaufverfahren, Videokonferenzen, etc.) herbeiführt.\newline
        (2) Erfolgt die Wahl per Brief so findet § 13 ausgenommen der Absätze 1, 5 und 9 Satz 2 entsprechende Anwendung. Wird die Wahl digital (online) durchgeführt, so ist sie über ein (online) Wahl- oder Versammlungs-Tool durchzuführen. Im Rahmen der hierfür zumutbaren technischen, personellen und finanziellen Möglichkeiten ist sicherzustellen, dass die Wahl ohne eine Möglichkeit zur Manipulation und unter Wahrung des Wahlgeheimnisses erfolgen kann.\newline
        (3) Alle Entscheidungen nach diesem Paragraphen werden von der Sitzungsleitung des Studierendenrates im Einvernehmen mit dem EDVReferat vorbereitet. Sie gelten als vom Studierendenrat bestätigt, wenn dieser nicht anders entscheidet.\newline
        & \\\hline
        \multicolumn{2}{|c|}{alles verschiebt sich}\\\hline
        § 22 & § 24 \\
        § 23 & § 25 \\
        § 24 & § 26 \\
        § 25 & § 27 \\
        § 26 & § 28 \\
        § 27 & § 29 \\
        § 28 & § 30 \\
        § 29 & § 31 \\
        § 30 & § 32 \\
        § 31 & § 33 \\
        § 32 & § 34 \\
        § 33 & § 35 \\
        § 34 & § 36 \\
        § 35 & § 37 \\\hline
        \multicolumn{2}{|c|}{§ 36 wird gestrichen}\\\hline
        § 36 & gestrichen \\
        § 37 & § 38 \\
        § 38 & § 39 \\
    \end{longtable}
}{
    Obwohl eine globale Pandemie die Möglichkeiten zum Wahlkampf stark beschränkt hat, haben sich mit der Online Wahl 20\% der studierenden Personen zur Stimmen Abgabe bringen lassen, ca 7 \% mehr als in Vergleichszahl im Jahr zuvor. Online Wahlen senken die Hürden für in ihrer Mobilität eingeschränkten Personen und durch ihre bequeme Durchführung kann sie helfen mehr Leute für Hochschulpolitik zu gewinnen. Diese höhere Wahlbeteiligung stärkt nicht nur die Legitimität des StuRa's, sondern animiert Studierende auch eher sich demokratisch zu engagieren.\\
    Das Landeshochschulgesetz selbst empfiehlt sogar die Online-Wahl in §9 Absatz 8 Satz 5 LHG:\\
    „Die Wahlordnung soll Regelungen treffen, welche schriftlichen Erklärungen in Wahlangelegenheiten durch einfache elektronische Übermittlung, durch mobile Medien oder in elektronischer Form abgegeben werden können.“\\
    Als demokratische Institution sollten wir das auch umsetzen.
}{
    \textbf{1. Lesung:}
    \ul{
        \li{Sind die Online Wahlen damit auch nach der Pandemie als Ergänzung.}
            \noli{\ul{
            \lii{Ja, wegen höherer Wahlbeteiligung und technologischer Sicherheit.
            }}}
        \li{Wie sieht das mit der Sicherheit aus, dass die Uni ID von jemand anderem}
            \noli{\ul{
            \lii{Es gibt eine Abwägung zwischen Sicherheit und Wahlbeteiligung. Und das kommt sehr sehr selten vor.
            }}}
            \noli{\ul{
            \lii{Mehr Technik macht das nicht zu mehr als einer digitalen Briefwahl. Letzten Endes ist es eine Abwägung zwischen Wahlbeteiligung und Sicherheit. Auch ist es weniger relevant bei dieser Wahl zu bescheißen als bei einer Europawahl etc.
            }}}
        \li{Auch das Bild auf dem Studienausweiß ist ohnehin oft schwer zu erkennen.}
        \li{ Wer legt fest ob die Onlinewahl möglich ist und wer zahlt das dann?}
            \noli{\ul{
            \lii{Die Wahlkomission soll festlegen, ob online gewählt werden sollte. Auch wird erwartet, dass eine Mehrheit die Online Wahlen nutzen wird.
            }}}
        \li{Bei Online Wahlen wie geht das mit dem Haushaltsplan zusammen, der erstellt werden muss.}
            \noli{\ul{
            \lii{Es soll ein Änderungsantrag auf nur online Wahlen erarbeitet werden.
            }}}
        \li{Analog und digital zusammen wählen ist irre. Das ist organisatorisch wahnsinnig aufwändig und teuer}
        \li{Wie kann man die Sicherheit der Software gewerkstelligen}
            \noli{\ul{
            \lii{Die Wahlkomission sollte das machen.
            }}}
        \li{Entweder kann man selbst diese Sachen festschreiben oder man kann sich an schon jetzige Vorschreibungen halten. Auch ist es schwer digital und online gleichzeitig zu wählen, wegen der online Formatierung. Man kann aber in Notwahllokale gehen in denen man sicherer digital wählen kann.}
        \li{Diese Notwahllokale gewährleisten nur eine sichere Stimmabgabe für die Leute dort. Man will aber das für jeden gewährleisten. Auch kann man so Stimmen kaufen.}
            \noli{\ul{
            \lii{Das ist bei einer Briefwahl nicht anders. Online-Wahlen sind nicht hundertprozentig sicher, aber andere Wahlmethoden auch nicht. 
            }}}
        \li{Aber euer Wohnheim ist nicht vor der Neuen Uni / Zentralmensa}
        \li{Online Wahl heißt nicht unbedingt online Wahlkampf. Deswegen hat das nicht wirklich Einfluss auf den Wahlkampf.}
            \noli{\ul{
            \lii{Wahlkmapf ist kein Nullsummenspiel. Onlinewahlkampf schränkt Präsenzwahlkampf nicht ein.
            }}}
    }
    \textbf{2. Lesung}
    \ul{
	\li{keine Fragen}
    }
}{
    \abstimmungsergebnis{
        \fullref{satzungen:4}
    }{
        tba%Ja
    }{
        tba%Nein
    }{
        tba%Enth
    }{
        tba%Ergebnis  \ul{\noli{\ul{\lii{test}}}}
    }
}
\fi

\diskussion{Satzungsänderungen}{1. Lesung}{AK Satzungen}{
    \begin{itemize}
        \item  Organisationssatzung (OrgS)
        \item  Wahlordnung (WahlO)
        \item  Digitalwahlordnung (DigWahlO)
        \item  Beitragsordnung (BeitrO)
        \item  Finanzordnung (FinO)
        \item  Schlichtungsordnung (SchliO)
        \item  Aufwandsentschädigungsordnung (AEO)
        \item  Geschäftsordnung des StuRa (GeschOStuRa)
    \end{itemize}
}{
    Wir wollen kurz vorstellen, worum es geht und wo Handlungsbedarf besteht. Dann wollen wir gerne weitere Anregungen sammeln, um auf der Grundlage Änderungsanträge für die folgende Sitzung zu erarbeiten bzw. fertigzustellen.
    Dieses Vorgehen ermöglicht dem StuRa eine Diskussion der Thematik in drei Sitzungen und damit auch vor der ersten Lesung mehr Leuten, sich zu beteiligen.\\
    Wir brauchen Änderungen und Anpassungen, weil sich konkreter Handlungsbedarf gezeigt hat. Einige Abschnitte sind auch inhaltlich schwierig, da sie Verfahren festschreiben, die einfach nicht durchdacht und realitätsfern sind - und teilweise noch nie so wie beschrieben durchgeführt wurden.\\
    Nicht zuletzt hat sich die Corona-Situation, auf die einige kurzfristige Änderungen zielten, anders entwickelt als gedacht und einige Verfahren wie Videokonferenzen und Online-Wahlen sollten nach den bisherigen Erfahrungen auch dauerhaft als Möglichkeiten in unsere Ordnungen und Satzungen aufgenommen werden. Zumindest sollten wir es diskutieren.
}{
    \textbf{1. Lesung:}
    \ul{
	\li{Man sollte die Satzungen überarbeiten, weil die Satzungen teils sich widersprechen, manche Sachen sind nicht geregelt und manche Sachen sind übermäßig kompliziert. }
    }
}