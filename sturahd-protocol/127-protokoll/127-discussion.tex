

1 Begrüßung
}

2 Tagesordnung

2.1 Änderungen an der To
\textbf{X. Lesung:}
\ul{
	\li{Top 10.3 zurückgezogen}
}

3 Protokolle
	\li{keine Änderungsanträge}
}

4 Berichte

4.1 Bericht des Vorsitzes
\textbf{X. Lesung:}
\ul{
	\li{Bezüglich Neutralitätsgebot: würde das nicht die Aktion des PoBi Referats betreffen?}
		\noli{\ul{
		\lii{Das mit der Diskussion geht dann leider nicht.
		}}}
	\li{AfD nicht einladen?}
		\noli{\ul{
		\lii{Offiziell dürfen wir das nicht. Nächste Woche kommt Jonathan vom fzs und da können wir auf einer besseren Faktenbasis reden.
		}}}
}

4.2 Bericht der Härtefallkommission
\textbf{X. Lesung:}
\ul{
	\li{Die Satzung verbietet Förderung für mehr als 3 Monate. Sollte amn da nicht die Satzung ändern?}
		\noli{\ul{
		\lii{Das Problem ist, dass der Stura eine Zwangskörperschaft des öffentlichen Rechts ist. Deswegen braucht es gute Begründungen das zu machen. Wenn das politisch gewollt ist, ist das eine ganz andere Sache.
		}}}
	\li{Könnte man das temporär ändern?}
		\noli{\ul{
		\lii{Darauf müsste das Gremienreferat antworten. Die höchste Antragslast war nach dem ersten Lockdown. Denen geht es nicht anders. Dass es dieses Notlagenstipendium gibt ist nur ein Pflaster und das sind Aufgaben für andere Institutionen. 
		}}}
		\noli{\ul{
		\lii{Ergänzend: Wie viele Studierende haben ihr Studium abgebrochen, weil sie nicht wussten, dass es dieses Stipendium gibt. Wir sind alle Studierende und die HFK ist auch ein Zeitaufwand. Deswegen wären neue Mitglieder ganz gut.
		}}}
	\li{Würden die Referierenden empfehlen die Satzung zu verändern.}
		\noli{\ul{
		\lii{Wir als Stura können das noch mehr bewerben, um Studierende auf diese Option aufmerksam zu machen. Es wäre vielleicht besser es publik zu machen anstatt die Satzung zu ändern.
		}}}
		\noli{\ul{
		\lii{Was kann eine VS mit dem Geld am besten anfangen. Ob man das für ein Notlagenstipendium ausgeben will ist fraglich. Aber es wären auch zusätzliche Mitglieder für die HFK nötig, um das zu managen.
		}}}
	\li{Könnten wir ein Statement zur Abschaffung der Studiengebühren machen?}
		\noli{\ul{
		\lii{Das allgemeinpolitische Mandat wurde der VS entzogen. Aber das ist sehr hochschulpolitisch. Deswegen könnten wir das machen.
		}}}
	\li{Es stehen Wahlen bevor und da kann man das gut ansprechen. Auf Bundesebene ist das sehr traurig.}
	\li{Es ist noch ein Anliegen die HFK und das Notlagenstipendium bekannt zu machen. Muss man dafür einen Antrag stellen?}
		\noli{\ul{
		\lii{Wir brauchen hier mehr Personen die das ganze Prozedere in der HFK regeln. Und die HFK kann kaum die ganze Arbeit stemmen.
		}}}
	\li{Könnte man mehr Leute dafür begeistern durch Kompensation?}
		\noli{\ul{
		\lii{An wen war die Frage gerichtet?
		}}}
		\noli{\ul{
		\lii{Was die Aufwandsentschädigung angeht, wird das auch noch hier besprochen werden in der nächsten Sitzung.
		}}}
}

GO Antrag auf Stellung des Antrags des SDS zu Burschenschaften
\textbf{X. Lesung:}
\ul{
	\li{Der Antrag zu Burschenschaften soll gestellt werden ohne den zweiten Satz}
		\noli{\ul{
		\lii{Der zweite Satz will die Mitglieder von Burschenschaften von Ämtern der VS ausschließen. Das ist nicht nur rechtswidrig, sondern auch nicht wirklich mit der Demokratie vereinbar.
		}}}
	\li{Es ist nicht wirklich mit der Demokratie vereinbar, wenn die Sitzungsleitung einfach entscheidet, dass ein Antrag nicht diskutiert wird.}
		\noli{\ul{
		\lii{Das ist wehrhafte Demokratie.
		}}}
}

GO Antrag auf Fortsetzen der Tagesordnung
\textbf{X. Lesung:}
\ul{
	\li{Das ist jetzt wichtig und wir wollen Klarheit über die To haben.}
		\noli{\ul{
		\lii{ist angenommen, Sitzung geht weiter
		}}}
}

4.3 Bericht des Referats für hochschulpolitische Vernetzung
\textbf{X. Lesung:}
\ul{
	\li{Der Studiomat zur Landtagswahl: Wer hat das erstellt und wie sollen die anderen Referate dabei helfen?}
		\noli{\ul{
		\lii{Der Studiomat ist bei der Lak entstanden. Die Thesen wurden dann auch von der LAK bestätigt und die Parteien hatten 4 Wochen Zeit die Fragen zu beantworten. Viele Studierende kommen auch für das Studium nach BW und die haben nicht notwendigerweise den Wunsch wählen zu gehen.
		}}}
	\li{Es ist gut auf die StuVe zuzugehen. Aber die Stimmen werden nicht in der Sitzung gesammelt sondern davor. Dafür braucht es eine größere Strategie.}
		\noli{\ul{
		\lii{Das Studierendenwerksgesetz hat sich geändert. Bisher stand drin wieviel studentische Vertrende es gab und jetzt nicht mehr. Deswegen gibt es jetzt eine neue Satzung des Studierendenwerks. 
		}}}
	\li{Ist es nicht eher besser wenn man mit einem anderen Vorschlag hingeht?}
		\noli{\ul{
		\lii{Das ist richtig. Aber wir wollen zuerst an das Studierendenwerk herantreten und da schon im ersten Vorschlag eine höhere Anzahl an studentischen Vertretenden reinbringen.
		}}}
}

4.4 Bericht Studierendenwerk
\textbf{X. Lesung:}
\ul{
	\li{Es wurde nicht im Stura diskutiert da eine Satzungsänderung einzubringen. Warum? und was ist die Strategie }
		\noli{\ul{
		\lii{Den Antrag gab es schon einmal. Die Kernforderung nach studentischer Einbringung in diesen Gremien gab es schon sehr oft, weswegen das als nicht sehr kontrovers betrachtet wurde. Wir werden versuchen müssen und sehr stark einzubringen. Wir arbeiten viel mit den anderen Studierendenvertretenden zusammen und da war es auch sehr effektiv mit einer Stimme zu sprechen.
		}}}
}

4.4.1 Go-Antrag auf Ausschluss der Öffentlichkeit
\textbf{X. Lesung:}
\ul{
	\li{ohne Gegenrede angenommen}
}

5 Satzungen

5.1 Neufassung der Studienfachschaftssatzung UFG/VA
\textbf{X. Lesung:}
\ul{
	\li{nicht genügend stimmberechtigte Mitglieder hier}
}

5.2 Fusion der Fachschaften Klassische Archäologie und Byzantinische Archäologie und Kunstgeschichte
\textbf{X. Lesung:}
\ul{
	\li{nicht genügend stimmberechtigte Mitglieder hier}
}

5.3 Satzung der neuen Fachschaft Klassische und Byzantinische Archäologie 
\textbf{X. Lesung:}
\ul{
	\li{nicht genügend stimmberechtigte Mitglieder hier}
}

5.4 Antrag zur Festschreibung von digitalen Wahlen in der regulären Wahlzeit
\textbf{X. Lesung:}
\ul{
}

5.5 Satzungsänderungen
\textbf{X. Lesung:}
\ul{
}

6 Wahlen
}

6.1 Annalena Wirth für das Referat für hochschulpolitische Vernetzung
	\li{keine Fragen}
	\li{angenommen mit 35/2/2}

6.2 Marc Baltrun für das Referat für hochschulpolitische Vernetzung
\textbf{X. Lesung:}
\ul{
	\li{Um welche Themen geht es im Referat und wie will man Leuten die Arbeit im fzs erklären}
		\noli{\ul{
		\lii{Das könnte man gut in der nächsten Sitzung gut klären weil der Vorsitzende des fzs zu uns in die Sitzung kommt.
		}}}
	\li{Was muss man mitbringen wenn man mitmachen wollte.}
		\noli{\ul{
		\lii{Es schadet nicht die Struktur des fzs zu kennen und wozu Ministerien da sind, aber grundsätzlich muss man nur Interesse mitbringen.
		}}}
	\li{angenommen mit 35/0/2}
}

6.3 Simon Kleinhanß für die Härtefallkommission
\textbf{X. Lesung:}
\ul{
	\li{keine Fragen}
	\li{gewählt mit 33/0/4}
}

6.4 Lucas Kelm für das Referat für internationale Studierende
\textbf{X. Lesung:}
\ul{
	\li{keine Fragen}
	\li{gewählt mit 28/0/3}
}

6.5 Florian Weiss für das Finanzreferat 
\textbf{X. Lesung:}
\ul{
	\li{Er hatte schon gesagt, dass er es nicht komplett weiterführen will und nurnoch dabei ist um Felix einzuarbeiten. Tritt er nach einer gewissen Zeit zurück? Ist es nicht besser die Stelle als vakant zu deklarieren, dass sich Leute eher dafür finden.}
		\noli{\ul{
		\lii{Der Wunsch dieses Amt weiterzuführen hat sich geändert. Jetzt hat er nicht sehr viel dagegen die Stelle weiterzuführen. Es hat sich aber schon jemand gefunden, die diese Stelle weiterführen würde. Vor dem Sommer hält er es hingegen nicht für schlau da Leute auszuwechseln.
		}}}
	\li{gewählt mit 30/1/2}
}

6.6 Victoria Engels für das Referat für Lehre und Lernen 
\textbf{X. Lesung:}
\ul{
	\li{Wie nimmst du die Arbeit des Sturas in puncto Inklusion von Menschen mit Behinderungen wahr? Und was ist dein Eindruck zu dem Gesundheitsreferat in diesem Punkt?}
		\noli{\ul{
		\lii{Es scheint beim Thema Barrierefreiheit keine klare Zuständigkeit bei den einzelnen Themen zu geben. Sie fand es sehr schwierig sich darüber einen Überblick zu verschaffen.
		}}}
}

6.7 Vionjan Vijeyaranjan  für die Vertretungsversammlung des StuWe 
\textbf{X. Lesung:}
\ul{
	\li{keine Fragen}
}

6.8 Uli Roth für das EDV-Referat 
\textbf{X. Lesung:}
\ul{
	\li{Wo ist deine Kandidatur?}
		\noli{\ul{
		\lii{Sie ist online jetzt.
		}}}
	\li{In welchen Organisationen ist er?}
		\noli{\ul{
		\lii{Er ist im VVN-BDA, SDS, evtl. Mitglied der Linken,
		}}}
	\li{Gibt es einen Bereich in der EDV in dem er sich gerne stärker oder weniger involvieren würde}
		\noli{\ul{
		\lii{Er kann sich alle Aspekte des EDV-Referats vorstellen.
		}}}
	\li{Hast du generell Zeit dafür?}
		\noli{\ul{
		\lii{Ab Ende Februar hat er Zeit
		}}}
	\li{Welche Programmiersprache(n) interessieren ihn besonders?}
		\noli{\ul{
		\lii{Python, C++, und im letzten Jahr hat er sich mit Python als Webbrowser auseinandergesetzt.
		}}}
	\li{Er ist ja auch dann in der RefKonf Mitglied}
		\noli{\ul{
		\lii{Ja da hat er sich auch schon darüber informiert.
		}}}
	\li{Was ist der Plan wenn man jetzt zu zweit ist.}
		\noli{\ul{
		\lii{Da gibt es direkt keine Antwort drauf. Aber in Coronazeiten geht die EDV-Arbeit nicht wirklich aus.
		}}}
	\li{Wird bei SDS-Veranstaltungen die StuRa-Technik dann auch eingesetzt.}
		\noli{\ul{
		\lii{Wenn es einen Antrag auf Nutzung der Geräte der VS gibt, dann werden diese bereit gestellt.
		}}}
	\li{Hat er an dem Antrag über Burschenschaften mitgeschrieben?}
		\noli{\ul{
		\lii{Nein, aber selbst wenn zählt für ihn die Beschlusslage des Stura's
		}}}
}

6.9 Alexander Riemer als Vertreter in der Kommission für die Marsilius-Studien
\textbf{X. Lesung:}
\ul{
	\li{keine Fragen}
}

6.10 Ole Klarhof als Vertreter in der Kommission für die Marsilius-Studien
\textbf{X. Lesung:}
\ul{
	\li{keine Fragen}
}

6.11 Christian Heusel für die M-N Gesamtfakultät
\textbf{X. Lesung:}
\ul{
	\li{Was sind deine politischen Interessen?}
		\noli{\ul{
		\lii{Ich bin in keiner politischen Organisation vertreten. Ich bin aber trotzdem in der Hochschulpolitk aktiv vertreten.
		}}}
}

6.12 Christoph Blattgerste für die M-N Gesamtfakultät
\textbf{X. Lesung:}
\ul{
	\li{Was sind deine politischen Interessen? Wie lange wird er dieses Amt übernehmen?}
		\noli{\ul{
		\lii{Er ist in keiner Liste oder politischen Gruppe aktiv. Er ist lange genug noch für eine Amtszeit an der Universität.
		}}}
}

6.13 Wahlvorschlag
\textbf{X. Lesung:}
\ul{
	\li{Mitglieder:}
		\noli{\ul{
		\lii{David Löw(zurückgezogen)
		}}}
		\noli{\ul{
		\lii{Annalena Wirth
		}}}
		\noli{\ul{
		\lii{Leon P. Köpfle
		}}}
		\noli{\ul{
		\lii{Magdalena Schwörer
		}}}
	\li{Stellvertreter*innen:}
		\noli{\ul{
		\lii{Julian Beier
		}}}
		\noli{\ul{
		\lii{Anna Scherer
		}}}
		\noli{\ul{
		\lii{Simon Kleinhanß
		}}}
		\noli{\ul{
		\lii{Christian Heusel
		}}}
	\li{In welchen relevanten politischen Gruppierungen sind alle hier?}
		\noli{\ul{
		\lii{Leon Köpfle: SPD
		}}}
		\noli{\ul{
		\lii{Annalena Wirth: SPD, Verdi
		}}}
		\noli{\ul{
		\lii{Magdalena Schwörer: -
		}}}
		\noli{\ul{
		\lii{Julian Beier: 
		}}}
		\noli{\ul{
		\lii{Anna Scherer: JEF, CDU, JU
		}}}
		\noli{\ul{
		\lii{Christian Heusel: -
		}}}
}

6.14 Nanina Föhr für die Härtefallkommission
\textbf{X. Lesung:}
\ul{
	\li{Ich wollte eigentlich fragen, wie gut du dich vorher informiert hast, was da so auf dich zukommt und wie du von der Kommission erfahren hast?}
		\noli{\ul{
		\lii{Es wurde schon telefoniert mit der HFK und da hat man sich gut ausgetauscht.
		}}}
}

6.15 Stefania Fiume für die QSM-Kommission  
\textbf{X. Lesung:}
\ul{
	\li{keine Fragen}
}

7 Diskussionen

7,1 Burschenschaften
\textbf{X. Lesung:}
\ul{
	\li{Der Antrag scheint ziemlich unmöglich. Aber dass man alle Heidelberger Burschenschaften ausschließt ist nicht gerechtfertigt. Auch ist problematisch, dass man }
		\noli{\ul{
		\lii{Eine Demokratie muss wehrhaft sein. Man will jetzt nicht jede Verbindung ausschließen. Aber bei Burschenschaften versteckt sich unter dem Deckmantel rechtsextremistisches Gedankengut, das oft auch zu Gewalt führt. Da kann man auch auf die Debatte mit dem EDV-Referat verweisen, dass man Equipment nicht an antisemitische Gruppen verleit.
		}}}
	\li{Es ist schwierig alle Burschenschaften auszuschließen.}
	\li{Die Argumentation ist nicht wirklich schlüssig, weil dieser Antrag nicht ein spezifisches Problem löst. Die volle Version des Antrags wäre auch sehr interessant. }
	\li{Gab es hier schon eine Kooperation mit Burschenschaften?}
		\noli{\ul{
		\lii{Nein gab es nicht seit 100 Jahren
		}}}
	\li{Den HSG's kann man nicht verbieten sich mit jemandem zu treffen und das ist nur eine Aufforderung. Der Vorfall der Normannia war jetzt nur das Einzige was man mitbekommen hat. Bei Burschenschaften gibt es aber sehr oft homophobe Redner:innen zum Beispiel. Dass es noch keine Kooperation mit Burschenschaften gab, hat bei früheren Anträgen zu Unvereinbarkeit auch keinen Unterschied gemacht.}
	\li{Der zweite Satz will die Mitglieder von Burschenschaften von Ämtern der VS ausschließen. Das ist nicht nur rechtswidrig, sondern auch nicht wirklich mit der Demokratie vereinbar. Deswegen hat die Sitzungsleitung diesen nicht zugelassen. Man sollte auch einmal die Blickrichtung wechseln und das beurteilen wenn man das aus Sicht von einer rechten Partei die eine linke Gruppierung ausschließen will.}
	\li{Bevor klar ist ob man etwas ausschließen sollte, sollte man schauen ob die Vereinigung verfassungsfeindlich ist.}
	\li{Dieser Antrag scheint inhaltlich unfertig. Wenn der Antrag gut argumentiert sollte man über ihn }
	\li{Den Blickwinkel zu wechseln scheint wie Gleichmacherei. Auch Gruppen die von rechten Gruppierungen nicht gemocht werden, sind Gruppierungen für die Rechte von Homosexuellen LGBTQ Frauen etc. eintreten.}
	\li{Man könnte den Antrag sehr gut verlängern. Wenn eine Burschenschaft verfassungsfeindlich agiert, ist nicht wirklich relevant hier wenn man das Handeln der Normannia betrachtet.}
	\li{Wir sind hier Studierende und sollten anderen versuchen zu verstehen.}
}

GO Antrag auf Schluss der Debatte
\textbf{X. Lesung:}
\ul{
}

8 Corona Sondersitzung
}

8.1 Online-Sprechstunden
	\li{keine Fragen}
	\li{angenommen mit 25/0/3}
}

8.2 WLAN (VERTAGT)
	\li{Der Änderungsantrag über Up und Downloadgeschwindigkeit wurde nicht geschrieben.}
		\noli{\ul{
		\lii{Ja das wurde übergangen. Das ist wegen Prüfungsstress untergegangen.
		}}}

GO Antrag auf Vertagung
\textbf{X. Lesung:}
\ul{
	\li{Angenmmen ohne Gegenrede}
}

8.3 Qualität der digitalen Lehre
	\li{keine Fragen}
	\li{angenommen mit 22/0/1}

8.3.1 Änderungsantrag zu 8.3
\textbf{X. Lesung:}
\ul{
	\li{keine Fragen}
	\li{angenommen mit 24/0/1}
}

GO Antrag auf Vorziehung von 10.2 Stura Terminen fürs Sommersemester
\textbf{X. Lesung:}
\ul{
	\li{ohne Gegenrede angenommen}
}

8.4 Mensa-Essen
	\li{keine Fragen}
	\li{angenommen mit 23/0/1}

8.4.1  Änderungsantrag zu 8.4
\textbf{X. Lesung:}
\ul{
	\li{keine Fragen}
	\li{angenommen mit 22/0/1}
}

8.5 Corona und Soziales
	\li{Es ist nicht sicher welche Notlagenfonds gemeint sind}
		\noli{\ul{
		\lii{Das bezieht sich auf die Notlagenfonds der Universität Heidelberg
		}}}
	\li{angenommen mit 20/0/4}

Änderungsantrag 8.5.1
\textbf{X. Lesung:}
\ul{
	\li{Keine Fragen}
	\li{angenommen mit 21/0/1}
}

8.6 Freischuss für Medizin (VERTAGT)

GO Antrag auf Vertagung von 8.6
\textbf{X. Lesung:}
\ul{
	\li{ohne Gegenrede angenommen}
}

9 Finanzanträge
	\li{keine Anträge zu diesem Thema}
}

10 Sonstiges

10.1 Wahl des stud. Senators für den Academic Council von 4EU+
\textbf{X. Lesung:}
\ul{
	\li{keine Fragen}
	\li{angenommen mit 23/1/1}
}

10.2 StuRa-Termine für das SoSe
\textbf{X. Lesung:}
\ul{
	\li{keine Fragen}
}