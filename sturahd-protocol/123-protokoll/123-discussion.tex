

2.1 Änderungen an der To

2.1.1 GoAntrag auf Aufnahme eines TOPs
\textbf{X. Lesung}
\ul{
}

2.1.1 GoAntrag auf Aufnahme eines TOPs
\textbf{X. Lesung}
\ul{
	\li{ohne Gegenrede angenommen}
}

2.2.2 Verschieben von Finanzanträgen vor die Wahlen
\textbf{X. Lesung}
\ul{
	\li{formale Gegenrede für Abstimmung}
		\noli{\ul{
		\lii{Angenommen mit 30/6/7
		}}}
}

Protokolle

Protokoll 122
\textbf{X. Lesung}
\ul{
}

Protokoll 122
\textbf{X. Lesung}
\ul{
	\li{Einwand auf Rechtsschreibfehler in letzten Protokollen}
		\noli{\ul{
		\lii{Man will darauf in der Zukunft genauer achten
		}}}
}

Berichte

Nextbike
\textbf{X. Lesung}
\ul{
}

Nextbike
\textbf{X. Lesung}
\ul{
	\li{Es gab ein öffentliches Treffen mit dem Verkehrsreferat bei dem keine:r da war und der Raum nicht geöffnet wurde.}
		\noli{\ul{
		\lii{Entschuldigung des Verkehrsreferats dafür
		}}}
}

Vorsitz
\textbf{X. Lesung}
\ul{
	\li{Sind in der Infomail an die Erstsemester auch Informationen dass sie BigBlueButton nutzen können.}
		\noli{\ul{
		\lii{Das ist nicht der Sinn der Mail. Diese soll weniger auf dieser sachlichen Ebene sein. Aber darüber gibt es noch Informationen an alle Studierenden.
		}}}
	\li{Wo ist genau der Unterschied zwischen dem bbb- StuRa Zugang und dem RefKonf-Zugang? Welchen soll man jetzt als studentische Gruppe nutzen?}
		\noli{\ul{
		\lii{Es gibt keinen RefKonf-Zugang. Man soll als studentische Gruppe BigBlueButton nutzen.
		}}}
}

Treffen zur StuRa-Sondersitzung zu Corona
\textbf{X. Lesung}
\ul{
	\li{Informationen an welche Personen man sich bei Problemen wenden kann und Lernorte sollte man coronakonform überdenken. Wie hoch sollte die Teilnehmendenzahl sein.}
		\noli{\ul{
		\lii{Es ist keine Vollversammlung geplant, sondern eine außerordentliche StuRa-Sitzung. Man will es im Rahmen des Stura's machen für Legitimation und Struktur.
		}}}
	\li{Wie sieht es mit Experten:innen/Dozent:innen}
		\noli{\ul{
		\lii{Das liegt in der Verantwortung des Vorbereitungsteams das zu erarbeiten und zu planen ob und in welchem Maße man das macht.
		}}}
	\li{Es sollten schnell Leute für das Vorbereitungsteam gefunden werden}
	\li{Zu Frage 2: Die Informationen sollen an die richtigen Leute weitergehen also an das Rektorat zum Beispiel. Die Dozent:innen sollen auch }
		\noli{\ul{
		\lii{Wichtiger sei es zu klären, ob die generelle Stimmung zum weitermachen ist oder ob da genug getan wurde.
		}}}
	\li{Was sind die Aufgaben des Vorbereitungsteams}
		\noli{\ul{
		\lii{Informationen zusammentragen, Jede:r soll ein oder zwei Themenfelder haben, in denen er:sie sich informieren soll. 
		}}}
		\noli{\ul{
		\lii{Will man das zum Austausch zwischen Fachschaften etc. nutzen oder sollten politische Forderungen erarbeitet werden? Darüber sollte man auch noch diskutieren.
		}}}
}

Meinungsbild über Einstellung zur Sondersitzung über Corona
\textbf{X. Lesung}
\ul{
	\li{deutlich mehr Stimmen dafür als dagegen}
}

Meinungsbild über die Richtung in die die Sondersitzung gehen soll 
\textbf{X. Lesung}
\ul{
		\noli{\ul{
		\lii{35 Stimmen für politische Forderungen zu erarbeiten 17 für inneruniversitären Austausch 
		}}}
}

GO Verfahrensvorschlag
\textbf{X. Lesung}
\ul{
		\noli{\ul{
		\lii{angenommen 
		}}}
}

2.2.3 Verschieben von Satzungen vor die Wahlen
\textbf{X. Lesung}
\ul{
	\li{ohne Gegenrede angenommen}
}

Satzungen

4.1 Änderung der Satzung der Studienfachschaft Philosophie
\textbf{X. Lesung}
\ul{
}

4.1 Änderung der Satzung der Studienfachschaft Philosophie
\textbf{X. Lesung}
\ul{
	\li{angenommen 55/1/0}
}

4.2 Änderung der Satzung der Studienfachschaft Pharmazie
\textbf{X. Lesung}
\ul{
		\noli{\ul{
		\lii{angenommen 55/0/0
		}}}
}

4.3 Änderung der Satzung der Studienfachschaft Japanologie 
\textbf{X. Lesung}
\ul{
		\noli{\ul{
		\lii{angenommen 55/0/0
		}}}
}

4.4 Neufassung der Studienfachschaftssatzung UFG/VA 
\textbf{X. Lesung}
\ul{
	\li{zu Punkt §5 (4): Mittel sollen nach Anzahl der Studierenden veranschlagt werden muss QSM-Referat mitgeteilt werden und QSM-Referat muss das feststellen. Das ist sehr schwer für das QSM-Referat.Zusätzlich schränkt sich die Fachschaft dadurch sehr ein. Deswegen wird zu Änderung von muss zu sollte geraten.}
		\noli{\ul{
		\lii{Das soll noch mit der Fachschaft besprochen werden
		}}}
	\li{Die Änderung soll am 01.01 in Kraft treten. Die nächste StuRa-Sitzung ist aber am 12.01. Deswegen muss das mit dem QSM-Referat koordiniert werden.}
}

4.5 Fusion der Fachschaften Klassische Archäologie und Byzantinische Archäologie und Kunstgeschichte
\textbf{X. Lesung}
\ul{
}

4.6 Satzung der neuen Fachschaft Klassische und Byzantinische Archäologie 
\textbf{X. Lesung}
\ul{
}

Finanzanträge

6.1 Unterstützung der Campus Debatte Heidelberg 
\textbf{X. Lesung}
\ul{
}

6.1 Unterstützung der Campus Debatte Heidelberg 
\textbf{X. Lesung}
\ul{
		\noli{\ul{
		\lii{angenommen 44/1/0
		}}}
}

6.2 Studieren ohne Grenzen
\textbf{X. Lesung}
\ul{
		\noli{\ul{
		\lii{angenommen 45/3/1
		}}}
}

6.3 Club für Wissenschaft und Kultur
\textbf{X. Lesung}
\ul{
	\li{Wie verteilt sich das Geld (4500€), was beantragt wird genau?}
		\noli{\ul{
		\lii{Es wird zweigleisig mit online und Präsenz geplant. Und das bestimmt die Verteilung der Gelder.
		}}}
		\noli{\ul{
		\lii{angenommen 40/2/5
		}}}
}

6.4 Konfliktbarometer
\textbf{X. Lesung}
\ul{
		\noli{\ul{
		\lii{angenommen mit 38/4/5
		}}}
}

6.5 Law NMUN
\textbf{X. Lesung}
\ul{
		\noli{\ul{
		\lii{angenommen mit 27/11/9
		}}}
}

6.5.1 Änderungsantrag
\textbf{X. Lesung}
\ul{
		\noli{\ul{
		\lii{angenommen mit 37/3/4
		}}}
}

6.9.1 Änderungsantrag
\textbf{X. Lesung}
\ul{
		\noli{\ul{
		\lii{angenommen mit 43/3/5
		}}}
}

6.6 Kritjur Zoom 
\textbf{X. Lesung}
\ul{
		\noli{\ul{
		\lii{angenommen mit 31/14/4
		}}}
}

6.7 Kritjur Vortrag Völkerstrafrecht
\textbf{X. Lesung}
\ul{
	\li{angenommen mit 42/3/5}
}

6.8 Klimagerechte Wege aus dem Kapitalismus
\textbf{X. Lesung}
\ul{
	\li{Die Anarchistische Gruppe Mannheim lehnt das Konzept von Staaten ab}
		\noli{\ul{
		\lii{Anwesenheit bei der Buchmesse ist hier eine gute Informationsquelle. 
		}}}
	\li{Steht die Anarchistische Gruppe auf dem Boden der freiheitlich demokratischen Grundordnung?}
		\noli{\ul{
		\lii{Nach eigenem Ermessen streben die Mitglieder nach einer freien und demokratischen Gesellschaft.
		}}}
	\li{Worum genau soll es in den Vorträgen gehen?}
		\noli{\ul{
		\lii{Bitte um Präzisierung der Themen, weil sie ja im Antrag enthalten sind.
		}}}
	\li{Die Homepage der Ananrchstischen Gruppe schien nah an Verschwörungstheorien}
		\noli{\ul{
		\lii{Die Anarchistische Gruppe Mannheim ist gegen die Querdenker und grenzt sich so klar davon ab und die Referent:innen stehen diesen auch nicht nahe.
		}}}
	\li{Die Anarchistische Gruppe Mannheim scheint eine sehr politische Gruppe zu sein und der StuRa ist ja eine neutrale Institution.}
		\noli{\ul{
		\lii{Die Positionen die in den Vorträgen vertreten werden sind auch schon sehr dem linken Spektrum zuzuordnen. Deswegen ist dieser Vortrag schon von Natur aus nicht neutral.
		}}}
	\li{Der StuRa steht durch diesen Antrag neben der AGM, was den StuRa als neutrale Institution }
	\li{Die Neutralität des StuRa's ist in der Vergangenheit gewesen, dass keine politische Richtung die grundgesetzkonform ist, ausgeschlossen werden darf. }
	\li{Die Veranstaltung beinhaltet keine neutrale Stimme, weil man hier ja schon aus dem Kapitalismus raus will und das nicht neutral ist.}
	\li{Die Vorträge liegen im Interesse der Studierenden und sollten deswegen gefördert werden. Es ist aber unmöglich alle Themen in den Vorträgen abzubilden. Ausschlusskriterium wäre dass die Referent:innen gegen die Leitlinien der VS verstoßen.}
	\li{Es ist schwierig wenn das Logo des StuRa's neben einer Gruppe wie der AGM steht.}
	\li{Je kontroverser die Themen sind, desto mehr Gegenwind sollten sie erhalten. }
	\li{Politische Neutralität im Sinne des Gesetzes ist sehr weit gefasst und nicht errreicht solange man nicht etwas macht wie Abgeordnete unterstützen. Aber wenn man das als Argument benutzt sei es vollkommen nachvollziehbar, wegen der kontroverseren Orientierung der AGM.}
	\li{abgelehnt mit 20/21/9}
}

6.9 Kritjur Werbemittel BAKJ-Kongress
\textbf{X. Lesung}
\ul{
		\noli{\ul{
		\lii{angenommen mit 34/5/7
		}}}
}

Wahlen

5.1 Kandidatur für das Referat für politische Bildung (2. Lesung) 
\textbf{X. Lesung}
\ul{
}

5.1 Kandidatur für das Referat für politische Bildung (2. Lesung) 
\textbf{X. Lesung}
\ul{
	\li{Wieviel Wissen gibt es bei den Aufgaben und Pflichten von Referent:innen}
		\noli{\ul{
		\lii{(Janek) Es gibt sehr viel Anträge schreiben. Der Vorteil ist, dass er von Felix sehr gut eingearbeitet werden kann. 
		}}}
		\noli{\ul{
		\lii{(Felix)Einreichen von Berichten, stellen von Anträgen
		}}}
	\li{Die Antworten waren nicht ganz zufriedenstellend.}
		\noli{\ul{
		\lii{(Felix) Die Frage war falsch verstanden worden. Sie wollen mehr Arbeitsteilung machen um die Aufgaben zu 
		}}}
	\li{Mit ihrer Wahl wird das Referat sehr männlich geprägt sein. Wie wollen sie Frauen mit ins Boot holen.}
		\noli{\ul{
		\lii{(Janek)Es sollen Frauen direkt angesprochen werden. Man ist sich bewusst, dass akademische Felder oft sehr männlich dominiert sind. Er will Frauen auch für das Referat zu begeistern.
		}}}
		\noli{\ul{
		\lii{(Felix)Bei Vorträgen will man eine Gleichstellung von Frauen und Männern erreichen. Er verweißt darauf, dass er es schon oft in seinem Umfeld Komilitoninnen dafür begeistern versucht hatte.
		}}}
		\noli{\ul{
		\lii{Felix wurde gewählt
		}}}
		\noli{\ul{
		\lii{Janek wurde gewählt
		}}}
}

5.2 Kandidatur für das Referat für Kultur und Sport (2. Lesung) 
\textbf{X. Lesung}
\ul{
	\li{Wie kann man die Arbeit während Corona gut führen?}
		\noli{\ul{
		\lii{Virtuelle Lesekreise sind möglich. Information zu Büchern könnte man bereitstellen. Das Frankfurter Stillenmuseum könnte man um Kooperationen anfragen, um Kultur für die Studierenden bereitzustellen.
		}}}
}

5.3 Kandidatur für das Referat Antirassismus (2. Lesung)
\textbf{X. Lesung}
\ul{
	\li{Jemand von der LHG war Mitglied in der Whatsapp Gruppe. In dieser wurde anscheinend gewaltverherrlichende Aussagen getätigt. }
		\noli{\ul{
		\lii{Diese Person war anscheinend als der Vorfall mit der Normannia war nicht begeistert darüber, dass man dagegen etwas übernehmen wollte. 
		}}}
}

5.4 Kandidatur für das Referat für Kultur und Sport (2. Lesung) 
\textbf{X. Lesung}
\ul{
	\li{keine Fragen}
}

5.5 Kandidatur für das Queerreferat (2. Lesung)
\textbf{X. Lesung}
\ul{
	\li{keine Fragen}
}

5.6 Senatskommission für die Verleihung der Bezeichnung apl. Prof. (1. Lesung)
\textbf{X. Lesung}
\ul{
	\li{In der Senatskommission ist eine Stelle für apl. Profs vorgesehen. Wie gneua würden sie das verteilen.}
		\noli{\ul{
		\lii{Nach Nachfrage war eine Teamkandidatur erlaubt. Tomke würde ansonsten stellvertretendes Mitglied werden.
		}}}
	\li{Wie genau würden sie sich einbringen wollen.}
		\noli{\ul{
		\lii{Sie würden sich mit der Lehre auseinandersetzen und die Studierenden am besten vertreten.
		}}}
		\noli{\ul{
		\lii{Sie wollen darauf schauen ob diese Person eine gute Lehre macht, aber durch die Arbeit selber zeigt sich auf was es ankommt.
		}}}
}

5.7 Zusammenfassung 
\textbf{X. Lesung}
\ul{
}

Diskussionen

8.1 Nicht-Einsehbarkeit der Teilnehmendenlisten auf Moodle ( 2. Lesung )
\textbf{X. Lesung}
\ul{
}

8.1 Nicht-Einsehbarkeit der Teilnehmendenlisten auf Moodle ( 2. Lesung )
\textbf{X. Lesung}
\ul{
		\noli{\ul{
		\lii{angenommen mit 
		}}}
}

8.1.1 Änderungsantrag zu Nicht-Einsehbarkeit der Teilnehmendenlisten auf Moodle
\textbf{X. Lesung}
\ul{
	\li{Das Anzeigen von Prüfungsergebnissen ist nicht auf Moodle. Die Emails sind nicht einsehbar. Zur einfacheren Kontaktierung der Kommiliton:innen ist das öffentliche Einsehen nötig.}
	\li{Die inhaltlichen Vorschläge sind nicht ganz }
		\noli{\ul{
		\lii{Die ganzen Missgeschicke der Dozent:innen sollten nicht mehr vorkommen und deswegen muss man das anprangern.
		}}}
	\li{Die Dozent:innen sollten nicht die Entscheidung haben die Daten öffentlich zu machen, sondern die Leute die die Daten selber }
	\li{Bei Erstsemester:innen gibt es viele, die ihr eigenes Profil schon sehr viel bearbeitet haben. Deswegen sollten wir betonen, dass es freiwillig ist.}
	\li{Denkt man dass das rechtlich problematischist oder moralisch. Man könnte den:die Datenschutzbeauftragte:n mal bitten sich damit zu beschäftigen. }
	\li{Bilder in Moodle sind hilfreich, weil die Gesichter leichter zu merken sind als die Namen. Es ist eine technische Sache die Zweitnamen auszublenden.}
	\li{Um wirklich arbeiten zu können benötigt es diese Daten. Da überwiegt die Funtionalität. Die Leute haben oft soziale Netzwerke, bei denen sie nicht so auf ihre Daten achten}
		\noli{\ul{
		\lii{Über HeiConf ist es möglich die Mailadresse auszutauschen. Zusätzlich verbietet der Antrag für die Studierenden nichts. Das Profil kann noch öffentlich gemacht werden.
		}}}
	\li{Wenn Daten anderer Leute benötigt werden, dann kann einfach gefragt werden. Im Antrag wird das Veröffentlichen nicht verboten. }
	\li{Abwägung zwischen Datenschutz und Praktikabilität sind auf persönlicher Ebene relevant. Auf rechtlicher Ebene auf der die Universität ist, nicht. Das Veröffentlichen von Notenlisten ist keine Abwägung sondern einfach verboten. Uterhaltenswert ist es zu wissen wer denn noch in der Vorlesung ist. }
	\li{Aus persönlicher Erfahrung war das ein Problem, dass Emailadressen nicht bekannt waren und das kann die Vorlesung in die Länge zu ziehen.}
	\li{Was ist der Weg nach diesem Antrag? Will man sich danach sich an das URZ wenden? Will man dagegen klagen?}
	\li{Man könnte den Antrag in den Senat einbringen, weil deren Entscheidung weitreichender ist.}
	\li{Die Vorsitzenden der VS bieten an sich mit dem URZ zu treffen und darüber zu reden.}
		\noli{\ul{
		\lii{diese Entscheidung wird begrüßt.
		}}}
		\noli{\ul{
		\lii{Was ist wenn man in einer Heiconf die Namen nicht sieht.
		}}}
		\noli{\ul{
		\lii{Bei Heiconf hat man die Möglichkeit zu bestimmen, wie man sich nennen will und deswegen geht das konform
		}}}
}