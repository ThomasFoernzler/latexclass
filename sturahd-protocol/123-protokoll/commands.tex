%edge margin
\geometry{
a4paper,
top=5cm,
bottom=4cm,
width=16cm
}
\pagestyle{scrheadings}
\hypersetup{
pdfborder={0 0 0},
colorlinks=true,
linkcolor=black,
filecolor=magenta,      
urlcolor=cyan,
}

\newcommand{\SetLogo}{
\tikz [remember picture,overlay]
\node [shift={(-46.15mm,-26.5mm)}]
      at (current page.north east)
         {\includegraphics[width=40mm, height=13mm]
         {../logo.png}};
}
\newcommand{\SetFooter}{
\tikz [remember picture,overlay]
\node [shift={(-65mm,30mm)}]
     at (current page.south east)
     {\parbox{8cm}{
          \begin{flushright}
               \textcolor{gray}{
               \eventnumber. SturaSitzung am \datum \\
               Alber-Ueberle-Straß e 3-5. 69120 Heidelberg (DE)}
          \end{flushright}

}};
}
\newcommand{\SetPageNumber}{
\tikz [remember picture,overlay]
\node [shift={(50mm,30mm)}]
     at (current page.south west)
     {\parbox{5cm}{\phantom{space}\\\textcolor{gray}{\raggedleft\normalfont \rmfamily Seite \thepage\ von \pageref*{LastPage}}}};
}
\newcommand{\SetLegislatur}{
     \tikz [remember picture,overlay]
     \node [shift={(50mm,-20mm)}]
     at (current page.north west)
     {\parbox{5cm}{\textcolor{gray}{8. Legislaturperiode}}};
}

\ihead*{\SetLegislatur}
\chead*{}
\ohead*{\SetLogo\SetPageNumber}

\ofoot*{\SetFooter}
\cfoot*{}
\ifoot*{}

\pagestyle{plain.scrheadings}

\let\oldmaketitle\maketitle
\def\maketitle{\oldmaketitle\thispagestyle{scrheadings}}

\newcommand{\kandidaturenseite}{\href{https://www.stura.uni-heidelberg.de/kandidaturen}{Kandidaturenwebsite}}

%generate a costum listcommand for compact discussions with the 2nd level being an arrow indicating a response
%     \ul{
%          \li{1 level}
%          \li{still 1 level}
%          \ul{\lii{2 level}}
%          \li{1 level again}
%     }
\newcommand{\ul}[1]{\begin{itemize}\setlength{\itemsep}{0pt}\setlength{\parskip}{0pt}\setlength{\parsep}{0pt}#1\end{itemize}}
\newcommand{\li}[1]{\item[\textbullet]{#1}} 
\newcommand{\lii}[1]{\item[\MVRightarrow]{#1}} 


\newcommand{\noli}[1]{\item[]{#1}} % blank item for deeper level of list (prevents error, when no items in 1 level but items in 2 level)
%\ul{
%     \ul{\lii{test}}
%} This would make an error
%\ul{
%     \noli{
%          \ul{\lii{test}}
%    }
%} This will not throw an error and keep the first level without an item bullet

\newcommand{\abstimmungsergebnis}[4]{%1titel,2ja,3nein,4enth
     \begin{center}
          \begin{tabular}{|m{9cm}|m{1cm}|m{1cm}|m{1cm}|}
               \hline
               TOP-Titel & Ja & Nein & Enth\\\hline
               #1 & #2 & #3 & #4\\\hline
          \end{tabular}
     \end{center}
}
\newcommand{\antrag}[9]{%1titel,2lesung,3antragssteller,4antragstext,5begründung,6diskussion,7ja,8nein,9enth
     \subsection{#1 (#2)}
     Antragssteller: #3
     \paragraph{Antragstext:}\phantom{spacer}\\
     #4
     \paragraph{Begründung:}\phantom{spacer}\\
     #5
     \paragraph{Diskussion:}\phantom{spacer}\\[1em]
     #6
     \paragraph{Abstimmungsergebnis}
     \abstimmungsergebnis{#1}{#7}{#8}{#9}
}
\newcommand{\diskussion}[6]{%1titel,2lesung,3antragssteller,4antragstext,5begründung,6diskussion
     \subsection{#1 (#2)}
     Antragssteller: #3
     \paragraph{Antragstext:}\phantom{spacer}\\
     #4
     \paragraph{Begründung:}\phantom{spacer}\\
     #5
     \paragraph{Diskussion:}\phantom{spacer}\\[1em]
     #6
}
\newcommand{\wahl}[5]{%1titel,2lesung,3kandidaten,4text,5diskussion
     \subsection{#1 (#2)}
     Kandidaten: #3
     \paragraph{Kandidaturtext:}\phantom{spacer}\\
     #4
     \paragraph{Diskussion:}\phantom{spacer}\\
     #5
}
\newcommand{\GOantrag}[8]{%1titel,2text,3begründung,4gegenrede,5ja,6nein,7enth,8angenommen/abgelehnt
     \subsubsection{GO-Antrag: #1}
     \paragraph{Antragstext:}\phantom{spacer}\\
     #2
     \paragraph{Begründung:}\phantom{spacer}\\
     #3
     \paragraph{Gegenrede:}\phantom{spacer}\\
     #4
     \begin{center}
          \begin{tabular}{|m{9cm}|m{1cm}|m{1cm}|m{1cm}|}
               \hline
               Ergebnis & Ja & Nein & Enth\\\hline
               \MVRightarrow\ #8 & #5 & #6 & #7\\\hline
          \end{tabular}
     \end{center}
}