\section{Diskussionen}

\antrag
{
    Nicht-Einsehbarkeit der Teilnehmendenlisten auf Moodle
}{
    2. Lesung
}{
    GHG Heidelberg
}{
    Der StuRa fordert, die Teilnehmendenlisten von den Moodle-Kursen für Studierende nicht-einsehbar zu
    machen.\\
    Alternativ kann den Studierenden auch die Möglichkeit gegeben werden, ihren Namen in einer
    Teilnehmendenliste nur einsehbar zu machen, wenn sie das ausdrücklich gestatten.
}{
    Da viele Studierende es als unangenehm empfinden, dass öffentlich einsehbar ist, welche Kurse sie
    besuchen, sollten die Teilnehmendenlisten nicht für Studierende öffentlich sein. Zudem sind wir der
    Meinung, dass gerade Zweitnamen, von denen einige eine ganze Menge haben, die Öffentlichkeit nichts
    angehen und privat sein sollten.\\
    An der Pädagogischen Hochschule Heidelberg ist es beispielsweise bereits möglich, dass der eigene
    Name nicht in der Teilnehmendenliste zu sehen ist. Daran sollte sich die Uni ein Beispiel nehmen.\\
    Ein ähnliches Recht auf Anonymität hat der StuRa bereits in der 5. Legislatur für alle Studierenden
    eingefordert:\\
    \url{https://www.stura.uni-heidelberg.de/fileadmin/Intern/Protokolle_und_Beschluesse/5/Beschluesse/Beschluesse_des_StuRa_5_Legislatur.pdf}
}{\textbf{1. Lesung:}
    \ul{
        \li{Manchmal ganz praktisch, wenn man Fragen an Kommilitonen hat kann man einfach E-Mails schreiben.}
            \ul{\lii{ Opt-In besser, E-Mails können schon nicht mehr eingesehen werden.}}
        \li{Meist mehr als 200 Leute könne Daten einsehen. In Gesellschaft wird sonst viel Wert auf Datenschutz gelegt.}
        \li{Moolde fragt immer nach weiteren Informationen zum Profil, für wen sollen die Infos sein?}
            \ul{\lii{Frage wird weitergeleitet}}
        \li{Toller Antrag zum  Studienalltag! Vielleicht schon zu spezifisch, nicht nur aufs Moodle begrenzt. Mehr Opt-in auf allen Ebenen der Uni. Frage nach Datensparsamkeit, Datenschutzschulungen des EDV-Referats sind passend zum Thema}
            \ul{\lii{Noch nicht auf andere Themen gestoßen, offen für Zusammenarbeit}}
        \li{Nutzen der Namen zweifehalft, wenn eh keine E-Mails abgerufen werden können}
        \li{Wird das Freiwillige freigeben von Daten in Antrag aufgenommen?}
            \ul{\lii{Bereits im Antrag}}
        \li{Bei einer Vorlesung konnte man einsehen, wer welche Klasuur nachschreiben muss, es werden Listen von Namen und dazugehörigen Matrikelnummern verschickt}
            \ul{\lii{Bitte meldet euch bei der GHG, dann kann das noch mit aufgenommen werden.}}
    }
}{tba}{tba}{tba}

\subsubsection{Änderungsantrag zu Nicht-Einsehbarkeit der Teilnehmendenlisten auf Moodle}
Antragssteller: Liste GHG
\paragraph{Antragstext:}\phantom{spacer}\\
Neuer Antragstext:\\
 Der StuRa fordert, dass der Anspruch von Studierenden auf Geheimhaltung ihrer
 personenbezogenen Daten gewahrt wird. Nur mit ausdrücklicher, freiwilliger Zustimmung
 der Studierenden sollten andere (auch Teilnehmende des gleichen Kurses) Einsicht in
 ihren vollständigen Namen, Matrikelnummer, Mail-Adresse o.Ä. bekommen können.\\
 Uns sind mehrere Beispiele bekannt, in denen das aktuell nicht der Fall ist:
    \begin{itemize}
        \item Frei einsehbare Teilnehmendenlisten mit vollständigen Namen in Moodle-Kursen
        \item Mails mit der Aufforderung zur Vervollständigung des Moodle-Profils
        \item Unachtsamkeit von Dozierenden in Mails an alle Kursteilnehmenden mit vollständigen Namen und/oder Matrikelnummern oder gar Noten
        \item Offen einsehbare Listen für Nachschreibetermine, in die sich Studierende eintragen sollen
    \end{itemize}
 Dies soll in Zukunft nicht mehr vorkommen.\\
 Insbesondere fordert der StuRa, die Teilnehmendenlisten von den Moodle-Kursen für
 Studierende nicht einsehbar zu machen. Alternativ kann den Studierenden auch die
 Möglichkeit gegeben werden, ihren Namen in einer Teilnehmendenliste nur einsehbar zu
 machen, wenn sie das ausdrücklich gestatten.\\
 Des weiteren sollte aus den (vorformulierten) automatisierten Mails beim Einschreiben
 in einen Moodle-Kurs direkt hervorgehen, dass weitere persönliche Angaben im eigenen
 Profil rein freiwillig und für den Kurs nicht notwendig sind.
\paragraph{Begründung:}\phantom{spacer}\\
Viele Studierende sehen ihr Recht auf informationelle Selbsbestimmung verletzt.\\
Durch die automatisierten E-Mails, welche beim Einschreiben in neue Moodle-Kurse versendet werden, fühlen sich viele Studierende, insbesondere Erstsemester, verunsichert, da unklar ist, aus welchen Gründen das Profil vervollständigt werden soll.
\paragraph{Diskussion:}\phantom{spacer}\\
Diskussion
\paragraph{Abstimmungsergebnis}
\abstimmungsergebnis{Änderungsantrag zu "Nicht-Einsehbarkeit der Teilnehmendenlisten auf Moodle"}{tba}{tba}{tba}
\antrag% TODO zurückgezogen
{
    Positionierung zu Verankerung von öffentlichem Tagen des Senats in der Verfahrensordnung der Ruprecht-Karls-Universität
}{
    2. Lesung
}{
    Juso Hochschulgruppe
}{
    Die Verfasste Studierendenschaft fordert eine Änderung der Verfahrensordnung der
    Ruprecht-Karls-Universität Heidelberg, dass diese öffentliches Tagen des Senats,
    ähnlich wie in §3 GeschO-RefKonf und §6 GeschO-StuRa, festschreibt.
}{
    Die Aufgaben des Senates sind im Landeshochschulgesetz festgelegt: "Der Senat entscheidet in
    Angelegenheiten von Forschung, Kunstausübung, künstlerischen Entwicklungsvorhaben, Lehre, Studium und
    Weiterbildung, soweit diese nicht durch Gesetz einem anderen zentralen Organ oder den Fakultäten
    zugewiesen sind." (§ 19 LHG). Die sehr weitreichenden Entscheidungen, die der Senat trifft,
    beeinflussen die Studierenden direkt. Deshalb sollten sie wenigstens eine Stimme haben können wenn
    diese getroffen werden.\\
    Auch braucht es Transparenz im Senat. Studierende sollten mitbekommen können, was gerade an ihrer
    Universität passiert, unter anderem auch damit sie bei der Senatswahl  gute Entscheidungen treffen
    können. Es scheint ein bisschen absurd, dass Studierende zwar ihre Vertrteter:innen wählen, aber dann
    aber nicht sehen, was diese in ihrem Namen abstimmen.\\
    Öffentliche Senatssitzungen sorgen dafür, dass sich Studierende mehr einbringen können, da sie die
    Funktionsweise des Senats im Detail kennen lernen können und sich so einbringen können, was zu
    besseren Entscheidungen führt. Dies nutzt auch dem Senat selbst.\\
    Wenn es natürlich Themen gibt, die sensibler sind,  dann würden Regelungen, wie sie z. B. der
    StuRa jetzt schon hat einen Ausschluss der Öffentlichkeit ermöglichen. Es spricht wenig gegen das
    öffentliche Tagen des Senats aber eine Menge dafür.
}{
    \textbf{1. Lesung:}
    \ul{
        \li{ LHG regelt die Nichtöffentlichkeit des Senats, legiglich Ausnahmen sind vorgesehen, BW eines der Wenigen BL die das so vorsehen, Konkreter Vorschlag zur Änderung der Verfahrensordnung}
    }
}{tba}{tba}{tba}