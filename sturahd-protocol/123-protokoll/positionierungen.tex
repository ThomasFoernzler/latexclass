\section{Diskussionen}

\antrag
{
    Nicht-Einsehbarkeit der Teilnehmendenlisten auf Moodle
}{
    2. Lesung
}{
    GHG Heidelberg
}{
    Der StuRa fordert, die Teilnehmendenlisten von den Moodle-Kursen für Studierende nicht-einsehbar zu
    machen.\\
    Alternativ kann den Studierenden auch die Möglichkeit gegeben werden, ihren Namen in einer
    Teilnehmendenliste nur einsehbar zu machen, wenn sie das ausdrücklich gestatten.
}{
    Da viele Studierende es als unangenehm empfinden, dass öffentlich einsehbar ist, welche Kurse sie
    besuchen, sollten die Teilnehmendenlisten nicht für Studierende öffentlich sein. Zudem sind wir der
    Meinung, dass gerade Zweitnamen, von denen einige eine ganze Menge haben, die Öffentlichkeit nichts
    angehen und privat sein sollten.\\
    An der Pädagogischen Hochschule Heidelberg ist es beispielsweise bereits möglich, dass der eigene
    Name nicht in der Teilnehmendenliste zu sehen ist. Daran sollte sich die Uni ein Beispiel nehmen.\\
    Ein ähnliches Recht auf Anonymität hat der StuRa bereits in der 5. Legislatur für alle Studierenden
    eingefordert:\\
    \url{https://www.stura.uni-heidelberg.de/fileadmin/Intern/Protokolle_und_Beschluesse/5/Beschluesse/Beschluesse_des_StuRa_5_Legislatur.pdf}
}{\textbf{1. Lesung:}
    \ul{
        \li{Manchmal ganz praktisch, wenn man Fragen an Kommilitonen hat kann man einfach E-Mails schreiben.}
            \ul{\lii{ Opt-In besser, E-Mails können schon nicht mehr eingesehen werden.}}
        \li{Meist mehr als 200 Leute könne Daten einsehen. In Gesellschaft wird sonst viel Wert auf Datenschutz gelegt.}
        \li{Moolde fragt immer nach weiteren Informationen zum Profil, für wen sollen die Infos sein?}
            \ul{\lii{Frage wird weitergeleitet}}
        \li{Toller Antrag zum  Studienalltag! Vielleicht schon zu spezifisch, nicht nur aufs Moodle begrenzt. Mehr Opt-in auf allen Ebenen der Uni. Frage nach Datensparsamkeit, Datenschutzschulungen des EDV-Referats sind passend zum Thema}
            \ul{\lii{Noch nicht auf andere Themen gestoßen, offen für Zusammenarbeit}}
        \li{Nutzen der Namen zweifehalft, wenn eh keine E-Mails abgerufen werden können}
        \li{Wird das Freiwillige freigeben von Daten in Antrag aufgenommen?}
            \ul{\lii{Bereits im Antrag}}
        \li{Bei einer Vorlesung konnte man einsehen, wer welche Klasuur nachschreiben muss, es werden Listen von Namen und dazugehörigen Matrikelnummern verschickt}
            \ul{\lii{Bitte meldet euch bei der GHG, dann kann das noch mit aufgenommen werden.}}
    }
}{tba}{tba}{tba}

\subsubsection{Änderungsantrag zu Nicht-Einsehbarkeit der Teilnehmendenlisten auf Moodle}
Antragssteller: Liste GHG
\paragraph{Antragstext:}\phantom{spacer}\\
Neuer Antragstext:\\
 Der StuRa fordert, dass der Anspruch von Studierenden auf Geheimhaltung ihrer
 personenbezogenen Daten gewahrt wird. Nur mit ausdrücklicher, freiwilliger Zustimmung
 der Studierenden sollten andere (auch Teilnehmende des gleichen Kurses) Einsicht in
 ihren vollständigen Namen, Matrikelnummer, Mail-Adresse o.Ä. bekommen können.\\
 Uns sind mehrere Beispiele bekannt, in denen das aktuell nicht der Fall ist:
    \begin{itemize}
        \item Frei einsehbare Teilnehmendenlisten mit vollständigen Namen in Moodle-Kursen
        \item Mails mit der Aufforderung zur Vervollständigung des Moodle-Profils
        \item Unachtsamkeit von Dozierenden in Mails an alle Kursteilnehmenden mit vollständigen Namen und/oder Matrikelnummern oder gar Noten
        \item Offen einsehbare Listen für Nachschreibetermine, in die sich Studierende eintragen sollen
    \end{itemize}
 Dies soll in Zukunft nicht mehr vorkommen.\\
 Insbesondere fordert der StuRa, die Teilnehmendenlisten von den Moodle-Kursen für
 Studierende nicht einsehbar zu machen. Alternativ kann den Studierenden auch die
 Möglichkeit gegeben werden, ihren Namen in einer Teilnehmendenliste nur einsehbar zu
 machen, wenn sie das ausdrücklich gestatten.\\
 Des weiteren sollte aus den (vorformulierten) automatisierten Mails beim Einschreiben
 in einen Moodle-Kurs direkt hervorgehen, dass weitere persönliche Angaben im eigenen
 Profil rein freiwillig und für den Kurs nicht notwendig sind.
\paragraph{Begründung:}\phantom{spacer}\\
Viele Studierende sehen ihr Recht auf informationelle Selbsbestimmung verletzt.\\
Durch die automatisierten E-Mails, welche beim Einschreiben in neue Moodle-Kurse versendet werden, fühlen sich viele Studierende, insbesondere Erstsemester, verunsichert, da unklar ist, aus welchen Gründen das Profil vervollständigt werden soll.
\paragraph{Diskussion:}\phantom{spacer}\\
\ul{
	\li{Das Anzeigen von Prüfungsergebnissen ist nicht auf Moodle. Die Emails sind nicht einsehbar. Zur einfacheren Kontaktierung der Kommiliton:innen ist das öffentliche Einsehen nötig.}
	\li{Die inhaltlichen Vorschläge sind nicht ganz }
		\noli{\ul{
		\lii{Die ganzen Missgeschicke der Dozent:innen sollten nicht mehr vorkommen und deswegen muss man das anprangern.
		}}}
	\li{Die Dozent:innen sollten nicht die Entscheidung haben die Daten öffentlich zu machen, sondern die Leute die die Daten selber }
	\li{Bei Erstsemester:innen gibt es viele, die ihr eigenes Profil schon sehr viel bearbeitet haben. Deswegen sollten wir betonen, dass es freiwillig ist.}
	\li{Denkt man dass das rechtlich problematischist oder moralisch. Man könnte den:die Datenschutzbeauftragte:n mal bitten sich damit zu beschäftigen. }
	\li{Bilder in Moodle sind hilfreich, weil die Gesichter leichter zu merken sind als die Namen. Es ist eine technische Sache die Zweitnamen auszublenden.}
	\li{Um wirklich arbeiten zu können benötigt es diese Daten. Da überwiegt die Funtionalität. Die Leute haben oft soziale Netzwerke, bei denen sie nicht so auf ihre Daten achten}
		\noli{\ul{
		\lii{Über HeiConf ist es möglich die Mailadresse auszutauschen. Zusätzlich verbietet der Antrag für die Studierenden nichts. Das Profil kann noch öffentlich gemacht werden.
		}}}
	\li{Wenn Daten anderer Leute benötigt werden, dann kann einfach gefragt werden. Im Antrag wird das Veröffentlichen nicht verboten. }
	\li{Abwägung zwischen Datenschutz und Praktikabilität sind auf persönlicher Ebene relevant. Auf rechtlicher Ebene auf der die Universität ist, nicht. Das Veröffentlichen von Notenlisten ist keine Abwägung sondern einfach verboten. Uterhaltenswert ist es zu wissen wer denn noch in der Vorlesung ist. }
	\li{Aus persönlicher Erfahrung war das ein Problem, dass Emailadressen nicht bekannt waren und das kann die Vorlesung in die Länge zu ziehen.}
	\li{Was ist der Weg nach diesem Antrag? Will man sich danach sich an das URZ wenden? Will man dagegen klagen?}
	\li{Man könnte den Antrag in den Senat einbringen, weil deren Entscheidung weitreichender ist.}
	\li{Die Vorsitzenden der VS bieten an sich mit dem URZ zu treffen und darüber zu reden.}
		\noli{\ul{
		\lii{diese Entscheidung wird begrüßt.
		}}}
		\noli{\ul{
		\lii{Was ist wenn man in einer Heiconf die Namen nicht sieht.
		}}}
		\noli{\ul{
		\lii{Bei Heiconf hat man die Möglichkeit zu bestimmen, wie man sich nennen will und deswegen geht das konform
		}}}
}
\paragraph{Abstimmungsergebnis}
\abstimmungsergebnis{Änderungsantrag zu "Nicht-Einsehbarkeit der Teilnehmendenlisten auf Moodle"}{tba}{tba}{tba}
\subsection{Positionierung zu Verankerung von öffentlichem Tagen des Senats in der Verfahrensordnung der Ruprecht-Karls-Universität (zurückgezogen)}
Der Antrag wurde vom Antragssteller zurückgezogen.