\section{Kandidaturen und Wahlen}

\wahl{Kandidatur für das Referat für politische Bildung}{2. Lesung:}{Felix Diener und Janek Kasperowski}
{
    Der Kandidaturtext findet sich auf der \kandidaturenseite.
}{
    \textbf{1. Lesung:}
    \ul{
        \li{Felix nochmal die Mitglidschaften nennen?}
        \ul{\lii{ver.di, Die linke.SDS, Rote Hilfe e.V.}}
        \li{Wie wollt ihr als Mitglieder einer linken HSG gewährleisten, dass die politische Bildung der Studierenden neutral stattfindet?}
        \ul{\lii{Jan: Diskurs mit anderen Gruppen und Privatpersonen}
            \lii{Felix: politische neutralität kann gewährleistet werden, es fehlt im Referat die Stimme der liberalen und konservativen Meinung}
        }
    }
    \textbf{2. Lesung:}
    \ul{
	\li{Wieviel Wissen gibt es bei den Aufgaben und Pflichten von Referent:innen}
		\noli{\ul{
		\lii{(Janek) Es gibt sehr viel Anträge schreiben. Der Vorteil ist, dass er von Felix sehr gut eingearbeitet werden kann. 
		}}}
		\noli{\ul{
		\lii{(Felix)Einreichen von Berichten, stellen von Anträgen
		}}}
	\li{Die Antworten waren nicht ganz zufriedenstellend.}
		\noli{\ul{
		\lii{(Felix) Die Frage war falsch verstanden worden. Sie wollen mehr Arbeitsteilung machen um die Aufgaben zu 
		}}}
	\li{Mit ihrer Wahl wird das Referat sehr männlich geprägt sein. Wie wollen sie Frauen mit ins Boot holen.}
		\noli{\ul{
		\lii{(Janek)Es sollen Frauen direkt angesprochen werden. Man ist sich bewusst, dass akademische Felder oft sehr männlich dominiert sind. Er will Frauen auch für das Referat zu begeistern.
		}}}
		\noli{\ul{
		\lii{(Felix)Bei Vorträgen will man eine Gleichstellung von Frauen und Männern erreichen. Er verweißt darauf, dass er es schon oft in seinem Umfeld Komilitoninnen dafür begeistern versucht hatte.
		}}}
    }
}

\wahl{Kandidatur für das Referat für Kultur und Sport}{2. Lesung:}{Jovana Perovic}
{
    Der Kandidaturtext findet sich auf der \kandidaturenseite.
}{
    \textbf{1. Lesung:}
    \ul{
        \li{Test der Theaterflatrate, überlegeungen zu Schwimmbadflatrate, würdest du an der Theaterflatrate weiterarbeiten und das mit der Schwimmbadflatrate anstossen?}
        \ul{\lii{Ja Theaterflatrate, wenn Schwimmbadflatrate von Studierenden gewünscht ist auch gerne}}
        \li{andere Projektideen}
        \ul{\lii{virtuellen Lesekreis, digitale Rundgänge in Museen}}
    \li{Mitgliedschaften?}
        \ul{\lii{politisch informiert und intereessiert aber bekennt sich nicht zu irgendwelchen poolitischem Gruppen}}
    }
    \textbf{2. Lesung:}
    \ul{
	\li{Wie kann man die Arbeit während Corona gut führen?}
		\noli{\ul{
		\lii{Virtuelle Lesekreise sind möglich. Information zu Büchern könnte man bereitstellen. Das Frankfurter Stillenmuseum könnte man um Kooperationen anfragen, um Kultur für die Studierenden bereitzustellen.
		}}}
    }
}

\wahl{Kandidatur für das Referat Antirassismus}{2. Lesung:}{Mithily Masilamany}
{
    Der Kandidaturtext findet sich auf der \kandidaturenseite.
}{
    \textbf{1. Lesung:}
    \ul{
        \li{siehst du anisemitismus als deinen aufgabenbereich und inwiefern beschäftigt sich das referat damit}
        \ul{\lii{ja ist aufgabe, bis jettzt keine meldungen diesbezügich, stellungsnaheme wegen Burschenscfhaftsvorfall, freuen uns über Mithilfe und Input}
        }
    }
    \textbf{2. Lesung:}
    \ul{
	\li{Jemand von der LHG war Mitglied in der Whatsapp Gruppe. In dieser wurde anscheinend gewaltverherrlichende Aussagen getätigt. }
		\noli{\ul{
		\lii{Keine solche Aussagen aufgefallen. 
		}}}
}
}

\wahl{Kandidatur für das Referat für Kultur und Sport}{2. Lesung:}{Chiara Citro}
{
    Der Kandidaturtext findet sich auf der \kandidaturenseite.
}{
    \textbf{1. Lesung:}
    \ul{
        \li{Theaterflatrate und Schwimmbadflatrate}
        \ul{\lii{Theaterflatrate coole Sache, umsetzung der Schwimmbadflatrate fraglich momentan}}
        \li{Schon mit joanna zusammengesetzt?}
        \ul{\lii{Wir haben uns bisher nicht zusammengesetzt, aber ich denke, dass das kein Problem sein wird. Ich gehe mal davon aus, dass wir zusammenarbeiten können, und es gibt grade in dem Bereich genug Themen, die wir abdecken können, entweder zu zweit oder einzeln, wenn es eine von uns nicht interessiert.}}
    }
    \textbf{2. Lesung:}
    \ul{
	\li{keine Fragen}
}
}

\wahl{Kandidatur für das Querreferat}{2. Lesung:}{Nel Nußberger und Mira Schwarzer}
{
    Der Kandidaturtext findet sich auf der \kandidaturenseite.
}
{
    \textbf{1. Lesung:}
    \ul{
        \li{Wir würdet ihr euch die zusammenarbeit mit anderen (autonomen) Referaten wünschen}
        \ul{\lii{Gerne weitervernetzen, manche Themen betreffen auch andere, offen für zusammenarbeit}}
    }
    \textbf{2. Lesung:}
    \ul{
	\li{keine Fragen}
}
}

\wahl{Senatskommission für die Verleihung der Bezeichnung apl. Prof.}{1. Lesung}{Huilin Guo und Tomke Arand}
{
    Der Kandidaturtext findet sich auf der \kandidaturenseite.
}
{
    \textbf{1. Lesung:}
    \ul{
	\li{In der Senatskommission ist eine Stelle für apl. Profs vorgesehen. Wie gneua würden sie das verteilen.}
		\noli{\ul{
		\lii{Nach Nachfrage war eine Teamkandidatur erlaubt. Tomke würde ansonsten stellvertretendes Mitglied werden.
		}}}
	\li{Wie genau würden sie sich einbringen wollen.}
		\noli{\ul{
		\lii{Sie würden sich mit der Lehre auseinandersetzen und die Studierenden am besten vertreten.
		}}}
		\noli{\ul{
		\lii{Sie wollen darauf schauen ob diese Person eine gute Lehre macht, aber durch die Arbeit selber zeigt sich auf was es ankommt.
		}}}
}
}

\subsection{Zusammenfassung}
\begin{center}
    \begin{tabular}{|p{6cm}|m{2cm}|m{1cm}|m{1cm}|m{1cm}|}
        \hline
        Kandidatur & Gewählt & Ja & Nein & Enth\\\hline
        Felix Diener & ja & 39 & 3 & 3\\\hline
        Janek Kasperowski & ja & 32 & 7 & 6\\\hline
        Jovana Perovic & ja & 32 & 2 & 0\\\hline
        Mithily Masilamany & ja & 29 & 8 & 7\\\hline
        Chiara Citro & ja & 40 & 1 & 2\\\hline
        Nel Nußberger & ja & 29 & 5 & 8\\\hline
        Mira Schwarzer & ja & 29 & 4 & 9\\\hline
        Huilin Guo & ausstehend & tba & tba & tba \\\hline
        Tomke Arand & ausstehend & tba & tba & tba \\\hline
    \end{tabular}
\end{center}