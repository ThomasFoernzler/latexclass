\section{Begrüßung durch die Sitzungsleitung}
Die Mitglieder  der  Sitzungsleitung begrüßendie  Mitglieder  des  Studierendenrats  und  alle Gäste.

\section{Tagesordnung und Ablauf}
\tableofcontents
\subsection{Änderungsanträge zur Tagesordnung}
Bis jetzt keine Änderungsanträge.% TODO Anträge einfügen
\subsection{GO-Antrag: Verfahrensvorschlag für Finanzanträge}

Bei dem Verfahrensvorschlag aus der 107. Stura-Sitzung sollen folgende Punkte geändert werden:\\[1em]
5. Das Vorgehen nach dem die Finanzanträge die zu einem der beiden Termine (Dezember und Juni) gestellt werden, sieht wie folgt aus:\\
\begin{itemize}
    \item[a] Mithilfe  einer  Bewertungswahl  wird  eine  Rangfolge  der  abzustimmenden Finanzanträge erstellt. Dabei kann jedem Antrag eine Punktzahl zwischen -2 und 2  gegeben  werden.    Die  Position  des  jeweiligen  Antrages  innerhalb  der Rangfolge ergibt sich aus dem arithmetischen Mittel aller gültigen abgegebenen Punkte für den jeweiligen Antrag. Wenn alle Finanzanträge die verfügbare Menge an Finanzmitteln nicht überschreiten, wird von der Abstimmung über das Ranking abgesehen.
    \item[b] Es werden Änderungsanträge zu allen Anträgen gesammelt. Über die Bewilligung der Finanzanträge wird in der Reihenfolge der Rangfolge nach  Buchstabe  a  abgestimmt.  Die  Bewilligung  der  Finanzantrag  beginnt  bei dem am höchsten gewerteten Finanzantrag und schreitet dann immer weiter zu den niedriger gewerteten Finanzanträgen fort und läuft so lange, bis die für den jeweiligen Bewilligungszeitraum eingeplanten Mittel aufgebraucht sind.
\end{itemize}
\paragraph{Begründung:}\phantom{spacer}\\
Die Erstellung eines Rankings ist in diesem Fall nicht nötig und online generell auch schwer durchführbar.\\
Die Ändeurngsanträge sollten alle gesammelt vorliegen, um die Vergleichbarkeit zwischen den Anträgen, welche ja der eigentliche Grund für diese Sitzung ist, nicht zu verlieren.\\

\section{Protokolle}
Protokolle werden nicht beschlossen, sie sind angenommen, wenn keine Änderungsanträge vorliegen.
Bitte bedenkt, dass das Protokoll zur Außendarstellung des StuRa beiträgt,und macht daher konkrete Vorschläge
für Ergänzungen. Am besten schickt ihr diese vor der Sitzung an die Sitzungsleitung, damit sie ggf. schon im 
Vorfeld der Sitzung eingepflegt werden können.
\subsection{Protokoll der 121. Stura-Sitzung}
Zu dem Protokoll liegen keine Änderungsanträge vor. %TODO Rechtschreibfehler
\subsection{Protokoll der 122. Stura-Sitzung}
Zu dem Protokoll liegen keine Änderungsanträge vor.