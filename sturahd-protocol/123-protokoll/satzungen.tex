\section{Satzungen und Ordnungen}

\antrag{Satzung der Studienfachschaft Philosophie}{4.Lesung}{Fachschaft Philosophie}
    {\begin{longtable}{|p{7.5cm}|p{7.5cm}|}
        \hline
        \multicolumn{2}{|c|}{Synopse}\\\hline
        Bisheriger Text & Neuer Text \\\hline
        \endfirsthead
        \hline
        Bisheriger Text & Neuer Text \\
        \hline
        \endhead
        \multicolumn{2}{|r|}{Weiter auf der nächsten Seite...}\\
        \hline
        \endfoot
        \hline
        \multicolumn{2}{c}{Ende der Synopse} \\
        \endlastfoot
        \multicolumn{2}{|c|}{§3}\\\hline
        (4) Er umfasst mindestens zwei Mitglieder. Sollten mehr als zwei Kandidat*innen
        aufgestellt werden, so gilt, dass die Anzahl der zu besetzenden Sitze der Zahl der
        Kandidat*innen entspricht, aber maximal vier beträgt.
    &   (4) Er umfasst bis zu vier, aber mindestens zwei Mitglieder.\newline 
        (5) Gewählt sind diejenigen Kandidierenden, die die meisten Stimmen erhalten, wobei
        jede*r Wahlberechtigte bis zu vier Stimmen, aber höchstens so viele Stimmen wie es
        Kandidierende gibt, hat. Bei vier oder weniger als vier Kandidierenden, kann für oder
        gegen jede*n Kandidierende*n gestimmt werden und gewählt sind diejenigen, die mehr
        Ja- als Nein-Stimmen erhalten. Im Übrigen gilt die Wahlordnung der
        Studierendenschaft.\\\hline
        (6) Die Mitglieder des Fachschaftsrates treffen sich in der Vorlesungszeit mindestens
        einmal im Monat zu einer Fachschaftsratssitzung:
        \begin{enumerate}
            \item[a] Diese Sitzung ist mit der Anwesenheit von 2/3 der Fachschaftsräte beschlussfähig.
            \item[b] Das Stura-Mitglied der Fachschaft ist bei diesen Sitzungen beratendes Mitglied.
            \item[c] Der Termin der Fachschaftsratssitzung des jeweiligen Monats wird in der letzten
            Fachschaftsvollversammlung des Vormonats festgelegt.
        \end{enumerate}
    &   (7) Die Mitglieder des Fachschaftsrates treffen bei Bedarf, mindestens aber zweimal
        im Semester, zu einer Fachschaftsratssitzung:
        \begin{enumerate}
            \item[a] Diese Sitzung ist mit der Anwesenheit von 2/3 der Mitglieder beschlussfähig.
            \item[b] Das Stura-Mitglied der Fachschaft ist bei diesen Sitzungen beratendes Mitglied.
            \item[c] Der Termin der Fachschaftsratssitzung wird von den Fachschaftsrät*innen
            festgelegt. Er muss in geeigneter Weise ortsüblich bekannt gemacht werden und in
            einer Fachschaftsvollversammlung angekündigt werden.
        \end{enumerate}\\\hline
        \multicolumn{2}{|c|}{§4 Arbeitskreise der Fachschaft}\\\hline
        (neu)
    &   (1) Die Fachschaft kann zur Bearbeitung bestimmter Themengebiete Arbeitskreise
        einrichten. Den Beschluss über die Einrichtung trifft die Fachschaftsvollversammlung.\newline
        (2) Ein Arbeitskreis kann jederzeit durch die Fachschaftsvollversammlung aufgelöst
        werden. Ein Arbeitskreis wird aufgelöst, wenn er dreizehn Monate nicht tagt.\newline
        (3) Die Mitwirkung an der Arbeit der Arbeitskreise richtet sich nach den gleichen
        Voraussetzungen wie die Teilnahme an Fachschaftsvollversammlungen.\newline
        (4) Die Fachschaftsvollversammlung entsendet für jeden Arbeitskreis zugleich einen
        Berichterstatter oder eine Berichterstatterin. Die Berichterstatter*in betreut den Arbeitskreis
        und berichtet regelmäßig in der Fachschaftsvollversammlung über dessen Arbeit. Die Amtszeit
        der Berichterstatter*in beträgt ein Jahr.\newline
        (5) Die Termine der Sitzungen der Arbeitskreise werden von der Berichterstatterin
        festgelegt. Die Sitzungen müssen mindestens zwei Tage im Voraus öffentlich und in
        geeigneter Weise ortsüblich bekannt gemacht werden.\\\hline
        \multicolumn{2}{|c|}{§5}\\\hline
        (1) Die Fachschaftsvollversammmlung beschließt einen Verwendungsvorschlag über einen 
        Teil der QSM oder die gesamten QSM. Dieser Beschluss muss mindestens eine Woche vor
        der Einreichungsfrist der Vorschläge gefasst werden.
        \begin{enumerate}
            \item[a] Der Verwendungsvorschlag der Fachschaftsvollversammlung muss bis zum 8. Januar
            für Vorschläge, die bis zum 15. Januar eingereicht werden sollen, und spätestens bis
            zum 8. Mai für Vorschläge, die bis zum 15. Mai eingereicht werden sollen, gefasst
            werden.
            \item[b] Der Verwendungsvorschlag muss nicht in ausgearbeiteter Form vorliegen, sondern
            lediglich das Interesse der Fachschaftsvollversammlung widerspiegeln.
            \item[c] Der Vorschlag ist für den Fachschaftsrat bindend.
            \item[d] Der Beschluss wird mit einfacher Mehrheit gefasst
        \end{enumerate}
    &   (1) Die Fachschaftsvollversammmlung beschließt einen Verwendungsvorschlag über einen
        Teil der QSM oder die gesamten QSM. Dieser Beschluss muss mindestens eine Woche vor
        der Einreichungsfrist der Vorschläge gefasst werden.
        \begin{enumerate}
            \item[a] Der Verwendungsvorschlag der Fachschaftsvollversammlung muss bis zum 8. Januar
            für Vorschläge, die bis zum 15. Januar eingereicht werden sollen, und spätestens bis
            zum 8. Mai für Vorschläge, die bis zum 15. Mai eingereicht werden sollen, gefasst
            werden.
            \item[b] Der Verwendungsvorschlag muss nicht in ausgearbeiteter Form vorliegen, sondern
            lediglich das Interesse der Fachschaftsvollversammlung widerspiegeln.
            \item[c] Der Beschluss wird mit einfacher Mehrheit gefasst.
        \end{enumerate}\\\hline
        \multicolumn{2}{|c|}{§8}\\\hline
        (neu) & Die Satzung tritt in Kraft am 27.06.2020.\\\hline
        \end{longtable}
        Erläuterung:\\
        \begin{enumerate}
            \item In §3 Abs. 4 wird ergänzt, dass bei der Wahl des Fachschaftsrates im Fall von
            weniger Kandidierenden als Plätze, für oder gegen jede einzelne Person gestimmt
            werden kann.
            Aus einem werden hier zwei Absätze, wodurch sich die darauffolgende Nummerierung
            verschiebt.
            \item In §3 Abs. 6 (neu Abs. 7) wird der Rhythmus der Fachschaftsratssitzungen
            reduziert (von einmal pro Monat auf zweimal pro Semester), sowie die Regelung
            zur Festlegung des Termins der Fachschaftsratssitzungen geändert.
            \item Nach §3 wird §4 eingefügt, der Arbeitskreise behandelt und der Fachschaft
            erlaubt, Arbeitskreise einzurichten. Die Nummerierung der nachfolgenden §
            verschiebt sich dementsprechend.
            \item In §5 Abs. 1 wird der Unterpunkt „Der Vorschlag ist für den Fachschaftsrat
            bindend“ in Bezug auf den Vorschlag der FSVV über die Verwendung der QSM
            gestrichen.
        \end{enumerate}
    }{
        \begin{enumerate}
            \item Hier handelt es sich um eine organisatorische Änderung die laut Gremienreferat rechtlich gewünscht ist.
            \item Die FSR-Sitzungen fanden in der Vergangenheit nicht in dem in der Satzung definierten Rhythmus statt, da bei nur vier FSRen auch ohne regelmäßige Sitzungen gute Kommunikation möglich ist. Die Regelung, dass die FSVV über die Termine der FSR-Sitzungen entscheidet, ist unpraktisch und wurde bisher meistens nicht umgesetzt. Durch die Regelungen, dass die von den FSRen festgelegten Termine immer in mindestens einer FSVV angekündigt werden müssen, stellen wir denselben Grad der Öffentlichkeit wie zuvor her.
            \item In der Vergangenheit hat die Fachschaft immer wieder Arbeitskreise einberufen, die dann vergessen oder ignoriert wurden. Es herrschte kein Überblick, was unter Anderem daran lag, dass Arbeitskreise nie in der Satzung definiert waren. Dies wollen wir nun ändern.
            \item Die Streichung des Satzes an dieser Stelle hat zwei Gründe.
                \begin{enumerate}
                    \item In der Vergangenheit wurde für FSVVen, in denen der QSM-Vorschlag ausgearbeitet wurde, häufig von einer kleinen Personengruppe unter ihren Freunden so sehr geworben, dass diese Personengruppe dann ein de facto alleiniges Entscheidungsrecht über die QSM hatte, obwohl ihre Wünsche nicht die Wünsche der Studienfachschaft widerspiegelten. Der Fachschaftsrat ist demokratisch legitimiert und von mehr Wählern bestätigt, als jemals bei einer FSVV über die QSM entscheiden werden. Der Fachschaftsrat soll sich weiterhin am Vorschlag der FSVV orientieren und diesem nur entgegenhandeln, wenn er das Gefühl hat, die Interessen der Studierenden werden in diesem Punkt nicht vom Vorschlag repräsentiert.
                    \item Der gestrichene Satz führt in Verbindung mit dem QSM-Verfahren in der Philosophie zu Verwirrungen. QSM funktioniert bei uns wie folgt:
                    In einer FSVV wird über den allgemeinen QSM-Vorschlag entschieden, also darüber, was finanziert werden soll. Unter anderem werden so fast jedes Jahr mindestens zwei Seminare über QSM finanziert.
                    In einer zweiten FSVV wird entschieden, in welchen Themenbereichen diese Seminare ausgeschrieben werden sollen. Insgesamt gibt es meistens mehr Ausschreibungen als Seminare, da nicht zu jeder Ausschreibung Bewerbungen von Lehrenden eingehen.
                    In einem dritten Schritt entscheidet dann der Fachschaftsrat darüber, welche Bewerber den Zuspruch erhalten und damit auch, welche der ausgeschriebenen Themen letztlich zu Stande kommen.
                    Der gestrichene Satz führte nun häufiger zu Verwirrungen, weil 
                    \begin{enumerate}
                        \item Personen denken, die Themen, die in der zweiten FSVV abgestimmt werden, werden sicher in Seminaren verwirklicht, wenn Bewerbungen eingehen. Dies ist nicht so und auch nicht wünschenswert, da schlechte Bewerbungen vom Fachschaftsrat im dritten Schritt abgelehnt werden sollten, um eine hohe Qualität der über QSM finanzierten Seminare sicherzustellen.
                        \item Personen denken, dass sie in der FSVV über die Bewerbungen abstimmen können. Dies geht aus Datenschutzgründen der Bewerber allerdings nicht. Die Verwaltung unseres Seminars erlaubt nur dem FSR, auf die Bewerbungen zuzugreifen.
                    \end{enumerate}
                \end{enumerate} 
        \end{enumerate}
    }{
        \textbf{1. Lesung}
        \ul{\li{Keine Fragen}}
        \textbf{2. Lesung}
        \ul{\li{Keine Fragen}}
        \textbf{3. Lesung}
        \ul{\li{Keine Fragen}}
        \textbf{4. Lesung}
        \ul{\li{Keine Fragen}}
    }{55}{1}{0}

\antrag{Satzung der Studienfachschaft Pharmazie}{4.Lesung}{Fachschaft Pharmazie}
    {\begin{longtable}{|p{7.5cm}|p{7.5cm}|}
        \hline
        \multicolumn{2}{|c|}{Synopse}\\\hline
        Bisheriger Text & Neuer Text \\\hline
        \endfirsthead
        \hline
        Bisheriger Text & Neuer Text \\
        \hline
        \endhead
        \hline
        \multicolumn{2}{|r|}{Weiter auf der nächsten Seite...}\\
        \hline
        \endfoot
        \hline
        \multicolumn{2}{c}{Ende der Synopse} \\
        \endlastfoot
        \multicolumn{2}{|c|}{§1}\\\hline
        (1) Die Studienfachschaft vertritt die Studierenden ihres Faches oder ihrer Fächer und
        entscheidet insbesondere über fachspezifische Fragen und Anträge.&
        (1) Die Studienfachschaft vertritt die Studierenden ihres Faches oder ihrer Fächer.
        Sie ist insbesondere für fachspezifische Fragen innerhalb der Zuständigkeit der
        Studierendenschaft nach § 2 der Organisationssatzung zuständig und entscheidet über
        diese Angelegenheiten eigenständig.\\
        (2) Die Zugehörigkeit zur Studienfachschaft ergibt sich aus der Liste in Anhang B.&
        (2) Die Zugehörigkeit zur Studienfachschaft ergibt sich aus der Liste in Anhang B
        der Organisationssatzung.\\
        (3) Die Studienfachschaft entsendet studentische Mitglieder in die in ihrem Bereich arbeitenden
        Gremien, oder beteiligt sich zumindest an einem gemeinsamen Wahlvorschlag für eben diese.&
        (3) Die Studienfachschaft stellt in der Regel die studentischen Mitglieder der in
        ihrem Bereich arbeitenden Gremien der Universität. Sie unterstützt – im Rahmen
        ihrer Neutralität – die Aufstellung von Wahlvorschlägen zu direkt gewählten
        Gremien der akademischen Selbstverwaltung.\\
        (4) Organe der Studienfachschaft sind die Fachschaftsvollversammlung und der Fachschaftsrat.
        &
        (4) Organe der Studienfachschaft sind die Fachschaftsvollversammlung und
        der Fachschaftsrat.\\\hline
        \multicolumn{2}{|c|}{§2}\\\hline
        (1) Die Fachschaftsvollversammlung ist die Versammlung der Mitglieder der Studienfachschaft.
        Sie tagt öffentlich, soweit gesetzliche Bestimmungen dem nicht entgegenstehen.&
        (1) Die Fachschaftsvollversammlung ist die Versammlung der Mitglieder der Studienfachschaft.\\
        (2) Rede-, antrags- und stimmberechtigt sind alle anwesenden Mitglieder der Studienfachschaft.&
        (2) Rede-, antrags- und stimmberechtigt sind alle anwesenden Mitglieder der Studienfachschaft.\\
        (3) Von jeder Sitzung ist ein Protokoll anzufertigen und öffentlich zugänglich zu machen.&
        (3) Von jeder Sitzung ist ein Protokoll anzufertigen und öffentlich zugänglich zu machen.\\
        (4) Beschlüsse werden mit einfacher Mehrheit gefasst.&
        (4) Beschluüsse werden mit einfacher Mehrheit getroffen und sind bindend für den Fachschaftsrat.\\
        (5) Die gefassten Beschlüsse sind bindend für den Fachschaftsrat.&
        (5) Fachschaftsvollversammlungen müssen unverzüglich vom Fachschaftsrat einberufen werden:
        \begin{enumerate}
            \item auf Antrag eines Mitglieds des Fachschaftsrates oder
            \item auf schriftlichen Antrag von einem Hundertstel der Mitglieder der
            Studienfachschaft.
        \end{enumerate}\\
        (6) Die Fachschaftsvollversammlung bestimmt aus ihrer Mitte mit einfacher Mehrheit
        zwei Kassenprüfer*innen. Die Kassenprüfung muss zum Ende der Amtszeit des Fachschaftsrates
        stattfinden. Die Kassenprüfer*innen beantragen bei der Fachschaftsvollversammlung die Entlastung
        des Fachschaftsrates.&
        (6) Die Einberufung einer Fachschaftsvollversammlung muss mindestens vier Tage zuvor öffentlich und
        in geeigneter Weise sowie ortsüblich bekannt gemacht werden.\\
        (7) Fachschaftsvollversammlungen finden am ersten Montag im Monat während der Vorlesungszeit statt.
        Die Studienfachschaft wird am Vortag öffentlich und in geeigneter Weise sowie ortsüblich durch den
        Fachschaftsrat daran erinnert. Zusätzlich können sie von 1/3 des Fachschaftsrats oder durch schriftlichen
        Antrag an den Fachschaftsrat von 1 \% der Mitglieder der Studienfachschaft einberufen werden.&
        (7) Die Fachschaftsvollversammlung gibt sich eine Geschäftsordnung. Beschlüsse zu
        Änderungen der Geschäftsordnung erfolgen mit Zweidrittelmehrheit aller anwesenden
        Mitglieder der Studienfachschaft.\\
        (8) Die zusätzliche Einberufung einer Fachschaftsvollversammlung durch den Fachschaftsrat muss mindestens
        fünf Tage zuvor öffentlich und in geeigneter Weise sowie ortsüblich bekannt gemacht werden.
    &   (deleted)\\\hline
        \multicolumn{2}{|c|}{§3 Fachschaftsrat}\\\hline
        (1) Der Fachschaftsrat wird in gleicher, direkter, freier und geheimer Wahl gewählt.
        Es findet Personenwahl statt.&
        (1) Der Fachschaftsrat wird in gleichen, direkten, freien und geheimen Wahlen
        gewählt. Es findet Personenwahl statt.\\
        (2) Alle Mitglieder der Studienfachschaft haben das aktive und passive Wahlrecht. Es gilt
        die Wahl- und Verfahrensordnung der Verfassten Studierendenschaft, oder eine vom
        Studierendenrat für die Wahlen der Fachschaftsräte erlassene eigene Wahlordnung.&
        (2) Alle Mitglieder der Studienfachschaft haben das aktive und passive Wahlrecht,
        ausgenommen derer nach § 60 Absatz 1 Satz 5 LHG.\\
        (3) Der Fachschaftsrat umfasst zwei Vorsitzende.&
        (3) Der Fachschaftsrat hat zwei Mitglieder.\\
        &(4) Gewählt sind die zwei Kandidierenden, die die meisten Stimmen erhalten, wobei jede*r
        Wahlberechtigte zwei Stimmen hat. Bei genau zwei oder weniger als zwei Kandidierenden,
        kann für oder gegen jeden Kandidierenden gestimmt werden und gewählt sind diejenigen, die mehr
        Ja-Stimmen als Nein-Stimmen erhalten. Im Übrigen gilt die Wahlordnung der Studierendenschaft.\newline
        (4) Die Amtszeit der Mitglieder des Fachschaftsrats beträgt ein Jahr. § 47 der
        Organisationssatzung gilt entsprechend.\\
        (4) Der Fachschaftsrat vertritt die Interessen der Mitglieder der Studienfachschaft.&
        (5) Der Fachschaftsrat vertritt die Interessen der Mitglieder der Studienfachschaft und
        führt die Beschlüsse der Fachschaftsvollversammlung aus.\\  
        (5) Zu den Aufgaben des Fachschaftsrats gehören:
        \begin{enumerate}
            \item[a] Einberufung und Leitung der Fachschaftsvollversammlung.
            \item[b] Ausführung und Koordination der Beschlüsse der Fachschaftsvollversammlung.
            \item[c] Führung der Finanzen.
            \item[d] Beratung und Information der Studienfachschaftsmitglieder.
            \item[e] Mitwirkung an der Lehrplangestaltung.
            \item[f] Austausch und Zusammenarbeit mit den Mitgliedern des Lehrkörpers in den betroffenen Studiengängen.
        \end{enumerate}&
        (6) Zu den Aufgaben des Fachschaftrats gehören:
        \begin{enumerate}
            \item[a] Einberufung und Leitung der Fachschaftsvollversammlung,
            \item[b] Ausführung der Beschlüsse der Fachschaftsvollversammlung,
            \item[c] Führung der Finanzen, Bestimmung des/der Fianzverantwortlichen,
            \item[d] Beratung und Information der Studienfachschaftsmitglieder,
            \item[e] Mitwirkung an der Lehrplangestaltung,
            \item[f] Austausch und Zusammenarbeit mit den Mitgliedern  des  Lehrkörpers  in  den  betroffenen Studiengängen. 
        \end{enumerate}\\
        (6) Die Amtszeit der Mitglieder des Fachschaftsrats beträgt ein Jahr.& \\
        (7) Fr das vorzeitige Ausscheiden aus dem Fachschaftsrat gilt § 35 OS. Außerdem scheidet eine Person
        aus dem StuRa aus, wenn sie nicht mehr für einen der Studiengänge, welche die Studienfachschaft
        vertritt, immatrikuliert ist.&
        (7) Der Fachschaftsrat gibt sich eine Geschäftsordnung. Beschlüsse zu Änderungen der Geschäftsordnung
        bedürfen der Zustimmung beider Mitglieder des Fachschaftsrats.\\
        (8) Im Falle des Ausscheidens eines Mitglieds des Fachschaftsrats rückt der jeweilige vorher zu Beginn
        der Amtszeit durch die Studienfachschaft gewählte Vertreter nach.&
        (8) Die  Aufgaben  des  Fachschaftsrats  kann  dieser unter seinen Mitgliedern aufteilen. Näheres regelt
        die Geschäftsordnung des Fachschaftsrats Pharmazie.\\\hline
        \multicolumn{2}{|c|}{§4 Beauftragte des Fachschaftsrats}\\\hline
        (neu) &
        (1) Die Aufgaben des Fachschaftsrats kann dieser an Mitglieder der Studienfachschaft delegieren.
        Dazu führt der Fachschaftsrat Ämter für Beauftragte ein, die durch den Fachschaftsrat besetzt werden.
        Im Fachschaftsrat bedarf es hierfür der Zustimmung beider  Mitglieder.
        Die Fachschaftsvollversammlung hat das Recht, Vorschläge für Beauftragte zu machen.\newline
        (2) Die Verantwortung für die Arbeit der Beauftragten trägt der Fachschaftsrat in seiner Gesamtheit.\newline
        (3) Der Fachschaftsrat kann Beauftragte jederzeit ihres Amtes entheben und ihre Aufgaben wieder an sich ziehen.
        Dazu bedarf es eines Beschlusses des Fachschaftsrats bei Zustimmung beider Fachschaftsratsmitglieder.\newline
        (4) Näheres regeln die Geschäftsordnung der Fachschaftsvollversammlung Pharmazie und die Geschäftsordnung des
        Fachschaftsrats Pharmazie.\\\hline
        \multicolumn{2}{|c|}{§4->5 Kooperation und Stimmführung im StuRa}\\\hline
        (1) Der Fachschaftsrat entsendet einen Vertreter*in der Fachschaft in den StuRa.&
        (1) Der Fachschaftsrat entsendet auf Grundlage eines Vorschlags der Fachschaftsvollversammlung
        Vertreter*innen der Fachschaft in den Studierendenrat. Vertretung ist möglich.\\
        (2) Die Amtszeit der Vertreter*in im StuRa beträgt ein Jahr.&
        (2) Die Amtszeit der Vertreter*in im StuRa beträgt ein Jahr.§ 47 der Organisationssatzung gilt entsprechend.\\
        (3) Für das vorzeitige Ausscheiden aus dem StuRa gilt § 35 OS. Außerdem scheidet eine Person aus dem StuRa aus,
        wenn sie nicht mehr für einen der Studiengänge, welche die Studienfachschaft vertritt, immatrikuliert ist.&
        ?\\
        (4) Die Studienfachschaft kann sich nach § 14 der Organisationssatzung der Studierendenschaft mit anderen
        Studienfachschaften zu einer Kooperation zusammenschließen.&
        (4) Die Studienfachschaft kann sich nach § 14 der Organisationssatzung der Studierendenschaft
        mit anderen Studienfachschaften zu einer Kooperation zusammenschließen.\\
        \hline
        \multicolumn{2}{|c|}{Alter §5 Entsendung in universitäre Gremien und die Qualitätssicherungsmittelkommission}\\\hline
        (1) Der Fachschaftsrat entsendet entsprechend der möglichen Anzahl von Vertretern im jeweiligen Gremium, Vertreter
        der Studienfachschaft auf Empfehlung der Fachschaftsvollversammlung in universitäre Gremien, in die die Studienfachschaft
        Mitglieder entsendet, insbesondere die „Qualitätssicherungsmittelkommission 2.0 (Quako 2.0) der Fächer Molekulare
        Biotechnologie und Pharmazie“ zwei studentische Vertreter.\newline
        (2)  Die entsandten Vertreter in der „Quako 2.0“ werden durch die Fachschaftsvoll-versammlung beauftragt, das Vorschlagsrecht
        für die studentischen Qualitäts-sicherungsmittel der Fachschaft Pharmazie auszuüben. Die Anträge werden an die 
        gemeinsame „Quako 2.0“ der Fachschaften Molekulare Biotechnologie und Pharmazie und des Institutes für Pharmazie und
        Molekulare Biotechnologie gerichtet. Für die Mittel der Fachschaft Molekulare Biotechnologie üben die beiden gewählten
        Vertreter das alleinige Vorschlagsrecht aus. Näheres zur Antragsstellung regelt die Geschäftsordnung der „Quako 2.0“.&
        (deleted)\\\hline
        \multicolumn{2}{|c|}{Neuer §6 Abwahl eines Mitglieds des Fachschaftsrates}\\\hline
        (new)&
        (1) Ein Mitglied des Fachschaftsrats kann von den Mitgliedern der Studienfachschaft vor Ablauf seiner Amtszeit
        abgewählt werden.\newline
        (2) Zur Einleitung des Abwahlverfahrens bedarf es eines schriftlichen Antrags von mindestens 5\% der Mitglieder
        der Studienfachschaft an die Fachschaftsvollversammlung. Die Durchführung einer Abstimmung über die Abwahl bedarf eines 
        Beschlusses  der  Fachschaftsvollversammlung  mit mindestens der Hälfte der anwesenden Stimmberechtigten. Dabei müssen
        mindestens 20 stimmberechtigte  Studienfachschaftsmitglieder in dieser Fachschaftsvollversammlung anwesend sein.\newline
        (3) Die Abstimmung zur Abwahl des Mitglieds des Fachschaftsrats  muss  mindestens  28  Tage  vorher in geeigneter 
        Weise bekannt gemacht werden. Die Abstimmung zur Abwahl wird zusammen mit dem Zentralen Wahlausschuss der Verfassten
        Studierendenschaft  vorbereitet.  Die  Abstimmung zur  Abwahl  wird  an  einem  Vorlesungstag  über einen Zeitraum von
        mindestens fünf aufeinanderfolgenden  Stunden  durchgeführt.  Bei der Abstimmung zur Abwahl haben alle
        Studienfachschaftsmitglieder das aktive Stimmrecht mit Ausnahme derer nach §60 Abs. 1 Satz  5  LHG. Eine  Briefwahl
        ist  nicht  möglich.  Alles Weitere  regelt  sinngemäß  die  Wahlordnung  der Verfassten Studierendenschaft.\newline
        (4)  Spricht  sich  in  der  Abstimmung  eine  einfache Mehrheit  der  teilnehmenden  Stimmberechtigten für  die 
        Abwahl  des  betreffenden  Mitglieds  des Fachschaftsrats  aus,  scheidet  es  mit  Ablauf  des Tages,  an  dem  der
        Wahlausschuss  der  Verfassten Studierendenschaft   die   Abwahl   feststellt,   aus seinem Amt. In Abweichung von 
        § 3 Absatz 4 dieser Satzung ist ein Verbleiben im Amt in kommissarischer    Funktion    nicht    möglich.    Die Nachwahl
        eines   Mitglieds   des   Fachschaftsrats erfolgt  gemäß  §  4  Absatz  3  der  Wahlordnung  der Verfassten Studierendenschaft.\\\hline
        \multicolumn{2}{|c|}{Neuer §7 Satzungsänderungen}\\\hline
        (new)&
        (1) Über Änderungen der Satzung der Studienfachschaft    Pharmazie    entscheidet    der Studierendenrat  nach  §§  17 
        Absatz  4,  34  und  37 Absatz 2 der Organisationssatzung.(2) Einen Antrag auf Änderung dieser Studienfachschaftssatzung 
        stellt der Fachschaftsrat an die Sitzungsleitung des Studierendenrates. Ein solcher Antrag bedarf eines Beschlusses der
        beiden Mitglieder des Fachschaftsrates  sowie  einer  Zweidrittelmehrheit der Anwesenden bei einer Fachschaftsvollversammlung.\\\hline  
        \multicolumn{2}{|c|}{Neuer §8 Inkrafttreten/Außerkrafttreten}\\\hline
        (new)&
        (1) Diese Satzung tritt mit Wirkung vom XX. Monat 2020 in Kraft. Zugleich tritt die Studienfachschaftssatzung
        vom 3. November 2014, 15. November 2016 und 5. Mai 2017 außer Kraft.\\
    \end{longtable}
    }{
        Grundsätzlich wollen wir die Satzung nach der gelebten Realität unserer Fachschaft formen. Die meisten Aufgaben in
        unserer Fachschaft werden nicht durch den Fachschaftsrat erledigt, sondern durch aus der Mitte der Fachschaftsvollversammlung
        gewählte Beauftragte. Das betrifft insbesondere auch die Führung der Finanzen, aber auch die Ausführung der Beschlüsse der
        FSVV in verschiedener Form. Da dies laut unserer Satzung eigentlich Aufgaben des FSR wären, wollen wir eine „rechtssichere“
        Formulierung finden, die Aufgaben an Studierende aus der FSVV zu übertragen.\\
        Dazu wollen wir gerne Ämter für Beauftragte einführen, die durch den FSR besetzt werden können. Die FSVV soll ein
        Vorschlagsrecht bekommen, analog der Formulierung für die Finanzverantwortlichen aus eurer Formulierungshilfe.
        Da die Aufgaben, die durch die Beauftragten erfüllt werden allerdings immer noch originäre Aufgaben des FSR sind,
        trägt dieser auch weiterhin die Verantwortung dafür. Er darf daher auch die Beauftragten ihres Amtes entheben,
        sollten diese ihre Aufgaben nicht erfüllen.\\
        Wir haben keine gesonderte Formulierung für einen oder eine Finanzverantwortliche:n in den Entwurf geschrieben.
        Die Führung der Finanzen ist eine Aufgabe des FSR, die wie andere Aufgaben auch an Beauftragte delegiert werden können.\\
        An diesen grundlegenden Rechtsrahmen würden wir dann zwei Geschäftsordnungen anschließen,
        die neben den Abläufen unserer Sitzungen auch die Wahlverfahren für Vorschläge für Ämter und die Ämter selbst regelt.
    }{
        \textbf{1. Lesung}
        \ul{\li{Keine Fragen}}
        \textbf{2. Lesung}
        \ul{\li{Keine Fragen}}
        \textbf{3. Lesung}
        \ul{\li{Keine Fragen}}
        \textbf{4. Lesung}
        \ul{\li{Keine Fragen}}
    }{55}{0}{0}
    
\antrag{Satzung der Studienfachschaft Japanologie}{4.Lesung}{Fachschaft Japanologie}
    {\begin{longtable}{|p{7.5cm}|p{7.5cm}|}
        \hline
        \multicolumn{2}{|c|}{Synopse}\\\hline
        Bisheriger Text & Neuer Text \\\hline
        \endfirsthead
        \hline
        Bisheriger Text & Neuer Text \\
        \hline
        \endhead
        \hline
        \multicolumn{2}{|r|}{Weiter auf der nächsten Seite...}\\
        \hline
        \endfoot
        \hline
        \multicolumn{2}{c}{Ende der Synopse} \\
        \endlastfoot
        \multicolumn{2}{|c|}{§3 Fachschaftsrat}\\\hline
        (7) Die Amtszeit der Mitglieder des Fachschaftsrats beträgt ein Jahr. Sie beginntim Sommersemesterund endet mit der
        Wahl des neuen Fachschaftsrats.&
        (7) Die Amtszeit der Mitglieder des Fachschaftsrats beträgt ein Jahr. Siebeginntim Wintersemesterund endet mit der
        Wahl des neuen Fachschaftsrats.\\
        (8) Die Wahlenzum Fachschaftsrat finden in der Regel während eines jeden Wintersemestersstatt. Es wird eine Zusammenlegung
        mit den Wahlen zum Fachrat angestrebt.&
        (8) Die Wahlenzum Fachschaftsrat finden in der Regel während eines jeden Sommersemesterstatt.Die Wahlen des Fachrats
        finden in der Regel während eines jeden Wintersemesters statt.\\
        (9) Die Organisationen der Wahlen werden von einem AK durchgeführt. Kandidaturen für den Fachschaftsratmüssen bis zum 
        Ende der Winterferienbei diesem eingereicht werden. Dies dient zur Sicherung der Chancengleichheit der einzelnen KandidatInnen.
        Sollte der Wahltermin nicht während oder bis Ende eines Wintersemestersdurchgeführt werden können, so kann die
        Fachschaftsvollversammlung beschließen, diesen zu verschieben. Die Frist zum Einreichen von Kandidaturen wäre in diesem
        Fall drei Wochen vor dem Wahltermin, um Absatz 13 gewährleisten zu können.&
        (9) Die Organisationen der Wahlen werden von einem AK durchgeführt. Die Kandidaturen für den Fachschaftsrat müssen bis Ende
        Mai bei diesem eingereicht werden. Dies dient zur Sicherung der Chancengleichheit der einzelnen KandidatInnen. Sollte der
        Wahltermin nicht während oder bis Ende einesSommersemestersdurchgeführt werden können, so kann die Fachschaftsvollversammlung
        beschließen, diesen zu verschieben. Die Frist zum Einreichen von Kandidaturen wäre in diesem Fall drei Wochen vor dem
        Wahltermin, um Absatz 13gewährleisten zu können.\\
    \end{longtable}
    }{
        Es hat sich herausgestellt, dass die Wahlen im Wintersemester und der Amtsbeginn im SS des FSR ungünstig für die Studenten
        sind. Dafür gibt es mehrere Gründe.\newline
        Für (7) \& (8):
        \begin{enumerate}
            \item Erfahrungsgemäß bewerben sich höhere Studenten für den FSR und gelangen in eine missliche Situation,
            in der sie plötzlich ihr Amt beenden müssen, weil sie ihren Auslandsstudium machen. (Fast alle Studenten in der
            Japanologie möchten in Japan ein Auslandsstudium machen).Dies steht dann im Konflikt mit der Satzung und den 
            "eigentlichen" Wahlen.
            \begin{itemize}
                \item Eine Wahl im SS und der Beginn im WS würde den Studenten eher passen, da die Amtszeit genau dann endet, wenn sie ihr Auslandsstudium beginnen (Ende SS, Anfang WS).
                \item Eine Wahl im SS würde auch den neuen Erstsemestler, die im WS kommen, genug Zeit geben sich in der Fachschaft zu engagieren und ihr Interesse am Amt des FSR wecken. Erfahrungsgemäß interessieren sich die meisten Studenten erst ab dem 2.Semester für die Fachschaft und kandidieren zum 3.Semester (Wahlen im 2.Semester SS) für das Amt im FSR.
            \end{itemize}
            \item Studenten, die ihr Auslandsstudium beendet haben, kommen Ende SS und zu Beginn des WS zurück.Nach der alten Satzung haben sie nicht die Möglichkeit sich für das Amt im FSR zu bewerben.Einerseits, weil die Ämter besetzt sind. Und andererseits, weil sie dann keine Zeit mehr haben werden das Amt voll auszuüben, weil sie in dieser Zeit ihr Studium beenden. (Nach dem Abwarten, bis zur nächsten Kandidatur.)
            \begin{itemize}
                \item Die Wahl im SS und der Amtsbeginn im WS würde auch den zurückkehrenden Studenten die Möglichkeit geben sich für das Amt zum FSR zu bewerben und es noch vor ihrem Abschluss vollständig auszuüben.
            \end{itemize}
            \item Der Fachrat der Japanologie wurde bisher immer im WS gewählt. Dies soll weiterhin so sein.
        \end{enumerate}
        Für (9) Anpassung der Änderung.
    }{
        \textbf{1. Lesung}
        \ul{\li{Keine Fragen}}
        \textbf{2. Lesung}
        \ul{\li{Keine Fragen}}
        \textbf{3. Lesung}
        \ul{\li{Keine Fragen}}
        \textbf{4. Lesung}
        \ul{\li{Keine Fragen}}
    }{55}{0}{1}

\antrag{Neufassung der Satzung der Studienfachschaft UFG/VA}{1.Lesung}{Fachschaft UFG/VA}
    {\begin{longtable}{|p{7.5cm}|p{7.5cm}|}
        \hline
        \multicolumn{2}{|c|}{Synopse}\\\hline
        Bisheriger Text & Neuer Text \\\hline
        \endfirsthead
        \hline
        Bisheriger Text & Neuer Text \\
        \hline
        \endhead
        \hline
        \multicolumn{2}{|r|}{Weiter auf der nächsten Seite...}\\
        \hline
        \endfoot
        \hline
        \multicolumn{2}{c}{Ende der Synopse} \\
        \endlastfoot
        \multicolumn{2}{|c|}{Präambel}\\\hline
        & 
        In dem Bestreben, der Fachschaftsarbeit an der Ruprecht-Karls Universität Heidelberg
        eine dauerhafte und bestimmte Grundlage zu geben, haben sich die Studierenden der
        Fächer Geoarchäologie, Ur- und Frühgeschichte sowie Vorderasiatische Archäologie als
        Fachschaft Ur- und Frühgeschichte und Vorderasiatische Archäologie (UFG/VA) folgende
        Satzung gegeben.\newline Die Fachschaft steht für ein Studium ein, in dem sich alle Studierenden individuell
        entfalten und das eigene Recht auf Selbstbestimmung – im Rahmen der Gesetze –
        ausleben kann. In unserem Einsatz für ein solches Studium sehen wir uns als politisch
        neutral und respektieren die Religionsfreiheit unserer Studierenden. Wir fühlen uns
        in unserem Engagement – im Rahmen der Gesetze – ausschließlich durch den freien
        Willen und die unverletzliche Würde des Menschen bestärkt und verpflichtet. Damit
        sich dieser Gedanke in seiner Lebendigkeit entfalten und unermüdlich, aufrichtig und
        frei innerhalb von Universität und Studierendenschaft wirken kann, geben wir uns
        folgende Satzung und nehmen im Rahmen der Erfüllung unserer Aufgaben nach § 65 LHG
        unser – begrenztes – politisches Mandat wahr. Zudem ist die Fachschaft darum bemüht,
        für ein besseres Miteinander von Studierenden und Institut und einen besseren
        Zusammenhalt der Studierenden zu sorgen. Begründung: Dies ist von der VS als
        Kernaufgabe der Fachschaften vorgegeben und hatte in der bisherigen Arbeit unserer
        Fachschaft auch eine wichtige Bedeutung.
        \\
        \multicolumn{2}{|c|}{§1 Allgemeines}\\\hline
        (1)  Die Studienfachschaft vertritt die Studierenden des Fachbereichs „ Ur-
        und  Frühgeschichte und Vorderasiatische Archäologie“ und entscheidet insbesondere
        über fachspezifische Fragen und Anträge.&
        (1)  Die Studienfachschaft (im Folgenden „Fachschaft”) vertritt die Studierenden des
        Fachbereichs „Ur-und Frühgeschichte und Vorderasiatische Archäologie“ sowie „Geoarchä
        ologie” und entscheidet insbesondere über fachspezifische Fragen und Anträge.\\
        (2)  Die Zugehörigkeit zur Studienfachschaft ergibt sich aus der Liste in
        Anhang B.&
        (2)  Die Zugehörigkeit zur Studienfachschaft ergibt sich aus der Liste in Anhang B.\\
        (3)  Die Studienfachschaft stellt die studentischen Mitglieder der in ihrem
        Bereich arbeitenden\newline (4) Gremien oder beteiligt sich zumindest an einem gemeinsamen
        Wahlvorschlag für ebendiese. &
        (3)  Die Studienfachschaft stellt die studentischen Mitglieder der in ihrem Bereich
        arbeitenden Gremien oder beteiligt sich zumindest an einem gemeinsamen Wahlvorschlag
        für ebendiese.\\
        (5) Organe der Studienfachschaft sind die Fachschaftsvollversammlung und der
        Fachschaftsrat. &
        (4)  Organe der Studienfachschaft sind die Fachschaftsvollversammlung und der
        Fachschaftsrat.\\
        \multicolumn{2}{|c|}{§2 Fachschaftsvollversammlung}\\\hline
        (1)  Die Fachschaftsvollversammlung ist die Versammlung der Mitglieder der
        Studienfachschaft. Sie tagt öffentlich, soweit gesetzliche Bestimmungen nicht
        entgegenstehen.&
        (1)  Die Fachschaftsvollversammlung ist die Versammlung der Mitglieder der
        Fachschaft.  Sie tagt öffentlich, soweit gesetzliche Bestimmungen nicht
        entgegenstehen.\\
        (2) Rede-, antrags- und stimmberechtigt sind alle anwesenden Mitglieder der
        Studienfachschaft.&
        (2)  Rede-, antrags- und stimmberechtigt sind alle anwesenden Mitglieder der
        Fachschaft. \\
        (3)  Beschlüsse werden mit einfacher Mehrheit gefasst.&
        (3)  Beschlüsse werden mit einfacher Mehrheit gefasst.\\
        (4)  Die gefassten Beschlüsse sind bindend für den Fachschaftsrat.&
        (4)  Die gefassten Beschlüsse sind bindend für den Fachschaftsrat.\\
        &
        (5) Die Fachschaftsvollversammlung bestimmt im Einvernehmen des Fachschaftsrats bis
        zu zwei Finanzverantwortliche der Fachschaft. Die Finanzverantwortlichen müssen
        eingeschriebene Studierende sein. Die Amtszeit beträgt in der Regel ein Jahr. \\
        &
        (6) Zum Ende der Amtszeit der Finanzverantwortlichen prüft der Fachschaftsrat deren
        Arbeit und beantragt anschließend die Entlastung der Finanzverantwortlichen in der
        Fachschaftsvollversammlung. Diese beschließt die Entlastung der
        Finanzverantwortlichen mit einfacher Mehrheit.\\
        &(7) Die Fachschaftsvollversammlung kann Abstimmungsempfehlungen für das StuRa-
        Mitglied beschließen. Diese sind nicht bindend.\\
        &(8) Die Fachschaftsvollversammlung bestimmt jeden November aus ihrer Mitte bis zu
        drei Personen, welche die Anträge für die Qualitätssicherungsnachfolgemittel (QSM)
        der Fachschaft vorbereiten (QSM-Kommission der Fachschaft). Näheres regelt § 5 dieser
        Satzung.\\
        (5)  Fachschaftsvollversammlungen müssen unverzüglich vom Fachschaftsrat
        einberufen werden:
        \begin{itemize}
        \item[5a]auf Antrag eines Drittels der Mitglieder des Fachschaftsrates oder
        \item[5b]auf schriftlichen Antrag von 1\% der Mitglieder der Studienfachschaft. 
        \end{itemize}&(9)  Fachschaftsvollversammlungen müssen unverzüglich vom Fachschaftsrat einberufen
        werden:
        \begin{itemize}
        \item[9a] auf Antrag eines Drittels der Mitglieder des Fachschaftsrates oder
        \item[9b] auf schriftlichen Antrag von 1\% der Mitglieder der Fachschaft. 
        \end{itemize}\\
        (6) Die Einberufung einer Fachschaftsvollversammlung muss mindestens fünf Tage
        vorher öffentlich und in geeigneter Weise bekannt gemacht werden.&
        (10) Die Einberufung einer Fachschaftsvollversammlung muss mindestens fünf Tage
        vorher öffentlich und in geeigneter Weise bekannt gemacht werden.\\
        &(11) Eine Fachschaftsvollversammlung ist beschlussfähig, wenn sie ordnungsgemäß
        einberufen wurde, mindestens die Hälfte der Fachschaftsräte und insgesamt mindestens
        2 Mitglieder der Fachschaft anwesend sind.\\
        \multicolumn{2}{|c|}{§3 Fachschaftsrat}\\\hline
        (1)  Der Fachschaftsrat wird in gleicher, direkter, freier und geheimer Wahl gewählt.
        Es findet Personenwahl statt.&(1)  Der Fachschaftsrat wird in gleicher, direkter, freier und geheimer Wahl gewählt.
        Es findet Personenwahl statt.\\
        (2)  Alle Mitglieder der Studienfachschaft haben das aktive und passive Wahlrecht.&(2)  Alle Mitglieder der Studienfachschaft haben das aktive und passive Wahlrecht.\\
        (3)  Der Fachschaftsrat umfasst mindestens zwei Mitglieder.&(3)  Der Fachschaftsrat umfasst mindestens zwei und maximal acht Mitglieder.\\
        (4)  Der Fachschaftsrat nimmt die Interessen der Mitglieder der Studienfachschaft wahr. & (4)  Der Fachschaftsrat nimmt die Interessen der Mitglieder der Fachschaft wahr.\\
        (5)  Zu den Aufgaben des Fachschaftsrats gehören:
        \begin{itemize}
        \item[5a]Einberufung und Leitung der Fachschaftsvollversammlung.
        \item[5b]Ausführung der Beschlüsse der Fachschaftsvollversammlung.
        \item[5c]Führung der Finanzen.
        \item[5d]Beratung und Information der Studienfachschaftsmitglieder.
        \item[5e]Mitwirkung an der Lehrplangestaltung.
        \item[5f]Austausch und Zusammenarbeit mit den Mitgliedern des Lehrkörpers des Fachbereichs Ur- und Frühgeschichte und Vorderasiatische Archäologie.    
        \end{itemize}&
        (5)  Zu den Aufgaben des Fachschaftsrats gehören:
        \begin{itemize}
        \item[5a]Einberufung und Leitung der Fachschaftsvollversammlung.
        \item[5b]Ausführung der Beschlüsse der Fachschaftsvollversammlung.
        \item[5c]Führung de Finanzen sowie Prüfung der Arbeit der Finanzverantwortlichen sowie
        Beantragung der Entlastung dieser
        \item[5d]Beratung und Information der Studienfachschaftsmitglieder.
        \item[5e]Mitwirkung an der Lehrplangestaltung.
        \item[5f]Austausch und Zusammenarbeit mit den Mitgliedern des Lehrkörpers des Fachbereichs Ur- und Frühgeschichte und Vorderasiatische Archäologie.    
        \item[5g]Unterstützung der QSM-Kommission der Fachschaft bei ihrer Arbeit. 
        \end{itemize}\\
        (6)  Die Amtszeit der Mitglieder des Fachschaftsrats beträgt ein Jahr.&(6)  Die Amtszeit der Mitglieder des Fachschaftsrats beträgt ein Jahr. Die Amtszeit
        beginnt zum 01. April eines jeden Jahres.*\\
        (7)  Für das vorzeitige Ausscheiden aus dem Fachschaftsrat gilt § 35 OS. Außerdem
        scheidet eine Person aus dem Fachschaftsrat aus, wenn sie nicht mehr für einen der
        Studiengänge, welche die Studienfachschaft vertritt, immatrikuliert ist.&(7)  Für das vorzeitige Ausscheiden aus dem Fachschaftsrat gilt die
        Organisationssatzung des StuRa.\\
        (8) Im Falle des Ausscheidens eines Mitglieds des Fachschaftsrats rückt die Person
        mit der nachfolgenden Stimmenzahl für die verbleibende Amtszeit des ausscheidenden
        Mitglieds in den Fachschaftsrat nach.&(8) Im Falle des Ausscheidens eines Mitglieds des Fachschaftsrats rückt die Person
        mit der nachfolgenden Stimmenzahl für die verbleibende Amtszeit des ausscheidenden
        Mitglieds in den Fachschaftsrat nach.\\
        \multicolumn{2}{|c|}{§4 Kooperation und Stimmführung im Studierendenrat}\\\hline
        (1)  Der Fachschaftsrat entsendet Vertreter/innen der Fachschaft in den
        Studierendenrat.&(1)  Der Fachschaftsrat entsendet ein Mitglied der Fachschaft in den Studierendenrat
        (StuRa).\\
        &(2) Der Fachschaftsrat entsendet zudem Stellvertreter*innen in den StuRa. \\
        (2)  Die Amtszeit der Vertreter/innen im StuRa beträgt ein Jahr.& (3)  Die Amtszeit der Entsandten im StuRa beträgt ein Jahr.\\
        (3)  Für das vorzeitige Ausscheiden aus dem StuRa gilt § 35 OS. & (4)  Für das vorzeitige Ausscheiden aus dem Studierendenrat gilt die
        Organisationssatzung des StuRa. \\
        &(5) Das StuRa-Mitglied und dessen Stellvertreter*innen können per Beschluss mit 2/3-
        Mehrheit in der Fachschaftsvollversammlung abberufen werden. \\
        &(6) Das StuRa-Mitglied und dessen Stellvertreter*innen stimmen nach bestem Wissen und
        Gewissen im Studierendenrat ab. \\
        &(7) Das StuRa-Mitglied und dessen Stellverterer*innen orientieren sich an den
        Abstimmungsempfehlungen der Fachschaftsvollversammlung. \\
        (4)  Die Studienfachschaft kann sich nach § 14 der Organisationssatzung der
        Studierendenschaft mit anderen Studienfachschaften zu einer Kooperation
        zusammenschließen.&
        (8)  Die Fachschaft kann sich nach § 14 der Organisationssatzung der
        Studierendenschaft mit anderen Fachschaften zu einer Kooperation zusammenschließen.\\
        \multicolumn{2}{|c|}{§5 Qualitätssicherungsnachfolgemittel}\\\hline
        &(1) Die Fachschaftsvollversammlung bestimmt jeden November aus ihrer Mitte bis zu
        drei Personen, welche die Anträge für die QSM vorbereiten. Diese bilden die
        QSM-Kommission der Fachschaft.\\
        &(2) Nach Bildung der QSM-Kommission wird das QSM-Referat über dessen Mitglieder
        informiert. \\
        &(3) Vorschläge für die Verwendung der QSM müssen bis spätestens zwei Wochen vor
        Antragsfrist bei der QSM-Kommission der Fachschaft eingereicht werden.\\
        &(4) Bei der Vergabe sind die Mittel auf UFG und VA getrennt, der Anzahl der
        Studierenden entsprechend, zu veranschlagen. Die Mittel der Geoarchäologie werden
        denen der UFG zugerechnet. \\
        &(5) Per Beschluss der QSM-Kommission der Fachschaft können die Mittel auch gemeinsam
        veranschlagt werden. Sollte die Kommission nur aus einer Person, oder nur Personen
        einer der Fächer bestehen, so muss dieser Beschluss vom Fachschaftsrat getroffen
        werden. \\
        &(6) Aufgaben der QSM-Kommission der Fachschaft sind: 
        \begin{itemize}
        \item[6a] Die vorzeitige Information über den zur Verfügung stehenden Betrag für die QSM;
        \item[6b]Die Vorbereitung der Anträge für die QSM in Rücksprache mit der Fachschaft;
        \item[6c]   Die Fristgerechte Einreichung der QSM-Anträge.
        \end{itemize}\\
        & Die Änderung dieser Satzung tritt zum 01. Januar 2021 in Kraft.\\
    \end{longtable}
}{
    Einige der Änderungen sind zur Lesbarkeit, andere wie die Einführung einer Fachschaftseigenen QSM-Kommission entspringen der Notwendigkeit. Ebenso haben wir die Geoarchäologie, die wir ja auch vertreten, endlich mitaufgenommen.
}{
    \textbf{1. Lesung}
}{tba}{tba}{tba}

\antrag{Fusion der Fachschaften Klassische Archäologie und Byzantinische Archäologie und Kunstgeschichte}{1.Lesung}{Fachschaft Byzantinische Archäologie und Kunstgeschichte, Fachschaft Klassische Archäologie}
    {\begin{longtable}{|p{7.5cm}|p{7.5cm}|}
        \hline
        \multicolumn{2}{|c|}{Synopse}\\\hline
        Bisheriger Text & Neuer Text \\\hline
        \endfirsthead
        \hline
        Bisheriger Text & Neuer Text \\
        \hline
        \endhead
        \hline
        \multicolumn{2}{|r|}{Weiter auf der nächsten Seite...}\\
        \hline
        \endfoot
        \hline
        \multicolumn{2}{c}{Ende der Synopse} \\
        \endlastfoot
        \multicolumn{2}{|c|}{Anhang D}\\\hline
        1. Ägyptologie                                                   & 1. Ägyptologie                                                   \\
        2. Alte Geschichte                                               & 2. Alte Geschichte                                               \\
        3. American Studies                                              & 3. American Studies                                              \\
        4. Anglistik                                                     & 4. Anglistik                                                     \\
        5. Assyriologie                                                  & 5. Assyriologie                                                  \\
        6. Byzantinische Archäologie und Kunstgeschichte                 &                                                                  \\
        7. Biologie                                                      & 6. Biologie                                                      \\
        8. Chemie und Biochemie                                          & 7. Chemie und Biochemie                                          \\
        9. Computerlinguistik                                            & 8. Computerlinguistik                                            \\
        10. Deutsch als Fremdsprache                                     & 9. Deutsch als Fremdsprache                                      \\
        11. Erziehung und Bildung                                        & 10. Erziehung und Bildung                                        \\
        12. Ethnologie                                                   & 11. Ethnologie                                                   \\
        13. Geographie                                                   & 12. Geographie                                                   \\
        14. Geowissenschaften                                            & 13. Geowissenschaften                                            \\
        15. Germanistik                                                  & 14. Germanistik                                                  \\
        16. Gerontologie \& Care                                         & 15. Gerontologie \& Care                                         \\
        17. Geschichte                                                   & 16. Geschichte                                                   \\
        18. Informatik                                                   & 17. Informatik                                                   \\
        19. Islamwissenschaft                                            & 18. Islamwissenschaft                                            \\
        20. Japanologie                                                  & 19. Japanologie                                                  \\
        21. Jura                                                         & 20. Jura                                                         \\
        22. Klassische Archäologie                                       & 21. Klassische und Byzantinische Archäologie                     \\
        23. Klassische Philologie                                        & 20. Klassische Philologie                                        \\
        24. Kunstgeschichte (Europäische)                                & 21. Kunstgeschichte (Europäische)                                \\
        25. Mathematik                                                   & 24. Mathematik                                                   \\
        26. Medizin Heidelberg                                           & 25. Medizin Heidelberg                                           \\
        27. Medizin Mannheim                                             & 26. Medizin Mannheim                                             \\
        28. Mittellatein/Mittelalterstudien                              & 27. Mittellatein/Mittelalterstudien                              \\
        29. Molekulare Biotechnologie                                    & 28. Molekulare Biotechnologie                                    \\
        30. Musikwissenschaft                                            & 29. Musikwissenschaft                                            \\
        31. Ostasiatische Kunstgeschichte                                & 30. Ostasiatische Kunstgeschichte                                \\
        32. Pharmazie                                                    & 31. Pharmazie                                                    \\
        33. Philosophie                                                  & 32. Philosophie                                                  \\
        34. Physik                                                       & 33. Physik                                                       \\
        35. Politikwissenschaft                                          & 34. Politikwissenschaft                                          \\
        36. Psychologie                                                  & 35. Psychologie                                                  \\
        37. Religionswissenschaft                                        & 36. Religionswissenschaft                                        \\
        38. Romanistik                                                   & 37. Romanistik                                                   \\
        39. Semitistik                                                   & 38. Semitistik                                                   \\
        40. Sinologie                                                    & 39. Sinologie                                                    \\
        41. Slavistik/Osteuropastudien                                   & 40. Slavistik/Osteuropastudien                                   \\
        42. Soziologie                                                   & 43. Soziologie                                                   \\
        43. Sport                                                        & 42. Sport                                                        \\
        44. Südasieninwissenschaften (Fachschaft am SAI)                 & 43. Südasieninwissenschaften (Fachschaft am SAI)                 \\
        45. Theologie (Evangelische)                                     & 44. Theologie (Evangelische)                                     \\
        46. Transcultural Studies (891)                                  & 45. Transcultural Studies (891)                                  \\
        47. Ur- und Frühgeschichte/Vorderasiatische Archäologie (UFG/VA) & 46. Ur- und Frühgeschichte/Vorderasiatische Archäologie (UFG/VA) \\
        48. Übersetzen und Dolmetschen (Fachschaft am IÜD)               & 47. Übersetzen und Dolmetschen (Fachschaft am IÜD)               \\
        49. Volkswirtschaftslehre (VWL)                                  & 48. Volkswirtschaftslehre (VWL)                                  \\
        \multicolumn{2}{|c|}{Anhang B}\\\hline
        (6) Byzantinische Archäologie und Kunstgeschichte (830, 8302, 8305, 8304)
 (Byzantinische Archäologie und Kunstgeschichte)\newline (22) Klassische Archäologie (831, 8317, 8312, 8315, 8314, 8347, 12N, 849) (Klassische
 Archäologie) & (21) Klassische und Byzantinische Archäologie (831, 8317, 8312, 8315, 8314, 8347,
 12N, 849) (Klassische Archäologie) und (830, 8302, 8305, 8304) (Byzantinische
 Archäologie und Kunstgeschichte)\\
    \end{longtable}
}{
    Nach der Fusion der Institute haben die beiden Fachschaften beschlossen, dass es für die Wahrnehmung der Vertretung der Studierenden der beiden Fächer leichter ist, sich zu einer FS zusammenzuschließen.
}{
    \textbf{1. Lesung}
    \ul{\li{Keine Fragen}}
}{tba}{tba}{tba}

\antrag{Satzung der neuen Fachschaft Klassische und Byzantinische Archäologie}{1.Lesung}{Fachschaft Byzantinische Archäologie und Kunstgeschichte, Fachschaft Klassische Archäologie}
{
    
}{
    Nach der Fusion der Institute haben die beiden Fachschaften beschlossen, dass es für die Wahrnehmung der Vertretung der Studierenden der beiden Fächer leichter ist, sich zu einer FS zusammenzuschließen.
}{
    \textbf{1. Lesung}
    \ul{\li{Keine Fragen}}
}{tba}{tba}{tba}