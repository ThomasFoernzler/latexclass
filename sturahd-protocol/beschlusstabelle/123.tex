\sitzung{123}{15.12.2020}
\beschluss{ Satzung der Studienfachschaft Philosophie }{
    \begin{longtable}{|p{10.5cm}|p{10.5cm}|}
        \hline
        \multicolumn{2}{|c|}{Synopse}\\\hline
        Bisheriger Text & Neuer Text \\\hline
        \endfirsthead
        \hline
        Bisheriger Text & Neuer Text \\
        \hline
        \endhead
        \multicolumn{2}{|r|}{Weiter auf der nächsten Seite...}\\
        \hline
        \endfoot
        \hline
        \multicolumn{2}{c}{Ende der Synopse} \\
        \endlastfoot
        \multicolumn{2}{|c|}{§3}\\\hline
        (4) Er umfasst mindestens zwei Mitglieder. Sollten mehr als zwei Kandidat*innen
        aufgestellt werden, so gilt, dass die Anzahl der zu besetzenden Sitze der Zahl der
        Kandidat*innen entspricht, aber maximal vier beträgt.
    &   (4) Er umfasst bis zu vier, aber mindestens zwei Mitglieder.\newline 
        (5) Gewählt sind diejenigen Kandidierenden, die die meisten Stimmen erhalten, wobei
        jede*r Wahlberechtigte bis zu vier Stimmen, aber höchstens so viele Stimmen wie es
        Kandidierende gibt, hat. Bei vier oder weniger als vier Kandidierenden, kann für oder
        gegen jede*n Kandidierende*n gestimmt werden und gewählt sind diejenigen, die mehr
        Ja- als Nein-Stimmen erhalten. Im Übrigen gilt die Wahlordnung der
        Studierendenschaft.\\\hline
        (6) Die Mitglieder des Fachschaftsrates treffen sich in der Vorlesungszeit mindestens
        einmal im Monat zu einer Fachschaftsratssitzung:
        \begin{enumerate}
            \item[a] Diese Sitzung ist mit der Anwesenheit von 2/3 der Fachschaftsräte beschlussfähig.
            \item[b] Das Stura-Mitglied der Fachschaft ist bei diesen Sitzungen beratendes Mitglied.
            \item[c] Der Termin der Fachschaftsratssitzung des jeweiligen Monats wird in der letzten
            Fachschaftsvollversammlung des Vormonats festgelegt.
        \end{enumerate}
    &   (7) Die Mitglieder des Fachschaftsrates treffen bei Bedarf, mindestens aber zweimal
        im Semester, zu einer Fachschaftsratssitzung:
        \begin{enumerate}
            \item[a] Diese Sitzung ist mit der Anwesenheit von 2/3 der Mitglieder beschlussfähig.
            \item[b] Das Stura-Mitglied der Fachschaft ist bei diesen Sitzungen beratendes Mitglied.
            \item[c] Der Termin der Fachschaftsratssitzung wird von den Fachschaftsrät*innen
            festgelegt. Er muss in geeigneter Weise ortsüblich bekannt gemacht werden und in
            einer Fachschaftsvollversammlung angekündigt werden.
        \end{enumerate}\\\hline
        \multicolumn{2}{|c|}{§4 Arbeitskreise der Fachschaft}\\\hline
        (neu)
    &   (1) Die Fachschaft kann zur Bearbeitung bestimmter Themengebiete Arbeitskreise
        einrichten. Den Beschluss über die Einrichtung trifft die Fachschaftsvollversammlung.\newline
        (2) Ein Arbeitskreis kann jederzeit durch die Fachschaftsvollversammlung aufgelöst
        werden. Ein Arbeitskreis wird aufgelöst, wenn er dreizehn Monate nicht tagt.\newline
        (3) Die Mitwirkung an der Arbeit der Arbeitskreise richtet sich nach den gleichen
        Voraussetzungen wie die Teilnahme an Fachschaftsvollversammlungen.\newline
        (4) Die Fachschaftsvollversammlung entsendet für jeden Arbeitskreis zugleich einen
        Berichterstatter oder eine Berichterstatterin. Die Berichterstatter*in betreut den Arbeitskreis
        und berichtet regelmäßig in der Fachschaftsvollversammlung über dessen Arbeit. Die Amtszeit
        der Berichterstatter*in beträgt ein Jahr.\newline
        (5) Die Termine der Sitzungen der Arbeitskreise werden von der Berichterstatterin
        festgelegt. Die Sitzungen müssen mindestens zwei Tage im Voraus öffentlich und in
        geeigneter Weise ortsüblich bekannt gemacht werden.\\\hline
        \multicolumn{2}{|c|}{§5}\\\hline
        (1) Die Fachschaftsvollversammmlung beschließt einen Verwendungsvorschlag über einen 
        Teil der QSM oder die gesamten QSM. Dieser Beschluss muss mindestens eine Woche vor
        der Einreichungsfrist der Vorschläge gefasst werden.
        \begin{enumerate}
            \item[a] Der Verwendungsvorschlag der Fachschaftsvollversammlung muss bis zum 8. Januar
            für Vorschläge, die bis zum 15. Januar eingereicht werden sollen, und spätestens bis
            zum 8. Mai für Vorschläge, die bis zum 15. Mai eingereicht werden sollen, gefasst
            werden.
            \item[b] Der Verwendungsvorschlag muss nicht in ausgearbeiteter Form vorliegen, sondern
            lediglich das Interesse der Fachschaftsvollversammlung widerspiegeln.
            \item[c] Der Vorschlag ist für den Fachschaftsrat bindend.
            \item[d] Der Beschluss wird mit einfacher Mehrheit gefasst
        \end{enumerate}
    &   (1) Die Fachschaftsvollversammmlung beschließt einen Verwendungsvorschlag über einen
        Teil der QSM oder die gesamten QSM. Dieser Beschluss muss mindestens eine Woche vor
        der Einreichungsfrist der Vorschläge gefasst werden.
        \begin{enumerate}
            \item[a] Der Verwendungsvorschlag der Fachschaftsvollversammlung muss bis zum 8. Januar
            für Vorschläge, die bis zum 15. Januar eingereicht werden sollen, und spätestens bis
            zum 8. Mai für Vorschläge, die bis zum 15. Mai eingereicht werden sollen, gefasst
            werden.
            \item[b] Der Verwendungsvorschlag muss nicht in ausgearbeiteter Form vorliegen, sondern
            lediglich das Interesse der Fachschaftsvollversammlung widerspiegeln.
            \item[c] Der Beschluss wird mit einfacher Mehrheit gefasst.
        \end{enumerate}\\\hline
        \multicolumn{2}{|c|}{§8}\\\hline
        (neu) & Die Satzung tritt in Kraft am 27.06.2020.\\\hline
        \end{longtable}
}
\newpage