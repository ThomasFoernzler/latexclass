\section{Infos, Termine, Berichte}

\subsection{Wahlen}
\begin{itemize}
    \item bis 15.12.2020: Anmeldung von Online-Wahlen
    \item 14.01.2021, 16:00: Ende des Kandidaturzeitraums
    \item 25.01.2021, 10:00 – 02.02.2021, 12:00: Online-Wahlen
\end{itemize}
\textbf{Was steht an?}
\begin{enumerate}
    \item FSR und FR Wahlen wen betrifft es?\\
    => siehe diese Tabelle: \url{https://www.stura.uni-heidelberg.de/wp-content/uploads/Wahlen_2020/Wahlen_WiSe_2020.pdf}\\
    => Link zur Bekanntmachung: \url{https://www.stura.uni-heidelberg.de/wp-content/uploads/Wahlen_2021/Bekanntgabe_Wahlen_FSR_FR_Winter_2020.pdf}
    \item Fusion der Archäologien \url{https://www.stura.uni-heidelberg.de/wp-content/uploads/Wahlen_2021/Satzungseinreicheaufforderung_Fusion_Byz-Klarch.pdf}
    \item  Satzungsüberarbeitung wir überarbeiten gerade die Wahlordnung und weitere damit zusammenhängende Satzungen, meldet euch wenn euch was auffällt
\end{enumerate}
\emph{Weitere Infos:}\\
\url{https://www.stura.uni-heidelberg.de/wahlen/}

\subsection{Bericht aus der Kommission der Marsilius-Studien Sommersemester 2020}
Als studentische Vertreterinnen in der Kommission der Marsilius-Studien am Marsilius-Kolleg würden wir, wie auch schon letztes Semester, gerne in der kommenden StuRa-Sitzung kurz über unsere Arbeit und Sitzungen der Kommission im Sommersemester 2020 berichten. Anbei findet ihr unseren Bericht für das Sommersemester 2020.
\paragraph{Sitzungen der Kommission}
Im Berichtzeitraum fand eine Sitzung,am 30.07.2020, statt, die ca. 1 Stunde dauerte. An den Sitzun-gen nahmen  folgende  Kommissionsmitglieder  teil:  Die  beidenneugewähltenDirektor*innendes Marsilius-Kollegs, einVertreter der Hochschullehrer*innen, die beidenstudentischen Vertreterin-nenund die stellvertretende studentische Vertreterin sowie ein Gast,der für die nächste Amtszeit als studentischer Vertreter kandidieren möchte. Mit beratender Stimme nahm auch derGeschäfts-führer des Marsilius-Kollegs teil.Die Atmosphäre in den Sitzungen war sehr konstruktiv, offen und auf Augenhöhe. Unsere Anregun-gen und Kritik wurden aufgenommen, konstruktiv diskutiert und,wenn möglich,auch umgesetzt. Alle Themen wurden so lange erörtert, bis ein einvernehmlicher Beschluss gefällt werden konnte.
\paragraph{Themen}
Zu Beginn dieser Sitzungstellten sich zuerst Frau Prof. Nüssel und Herr Prof. Boutros als neues Direktorium des Marsilius-Kollegs vor und es folgte eine kurze Vorstellungsrunde. Anschließend wurden  die Veranstaltungen  des  zurückliegenden  Semesters  besprochen  und  ggf.  auch  aufgetretene Probleme diskutiert. Trotz der Beschränkungen konnten drei Brückenseminare erfolgreich stattfinden. Der Erfolg der Umsetzung dieser Seminare war stark von der Konzeption des Seminars so wie von den Teilnehmenden abhängig. Die Geschäftsstelle berichtetevondemFeedback der Seminarleiter*innen, die von einersehr positivenZusammenarbeit mit denTeilnehmer*innen sprachen und bedankte sich bei allen Dozierenden für ihr Engagement und zeitlichen Mehraufwand. Wie bereits in vorherigen Semestern waren auch hier teilweise die Teilnehmerzahlen sehr gering. Die Kommission  diskutierte  hier  nun  über  eine  einheitliche  und  festgeschriebene  Verankerung  der  Marsilius-Studien in den Prüfungsordnungen, um einen höheren Anreiz zu schaffen und eine höhere Verbind-lichkeit  zu  gewährleisten.  Frau Prof.  Nüssel  und Herr  Prof.  Boutros werden dieses Thema  mit  der Prorektorin für Lehre besprechen und anschließend in die notwendigen Gremien einbringen. Dies wurde von den studentischen Vertreterinnen sehr begrüßtund das Weiternbeschlossen, dass die studentische Vertretung in diesenProzess miteinbezogen werden soll. Bei derSitzung wurden darüber hinausdie schriftlichen Vorschläge für Brückenveranstaltungen des Folgesemesters  diskutiert. Auch  diese  werden  höchstwahrscheinlich  in  Online-Formaten  stattfinden. Wie bereits im vorherigen Semester wurdebesonders auf die für die Marsilius-Studien konstituierende  Interdisziplinarität  geachtet.  Auch  organisatorische  Aspekte (z.B.  Veranstaltungsrhythmus, Terminierung von Blockterminen, Arbeitsanforderungen) wurden ausführlich erörtert mit dem Ziel, dass sich die Veranstaltungen möglichst gut mit dem Hauptstudium vereinbaren lassen. Auchlassensich leider aus den studentischen Rückmeldungen der letzten Semester und den bisherigen Erfahrungen mit  den  Brückenseminaren keine  einheitlichen  Empfehlungen  ableiten. Dies  liegt  im Wesentlichenan der Vielfalt und unterschiedlichen Ausgestaltung der angebotenen Seminare. Bei den Seminarvorschlägen hat die Kommission Änderungen oder Ergänzungen vorgeschlagen, die von den Seminarleiter*innen (soweit möglich) aufgenommen wurden. Trotz der aktuellen Veränderungen bemüht sich das Marsilius-Kolleg weiterhin,aktuelle und span-nende Seminarangebote zu vielfältigen Themen anzubieten und nimmt Vorschläge immer gut und sehr gerne auf. Zuletzt  wurden  wir  drei  studentischen  Vertreterinnen  verabschiedet  und  wir  bedanken  uns  auch hiermit beim StuRa und der ganzen Studierendenschaft für die Möglichkeit, sich in den Marsilius-Studien für Vielfalt und Interdisziplinarität einzusetzen.
\paragraph{Ausblick}
Die von Studierenden initiierten Marsilius-Studien sind ein kleines, aber sehr sinnvolles Angebot für die Studierenden der Uni Heidelberg. Das Angebot sollte aus unserer Sicht mittelfristig ausgebaut werden,  um  mit  einem  noch  breiteren  Programm –insbesondere  auch  zu  kontroversen  gesellschaftspolitischen  Themen –noch  mehr  Studierende  zu  erreichen.  Der  Beitrag  der  studentischen Vertreter*innen  in  der  Kommission  ist  hierfür  sehr  wichtig,  um  die  Angebote  für  Studierende  attraktiv und mit sonstigencurricularenVerpflichtungen vereinbar zu gestalten. Wir hoffen, dass auch im nächsten Semester motivierte Studierende dieses Amt übernehmen und ausbauen werden.