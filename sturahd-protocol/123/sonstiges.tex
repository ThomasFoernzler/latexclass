\section{Sonstiges}
\antrag{Corona-Vollversammlung}{2. Lesung}{Leonard Späth (SDS Heidelberg)}
{
    Der StuRa beschließt die Organisation einer (satzungstechnisch inoffiziellen) Vollversammlung
    für alle Studierenden zur Lage der Krise an den Hochschulen und in der Gesellschaft.
    Diese wird mit finanzieller und ideeller Unterstützung der Referatekonferenz von Interessierten
    Studierenden organisiert und durchgeführt. 
}{
    Die Krise hat schon vorher bestehende Missstände weiter verschärft. Soziale Ungleichheiten haben
    massiv zugenommen, die Ungerechtigkeiten im Bildungssystem treten verstärkt offen zu Tage, viele
    kleinere Kulturbetriebe haben Existenznöte.\\
    Die gesellschaftlichen Problematiken zeigen sich auch an den Hochschulen. Während es Konzernhilfen
    in Milliardenhöhe gibt, werden Tausende Studierende mit ihren finanziellen Problemen alleine
    gelassen. Während Produktion weiter anläuft, wird noch nicht einmal die Möglichkeit gegeben,
    kleinere Seminare an der Uni in Präsenz stattfinden zu lassen. Während Erstsemester in ein
    teilweise miserabel organisiertes Digitalsemester eingeladen werden, wird weiter an der
    Prüfungsfixierung festgehalten, ohne Studierenden mit an die Pandemie angepassten Prüfungsbedingungen
    entgegen zu kommen.\\
    Aus diesem Grund braucht es ein Informations, - Diskussions, und Austauschangebot an alle 
    Studierende, in der wir gemeinsam mit ebenfalls von Pandemie besonders betroffenen, wie etwa
    Künstler*innen oder überlasteten Beschäftigten im Gesundheitssystem diskutieren wollen, wie man eine
    soziale, emanzipatorische, und gesundheitlich verantwortungsbewusste Lösung für die Krise finden
    kann.\\
    Eine Vollversammlung bietet dafür die Möglich. Diese sieht unsere Satzung offiziell nicht vor.
    Allerdings ist es trotzdem möglich, als Studierendenvertretung und Fachschaften zusammen einzuladen.
    Diese könnte (je nach Situation) in Präsenz, online oder auch als Hybridformat stattfinden.\\
    Ein möglicher Tagungsablauf für diese wäre:\\
    \begin{itemize}
        \item Inputs von Studierendenseite zur aktuellen sozialen Situation, der Arbeit an der Hochschule
            für die Bewältigung der Pandemie usw.
        \item Eindrücke und Grußworte von Gästen aus Kulturbetrieben, dem Gesundheitswesen, der
            Wissenschaft (zum Beispiel Sozialforschung) oder Schulen und Kindergärten.
        \item Je nach Teilnehmerzahl Diskussion und Erarbeitung von Handlungsperspektiven als
            Studierendenschaft im Plenum oder in Kleingruppen.
        \item Gemütlicher Ausklang (Socialising)
    \end{itemize}
    Die Studierendenschaft sollte dies unterstützen durch:
    \begin{itemize}
        \item Finanzielle Unterstützung durch Social-Media und Printwerbung
        \item Verbreitung über ihre Kanäle
        \item Sonstige ideelle Beratung und Unterstützung (Raumanfragen, digitale Infrastruktur)
    \end{itemize}
    Das ist nur ein erster Vorschlag. Zentral ist aber, dass wir Analyse, Aktion und das Soziale
    Miteinander zusammen bringen.\\
    Dieser grobe Tagesordnungsvorschlag wird die Möglichkeit gegen ein Angebot an alle Studierenden
    zu machen. Zugleich könnten wir unsere studentischen Kämpfe für bessere Ausfinanzierung und
    Prüfungsbedingungen mit den aktuellen Herausforderungen in der Pandemiezeit finden. Und nicht
    zuletzt geben wir auch ein Angebot zum sozialen Austausch, der doch gerade jetzt, wichtiger denn je ist.
}{\textbf{1. Lesung:}
    \ul{
        \li{Momentan nur Digital möglich, Organe der VS müssen zusammenarbeiten, AkLele, konkreter Fahrplan von Nöten}
        \li{Präsenz. Hybrid oder Online? Schon im Januar, Durchführung in Präsenz nicht absehbar}
            \ul{\lii{im Januar digital}}
        \li{Was sollen wir aus der Vollversammlung mitnehmen, bis jetzt nur politische Ziele, nichts für uns an der Uni}
            \ul{\lii{Bis jetzt gab es relativ wenig Auseinandersetzung mit Coona und den Hintergründen}
                \lii{Ziele: Solidarsemesterkampange, Demokratieabbau wegen wegfallenden Senatssitzungen, mehr Einsicht in Handlungen des Rektorats}
                \lii{Studis zum mitarbeiten erreichen, Selbstermächtigungsgedanke}}
        \li{Antrag enthält auch Finanzposten, über welchen Rahmen reden wir hier?}
            \ul{\lii{mehr als 100€ facebook , 100€ flyer, 100€ essen wäre es nicht}}
        \li{Wer setzt das denn um, die Referate?}
            \ul{\lii{Zusammenarbeit mit Referaten und Gruppen, mit PoBi-referat oder aklele Mail über die Liste um Interessierte zu finden}}
        \li{Wäre es möglich, eine Sondersitzung des Stura als Vollversammlung zu vermarkten}
            \ul{\lii{Ist nicht das gleiche wie eine StuRa sitzung, etikett des StuRa soll wegewischt werden, unabhängig von den Strukturen der Vs, informelles Format bessere Wahl}
                \lii{Extra Sitzung zu möglichen Corona-Hilfsangeboten, müsste gut vorbereitet sein}}
        \li{Anzahl Teilnehende?}
            \ul{\lii{50 - 200 Leute (Vermutung)}}
        \li{was sollen die 200 Leute machen? Diskutieren oder Dinge beschließen? was soll das Bringen? Diskussiojne über ausgearbeitete Anträge, ansonsten Chaos, kein roter Faden}
        \li{Man kann nicht zu Vollversammlung einladen, müssen wir mit mehr als 200 Leuten rechnen (28000 Studierende), man könnte es auch einfach nicht Vollversammlung sonder Workshop oder Austauschtreffen nennen, Begriff der Vollversammlung nicht ganz trivial}
        \li{Frage scheint eher um Format zu gehen, Ausserordentliche Stur-Sitzung mit einem Austausch-TOP, explizit auch nicht-reguläre Mitglieder einladen.}
        \li{Meinungsbild anstatt beschluss an dieser Stelle, bei Mehrheit Mail-Verteiler / Telegramgruppe, unter Leitung des Vorsitz, 5-10 Leute notwendig für Orga}
            \ul{\lii{Bewerbung der StuRa-Sitzung wichtig, offenes Format,Diskussion anhand von ausgearbeiteten Anträgen sinnvoll, vorformulierter Diskussionsgegenstand}}
    }
}{tba}{tba}{tba}