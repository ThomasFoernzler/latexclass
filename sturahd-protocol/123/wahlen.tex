\section{Kandidaturen und Wahlen}

\wahl{Kandidatur für das Referat für politische Bildung}{2. Lesung}{Felix Diener und Janek Kasperowski}
{
    Der Kandidaturtext findet sich auf der \kandidaturenseite.
}{
    \textbf{1. Lesung:}
    \ul{
        \li{Felix nochmal die Mitglidschaften nennen?}
        \ul{\lii{ver.di, Die linke.SDS, Rote Hilfe e.V.}}
        \li{Wie wollt ihr als Mitglieder einer linken HSG gewährleisten, dass die politische Bildung der Studierenden neutral stattfindet?}
        \ul{\lii{Jan: Diskurs mit anderen Gruppen und Privatpersonen}
            \lii{Felix: politische neutralität kann gewährleistet werden, es fehlt im Referat die Stimme der liberalen und konservativen Meinung}
        }
    }
}

\wahl{Kandidatur für das Referat für Kultur und Sport}{2. Lesung}{Jovana Perovic}
{
    Der Kandidaturtext findet sich auf der \kandidaturenseite.
}{
    \textbf{1. Lesung:}
    \ul{
        \li{Test der Theaterflatrate, überlegeungen zu Schwimmbadflatrate, würdest du an der Theaterflatrate weiterarbeiten und das mit der Schwimmbadflatrate anstossen?}
        \ul{\lii{Ja Theaterflatrate, wenn Schwimmbadflatrate von Studierenden gewünscht ist auch gerne}}
        \li{andere Projektideen}
        \ul{\lii{virtuellen Lesekreis, digitale Rundgänge in Museen}}
    \li{Mitgliedschaften?}
        \ul{\lii{politisch informiert und intereessiert aber bekennt sich nicht zu irgendwelchen poolitischem Gruppen}}
    }
}

\wahl{Kandidatur für das Referat Antirassismus}{2. Lesung}{Mithily Masilamany}
{
    Der Kandidaturtext findet sich auf der \kandidaturenseite.
}{
    \textbf{1. Lesung:}
    \ul{
        \li{siehst du anisemitismus als deinen aufgabenbereich und inwiefern beschäftigt sich das referat damit}
        \ul{\lii{ja ist aufgabe, bis jettzt keine meldungen diesbezügich, stellungsnaheme wegen Burschenscfhaftsvorfall, freuen uns über Mithilfe und Input}
        }
    }
}

\wahl{Kandidatur für das Referat für Kultur und Sport}{2. Lesung}{Chiara Citro}
{
    Der Kandidaturtext findet sich auf der \kandidaturenseite.
}{
    \textbf{1. Lesung}
    \ul{
        \li{Theaterflatrate und Schwimmbadflatrate}
        \ul{\lii{Theaterflatrate coole Sache, umsetzung der Schwimmbadflatrate fraglich momentan}}
        \li{Schon mit joanna zusammengesetzt?}
        \ul{\lii{Wir haben uns bisher nicht zusammengesetzt, aber ich denke, dass das kein Problem sein wird. Ich gehe mal davon aus, dass wir zusammenarbeiten können, und es gibt grade in dem Bereich genug Themen, die wir abdecken können, entweder zu zweit oder einzeln, wenn es eine von uns nicht interessiert.}}
    }
}

\wahl{Kandidatur für das Querreferat}{2. Lesung}{Nel und Mira Schwarzer}
{
    Der Kandidaturtext findet sich auf der \kandidaturenseite.
}
{
    \textbf{1. Lesung:}
    \ul{
        \li{Wir würdet ihr euch die zusammenarbeit mit anderen (autonomen) Referaten wünschen}
        \ul{\lii{Gerne weitervernetzen, manche Themen betreffen auch andere, offen für zusammenarbeit}}
    }
}

\wahl{Senatskommission für die Verleihung der Bezeichnung apl. Prof.}{1. Lesung}{Huilin Guo und Tomke Arand}
{
    Der Kandidaturtext findet sich auf der \kandidaturenseite.
}
{
    \textbf{1. Lesung:}
}

\subsection{Zusammenfassung}
\begin{center}
    \begin{tabular}{|p{6cm}|m{2cm}|m{1cm}|m{1cm}|m{1cm}|}
        \hline
        Kandidatur & Gewählt & Ja & Nein & Enth\\\hline
        Felix Diener & ausstehend & tba & tba & tba\\\hline
        Janek Kasperowski & ausstehend & tba & tba & tba\\\hline
        Jovana Perovic & ausstehend & tba & tba & tba\\\hline
        Mithily Masilamany & ausstehend & tba & tba & tba\\\hline
        Chiara Citro & ausstehend & tba & tba & tba\\\hline
        Nel & ausstehend & tba & tba & tba\\\hline
        Mira Schwarzer & ausstehend & tba & tba & tba\\\hline
    \end{tabular}
\end{center}