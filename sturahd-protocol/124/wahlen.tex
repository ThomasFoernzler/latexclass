\section{Kandidaturen und Wahlen}

\wahl{Senatskommission für die Verleihung der Bezeichnung apl. Prof.}{2. Lesung}{Huilin Guo und Tomke Arand}
{
    Der Kandidaturtext findet sich auf der \kandidaturenseite.
}
{
    \textbf{1. Lesung:}
    \ul{
	\li{In der Senatskommission ist eine Stelle für apl. Profs vorgesehen. Wie gneua würden sie das verteilen.}
		\noli{\ul{
		\lii{Nach Nachfrage war eine Teamkandidatur erlaubt. Tomke würde ansonsten stellvertretendes Mitglied werden.
		}}}
	\li{Wie genau würden sie sich einbringen wollen.}
		\noli{\ul{
		\lii{Sie würden sich mit der Lehre auseinandersetzen und die Studierenden am besten vertreten.
		}}}
		\noli{\ul{
		\lii{Sie wollen darauf schauen ob diese Person eine gute Lehre macht, aber durch die Arbeit selber zeigt sich auf was es ankommt.
		}}}
    }
}
\wahl{Kandidatur für das Referat für Finanzen}{1. Lesung:}{Felix Mehra}
{
    Der Kandidaturtext findet sich auf der \kandidaturenseite.
}{
    \textbf{1. Lesung:}
}
\wahl{Kandidatur für das Referat für Soziales}{1. Lesung:}{Nadja Hartmann}
{
    Der Kandidaturtext findet sich auf der \kandidaturenseite.
}{
    \textbf{1. Lesung:}
}

\subsection{Zusammenfassung}
\begin{center}
    \begin{tabular}{|p{6cm}|m{2cm}|m{1cm}|m{1cm}|m{1cm}|}
        \hline
        Kandidatur & Gewählt & Ja & Nein & Enth\\\hline
        Huilin Guo & ausstehend & tba & tba & tba \\\hline
        Tomke Arand & ausstehend & tba & tba & tba \\\hline
        Felix Mehra & ausstehend & tba & tba & tba \\\hline
        Nadja Hartmann & ausstehend & tba & tba & tba \\\hline
    \end{tabular}
\end{center}