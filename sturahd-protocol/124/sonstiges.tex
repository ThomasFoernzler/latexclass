\section{Sonstiges}
\antrag{Beschluss der Beitragshöhe der Mitgliedschaft des StuRa auf Ebene der Fachschaft Medizin Heidelberg in der Bundesvertretung der Medizinstudierenden in Deutschland e.V.}{1. Lesung}{}{
    Der StuRa beschließt die Genehmigung der Änderung der Beitragshöhe der Mitgliedschaft
 der Fachschaft Medizin Heidelberg in der Bundesvertretung der Medizinstudierenden in
 Deutschland e.V. durch die Vollversammlung der Fachschaft Medizin Heidelberg.
}{
    Der Studierendenrat ist auf Ebene der Fachschaft Medizin Heidelberg seit 02.04.2014 Mitglied in der Bundesvertretung der Medizinstudierenden in Deutschland e.V. (bvmd). Für diese Mitgliedschaft wird eine jährliche Beitragszahlung aus den der Fachschaft Medizin Heidelberg zugewiesenen Mitteln fällig. Die zentralen Gelder der Verfassten Studierendenschaft bleiben von dieser Ausgabe unberührt. Die Höhe dieser Beitragszahlung liegt nach § 4.1.2 Abs. 7 der Satzung der bvmd im Ermessen der Lokalvertretungen (in diesem Fall der Fachschaft Medizin Heidelberg). Es sollen 3\% des jährlich den Fachschaften zur Verfügung stehenden Finanzvolumen hierfür aufgewendet werden, die Mitgliedschaft in der bvmd ist jedoch in keinem Fall mit einer finanziellen Verpflichtung verbunden. In den letzten Jahren zahlte der StuRa aus den der Fachschaft Medizin Heidelberg zugewiesenen Geldern einen Beitrag in Höhe von 350 €, was unter dem von der bvmd angestrebten Satz von 3\% (für 2020 wären das bspw. 489,47 €) liegt. Die Vollversammlung der Fachschaft Medizin Heidelberg hat in ihrer Sitzung am 15.10.2020 beschlossen diesen Mitgliedsbeitrag einmalig im Jahr 2020 auf 2000 € zu erhöhen. Für das Finanzjahr 2021 sind wieder die üblichen 350 € veranschlagt. Hintergrund dieser einmaligen Erhöhung des Mitgliedsbeitrags ist die finanzielle Lage der bvmd, die jedoch nicht auf Misswirtschaft, sondern auf Folgen der Corona-Pandemie beruht. Die Arbeit der bvmd wird vor allem durch die Austauschprogramme und durch das Erheben von Teilnahmebeiträgen an Präsenzveranstaltungen finanziert. Auch war ein Teil der eingeworbenen Sponsorengelder an Veranstaltungen gebunden. Durch die Einschränkungen der globalen Pandemie konnten somit eine Finanzierung durch diese Quellen nicht gewährleistet werden. Da der Fachschaft Medizin Heidelberg im Finanzjahr 2020 noch einiger Spielraum durch nicht abgerufene Gelder blieb, wurde dieser auf Vorschlag des Fachschaftsrats und durch Beschluss der Vollversammlung genutzt, um den Mitgliedsbeitrag einmalig zu erhöhen. Das Finanzreferat hat nach Einreichung des Abrechnungsformular darauf hingewiesen, dass dieser Beschluss durch den Studierendenrat bestätigt werden muss.\\
    Die bvmd bildet den Zusammenschluss aller 39 medizinischen Fachschaften in Deutschland und repräsentiert somit mehr als 90.000 Studierende. Sie vertritt die Interessen der Medizinstudierenden in der Bundesrepublik Deutschland uns ist deren Stimme vor nationalen Gremien der Hochschul- und Gesundheitspolitik. Auf internationaler Ebene arbeitet die bvmd im europäischen Verband der European Medical Students Associaton (EMSA) und im weltweiten Dachverband International Federation of Medical Students’ Associations (IFMSA). Die Arbeit ist ehrenamtlich und der Verein ist als gemeinnützig anerkannt. Die wichtigste Säule der Verbandsarbeit neben der Mitgliederversammlung sind die ständigen Arbeitsgruppen und weitere Projekte, die zu verschiedenen Themen in Form von Aufklärungs- und Informationskampagnen, Workshops, Umfragen und Veröffentlichungen arbeiten. Der Austausch empfängt und entsendet im Rahmen des Netzwerkes jährlich über 800 Medizinstudierende aus und in alle Welt. Die ständigen Arbeitsgruppen untergliedern sich in folgende Bereiche: AG Europäische Integration, koordiniert die Zusammenarbeit mit EMSA, AG Famulaturaustausch (Standing Committee on Professional Exchange - SCOPE), AG Forschungsaustausch (Standing Committee on Research Exchange - SCORE), AG Gesundheitspolitik (Standing Committee on Health Policy - SCOHP), AG Medizin und Menschenrechte (Standing Committee on Human Rights and Peace - SCORP), AG Medizinische Ausbildung (Standing Committee on Medical Education - SCOME), AG Public Health (Standing Committee on Public Health - SCOPH) und AG Sexualität und Prävention. Darüber hinaus unterstützt die bvmd die Arbeit in lokalen Projekten wie MSV, Aufklärung Organspende, FirstAidForAll, Teddybärenkrankenhaus, UniHilft, Viola, Aufklärung gegen Tabak, SegMed und vielen weiteren massiv.
}{
    \textbf{1. Lesung:}
}{tba}{tba}{tba}
\antrag{Überwindung des Einspruchs des Finanzreferats zur Finanzentscheidung der Fachschaft Medizin Heidelberg betreffend Antrag 2020.621.31}{1. Lesung}{}{
    Der StuRa beschließt die Überwindung des Einspruchs des Finanzreferats zur
 Finanzentscheidung der Fachschaft Medizin Heidelberg zu Antrag 2020.621.31
}{
    Der Arbeitskreis Public Relations der Fachschaft Medizin Heidelberg hat im Dezember 2020 beim Finanzreferat ein Abrechnungsformular zur Beschaffung von Mund-Nasen-Schützen als Dozierendengeschenke eingereicht. Dieser Antrag wurde vom Finanzreferat mit der Begründung abgelehnt Fachschaftsgelder seien nicht dazu da Dozierende zu finanzieren und es zwar Geld für Dankesgeschenke gäbe, die Grenze jedoch weit unter der beantragten läge. Im Namen der Fachschaft Medizin Heidelberg möchte ich als Finanzverantwortlicher nun den StuRa bitten diesen Einspruch zu überwinden und uns die Ausgabe nachträglich zu genehmigen. Im Folgenden möchte ich darlegen, wieso uns diese Ausgabe so wichtig ist.\\
    Zum Ende des Jahres bedankt sich die Fachschaft Medizin im Namen aller Studierenden bei den verschiedenen Dozierenden, Sekretär*innen der Lehrsekretariate, Mitarbeiter*innen der Hausverwaltung/des Sicherheitsdienstes/Poststelle/UniShop, ... für die Zusammenarbeit im letzten Jahr. Ohne das Engagement dieser Personen wäre die Fachschaftsarbeit gar nicht erst möglich gewesen. Gerade in diesem doch etwas anderem Jahr waren wir verstärkt auf Unterstützung durch die Dozierenden und Verwaltungsmitarbeitenden angewiesen, um nicht nur Lehre in anderen Formaten durchführen zu können, sondern auch um Fachschaftsarbeit weiter am Laufen halten zu können. Beispielsweise konnte nur durch das großzügige Bereitstellen von Räumlichkeiten in Theoretikum, Klinikum und weiteren Gebäuden der Universität eine Einführungswoche für die neuen Erstsemester, zumindest in gewissem Umfang, in Präsenz stattfinden. Hiervon profitierten vor allem die neuen Studierenden, da so das soziale Miteinander und Kennenlernen, vor allem mit Ausblick auf ein digitales Wintersemester, deutlich erleichtert werden konnte. Auch von Seiten der Lehrenden und des Studiendekanats haben wir im vergangenen Semester deutliches Entgegenkommen erfahren dürfen, was allen Studierenden in diesem Umfang zugutekommt, dass primär eine Verzögerung des Studienablaufs weitestgehend verhindert werden konnte. Dies wurde bspw. durch eine Flexibilität in Anwesenheitspflicht und das vermehrte Bereitstellen von digitalen Lerninhalten ermöglicht. Bei den Geschenken soll es explizit nicht darum gehen Chefärzt*innen o.ä. noch ein Weihnachtschmankerl zu geben, sondern denen Personen zu danken, die das System Studium am Laufen halten, sowie die Schnittstellen von Lehrkörper, Verwaltung und Studierendenschaft darstellen. Durch diese Geschenke lenken wir den Fokus der Dozierenden und Beschenkten auch wieder auf die Fachschaft und deren Rolle als Studierendenvertretung und die Belange der Studierenden. Beispielsweise konnte so im letzten Jahr im Bereich der Medizinischen Psychologie/Soziologie ausgehend von der Übergabe des Dozierendengeschenkes ein Evaluationsgespräch erreicht werden, dem eine komplette Umstellung des Kurses folgte, von der alle Studierenden profitierten. Ähnlich lief es bei der Medizinischen Terminologie, die durch die Dozierendengeschenke auf die Fachschaft zuging, um den Kurs zu evaluieren und hier beispielsweise im Anschluss an ein Gespräch die Klausur überarbeitete und somit eine stärkere Trennschärfe erreicht werden konnte. Auch in diesem Jahr haben wir bisher wieder durchwegs positive Rückmeldungen zu den Dozierendengeschenken bekommen und hoffen auch wieder hier diese Konversationen als Startpunkte für weitergehende Gespräche nutzen zu können, um Bereich in denen wir Entwicklungspotenzial sehen weiter verbessern zu können. Auch von diesen Maßnahmen profitieren alle Studierenden. Darüber hinaus wurde uns von einigen Beschenkten schon rückgemeldet, die Maske zu benutzen, was auch visuell für Außenstehende die Verbindung zwischen Lehre/Verwaltung und Fachschaft deutlich macht.\\
    In diesem Jahr wurden der aktuellen Thematik angepasst Gesichtsmasken mit Fachschaftslogo beschafft. Wir erhoffen uns, dass diese durch die Beschenkten durch deren Schlichtheit auch benutzt werden und so die Verknüpfung zwischen Lehrenden/Fachschaft/Verwaltung zum Ausdruck gebracht wird. Bei der Beschaffung wurde darauf geachtet qualitativ hochwertige Masken zu beschaffen. Die beschafften Masken stellen in ihrer Qualitätsklasse die günstigsten dar. Entsprechende Vergleichsangebote wurden mit dem Abrechnungsformular eingereicht. Die Gesichtsmasken kosten uns 4,13 € das Stück, dazu erhält jede*r Beschenkte einen Schokololli für 0,18€, sowie eine persönlich beschriftete Dankeskarte a 0,07€ und dieses Jahr zur Vermeidung persönlicher Kontakte einen Briefumschlag für 0,18€. Insgesamt erhält jede*r Beschenkte somit ein Geschenk im Gegenwert von 4,56€. Im Vergleich hierzu wurden 2019 4,78€ (hochwertigere Schokonikoläuse, und nicht Lollis), im Jahr 2018 4,74€ pro Geschenk ausgegeben. Die Anträge in beiden Jahren wurden ohne Einspruch des Finanzreferats genehmigt. Deshalb sehen wir hier nicht die Gefahr einer Individualförderung oder die Finanzierung von Dozierenden, sondern sehen vor allem den großen Benefit der besseren Verbindung von Studierendenschaft und Lehrkörper sowie Universitätsverwaltung.\\
    Der Fachschaftsrat hat die Ausgabe aus den der Fachschaft Medizin Heidelberg zugewiesenen Mitteln am 20.11.2020 bewilligt. Die Bewilligung der ausgegebenen 619,38€ erfolgte auf Basis, der im Budgetplan 2020 veranschlagten, 1000€. Weder dem Finanzverantwortlichen noch dem Fachschaftsrat war das Überschreiten einer Grenze für Dankesgeschenke bewusst. Leider wurde versäumt dies mit dem Finanzreferat im Vorhinein abzuklären.\\
}{
    \textbf{1. Lesung:}
}{tba}{tba}{tba}