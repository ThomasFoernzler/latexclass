\section{Satzungen und Ordnungen}

\antrag{Satzung der Studienfachschaft Philosophie}{3.Lesung}{Fachschaft Philosophie}
    {\begin{tabular}{|p{7.5cm}|p{7.5cm}|}
        \hline
        \textbf{Bisheriger Text:} & \textbf{Neuer Text:} \\\hline
        \S 3 \newline (4) Er umfasst mindestens zwei Mitglieder. Sollten mehr als zwei Kandidat*innen
        aufgestellt werden, so gilt, dass die Anzahl der zu besetzenden Sitze der Zahl der
        Kandidat*innen entspricht, aber maximal vier beträgt.
    &   \S 3 \newline (4) Er umfasst bis zu vier, aber mindestens zwei Mitglieder. \newline 
        (5) Gewählt sind diejenigen Kandidierenden, die die meisten Stimmen erhalten, wobei
        jede*r Wahlberechtigte bis zu vier Stimmen, aber höchstens so viele Stimmen wie es
        Kandidierende gibt, hat. Bei vier oder weniger als vier Kandidierenden, kann für oder
        gegen jede*n Kandidierende*n gestimmt werden und gewählt sind diejenigen, die mehr
        Ja- als Nein-Stimmen erhalten. Im Übrigen gilt die Wahlordnung der
        Studierendenschaft.\\\hline
        
        \end{tabular}
    }{%
        Begründung siehe Anhang
    }{%
        Diskussion
    }{tba}{tba}{tba}
\antrag{Satzung der Studienfachschaft Pharmazie}{3.Lesung}{Fachschaft Pharmazie}
    {
        Antragstext siehe Anhang
    }{
        Begründung siehe Anhang
    }{
        Diskussion
    }{tba}{tba}{tba}
    
\antrag{Satzung der Studienfachschaft Japanologie}{3.Lesung}{Fachschaft Japanologie}
    {
        Antragstext siehe Anhang
    }{
        Begründung siehe Anhang
    }{
        Diskussion
    }{tba}{tba}{tba}
