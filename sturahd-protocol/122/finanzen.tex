\newcommand{\finanzantrag}[5]{%1title2antragssteller3antragstext4begründung5diskussion
    \subsection{#1}
    Antragssteller: #2
    \paragraph{Antragstext:}\phantom{spacer}\\
    #3
    \paragraph{Antragsbeschreibung:}\phantom{spacer}\\
    #4
    \paragraph{Diskussion:}\phantom{spacer}\\
    #5
}
\section{Finanzanträge}

\finanzantrag{Unterstützung der Campus Debatte Heidelberg}{Rederei e.V.}
{
    Der StuRa unterstützt Die Rederei Heidelberg e.V. bei der Ausrichtung der Campus Debatte Heidelberg vom 26.03 bis 28.03 2021.
}{
    \textbf{Infos zum Antragssteller:}\\
    Die Rederei e.V. ist ein in Heidelberg ansässiger Debattierclub. Seit 2001 vermitteln wir argumentative und rhetorische Fähigkeiten an Studierende aller Fachrichtungen. Wir glauben, dass Debattenkultur allen Menschen helfen kann, einen sachlichen und ergebnisoffenen Diskurs zu führen. Unsere Veranstaltungen sind nicht auf Studierende begrenzt. Auch andere junge Erwachsene wie SchülerInnen, Azubis oder Berufstätige sind bei uns willkommen. Wir treffen uns zwei Mal wöchentlich (aktuell über Zoom) für Debatten- und Trainingsabende und bieten auch darüber hinaus Seminare und Trainingseinheiten zur Verbesserung debattierrelevanter Fähigkeiten an.\\[1em]
    \textbf{Projektbeschreibung:}\\
    Die Campus Debatte Heidelberg ist Teil der Campus-Debatten-Turnierserie.  Diese besteht aus vier jährlich stattfinden Turnieren, welche nach der deutschsprachigen Debattiermeisterschaft die größten deutschsprachigen Debattierturniere sind. Hier messen sich die besten Debattierenden aus Deutschland, Österreich und der Schweiz im argumentativen Wettstreit über verschiedenste Themen. Von Politik und internationale Beziehungen über gesellschaftliche Fragen bis hin zu philosophischen Dilemmata ist das Themenfeld sehr weit. Über drei Tage finden fünf Vorrunden sowie die Halbfinals und das Finale statt.  Das Finale der Campus Debatte wird öffentlichkeitswirksam beworben und steht allen Interessierten offen. Die Veranstaltung ist kostenfrei und soll einen Einblick darin geben, wie ein geordneter, respektvoller und argumentativ hochwertiger Diskurs aussehen kann.\\
    Als Kooperationspartner haben wir den Dachverband VDCH, Verband der Debattierclubs an Hochschulen, an unserer Seite. Hierüber erhalten wir einen großen Teil der nötigen Fördergelder. Wir haben außerdem bereits die SRH als Partner gewinnen können, welche uns ihre Räumlichkeiten für die Vorrunden des Turniers zur Verfügung stellt. Im Gegenzug werden wir dort einzelne Trainings abhalten, um Studierende der SRH mit dem Debattieren vertraut zu machen.\\[1em]
    \textbf{Wer kann teilnehmen:}\\
    Teilnehmen kann, wer einem der über 70 Debattierclubs (davon zwei in Heidelberg und einer in Mannheim) angehört, die Mitglieder des Verbands der Debattierclubs an Hochschulen sind. Diese sind in Deutschland, Österreich und der Schweiz ansässig.\\
    Insgesamt werden etwa 100 Studierende an der Campus Debatte teilnehmen, etwa weitere 150 nicht debattierende Interessierte erwarten wir nach bisherigen Erfahrungen zum öffentlichen Finale. Wir werden selbst neben der Organisation hoffentlich noch einigen eigenen Teams aus jeweils drei Studierenden der Universität Heidelberg die Chance geben können, an dem Turnier teilzunehmen.\\
    Außerdem werden Teams des anderen Heidelberger Debattierklubs sowie des Mannheimer Debattierclubs antreten.\\[1em]
    \textbf{Erwartete Ergebnisse:}\\
    Wie auch in den letzten Jahren ist es unser Ziel, die besten Debattierenden Deutschlands zu finden und zu küren. Im Vordergrund steht uns aber auch, in Heidelberg durch die öffentliche Finalveranstaltung ein Bewusstsein für Debattenkultur zu schaffen. Gerne würden wir auch über die Region hinaus Debattierclubs stärken, indem wir auf diese aufmerksam machen. Wir glauben, dass eine Debattenkultur, in der das beste Argument Gehör findet, einen wichtigen Gegenpol zum aktuellen politischen Klima darstellt. Durch eine weitere Verbreitung des Debattiersports glauben wir, dass junge Erwachsene im Privaten wie im Beruflichen respektvoller miteinander zu diskutieren lernen.\\[1em]
    \textbf{Antragsbegründung:}\\
    Die vier Turniere der Campus Debatten-Serie sind nach der deutschen Debattiermeisterschaft die wichtigsten Veranstaltungen der studentischen Debattierszene im deutschsprachigen Raum. Sie bieten neben dem kompetitiven Turnier einen Ort zur Vernetzung und zum Treffen wichtiger Entscheidungen für das kommende Jahr.\\
    Neben einem kompetitiven Charakter hat ein solches Turnier aber auch einen höchst integrativen Charakter, da die Teilnahme nicht an irgendwelche Qualifikationen gebunden ist, sondern jedem Mitglied eines der vielen Debattierclubs offensteht. Um die Teilnahme nun tatsächlich allen zu ermöglichen, sind wir auf zahlreiche Sponsoren angewiesen. Diese ermöglichen es Jahr für Jahr, den Teilnahmebeitrag in einem angemessenen Rahmen zu halten. Aktuell planen wir hier mit 30€ pro Person.\\[1em]
    \emph{Zur Begründung der Unterstützung nicht-Heidelberger Studis:}\\
    Wir denken, diese Unterstützung beruht auf einem Geben-und-Nehmen-Prinzip. So war es auch in den letzten Jahren der Fall, dass die StuRas, StuPas, Astas (und was es sonst noch so gibt) der Ausrichteruniversitäten diese Turniere stets unterstützt haben. Von dieser Unterstützung profitieren jährlich viele Heidelberger Studierenden, sodass wir glauben, dass es legitim ist, dass in diesem Jahr die Verfasste Studierendenschaft Heidelbergs die Studierenden anderen Universitäten bei ihrem Aufenthalt in Heidelberg unterstützt.\\[1em]
    \textbf{Finanzvolumen:}\\
    Wir beantragen beim StuRa Unterstützung in Höhe von 2000€.\\[1em]
    Ansonsten beantragen wir keine Mittel bei der verfassten Studierendenschaft. Der Teilnahmebeitrag wird am Ende so gewählt werden, dass er die restlichen noch nicht gedeckten Kosten abdeckt. Zurzeit planen wir mit ungefähr 30 €. Es laufen aktuell noch Anfragen bei der Leonie-Wild-Stiftung, der Manfred-Lautenschläger-Stiftung und der Stadt-Heidelberg-Stiftung sowie beim Lions Club Heidelberg – etwa 30 weitere Anfragen wurden inzwischen leider abgelehnt. Sollten wir weitere Unterstützer finden können, werden wir am Ende auch weniger Geld vom StuRa nutzen. Dies war bereits vor 2 Jahren bei der von uns ausgerichteten deutschsprachigen Debattiermeisterschaft der Fall, als wir vom StuRa 5000€ bewilligt bekamen und am Ende nur etwa 1500€ in Anspruch genommen haben.\\[1em]
    \begin{tabular}{l l}
        Wieviel beantragt ihr beim Studierendenrat?                             & 2000€                     \\
        Wieviel wird bei der Verfassten Studierendenschaft insgesamt beantragt? & Keine weiteren Anträge    \\
        Wieviel wird über Mittel weiterer Stellen finanziert?                   & Bisher 7176,39€           \\
        Habt ihr Einnahmen bei der Veranstaltung?                               & Nein                      \\
        Wie hoch ist das Gesamtvolumen des Projekts                             & Aktuelle Planung: 12.128€ \\
    \end{tabular}
    \newline
    \vspace*{2em}
    \newline
    \begin{tabular}{c c p{10cm}}
        Verpflegung & 2550€ & \\
        Unterkunft & 6900€ & Hier würden wir die StuRa-Unterstützung anrechnen.\\
        Finale & 430€ & Ein öffentliches Finale (inklusive der Ehrenjury, für welche Reisekosten unter dem Punkt Transport anfallen) ist eine der Bedingungen der Hauptförderer unseres Dachverbandes (dies sind die Zeit-Stiftung Ebelin und Gerd Bucerius sowie die Karl-Schlecht-Stiftung). \\
        Transport & 700€ & \\
        Socials & 400€ & Werden coronabedingt ggf ausfallen und sind noch nicht geplant. Das Geld ist für eventuelle Raummieten eingeplant. Kosten vor Ort (z.B. Getränke) müssten die Teilnehmenden hier selbst zahlen. \\
        Sonstiges & 648€ & Hierunter fallen Druckkosten, eine Veranstaltungsversicherung und ein Sicherheitspuffer. Sollte der Sicherheitspuffer nicht gebraucht werden, werden wir entsprechend weniger Förderung in Anspruch nehmen. \\ 
        \textbf{Gesamt} & \textbf{12.128€} & \\       
    \end{tabular}

}{
    Diskussion
}
