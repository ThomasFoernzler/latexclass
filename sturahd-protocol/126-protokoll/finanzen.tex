\newcommand{\finanzantrag}[4]{%1title2antragstext3begründung4diskussion
    \subsection{#1}
    \paragraph{Antragstext:}\phantom{spacer}\\
    #2
    \paragraph{Antragsbeschreibung:}\phantom{spacer}\\
    #3
    \paragraph{Diskussion:}\phantom{spacer}\\
    #4
}
\section{Finanzanträge}
\finanzantrag{Globaler Klimastreik organisiert vom Ökoreferat und FFF Heidelberg \label{finanz:1}(2. Lesung)}{
    Der StuRa untersützt den globalen Klimastreik, der im Sommersemester 2021 von Fridays
    for Future Heidelberg mit Unterstützung des Referats für Ökologie und Nachhaltigkeit
    organisiert wird.\\
    Das Referat für Ökologie und Nachhaltigkeit ("Ökoreferat") plant mit der
    Arbeitsgruppe "Students For Future" (AG SFF) der vom StuRa unterstützten Gruppe
    "Fridays For Future Heidelberg" (FFF HD) einen globalen Klimastreik im Sommersemester
    2021 durchzuführen.\\
    Bei einem globalen Klimastreik werden weltweit Großdemonstrationen veranstaltet, um
    auf die Bedrohung der Klimakrise hinzuweisen und angemessene Maßnahmen von der
    Politik zu fordern. In Heidelberg werden je nach Corona-Lage 3000-5000 Menschen
    erwartet. Aus diesem Grund benötigt die Demonstration gut strukturierte Organisation,
    Bewerbung, Logistik und Technik.\\
    Die Kosten für die Demonstration werden mit 7000€ veranschlagt, wobei sich dieser
    Preis in der Planung noch konkretisieren wird. Bei den vergangenen
    Großdemonstrationen wurden zwischen 2500€ und 3500€ Spenden gesammelt, womit bei
    dieser Demonstration wieder gerechnet wird.\\
    Um die verbleibenden Kosten zu decken werden deshalb 3500€ beschlossen.
}{
    \textbf{Projektbeschreibung und Antragsbegründung:}\\
    \begin{itemize}
        \item \textbf{Was ist euer Projekt?} Fridays for Future wird eine Großdemonstration für Klimagerechtigkeit in Heidelberg organisieren. Dabei wird es eine große Kundgebung mit wertvollem Programm und einen Demozug geben.+
        \item \textbf{An wen richtet sich euer Vorhaben?} Die Demonstration richtet sich an alle Menschen aus Heidelberg und Umgebung. Insbesondere Studierende sind erfahrungsgemäß stark vertreten unter den Teilnehmenden. Außerdem profitieren die Studierenden im Orgateam von Fridays for Future, da sie sehr viel praktische Dinge lernen, wie man Demos organisiert und politisch aktiv wird.
        \item \textbf{ Warum sollte euch die Verfasste Studierendenschaft finanziell unterstützen?} Fridays for Future Heidelberg besteht zum großen Teil aus Studierenden, die durch FFF politisch sehr aktiv geworden sind. Auf Demos von FFF kommen viele Studierende, die dadurch die Möglichkeit bekommen, ihr demokratisches Demonstrationsrecht wahrzunehmen und auf den Kundgebungen durch Redebeiträge viel zu lernen. Außerdem organisiert Fridays for Future im Rahmen der AG Students for Future auch im universitären Kontext viele Projekte wie die Public Climate School, die 2019 viel besucht wurde, arbeitet mit den Scientists for Future zusammen, hat Forderungen an die Uni formuliert und ist weiterhin im Kontakt, um diese umzusetzen. Das alles wäre nicht möglich ohne eine entsprechende Legitimation und Bekanntheit durch große Proteste auf der Straße.
        \item \textbf{ Gibt es bereits ähnliche Projekte?} Es gab bereits einige Großdemonstrationen von Fridays for Future in den letzten 2 Jahren, von denen 2 vom organisatorischen und technischen Aufwand so intensiv sein werden wie die geplante Großdemonstration im Sommersemester 2021.
    \end{itemize}
    \textbf{Finanzvolumen des Antrags:}\\
    \begin{tabular}{l l}
        Wieviel beantragt ihr beim Studierendenrat?                             & 3500€ \\
        Wieviel wird bei der Verfassten Studierendenschaft insgesamt beantragt? & 3500€ \\
        Wieviel wird über Mittel weiterer Stellen finanziert?                   & Nicht solange wir mit dem Geld auskommen\\
        Habt ihr Einnahmen bei der Veranstaltung?                               & Spenden: ca 2000€ - 4000€ \\
        Wie hoch ist das Gesamtvolumen des Projekts                             & ca. 7000€\\
    \end{tabular}
    \newline
    \textbf{Verwendungszweck:}\\
    \begin{longtable}{p{3cm} p{1cm} p{11cm}}
        \endfirsthead
        \endhead
        \endfoot
        \endlastfoot
        \textbf{Verwendungszweck} & \textbf{Kosten} & \textbf{Begründung} \\
        3 Großflächenbanner & 500€ & Genehmigung und Druck von Großflächenbannern zur großflächigen Bewerbung\\
        Plakate & 1000€ & Druck von Plakaten zum selbst aufhängen und Druck von Plakaten für offizielle Plakatwände in Heidelberg und Aufhängen lassen durch Firma\\
        Flyer & 200€ & Umweltfreundliche Flyer zur Bewerbung\\
        Aufwandsentschädigung und Fahrtkosten für Musikacts und Redner*innen & 700€ & Je nach Anreise und Zeitaufwand zahlen wir Musiker*innen zwischen 50€ und 250€ und Redner*innen die Fahrtkosten (meistens gering, da aus der Region)\\
        Bühne und Technik & 4200€ & Um auf der Neckarwiese mehrere Tausend Menschen zu beschallen, die zusätzlich noch Corona-konformen Abstand halten, benötigen wir eine professionelle Bühne mit Soundtechnik.\\
        Awareness-Kits & 50€ & Wir stellen für Teilnehmende Wasser, Ohrstöpsel, Taschentücher, Traubenzucker etc zur Verfügung, damit Menschen bei dringendem Bedarf darauf zurückgreifen können.\\
        Gebärdendolmetschung & 300€ & Damit gehörlose Menschen die Redebeiträge verstehen können, benötigen wir Gebärdendolmetscher*innen\\
        Material zur Durchführung der Demo & 50€ & Kreide, Flatterband etc.\\
        \textbf{Gesamt} & \textbf{7000€} & - \\  
    \end{longtable}    \emph{Weitere Informationen:}\\
    Die Kosten sind eine Abschätzung für die Kosten und werden in der Planung noch konkreter werden. Wir beantragen einen Anteil von 3500€, wobei diser Betrag gekürzt werden kann, wenn es nicht anders geht. Wenn das Geld nicht reicht, können wir einen Antrag auf Bundesmittel von Fridays for Future stellen, wobei nicht sicher ist ob und in welcher Höhe diese Mittel zu dem Zeitpunkt dann verfügbar sein werden, wir nur einen kleinen Teil dadurch decken könnten und deshalb darauf verzichten wollen.
}{
    \textbf{1. Lesung:}
    \ul{
	\li{Ist es ratsam das während Corona in Präsenz zu machen? könnte es stattdessen auch Online stattfinden?}
		\noli{\ul{
		\lii{Grundsätzlich ist es immer wichtig auf den Klimawandel hinzuweißen. Onlinestreiks erreichen nicht sehr viele Menschen. Es wird ein umfassendes Klimakonzept geben. Aber wenn es unklug ist, dann wird er auch nicht in Präsenz stattfinden.
		}}}
	\li{Könnte der Antrag auch gekürzt werden oder kann die Verantaltung dann nicht mehr stattfinden?}
		\noli{\ul{
		\lii{Es wird auf der Demo dafür Spenden gesammelt. Das Geld ist so kalkuliert, dass es reicht, aber die Kosten sind deswegen höher kalkuliert. Nicht benötigtes Geld würde nicht beansprucht werden.
		}}}
    }
    \textbf{2. Lesung:}
    \ul{
        \li{Keine Fragen}
    }
}
\abstimmungsergebnis{
    \fullref{finanz:1}
}{
    22
}{
    2
}{
    3
}{
    \ul{\noli{\ul{\lii{Angenommen}}}}
}