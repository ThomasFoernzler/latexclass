

1 Begrüßung
}

2 Tagesordnung

2.1 Änderungen an der TO
\textbf{X. Lesung:}
\ul{
}

3 Protokolle
	\li{vertagt auf nächste Woche}
}

4 Berichte

Vorsitz
\textbf{X. Lesung:}
\ul{
	\li{Für welche Studienänge gelten die Gebühren}
		\noli{\ul{
		\lii{Das wird sich noch zeigen
		}}}
		\noli{\ul{
		\lii{Latinum/Graecum ist Studienvoraussetzung
		}}}
	\li{FS Sport hat viele Theologie Studierende. Für Theologie ist Graecum notwendig. }
		\noli{\ul{
		\lii{Man kann Studiienfächer wechseln und so ist es noch nicht ganz klar wie gezahlt werden muss. 
		}}}
	\li{Studienfach kurzfristig wechseln geht und so könnte man das abfedern.}
		\noli{\ul{
		\lii{Es stellt ein Problem dar dass man in manchen Studienfächern diese Gebühren zahlt und in manchen nicht.
		}}}
}

Bericht Referat für hochschulpolitische Vernetzung
\textbf{X. Lesung:}
\ul{
	\li{}
		\noli{\ul{
		\lii{Im Frühling ist sehr viel Zeit für generelle Debatten. Hier wird die inhaltliche Ausrichtung und interne Strukturierung gesetzt.
		}}}
}

 Bericht über das Treffen mit der Geschäftsfürung des Studierendenwerks am 01. Dezember
\textbf{X. Lesung:}
\ul{
	\li{Keine Fragen}
}

Bericht zur Corona Sondersitzung
\textbf{X. Lesung:}
\ul{
	\li{Keine Fragen}
}

Bericht Referat für politische Bildung
\textbf{X. Lesung:}
\ul{
	\li{Das gibt es auch an anderen Unis (Mannheim z.B.). Landtagswahlen sind die Wahlen die über Hochschulpolitik entscheiden. Die Organisation kann sehr gut neutral sein. Weil die Hochschulgruppen nicht komplett auf der Linie der Hauptparteien sind, ist es problematisch diese zu vertreten.}
	\li{Bei der LHG gibt es keine offiziellen Verbindung z. B.. Deswegen ist es nicht möglich die FDP für sie z. B. zu repräsentieren. }
	\li{Manche Parteien haben keine Hochschulgruppen. Das könnte diese ausschließen.}
		\noli{\ul{
		\lii{Dieser Punkt bereitet Bauchschmerzen. Aber Beschlüsse des Sturas in puncto Antirassismus/Antisexismus stehen ziemlich direkt gegen die HSG-lose Partei AfD.  
		}}}
		\noli{\ul{
		\lii{Es geht nicht darum die Parteien zu vertreten, sondern darum über die Parteien Auskunft zu geben. 
		}}}
	\li{Kann man die AfD einladen und rausschmeißen wenn sie rassistisch sind}
		\noli{\ul{
		\lii{Der Rassismus der AfD ist unterschwelliger
		}}}
	\li{Kann man die AfD nicht einfach ausschließen?}
		\noli{\ul{
		\lii{Die AfD will die VS abschaffen und viele der Äußerungen der AfD sind rassistisch. Beides würde einen Ausschluss rechtfertigen. 
		}}}
		\noli{\ul{
		\lii{
		}}}
}

GO Diskussion zu TO machen
\textbf{X. Lesung:}
\ul{
	\li{Das müsste man in der nächsten Sitzung wegen inhaltlicher Tiefe machen.}
	\li{Die Frage die sich gestellt hat für das Referat für PoBi, ob man das mit den Hochschulgruppen repräsentativ macht hat isch geklärt.}
}

 Bericht zur Neugestaltung des Campus Neuenheimer Feld im Rahmen des Masterplanverfahrens
\textbf{X. Lesung:}
\ul{
}

Bericht des Antirassismusreferats
\textbf{X. Lesung:}
\ul{
	\li{Im neuen LHG ist ein:e Antidiskriminierungsbeauftragt:e vorgesehen. Hat das Referat schon davon gehört?}
		\noli{\ul{
		\lii{Nein
		}}}
}

5 Satzungen

5.1 Neufassung der Studienfachschaftssatzung UFG/VA
\textbf{X. Lesung:}
\ul{
	\li{In die nächste Sitzung verschoben}
}

5.2 Fusion der Fachschaften Klassische Archäologie und Byzantinische Archäologie und Kunstgeschichte
\textbf{X. Lesung:}
\ul{
	\li{In die nächste Sitzung verschoben}
}

5.3 Satzung der neuen Fachschaft Klassische und Byzantinische Archäologie 
\textbf{X. Lesung:}
\ul{
	\li{In die nächste Sitzung verschoben}
}

5.4 Antrag zur Festschreibung von Digitalen Wahlen in der regulären Wahlzeit
\textbf{X. Lesung:}
\ul{
	\li{Sind die Online Wahlen damit auch nach der Pandemie als Ergänzung.}
		\noli{\ul{
		\lii{Ja, wegen höherer Wahlbeteiligung und technologischer Sicherheit.
		}}}
	\li{Wie sieht das mit der Sicherheit aus, dass die Uni ID von jemand anderem}
		\noli{\ul{
		\lii{Es gibt eine Abwägung zwischen Sicherheit und Wahlbeteiligung. Und das kommt sehr sehr selten vor.
		}}}
		\noli{\ul{
		\lii{Mehr Technik macht das nicht zu mehr als einer digitalen Briefwahl. Letzten Endes ist es eine Abwägung zwischen Wahlbeteiligung und Sicherheit. Auch ist es weniger relevant bei dieser Wahl zu bescheißen als bei einer Europawahl etc.
		}}}
	\li{Auch das Bild auf dem Studienausweiß ist ohnehin oft schwer zu erkennen.}
	\li{ Wer legt fest ob die Onlinewahl möglich ist und wer zahlt das dann?}
		\noli{\ul{
		\lii{Die Wahlkomission soll festlegen, ob online gewählt werden sollte. Auch wird erwartet, dass eine Mehrheit die Online Wahlen nutzen wird.
		}}}
	\li{Bei Online Wahlen wie geht das mit dem Haushaltsplan zusammen, der erstellt werden muss.}
		\noli{\ul{
		\lii{Es soll ein Änderungsantrag auf nur online Wahlen erarbeitet werden.
		}}}
	\li{"Analog und digital zusammen wählen ist irre. Das ist organisatorisch wahnsinnig aufwändig und teuer"}
	\li{Wie kann man die Sicherheit der Software gewerkstelligen}
		\noli{\ul{
		\lii{Die Wahlkomission sollte das machen.
		}}}
	\li{Entweder kann man selbst diese Sachen festschreiben oder man kann sich an schon jetzige Vorschreibungen halten. Auch ist es schwer digital und online gleichzeitig zu wählen, wegen der online Formatierung. Man kann aber in Notwahllokale gehen in denen man sicherer digital wählen kann.}
	\li{Diese Notwahllokale gewährleisten nur eine sichere Stimmabgabe für die Leute dort. Man will aber das für jeden gewährleisten. Auch kann man so Stimmen kaufen.}
		\noli{\ul{
		\lii{Das ist bei einer Briefwahl nicht anders. Online-Wahlen sind nicht hundertprozentig sicher, aber andere Wahlmethoden auch nicht. 
		}}}
	\li{"Aber euer Wohnheim ist nicht vor der Neuen Uni / Zentralmensa :D"}
	\li{Online Wahl heißt nicht unbedingt online Wahlkampf. Deswegen hat das nicht wirklich Einfluss auf den Wahlkampf.}
		\noli{\ul{
		\lii{Wahlkmapf ist kein Nullsummenspiel. Onlinewahlkampf schränkt Präsenzwahlkampf nicht ein.
		}}}
}

Go Antrag auf Vorziehung von 10.1 und 10.2 auf 5.6 und 5.7
\textbf{X. Lesung:}
\ul{
	\li{Die Begründung, dass man morgen früh aufstehen muss ist nicht hinreichend weil es nicht nur den Mediziner:innen so geht.}
	\li{Die Antragsteller:innen sind nicht Stura-Mitglieder und sind nur für diese Anträge hier.}
	\li{angenommen mit 27/11/6 }
}

5.6.Beschluss der Beitragshöhe der Mitgliedschaft des StuRa auf Ebene derFachschaft Medizin Heidelberg in der Bundesvertretung der Medizinstudierenden in Deutschland e.V. (2. Lesung)
\textbf{X. Lesung:}
\ul{
	\li{Die Gelder kommen nicht von der Allgemeinheit sondern aus der Fachschaft Medizin. }
}

5.7 Überwindung des Einspruchs des Finanzreferats zur Finanzentscheidung der Fachschaft Medizin Heidelberg betreffend Antrag 2020.621.31 (2. Lesung)
\textbf{X. Lesung:}
\ul{
	\li{Wurde bei der Zahlung Rücksprache gehalten, ob das generell okay wäre oder ob es nicht okay ist.}
	\li{Man muss schauen, ob das gut für die Außenwirkung ist und ob man diese Gelder dafür aufwenden wollen.}
	\li{Wenn das gängige Praxis war hat der Rechnungshof das mal kommentiert.}
		\noli{\ul{
		\lii{Damit hatte der Rechnungshof bis jetzt kein Problem.
		}}}
}

5.8 Diskussion zu Satzungsänderungen
\textbf{X. Lesung:}
\ul{
	\li{Man sollte die Satzungen überarbeiten, weil die Satzungen teils sich widersprechen, manche Sachen sind nicht geregelt und manche Sachen sind übermäßig kompliziert. }
}

6 Wahlen

6.1.  Kandidatur für das Referat für Finanzen (2. Lesung:)
\textbf{X. Lesung:}
\ul{
	\li{Keine Fragen}
	\li{gewählt mit 39/1/3}
}

6.2.  Kandidatur für das Referat für Soziales (2. Lesung:)
\textbf{X. Lesung:}
\ul{
	\li{Keine Fragen}
	\li{gewählt mit 40/1/0}
}

6.3.  Kandidatur für das Referat für hochschulpolitische Vernetzung (1. Lesung:)
\textbf{X. Lesung:}
\ul{
	\li{Gab es Kommunikation mit Marc, welcher auch dafür kandidiert.}
		\noli{\ul{
		\lii{Ja das wurde gemacht über die Arbeit in den letzten Jahren und wie die Arbeit aufgeteilt werden soll.
		}}}
}

6.4.  Kandidatur für das Referat für hochschulpolitische Vernetzung (1. Lesung:)
\textbf{X. Lesung:}
\ul{
	\li{Keine Fragen}
}

6.5.  Kandidatur für die Härtefallkommission (1. Lesung:)
\textbf{X. Lesung:}
\ul{
	\li{Keine Fragen}
}

6.6.  Kandidatur für das Referat für internationale Studierende (1. Lesung:)
\textbf{X. Lesung:}
\ul{
	\li{Keine Fragen}
}

6.7.  Kandidatur für das Finanzreferat (1. Lesung:)
\textbf{X. Lesung:}
\ul{
	\li{Keine Fragen}
}

7 Diskussionen

7.1. heiConf für alle Studierenden öffnen (zurückgezogen)
\textbf{X. Lesung:}
\ul{
}

7.2. Öffentliche Beratung über den Stand der geplanten Fuß- und Radbrücke über den Neckar
\textbf{X. Lesung:}
\ul{
	\li{Was ist mit dem zweiten Vorschlag?}
		\noli{\ul{
		\lii{wurde zurückgezogen
		}}}
	\li{An wen wurde der Preis vergeben?}
		\noli{\ul{
		\lii{Der Preis war Teil eines Vorauswahlverfahren. Insgesamt gab es 3 Verfahren. 
		}}}
	\li{Barrierefreiheit?}
		\noli{\ul{
		\lii{Bei dritter Brücke konnte sich Rad und Gehverkehr vermischen. dann wurden zwei Wege dafür geschaffen. Barrierefreiheit scheint bei beiden gut zu sein.
		}}}
}

8 Corona-Vollversammlung Beschlüsse

8.1 Lernräume und -orte (1. Lesung:)
\textbf{X. Lesung:}
\ul{
	\li{Keine Fragen}
	\li{angenommen mit 27/0/1}
}

8.1.1.  Änderungsantrag zu 8.1 : Lernräume und -orte (1. Lesung)
\textbf{X. Lesung:}
\ul{
	\li{Im Antrag steht hinzufügen, jetzt hieß es ersetzen. Was ist gemeint?}
		\noli{\ul{
		\lii{ersetzen ist gemeint. Das war ein redaktioneller Fehler in den Sitzungsunterlagen.
		}}}
	\li{angenommen mit //}
}

8.2.  Bibliotheken (1. Lesung)
\textbf{X. Lesung:}
\ul{
	\li{Keine Fragen}
	\li{angenommen mit 29/0/1}
}

8.2.1.  Änderungsantrag zu 8.2 : Bibliotheken (1. Lesung)
\textbf{X. Lesung:}
\ul{
	\li{Keine Fragen}
	\li{Durch Antragssteller:in angenommen}
}

8.3.  Corona-Freischuss! (1. Lesung)
\textbf{X. Lesung:}
\ul{
	\li{Ersetzen die Änderungsanträge den Originalantrag? Das ist nicht ganz klar wie sich die Anträge aufeinander beziehen. Bei dem Antrag für internationale Studierende ist der Punkt, dass internationale Studierende mehr Zeit haben problematisch. Weil es ja schon am Anfang nachgewiesen worden sollte, dass sie der Sprache mächtig ist.}
		\noli{\ul{
		\lii{Man könnte es so regeln, dass bei C1, C2 wie gut die Studierenden in der Sprache sind. Die Anträge sind durcheinander wegen Zeitdruck
		}}}
	\li{Abstufung ist nicht möglich. Auch ist es kaum umzusetzen im Hinblick auf Prüfungsordnungen.}
		\noli{\ul{
		\lii{Medizin/ Jura sollen eigenständig sein und der Rest ein Antrag.
		}}}
	\li{angenommen mit 23/2/1}
}

8.3.1.  Änderungsantrag zu Antrag 8.3
\textbf{X. Lesung:}
\ul{
	\li{angenommen mit 28/1/2}
}

8.3.2.  Englisch/ internationale Studierenden/ mehr Zeit für Prüfungen
\textbf{X. Lesung:}
\ul{
	\li{angenommen mit 16/3/6}
}

8.3.3.  Mehr Zeit für Jura
\textbf{X. Lesung:}
\ul{
	\li{Durch "https://www.justiz-bw.de/site/pbs-bw-rebrush-jum/get/documents_E1441090213/jum1/JuM/Justizministerium%20NEU/Pr%C3%BCfungsamt/Hinweise%20zur%20Staatspr%C3%BCfung%20in%20der%20Ersten%20juristischen%20Pr%C3%BCfung/Hinweis%20Homepage%20Coronavirus%202019-01-21.pdf " ist der Antrag obsolet. }
	\li{angenommen mit 20/1/4}
}

8.3.4.  Freischuss für Medizin
\textbf{X. Lesung:}
\ul{
}

8.4.  Klausurenphase (1. Lesung)
\textbf{X. Lesung:}
\ul{
	\li{Keine Fragen}
	\li{angenommen mit 22/0/1}
}

8.4.1.  Änderungsantrag zu Antrag 8.4
\textbf{X. Lesung:}
\ul{
	\li{Keine Fragen}
	\li{angenommen mit 26/0/1}
}

8.5.  Online-Sprechstunden (1. Lesung)
\textbf{X. Lesung:}
\ul{
	\li{Keine Fragen}
}

8.6.  Wlan (1. Lesung)
\textbf{X. Lesung:}
\ul{
	\li{Geschwindigkeit ist schwierig weil Down- und Uploadgeschwindigkeit verschieden sind.}
		\noli{\ul{
		\lii{Stimmt deswegen Änderungsantrag für 10 Mbit/s Uploadgeschwindigkeit. 
		}}}
}

8.6.1.  Änderungsantrag zu Antrag 8.6
\textbf{X. Lesung:}
\ul{
	\li{Keine Fragen}
}

8.7.  Qualität der digitalen Lehre (1. Lesung)
\textbf{X. Lesung:}
\ul{
	\li{Keine Fragen}
}

8.7.1.  Änderungsantrag zu Antrag 8.7
\textbf{X. Lesung:}
\ul{
	\li{Keine Fragen}
}

8.8.  Mensa-Essen (1. Lesung)
\textbf{X. Lesung:}
\ul{
	\li{Keine Fragen}
}

8.8.1.  Änderungsantrag zu Antrag 8.8
\textbf{X. Lesung:}
\ul{
	\li{Keine Fragen}
}

8.9.  Studierende mit Kind (1. Lesung)
\textbf{X. Lesung:}
\ul{
	\li{Keine Fragen}
	\li{Wenn es Studierende mit Kind gibt, die ja davon betroffen sind, dann kann man sich gerne an Henrike wenden, damit deren Interessen vertreten sind.}
	\li{angenommen mit 25/0/1}
}

8.10.  Corona und Soziales (1. Lesung)
\textbf{X. Lesung:}
\ul{
	\li{Welche Notlagenfonds sind gemeint? Das sollte man präzisieren.}
		\noli{\ul{
		\lii{Es geht um die universitären und nicht die der VS. Das kann abgeändert werden.
		}}}
}

GO  Dringlichkeit für 8.1
\textbf{X. Lesung:}
\ul{
	\li{angenommen mit 32/2/0}
}

GO Dringlichkeit für 8.2
\textbf{X. Lesung:}
\ul{
	\li{angenommen mit 32/2/0}
}

GO  Dringlichkeit für 8.3, 8.3.2, 8.3.3
\textbf{X. Lesung:}
\ul{
	\li{angenommen mit 27/8/4}
}

GO Dringlichkeit für 8.4
\textbf{X. Lesung:}
\ul{
	\li{angenommen mit 33/4/1}
}

GO Dringlichkeit für 8.9
\textbf{X. Lesung:}
\ul{
	\li{angenommen mit 31/4/3}
}

9.  Finanzanträge

 9.1 Globaler Klimastreik organisiert vom Ökoreferat und FFF Heidelberg (2.Lesung)
\textbf{X. Lesung:}
\ul{
	\li{angenommen mit 22/2/3}
}

10. Sonstiges

10.3.   Wahl des stud. Senators für den Academic Council von 4EU+  (1. Lesung)
\textbf{X. Lesung:}
\ul{
	\li{Keine Frage}
}