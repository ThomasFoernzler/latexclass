\section{Kandidaturen und Wahlen}
Die Kandidaturtexte sind aus Datenschutzgründen nur auf der Kandidaturenseite \url{https://www.stura.uni-heidelberg.de/kandidaturen} einzusehen, welche nur aus dem Universitätsnetzwerk oder mit dem VPN der Universität besucht werden kann.

\wahl{Kandidatur für das Referat für Finanzen}{2. Lesung:}{Felix Mehra}
{
    Der Kandidaturtext findet sich auf der \kandidaturenseite.
}{
    \textbf{1. Lesung:}
	\ul{
	\li{Ist Felix Mitglied in Parteien oder ähnlichen Organisationen?}
		\noli{\ul{
		\lii{Nein
		}}}
	\li{hat Felix Erfahrungen mit Fachschaftsarbeit und -strukturen?}
		\noli{\ul{
		\lii{Jein, In BWL nicht, aber in seinem Medizinstudium schon aber nicht allzu aktiv.
		}}}
	\li{Kann Felix sich vorstellen, das Referat mit jemandem jetzt in dem Referat zu machen}
		\noli{\ul{
		\lii{Ja
		}}}
	\li{Wäre er bereit sich auf die Suche nach einem weiblichen Mitglied des Finanzreferats zu machen?}
		\noli{\ul{
		\lii{Das hat er schon aber in seinem Umfeld gibt es keine die Zeit hat.
		}}}
	}
	\textbf{2. Lesung:}
	\ul{
	\li{Keine Fragen}
	}
}

\wahl{Kandidatur für das Referat für Soziales}{2. Lesung:}{Nadja Hartmann}
{
    Der Kandidaturtext findet sich auf der \kandidaturenseite.
}{
    \textbf{1. Lesung:}
	\ul{\li{Keine Fragen}}
	\textbf{2. Lesung:}
    \ul{\li{Keine Fragen}}
}

\wahl{Kandidatur für das Referat für hochschulpolitische Vernetzung}{1. Lesung:}{Annalena Wirth}
{
    Der Kandidaturtext findet sich auf der \kandidaturenseite.
}{
	\textbf{1. Lesung:}
	\ul{
	\li{Gab es Kommunikation mit Marc, welcher auch dafür kandidiert.}
		\noli{\ul{
		\lii{Ja das wurde gemacht über die Arbeit in den letzten Jahren und wie die Arbeit aufgeteilt werden soll.
		}}}
	}
}

\wahl{Kandidatur für das Referat für hochschulpolitische Vernetzung}{1. Lesung:}{Marc Baltrun}
{
    Der Kandidaturtext findet sich auf der \kandidaturenseite.
}{
	\textbf{1. Lesung:}
	\ul{\li{Keine Fragen}}
}

\wahl{Kandidatur für die Härtefallkommission}{1. Lesung:}{Simon Kleinhanß}
{
    Der Kandidaturtext findet sich auf der \kandidaturenseite.
}{
	\textbf{1. Lesung:}
	\ul{\li{Keine Fragen}}
}

\wahl{Kandidatur für das Referat für internationale Studierende}{1. Lesung:}{Lucas Kelm}
{
    Der Kandidaturtext findet sich auf der \kandidaturenseite.
}{
	\textbf{1. Lesung:}
	\ul{\li{Keine Fragen}}
}

\wahl{Kandidatur für das Finanzreferat}{1. Lesung:}{Florian Weiss}
{
    Der Kandidaturtext findet sich auf der \kandidaturenseite.
}{
	\textbf{1. Lesung:}
	\ul{\li{Keine Fragen}}
}

\subsection{Zusammenfassung}
\begin{center}
    \begin{tabular}{|p{6cm}|m{2cm}|m{1cm}|m{1cm}|m{1cm}|}
        \hline
        Kandidatur & Gewählt & Ja & Nein & Enth\\\hline
        Felix Mehra & ausstehend & 39 & 1 & 3 \\\hline
		Nadja Hartmann & ausstehend & 40 & 1 & 0 \\\hline
		Annalena Wirth & ausstehend & tba & tba & tba \\\hline
		Marc Baltrun & ausstehend & tba & tba & tba \\\hline
		Simon Kleinhanß & ausstehend & tba & tba & tba \\\hline
		Lucas Kelm & ausstehend & tba & tba & tba \\\hline
		Florian Weiss & ausstehend & tba & tba & tba \\\hline
    \end{tabular}
\end{center}