\section{Satzungen}
\subsection{Neufassung der Satzung der Studienfachschaft UFG/VA}\label{appendix:1}
    \myparagraph{Präambel}
    In dem Bestreben, der Fachschaftsarbeit an der Ruprecht-Karls Universität Heidelberg
    eine dauerhafte und bestimmte Grundlage zu geben, haben sich die Studierenden der
    Fächer Geoarchäologie, Ur- und Frühgeschichte sowie Vorderasiatische Archäologie als
    Fachschaft Ur- und Frühgeschichte und Vorderasiatische Archäologie (UFG/VA) folgende
    Satzung gegeben.\newline Die Fachschaft steht für ein Studium ein, in dem sich alle Studierenden individuell
    entfalten und das eigene Recht auf Selbstbestimmung – im Rahmen der Gesetze –
    ausleben kann. In unserem Einsatz für ein solches Studium sehen wir uns als politisch
    neutral und respektieren die Religionsfreiheit unserer Studierenden. Wir fühlen uns
    in unserem Engagement – im Rahmen der Gesetze – ausschließlich durch den freien
    Willen und die unverletzliche Würde des Menschen bestärkt und verpflichtet. Damit
    sich dieser Gedanke in seiner Lebendigkeit entfalten und unermüdlich, aufrichtig und
    frei innerhalb von Universität und Studierendenschaft wirken kann, geben wir uns
    folgende Satzung und nehmen im Rahmen der Erfüllung unserer Aufgaben nach § 65 LHG
    unser – begrenztes – politisches Mandat wahr. Zudem ist die Fachschaft darum bemüht,
    für ein besseres Miteinander von Studierenden und Institut und einen besseren
    Zusammenhalt der Studierenden zu sorgen. Begründung: Dies ist von der VS als
    Kernaufgabe der Fachschaften vorgegeben und hatte in der bisherigen Arbeit unserer
    Fachschaft auch eine wichtige Bedeutung.
    \\
    \paragraph{§1 Allgemeines}
    \begin{enumerate}
        \item[(1)] {Die Studienfachschaft (im Folgenden „Fachschaft”) vertritt die Studierenden des Fachbereichs „Ur-und Frühgeschichte und Vorderasiatische Archäologie“ sowie „Geoarchäologie” und entscheidet insbesondere über fachspezifische Fragen und Anträge.}
        \item[(2)] {Die Zugehörigkeit zur Studienfachschaft ergibt sich aus der Liste in Anhang B.}
        \item[(3)] {Die Studienfachschaft stellt die studentischen Mitglieder der in ihrem Bereich        arbeitenden Gremien oder beteiligt sich zumindest an einem gemeinsamen Wahlvorschlag        für ebendiese.}
        \item[(4)] {Organe der Studienfachschaft sind die Fachschaftsvollversammlung und der        Fachschaftsrat.}
    \end{enumerate}
    \paragraph{§2 Fachschaftsvollversammlung}
    \begin{enumerate}
        \item[(1)] {Die Fachschaftsvollversammlung ist die Versammlung der Mitglieder der        Fachschaft.  Sie tagt öffentlich, soweit gesetzliche Bestimmungen nicht        entgegenstehen.}
        \item[(2)] {Rede-, antrags- und stimmberechtigt sind alle anwesenden Mitglieder der        Fachschaft.}
        \item[(3)] {Beschlüsse werden mit einfacher Mehrheit gefasst.}
        \item[(4)] {Die gefassten Beschlüsse sind bindend für den Fachschaftsrat.}
        \item[(5)] {Die Fachschaftsvollversammlung bestimmt im Einvernehmen des Fachschaftsrats bis        zu zwei Finanzverantwortliche der Fachschaft. Die Finanzverantwortlichen müssen        eingeschriebene Studierende sein. Die Amtszeit beträgt in der Regel ein Jahr.}
        \item[(6)] {Zum Ende der Amtszeit der Finanzverantwortlichen prüft der Fachschaftsrat deren        Arbeit und beantragt anschließend die Entlastung der Finanzverantwortlichen in der        Fachschaftsvollversammlung. Diese beschließt die Entlastung der        Finanzverantwortlichen mit einfacher Mehrheit.}
        \item[(7)] {Die Fachschaftsvollversammlung kann Abstimmungsempfehlungen für das StuRa-        Mitglied beschließen. Diese sind nicht bindend.}
        \item[(8)] {Die Fachschaftsvollversammlung bestimmt jeden November aus ihrer Mitte bis zu        drei Personen, welche die Anträge für die Qualitätssicherungsnachfolgemittel (QSM)        der Fachschaft vorbereiten (QSM-Kommission der Fachschaft). Näheres regelt § 5 dieser        Satzung.}
        \item[(9)] {Fachschaftsvollversammlungen müssen unverzüglich vom Fachschaftsrat einberufen        werden:
            \begin{enumerate}
                \item[9a]auf Antrag eines Drittels der Mitglieder des Fachschaftsrates oder
                \item[9b]auf schriftlichen Antrag von 1\% der Mitglieder der Fachschaft.
            \end{enumerate}
        }
        \item[(10)]{Die Einberufung einer Fachschaftsvollversammlung muss mindestens fünf Tage        vorher öffentlich und in geeigneter Weise bekannt gemacht werden.}
        \item[(11)]{Eine Fachschaftsvollversammlung ist beschlussfähig, wenn sie ordnungsgemäß        einberufen wurde, mindestens die Hälfte der Fachschaftsräte und insgesamt mindestens        2 Mitglieder der Fachschaft anwesend sind.} 
    \end{enumerate}
    \paragraph{§3 Fachschaftsrat}
    \begin{enumerate}
        \item[(1)] { Der Fachschaftsrat wird in gleicher, direkter, freier und geheimer Wahl gewählt.        Es findet Personenwahl statt.}
        \item[(2)] {Alle Mitglieder der Studienfachschaft haben das aktive und passive Wahlrecht. }
        \item[(3)] {Der Fachschaftsrat umfasst mindestens zwei und maximal acht Mitglieder. }
        \item[(4)] {Der Fachschaftsrat nimmt die Interessen der Mitglieder der Fachschaft wahr. }
        \item[(5)] {Zu den Aufgaben des Fachschaftsrats gehören:
            \begin{enumerate}
                \item[5a]Einberufung und Leitung der Fachschaftsvollversammlung.
                \item[5b]Ausführung der Beschlüsse der Fachschaftsvollversammlung.
                \item[5c]Führung de Finanzen sowie Prüfung der Arbeit der Finanzverantwortlichen sowie Beantragung der Entlastung dieser
                \item[5d]Beratung und Information der Studienfachschaftsmitglieder.
                \item[5e]Mitwirkung an der Lehrplangestaltung.
                \item[5f]Austausch und Zusammenarbeit mit den Mitgliedern des Lehrkörpers des Fachbereichs Ur- und Frühgeschichte und Vorderasiatische Archäologie.
                \item[5g]Unterstützung der QSM-Kommission der Fachschaft bei ihrer Arbeit.
            \end{enumerate}
        }
        \item[(6)] {Die Amtszeit der Mitglieder des Fachschaftsrats beträgt ein Jahr. Die Amtszeit        beginnt zum 01. April eines jeden Jahres.}
        \item[(7)] {Für das vorzeitige Ausscheiden aus dem Fachschaftsrat gilt die        Organisationssatzung des StuRa.}
        \item[(8)] {Im Falle des Ausscheidens eines Mitglieds des Fachschaftsrats rückt die Person        mit der nachfolgenden Stimmenzahl für die verbleibende Amtszeit des ausscheidenden        Mitglieds in den Fachschaftsrat nach.}
    \end{enumerate}
    \paragraph{§4 Kooperation und Stimmführung im Studierendenrat}
    \begin{enumerate}
        \item[(1)] {Der Fachschaftsrat entsendet ein Mitglied der Fachschaft in den Studierendenrat        (StuRa).}
        \item[(2)] {Der Fachschaftsrat entsendet zudem Stellvertreter*innen in den StuRa.}
        \item[(3)] {Die Amtszeit der Entsandten im StuRa beträgt ein Jahr.}
        \item[(4)] {Für das vorzeitige Ausscheiden aus dem Studierendenrat gilt die        Organisationssatzung des StuRa.}
        \item[(5)] {Das StuRa-Mitglied und dessen Stellvertreter*innen können per Beschluss mit 2/3-        Mehrheit in der Fachschaftsvollversammlung abberufen werden.}
        \item[(6)] {Das StuRa-Mitglied und dessen Stellvertreter*innen stimmen nach bestem Wissen und        Gewissen im Studierendenrat ab.}
        \item[(7)] {Das StuRa-Mitglied und dessen Stellverterer*innen orientieren sich an den        Abstimmungsempfehlungen der Fachschaftsvollversammlung.  }
        \item[(8)] {Die Fachschaft kann sich nach § 14 der Organisationssatzung der        Studierendenschaft mit anderen Fachschaften zu einer Kooperation zusammenschließen.}  
    \end{enumerate}
    \paragraph{§5 Qualitätssicherungsnachfolgemittel}
    \begin{enumerate}
        \item[(1)] {Die Fachschaftsvollversammlung bestimmt jeden November aus ihrer Mitte bis zu        drei Personen, welche die Anträge für die QSM vorbereiten. Diese bilden die        QSM-Kommission der Fachschaft.}
        \item[(2)] {Nach Bildung der QSM-Kommission wird das QSM-Referat über dessen Mitglieder        informiert.}
        \item[(3)] {Vorschläge für die Verwendung der QSM müssen bis spätestens zwei Wochen vor        Antragsfrist bei der QSM-Kommission der Fachschaft eingereicht werden.}
        \item[(4)] {Bei der Vergabe sind die Mittel auf UFG und VA getrennt, der Anzahl der        Studierenden entsprechend, zu veranschlagen. Die Mittel der Geoarchäologie werden        denen der UFG zugerechnet.}
        \item[(5)] {Per Beschluss der QSM-Kommission der Fachschaft können die Mittel auch gemeinsam        veranschlagt werden. Sollte die Kommission nur aus einer Person, oder nur Personen        einer der Fächer bestehen, so muss dieser Beschluss vom Fachschaftsrat getroffen        werden.}
        \item[(6)] {Aufgaben der QSM-Kommission der Fachschaft sind:
            \begin{enumerate}
                \item[6a]Die vorzeitige Information über den zur Verfügung stehenden Betrag für die QSM;
                \item[6b]Die Vorbereitung der Anträge für die QSM in Rücksprache mit der Fachschaft;
                \item[6c]Die Fristgerechte Einreichung der QSM-Anträge.  
            \end{enumerate}
        }
    \end{enumerate}
    Die Änderung dieser Satzung tritt zum 01. Januar 2021 in Kraft.
\subsection{Fusion der Fachschaften Klassische Archäologie und Byzantinische Archäologie und Kunstgeschichte}\label{appendix:2}
    \myparagraph{Anhang B: Liste der Studienfachschaften (Studienfachschaftslistenanhang)}
    Die  Ziffern  und  Namen  in  den  Klammern  hinter  dem  jeweiligen  Studienfachschafts-namen bezeichnen die zugeordneten Studiengänge nach der Studierendenstatistik der Zentralen Universitätsverwaltung.
    \begin{enumerate}[noitemsep]
        \item Ägyptologie (1, 15, 886) (Ägyptologie, Papyrologie) 
        \item Alte Geschichte (272, 2722, 2725, 2724) (Alte Geschichte) 
        \item American Studies (838) (American Studies) 
        \item Anglistik (8, 835, 8357, 8352, 8355, 8354, 836, 837, 83, 97, 9222, 9232, 9242) (Englische Philologie, English Studies/Anglistik)
        \item Assyriologie (821, 8217, 8215, 8214, 9147) (Assyriologie) 
        \item Byzantinische Archäologie und Kunstgeschichte (830, 8302, 8305, 8304) (Byzantinische Archäologie und Kunstgeschichte) 
        \item Biologie (26, 933, 881, 843) (Biologie, Biowissenschaften, Molecular Biosciences) 
        \item Chemie - Biochemie (32, 25) (Chemie, Biochemie)
        \item Computerlinguistik (160, 1607, 1602, 1605, 1604, 927) (Computerlinguistik, ) 
        \item Deutsch als Fremdsprache (826, 8267, 827, 8272, 828, 8282, 901, 9017, 9012, 9015, 9014, 939, 940, 950) (Deutsch als Fremdsprachenphilologie, Deutsch als Zweitsprache, Germanistik im Kulturvergleich) 
        \item Erziehung und Bildung (52, 868, 890, 920, 9202, 9205, 9204, 190) (Berufs- und Organisationsbezogene Beratungswissenschaft, Bildungswissenschaft, Pädagogik/Erziehungswissenschaft,) 
        \item Ethnologie (173, 1737, 1732, 1734) (Ethnologie)13.  Geographie (50, 502, 505, 504, 892, 9112, 9115) (Geographie, Governance of Risk and Resources) 
        \item Geowissenschaften (39, 65, 111) (Geowissenschaften) 
        \item Germanistik (67, 672, 675, 674, 929) (Germanistik, Editionswissenschaften und Textkritik) 
        \item Gerontologie \& Care (863, 864, 867, 9676) (Gerontologie, Gesundheit und Care, Gesundheit und Gesellschaft[Care], Gerontologie) 
        \item Geschichte (68, 687, 682, 685, 684, 273, 2735, 2734, 840, 842, 8422, 918, 935) (Mittlere und Neue Geschichte, Osteuropäische Geschichte, Deutsch-Französischer Master in Geschichtswissenschaften, Global History, Historische Grundwissenschaften) 
        \item Informatik (79, 879, 889) (angewandte Informatik, Informatik) 
        \item Islamwissenschaft (81, 883, 884, 8857, 8852, 8854, 930) (Iranistik, Islamic Studies/Islamwissenschaft, Nah- und Mitteloststudien) 
        \item Japanologie (85, 853, 8537, 8532, 8534) (Japanologie, Ostasienwissenschaften Schwerpunkt Japanologie) 
        \item Jura (135, 873, 874, 8732, 932) (International Law [LL.M.], öffentliches Recht, Rechtswissenschaft [inkl. Legum Magister], Unternehmensstrukturierung [LL.M.]) 
        \item Klassische Archäologie (831, 8317, 8312, 8315, 8314, 8347, 12N, 849) (Klassische Archäologie) 
        \item Klassische Philologie (70, 95, 912, 9122, 9125, 9124, 913, 9132, 9135, 9134, 951) (Klassische Philologie: Gräzistik, Klassische Philologie: Latinistik, Klassische und Moderne Literaturwissenschaft) 
        \item Kunstgeschichte (Europäische) (92, 927, 922, 924, 915) (Europäische Kunstgeschichte [inkl. BA int. Verlaufsvariante], Kunstgeschichte und Museologie) 
        \item Mathematik (105, 875, 934) (Mathematik, Scientific Computing) 
        \item Medizin Heidelberg (247, 804, 806, 869, 871, 876, 878, 887, 949, 893, 895) (Advanced Physical Methods ind Radiotherapy, Clinical Medical Physics, International Health, Interprofessionelle Gesundheitsversorgung, Kinder- und Jugendpsychatrie, Medical Biometry/Biostatistics, Medical Education, Humanmedizin, Medizinische Informatik, Scientarum Humanarum, Versorgungsforschung und Implentierungswissenschaft im Gesundheitswesen,) 
        \item Medizin Mannheim (805, 877, 938, 945, 946) (Biomedical Engineering, Health Economics, Medical Physics with distinction in Radiotherapy and Biomedical optics, Humanmedizin, Translational Medical Research) 
        \item Mittellatein/Mittelalterstudien (818, 917) (Lateinische Philologie des Mittelalters und der Neuzeit, Mittelalterstudien) 
        \item Molekulare Biotechnologie (802, 803) (Molekulare Biotechnologie) 
        \item Musikwissenschaft (114, 1147, 1142, 1145, 1144) (Musikwissenschaft) 
        \item Ostasiatische Kunstgeschichte (850, 8502, 853, 8537, 8532, 8534) (Kunstgeschichte Ostasiens, Ostasienwissenschaften Schwerpunkt Kunstgeschichte) 
        \item Pharmazie (126) (Pharmazie) 
        \item Philosophie (127, 1277, 1272, 1275, 1274, 9217) (Philosophie) 
        \item Physik (14, 128, 888) (Astronomie und Astrophysik, Physik, technische Informatik) 
        \item Politikwissenschaft (129, 1297,1292, 1295, 1294, 882, 931, 829) (Politikwissenschaft, Politikwissenschaften/Wirtschaftswissenschaften, Non-Profit Management und Governance) 
        \item Psychologie (132, 1322) (Psychologie) 
        \item Religionswissenschaft (136, 1367, 1362, 1364) (Religionswissenschaft) 
        \item Romanistik (59, 84, 137, 150, 855, 856, 896, 897, 899, 904, 9047, 9042, 9045, 9044, 905, 9057, 9052, 9055, 9054, 906, 9067, 9062, 9065, 9064, 9072, 9075, 9074, 9082, 9084, 9092, 9095, 9094, 9102, 948, 9482) (Romanische Philologie, Romanistik: Französisch, Transkulturelle Studien. Literaturen und Sprachkontakte im frankophonen Raum, Romanistik: Italienisch, Italien im Kontakt – Literatur, Künste, Sprachen, Kulturen, Romanistik: Portugiesisch, Romanistik: Spanisch, Iberoamerikanische Studien. Kontakt – Theorien und Methoden) 
        \item Semitistik (820, 8202, 8205, 8204) (Semitistik)
        \item Sinologie (145, 1452, 858, 860, 861, 916, 853, 8537, 8532, 8534) (Klassische Sinologie, Moderne Sinologie, Sinologie [Chinese Studies], Ostasienwissenschaften Schwerpunkt Sinologie) 
        \item Slavistik/Osteuropastudien (139, 146, 964, 1467, 1462, 1465, 1464, 865, 8652, 8654, 866, 8665, 8664) (Slavistik, Slavische und Osteuropäische Studien) und (8447, 8442, 8445, 8444) (Osteuropa- Ostmitteleuropastudien) 
        \item Soziologie (149, 1492) (Soziologie) 
        \item Sport (29, 295, 872, 898, 9377, 947) (Sportwissenschaft, Sportwissenschaft mit Schwerpunkt Prävention und Rehabilitation)  
        \item Südasienwissenschaften (Fachschaft am SAI) (841, 8412, 8415, 8414, 845, 846, 852, 8527, 8522, 8524, 902, 9022, 9025, 9024, 903, 9032, 9035, 9034, 926, 851, 969) (Kommunikation, Literatur und Medien in Südasiatischen Neusprachen, Neuere Sprachen und Literaturen Südasiens [Moderne Indologie], Kultur und Religionsgeschichte Südasiens [Klassische Indologie], Health and Society in South Asia, Politikwissenschaft Südasiens) 
        \item Theologie (Evangelische) (53, 161, 848, 859, 862, 925, 928, 73, 9252, 9255, 9254, 900, 854) (Christentum und Kultur, Diakoniewissenschaft, Diakonie- Führungsverantwortung in christlich-sozialer Praxis, Doctor of Philosophy PhD, Evangelische Theologie [alle Examen], Magister Theologiae, Management, Ethik und Innovation im Non-Profit-Bereich, Unternehmensführung im Wohlfahrtsbereich) 
        \item Transcultural Studies (891) (Transcultural Studies) 
        \item Ur- und Frühgeschichte/Vorderasiatische Archäologie (UFG/VA) (548, 5482, 5485, 5484, 832, 8327, 8322, 8325, 8324, 9197, 894) (Ur- und Frühgeschichte, Vorderasiatische Archäologie, Geoarchäologie) 
        \item Übersetzen und Dolmetschen (Fachschaft am IÜD) (810, 811, 812, 813, 814, 815, 817, 822, 823) (Konferenzdolmetschen [alle Sprachen], Translation Studies for Information Technologies, Übersetzungswissenschaft [alle Sprachen]  49.  Volkswirtschaftslehre (VWL) (175, 184, 880, 8802, 936) (Economics (Politische Ökonomik), Economics, Volkswirtschaftslehre,) 
        \item Zahnmedizin (185) (Zahnmedizin)
    \end{enumerate}
    \myparagraph{Anhang D: Abweichende Regelungen für Studienfachschaften (ARS)}
    Studienfachschaften können beim Studierendenrat nach dem Studien-fachschaftskonstitutionsanhang (Anhang A) vom Studienfachschaftsregelmodell (Anhang C) abweichende Regelungen beantragen. Diese werden hier aufgeführt: 
    \begin{enumerate}[noitemsep]
        \item Ägyptologie                                                                                                                                                                                                  \\
        \item Alte Geschichte                                                                                                                                                                                              \\
        \item American Studies                                                                                                                                                                                             \\
        \item Anglistik                                                                                                                                                                                                    \\
        \item Assyriologie                                                                                                                                                                                                 \\
        \item Biologie                                                                                                                                                                                                     \\
        \item Chemie und Biochemie                                                                                                                                                                                         \\
        \item Computerlinguistik                                                                                                                                                                                           \\
        \item Deutsch als Fremdsprache                                                                                                                                                                                     \\
        \item Erziehung und Bildung                                                                                                                                                                                       \\
        \item Ethnologie                                                                                                                                                                                                  \\
        \item Geographie                                                                                                                                                                                                  \\
        \item Geowissenschaften                                                                                                                                                                                           \\
        \item Germanistik                                                                                                                                                                                                 \\
        \item Gerontologie  Care                                                                                                                                                                                          \\
        \item Geschichte                                                                                                                                                                                                  \\
        \item Informatik                                                                                                                                                                                                  \\
        \item Islamwissenschaft                                                                                                                                                                                           \\
        \item Japanologie                                                                                                                                                                                                 \\
        \item Jura                                                                                                                                                                                                        \\
        \item Klassische und Byzantinische Archäologie                                                                                                                                                                    \\
        \item Klassische Philologie                                                                                                                                                                                       \\
        \item Kunstgeschichte (Europäische)                                                                                                                                                                               \\
        \item Mathematik                                                                                                                                                                                                  \\
        \item Medizin Heidelberg                                                                                                                                                                                          \\
        \item Medizin Mannheim                                                                                                                                                                                            \\
        \item Mittellatein/Mittelalterstudien                                                                                                                                                                             \\
        \item Molekulare Biotechnologie                                                                                                                                                                                   \\
        \item Musikwissenschaft                                                                                                                                                                                           \\
        \item Ostasiatische Kunstgeschichte                                                                                                                                                                               \\
        \item Pharmazie                                                                                                                                                                                                   \\
        \item Philosophie                                                                                                                                                                                                 \\
        \item Physik                                                                                                                                                                                                      \\
        \item Politikwissenschaft                                                                                                                                                                                         \\
        \item Psychologie                                                                                                                                                                                                 \\
        \item Religionswissenschaft                                                                                                                                                                                       \\
        \item Romanistik                                                                                                                                                                                                  \\
        \item Semitistik                                                                                                                                                                                                  \\
        \item Sinologie                                                                                                                                                                                                   \\
        \item Slavistik/Osteuropastudien                                                                                                                                                                                  \\
        \item Soziologie                                                                                                                                                                                                  \\
        \item Sport                                                                                                                                                                                                       \\
        \item Südasieninwissenschaften (Fachschaft am SAI)                                                                                                                                                                \\
        \item Theologie (Evangelische)                                                                                                                                                                                    \\
        \item Transcultural Studies (891)                                                                                                                                                                                 \\
        \item Ur- und Frühgeschichte/Vorderasiatische Archäologie (UFG/VA)                                                                                                                                                \\
        \item Übersetzen und Dolmetschen (Fachschaft am IÜD)                                                                                                                                                              \\
        \item Volkswirtschaftslehre (VWL)
    \end{enumerate}
\subsection{Schlichtungsordnung}\label{appendix:3}
    \paragraph{I Organisation der Schlichtungskommission}
        \myparagraph{§1 Aufgaben}
            Die Schlichtungskommission (SchliKo) ist ein, den übrigen zentralen Organen der Studierendenschaft gegenüber, selbständiges und unabhängiges Organ für die Durchführung von Schlichtungsverfahren und Wahlprüfungen sowie zur Behandlung sonstiger, ihr übertragener Beschwerden.
        \myparagraph{§2 Geschäftsordnung}
            Die Schlichtungskommission kann sich bei Bedarf und eine Geschäftsordnung geben.
    \paragraph{II Sitzungen und Abstimmungen}
        \myparagraph{§3 Öffentlichkeit der Sitzung}
            Die Sitzungen der Schlichtungskommission sind grundsätzlich öffentlich. Die Öffentlichkeit oder Teile der Öffentlichkeit kann im Einzelfall mit einer Mehrheit von zwei Dritteln ausgeschlossen werden.
        \paragraph{§4 Terminierung der Sitzungen}
            \begin{itemize}
                \item[(1)] Die Schlichtungskommission tagt nach ihrer Anrufung binnen zwei Wochen, in der vorlesungsfreien Zeit binnen vier Wochen (§ 33 Absatz 2 OrgS). Dies gilt auch, wenn im Falle des § 7 Absatz 1 Nummer 4 das Verfahren automatisch beginnt (§ 12 Absatz 1Satz 1). Eine Sitzung ist nach Möglichkeit so zu terminieren, dass alle Beteiligten die Möglichkeit der Teilnahme haben.
                \item[(2)] Ferner sind Sitzungen der Schlichtungskommission    so    zu    terminieren,    dass mehrere   Verfahren   in   einer   Sitzung   abgehandelt werden können, wenn dies die Vorgaben des Absatz 1 und die Antragslage ermöglichen. 
            \end{itemize}
        \myparagraph{§5 Einberufung}
            Ein Mitglied der Schlichtungskommission lädt zu denSitzungen ein. Dies geschieht durch Bekanntgabe desSitzungstermins und Veröffentlichung der Einladung auf der Webpräsenz des Studierendenrates. Die Beteiligten sind per E-Mail zu unterrichten, wenn der Schlichtungskommission entsprechende Kontaktdaten vorliegen. Die Einladung muss mit einer Frist von mindestens vier Tagen erfolgen.
        \paragraph{§6 Beschlussfähigkeit und Abstimmungsregeln}
            \begin{itemize}
                \item[(1)] Die Schlichtungskommission ist beschlussfähig, wenn mindestens drei ihrer Mitglieder anwesend und stimmberechtigt sind. Die Beschlussfähigkeit muss jederzeit gegeben sein (§ 33 Absatz 5 OrgS).
                \item[(2)] Die Schlichtungskommission entscheidet mit der Mehrheit der abgegebenen Stimmen. Bei Stimmengleichheit
                    \begin{itemize}
                        \item[1.] im Rahmen der Verfahren nach § 7 Absatz 1 Nummer 1 und 5, Absatz 2 Nummer 2, 3 und 4 ist die Beschwerde zurückgewiesen; sodass das Handeln oder die Entscheidungen des Organs nicht beanstandet werden bzw. die Wahl als ordnungsgemäß anerkannt wird.
                        \item[2.] im Rahmen der Verfahren nach § 7 Absatz 1 Nummer 2, 3 und 4, Absatz 2 Nummer 1 ist die Abstimmung zu wiederholen. Bleibt esbei Stimmengleichheit, so entscheidet die Stimme desjenigen Mitgliedes der Schlichtungskommission, das ihr am längsten angehört. Wenn demnach mehreren Mitgliedern gleichermaßen in Frage kommen, so sind unter diesen nacheinander folgende weitere Kriterien zu berücksichtigen, bis ein einziges Mitglied feststeht, dem die Entscheidung zufällt:
                            \begin{itemize}
                                \item[a)] Längste Zugehörigkeit zu zentralem Organ der Studierendenschaft (Summe);
                                \item[b)] Zuordnung zu dem Geschlecht, das in der Schlichtungskommission am wenigsten vertreten ist;
                                \item[c)] Durch Los ausgewählt.
                            \end{itemize}
                        \item[3.] im Rahmen von internen Fragen (Geschäftsordnung, Verfahrensfragen, etc.) gilt der Antrag als abgelehnt.
                    \end{itemize}
            \end{itemize}
    \paragraph{III Verfahren vor der Schlichtungskommission}
        \paragraph{§7 Verfahrensarten}
            \begin{itemize}
                \item[(1)]Die Schlichtungskommission ist zuständig bei:
                    \begin{enumerate}
                        \item Beschwerden, die von jedem*jeder Studierenden mit der Behauptung erhoben werden können, die Studierendenschaft habe in einem konkreten Einzelfall ihre Zuständigkeit gemäß § 65 Absatz2 bis 4 LHG überschritten (§ 31 Absatz 1 Nummer 1 OrgS);
                        \item Streitigkeiten über die Kompetenzen von Organen derStudierendenschaft (§ 31 Absatz 1 Nummer 2 OrgS);
                        \item Uneinigkeit darüber, ob eine grundsätzliche Angelegenheit im Sinne von § 8 Absatz 3 OrgS vorliegt (§ 31 Absatz 1 Nummer 5 OrgS);
                        \item Wahlverfahren nach § 29 Absatz 6 OrgS, wenn der Studierendenrat bei zwei aufeinanderfolgenden Vorschlägen von Seiten des autonomen Referates keine*n Referent*in wählt (§ 31 Absatz 1Nummer 4 OrgS);
                        \item Einsprüchen gegen die Ordnungsmäßigkeit der Sitzungen von Organen der Studierendenschaft (§ 31 Absatz 1 Nummer 3 OrgS) und ihrer Beschlüsse sowie Einsprüche gegen Wahlen durch den Studierendenrat (§ 31 WahlO),
                        \item Streitigkeiten innerhalb der Strukturen der Studierendenschaft, wenn sie auf Wunsch eines Betroffenen zur Streitschlichtung angerufen wird. Auf diese formlose Streitschlichtung finden die Bestimmung dieser Satzung nur Anwendung, wenn sie ihrem Wesen nach darauf anwendbar sind.
                    \end{enumerate}
                \item[(2)]  Als Wahlprüfungsausschuss ist die Schlichtungskommission zuständig für: 
                    \begin{enumerate}
                        \item Die Überprüfung der Unterschriftenlisten bei Urabstimmungen (§ 31Absatz 2 Nummer 2 OrgS);
                        \item Die Entscheidung von Beschwerden gegen die Nichtzulassung von Urabstimmungen durch den Wahlausschuss (§ 6 Absatz 8 OrgS); sowie die Entscheidung von Beschwerden gegen die vom Wahlausschuss festgelegte Abstimmungsfrage (§ 8a Absatz 3Satz 3 WahlO);
                        \item Die Entscheidung von Einsprüchen gegen Wahlen und Urabstimmungen (§ 31 Absatz 2 Nummer 1 OrgS, § 20 WahlO Absatz 2);
                        \item Die Entscheidung von Beschwerden Betroffener gegen die Feststellung des Wahlausschusses, dass ein gewähltes Mitglied oder ein*e Amtsträger*in sein*ihr Amt beziehungsweise Mitgliedschaft verloren hat (§ 47 Absatz 2 Nummer 4 i.V.m Absatz 3 und § 19 Absatz 3 Satz 3 WahlO).
                    \end{enumerate}
            \end{itemize}
        \paragraph{§8 Bestimmungen für alle Verfahren}
            \begin{itemize}
                \item[(1)] Sofern nicht ausdrücklich etwas anders bestimmt ist, wird die Schlichtungskommission nur nach Anrufung tätig. Hierzu bedarf es eines schriftlichen Antrages (§ 21 Absatz 1), der zu begründen ist. Wenn eine Frist bestimmt ist in der die Schlichtungskommission anzurufen, die Anfechtung vorzunehmen oder die Beschwerde bzw. der Einspruch zu erheben ist, so muss der Antrag in dieser Frist der Schlichtungskommission zugehen.
                \item[(2)] Beteiligte an einem Verfahren sind – sofern für das betreffende Verfahren vorgesehen – der*die Antragsteller*in (Absatz 1), der*die Antragsgegner*in und weitere Beteiligte. Organe werden durch ihre Vorsitzenden vertreten. Der Wahlausschuss ist in den Verfahren nach § 7Absatz 2 immer weiterer Beteiligter. Bei Rechtsfragen soll die Schlichtungskommission die Stellen der Studierendenschaft, die sich in der Hauptsache damit beschäftigen, als weitere Beteiligte hinzuziehen. Im Übrigen ergibt sich die Beteiligteneigenschaft aus den Vorschriften zu den einzelnen Verfahren.
                \item[(3)] Die Schlichtungskommission gibt in allen Verfahren den Beteiligten die Möglichkeit zur Stellungnahme und zur Darlegung ihrer Sicht hinsichtlich der Sach- und Rechtslage. Sie kann dieBeteiligten bitten, bereits in Vorbereitung auf dieSitzung schriftlich Stellung zu etwaigen Nachfragen zu nehmen.
                \item[(4)] Die Schlichtungskommission kann von der Anberaumung einer Sitzung (§ 4 Absatz 1) absehen, wenn der Antrag offensichtlich missbräuchlich gestellt wurde oder der Antrag unzulässig ist, insbesondere weil die Schlichtungskommission nicht zuständig oder dieVorgaben des Absatz 1 nicht erfüllt sind oder wenn der Antrag evident unbegründet ist. Diese Entscheidungen kann sie im Umlaufverfahren treffen.
                \item[(5)] Die verbindliche Entscheidung oder Empfehlung der Schlichtungskommission wird allen Beteiligten nach Beschlussfassung unterbreitet.
            \end{itemize}
        \paragraph{§9 Verfahren in den Fällen des § 7 Absatz 1 Nummer 1}
            \begin{itemize}
                \item[(1)] Beschwerden mit der Behauptung, die Studierendenschaft hätte in einem konkreten Einzelfall ihre Zuständigkeit gemäß § 65 Absatz 2 bis 4 LHG überschritten, können von jedem*jeder Studierenden erhoben werden. Die Beschwerde muss die Maßnahme, durch die die Überschreitung erfolgt seinsoll, bezeichnen. Sie muss bei der Schlichtungskommission binnen sechs Monaten ab der Überschreitung der Befugnisse erhoben werden. Dauert die Überschreitung an (bspw. durch eine fortwährende Handlung der Studierendenschaft, Satzungsbestimmungen oder den Inhalt einer Positio-nierung etc.), so ist der Zeitpunkt der ersten Überschreitung maßgebend. Die Beschwerde kann nur erheben, wer zum Zeitpunkt der Überschreitung imma-trikuliert war und zum Zeitpunkt der Einreichung der Beschwerde immatrikuliert ist. Bei einer andauernden Überschreitung kann die Beschwerde auch von Neuimmatrikulierten binnen sechs Monaten ab ihrer Immatrikulation erhoben werden.
                \item[(2)] Antragsgegner*in ist das Organ, welches die behauptete Zuständigkeitsüberschreitung zu verantworten hat. Ist ein solches nicht zu ermitteln, sind die Vorsitzenden der Studierendenschaft Antragsgegner*innen.
                \item[(3)]  Stellt die Schlichtungskommission eine Überschreitung der Kompetenzen der Studierenschaft fest, so ordnet sie – sofern sie noch andauert – deren Einstellung an und dass sie in Zukunft zu unterbleiben hat. Die Anordnung ist für alle Organe der Studierendenschaft verbindlich.
            \end{itemize}
        \paragraph{§10 Verfahren in den Fällen des § 7 Absatz 1 Nummer 2}
            \begin{itemize}
                \item[(1)] Ist zwischen Organen der Studierendenschaft die Zuständigkeit oderKompetenz streitig, so kann die Schlichtungskommission von jedembetroffenen Organ mit der Bitte um Ausspruch einer Empfehlung angerufen werden. In dem Antrag ist genau zu bezeichnen, um welche Kompetenz es sich handelt und was die unterschiedlichen vertretenen Auffassungen sind. Bei Kollegialorganen wird die Anrufung durch einfache Mehrheit beschlossen. JedesMitglied dieses Kollegialorganes kann die Anrufung jedoch aucheinzeln vornehmen, wenn es der Meinung ist, ein anderes Organ verletze dasOrgan, dem es angehört, in seinen Rechten.
                \item[(2)] Wurde die Anrufung von einem Mitglied eines betroffenen Organs vorgenommen, so ist das Organ, dem es angehört, selbst weiterer Beteiligter. Das andere Organ, das in die Kompetenzstreitigkeit verwickelt ist, istebenfalls weiterer Beteiligter.
                \item[(3)] Die Schlichtungskommission gibt eine Empfehlung ab. Die beteiligtenOrgane sind nachdrücklich gehalten, die Empfehlung zu befolgen.
            \end{itemize}
        \paragraph{§11 Verfahren in den Fällen des § 7 Absatz 1 Nummer 3}
            \begin{itemize}
                \item[(1)]Besteht Uneinigkeit darüber, ob es sich bei einer Angelegenheit im Rahmen einer Urabstimmung um eine ‚grundsätzliche Angelegenheit‘ handelt, so entscheidet auf Antrag eines*einer jeden Studierenden die Schlichtungskommission. Eine Frist, innerhalb der die Frage der Schlichtungskommission vorgelegt werden kann, gibt es nicht, die Schlichtungskommission kann den Antrag jedoch als unerheblich zurückweisen, wenn dem Ergebnis der Urabstimmung aufgrund von Zeitablauf keine praktische Bedeutung mehr zukommt.
                \item[(2)]Die Referatekonferenz und der Wahlausschuss sind weitere Beteiligte.
                \item[(3)]Die Schlichtungskommission stellt durch Beschluss fest, ob eine„grundsätzliche Angelegenheit“ vorliegt und die Urabstimmung damit bindendoder nicht bindend ist.
            \end{itemize}    
        \paragraph{§12 Verfahren in den Fällen des § 7 Absatz 1 Nummer 4}
            \begin{itemize}
                \item[(1)] Wählt der Studierendenrat zweimal nacheinander keine*n Referent*in für ein autonomes Referat, obwohl von Seiten des jeweiligen autonomen Referates Vorschläge unterbreitet wurden, so findet automatisch ein Schlichtungsverfahren statt. Die Vorschläge / Wahlen gelten jedoch nicht als nacheinander erfolgt, wenn zwischen dem erstem Vorschlag / der ersten Wahl und dem zweitem Vorschlag / der zweiten Wahl mehr als vier Monate liegen.
                \item[(2)]    Der Studierendenrat und das betroffene autonome Referat gelten beide als Antragsteller*innen.
                \item[(3)]    Die Schlichtungskommission erarbeitet eine Empfehlung.§ 13Verfahren in den Fällen des § 7 Absatz 1 Nummer 5(1)    Gegen die Ordnungsmäßigkeit von Sitzungen von Organen der Studierendenschaft kann Einspruch erhoben werden. Insbesondere wegen nicht ordnungsgemäßer Einberufung der Sitzung, aber auch wegen allen anderen Gründen, die die Ordnungsmäßigkeit und Rechtmäßigkeit der Sitzung betreffen. Der Einspruch kann auch nur bezüglich der Fehler bei einzelnen Beschlüssen erhoben werden, insbesondere in Bezug auf Unregelmäßigkeiten bei Abstimmungen.Der Einspruch ist bis sieben Tage nach der Genehmigung des Protokolls eben dieser Sitzung zu erheben. Ist eine solche Genehmigung des Protokollsin dem entsprechenden Organ nicht üblich, kann der Einspruch binnen einerWoche nach der Sitzung erhoben werden.Der Einspruch kann von jedem Mitglied des Organes und von jedem ordentlich stimmberechtigten Mitglieddes Studierendenrates erhoben werden. In dem Einspruch ist zu bezeichnen, worin der Fehler der Sitzung oder des Beschlusses bestehen soll.(2)    Das betroffene Organ ist Antragsgegner.(3)    Die Schlichtungskommission stellt fest, ob die Sitzung oder einzelne Beschlüsse eines Organs ordnungsgemäß waren. Stellt die Schlichtungskommission fest, dass die Ordnungsmäßigkeit nicht gegeben war, so unterbreitet sie dem jeweiligen Organ eine Empfehlung. Die Empfehlung hat vorzusehen, dass das entsprechende Organ die gesamte Sitzung oder einzelne gefasste Beschlüsse oder vorgenommene Wahlen für ungültig erklären und aufheben soll, wenn dies rechtlich aufgrund der Fehler angebracht erscheint. Das jeweilige Organ ist nachdrücklich gehalten, die Empfehlung zu befolgen. Die Empfehlung ist zwingendin der nächsten Sitzung des jeweiligen Organs vor Einstieg in die Tagesordnung abschließend zu behandeln. Sofern das jeweilige Organ nicht mit absoluter Mehrheit etwas anders beschließt, gilt die Empfehlung als angenommen
                \item[(4)]    Die jeweiligen Anträge oder Kandidaturen der für ungültig erklärten und aufgehobenen Sitzung sowie etwaige Beschlüsse oder Wahlen gelten für die Sitzung nach Absatz 3 Satz 5, als fristgerecht eingereicht,geladen, sodass unmittelbar erneut abgestimmt oder gewählt werden kann.Anm.: Ohne eine solche Empfehlung der Schlichtungskommission kann ein Organ nicht einfacheine ganze Sitzung oder einzelne Beschlüsse oder Wahlen für ungültig erklären und aufheben! Das jeweilige Organ kann nur im Rahmen der regulär geltenden (Verfahrens-)Vorschriften Beschlüsse fassen, die auch vorangegangene Beschlüsse ändern oder Abwahlen vornehmen können.
                \item[(5)]    Für Beschlüsse, die Organe der Studierendenschaft außerhalb von Sitzungen (Umlaufverfahren) treffen, gelten die Absätze 1 und 2 entsprechend. Anstelle des Sitzungsprotokolls (Absatz 1 Satz 4) tritt das Protokoll, in dem der Beschlussbekanntgegeben wird. Ist dies nicht üblich, gilt Absatz 1 Satz 5 entsprechend.
                \item[(6)]    Für die vom Studierendenrat vorzunehmenden Wahlen gelten die Absätze 1 und 2 entsprechend. DieSchlichtungskommission kann hier jedoch eine Wiederholungswahl zwingend anordnen.
            \end{itemize}
        \myparagraph{§ 14 Verfahren in den Fällen des § 7 Absatz 2 Nummer 1}
            Die Schlichtungskommission prüft von Amts wegen dieUnterschriftenlisten für Urabstimmungen. Kommt sie zu dem Ergebnis, dass die Listen fehlerhaft und für die Zulassung des Antrags auf Urabstimmung ungeeignet oder unzureichend sind, so weist sie den Wahlausschuss an, die Urabstimmung nicht zuzulassen.
        \paragraph{§15 Verfahren in den Fällen des § 7 Absatz 2 Nummer 2}
            \begin{itemize}
                \item[(1)]    Gegen die Entscheidung des Wahlausschusses, eine Frage zur Urabstimmung nicht zuzulassen, können die Antragsteller der Urabstimmung Beschwerde bei der Schlichtungskommission erheben. Die Beschwerde ist spätestens am dritten Tag nachdem der Wahlausschuss die Antragsteller*innen von der Nichtzulassung in Kenntnisse gesetzt hat bei der Schlichtungskommission zu erheben.
                \item[(2)]    Gibt die Schlichtungskommission der Beschwerde statt, so ordnet sie die Zulassung zur Urabstimmungan.
                \item[(3)]    Für die Entscheidung von Beschwerden gegen die vom Wahlausschuss festgelegte Abstimmungsfrage gelten die Absätze 1 und 2 entsprechend. Gibt die Schlichtungskommission der Beschwerde statt, wird der ursprüngliche Text / die ursprüngliche Abstimmungsfrage wiederhergestellt. Sie kann der Beschwerde auch nur teilweise stattgeben und anordnen, dass derWahlausschuss näher zu bezeichnenden Verbesserungen an der von ihm festgelegten Abstimmungsfrage vorzunehmen hat.
            \end{itemize}
        \paragraph{§16 Verfahren in den Fällen des § 7 Absatz 2 Nummer 3}
            \begin{itemize}
                \item[(1)]    Die Schlichtungskommission prüft die Wahlen gemäß § 20 Absatz 2WahlO. Jedes Mitglied der Studierendenschaft kann die Wahl binnen einundzwanzig Tagen ab der Bekanntmachung der Ergebnisse bei der Schlichtungskommission anfechten. Die Wahlprüfung findet spätestens einunddreißig Tage nach der Bekanntmachung der Ergebnisse statt.
                \item[(2)]    Zur Wahlprüfung werden der Schlichtungskommission vom Wahlausschuss die Niederschrift über das Gesamtergebnis (§ 17 WahlO) und die Bekanntmachung des Ergebnisses (§ 18 WahlO) sowie auf Antrag sämtlichen Wahlraumberichte(§ 16 WahlO), sonstige Protokolle, Zähllisten, Stimmzettel, etc. bereitgestellt. Bei digitalen Wahlen werden die dort entsprechend vorliegenden Unterlagen und digitalen Daten zur Verfügung gestellt.
                \item[(3)]    Stellt die Schlichtungskommission Fehler oder Unregelmäßigkeiten bei der Wahl fest, die aber weder das Ergebnis beeinflusst haben noch die Wahl allgemein als den Wahlgrundsätzen und den Vorschriften entsprechend in Frage stellen, so benennt sie diese Fehler oder Unregelmäßigkeiten in ihrem Beschluss ausdrücklich.Stellt die Schlichtungskommission Fehler oder Unregelmäßigkeiten bei der Wahl fest, die mandatsrelevant sind (die das Ergebnis der Mandatsvergabe hätten verändern können) oder so gelagert sind, dass die Wahl nicht mehr als den Wahlgrundsätzen und den Vorschriften entsprechend gelten kann, soerklärt sie die Wahl oder ggf. den betroffenen Teil der Wahl für ungültig und ordnet eine Neuwahl an.Bestehen lediglich Zweifel an der Auszählung der Stimmen, so kann sie eine(Teil-)Neuauszählung anordnen.Ferner kann sie die einfache Berichtigung von Fehlern in Dokumenten nach Absatz 2 anordnen, die nicht aufFehler oder Unregelmäßigkeiten bei der Wahl oder Auszählung schließen lassen.
                \item[(4)]    Die Absätze 1 bis 3 sind auf Urabstimmungen entsprechen anzuwenden. Anstelle der Mandatsrelevanz tritt die Relevanz für das Ergebnis(also Annahme / Ablehnung / Quorum erreicht / etc.).
                \item[(5)]    Die Wahlprüfung nach den Absätzen 1 bis 4 ist Teil des Wahl- bzw. Abstimmungsverfahrens. Die Entscheidung der Schlichtungskommission hebt die Gültigkeit der Wahl oder nur des bekanntgegebenen Wahler- bzw. Abstimmungsergebnisses (vgl. § 20 Absatz 1 WahlO) auf. Anordnungen nachAbsatz 3 sind für den Wahlausschuss bindend, er hatihnen zeitnah nachzukommen. Im Falle von Anordnungen nach Absatz 3 Satz 3 und 4 hat der Wahlausschuss nach der Neuauszählung bzw. Berichtigung die entsprechenden Ergebnisse erneut bekanntzumachen. Diese erneute Bekanntmachung kann erneut unter den gleichen Bedingungen angefochten werden.
            \end{itemize}
        \paragraph{§17 Verfahren in den Fällen des § 7 Absatz 2 Nummer 4}
            \begin{itemize}
                \item[(1)]    Gegen die Feststellung des Wahlausschusses, dass eine Person ihr Amt verloren hat, weil bei ihr die Voraussetzungen der Wählbarkeit entfallen sind odersie aus gesundheitlichen Gründen nicht mehr in der Lage ist, das Amt auszuführen, oder rechtliche Gründe dem entgegenstehen, kann die Person, die durch die Feststellung ihr Amt verlieren würde, Beschwerde einlegen. Die Beschwerde muss spätestens am zehnten Tag, nachdem der Wahlausschuss die Person von der Feststellung in Kenntnis gesetzt hat, bei der Schlichtungskommission erhoben werden. Die Schlichtungskommission hört den Wahlausschuss zur Beschwerde.
                \item[(2)]    Bis zum Abschluss des Verfahrens behält die Person ihr Amt und die damit verbundenen Rechte undPflichten.
                \item[(3)]    Gibt die Schlichtungskommission der Beschwerde statt, so ist dieEntscheidung des Wahlausschusses aufgehoben.IV Protokolle der Schlichtungskommission
            \end{itemize}
        \paragraph{§18    Protokolle}
            \begin{itemize}
                \item[(1)]    Über jede Sitzung der Schlichtungskommissionwird ein Protokoll angefertigt. Das angefertigte Protokoll ist nach der Sitzung von dem*der protokollführenden Person zu unterschreiben. Die Protokolle werden archiviert.
                \item[(2)]    Ein Protokoll enthält mindestens:
                    \begin{itemize}
                        \item[1.]  Datum, Beginn und Ende der Sitzung,
                        \item[2.]  Liste der anwesenden Mitglieder sowie der Beteiligten,
                        \item[3.]  die gefassten Beschlüsse mit
                            \begin{itemize}
                                \item[a)] dem Wortlaut der bindenden Entscheidung bzw. der Empfehlung;
                                \item[b)] den Gründen und Erwägungen für die Entscheidung bzw. dieEmpfehlung sowie bei bindenden Entscheidungen oderEmpfehlungen auf Grundlage rechtlicher Fragen die rechtlichenErwägungen.
                            \end{itemize}
                    \end{itemize}
                \item[(3)]    Das Protokoll wird im Umlaufverfahren von den Mitgliedern der Schlichtungskommission genehmigt. Das Protokoll ist nach seinem Beschluss auf der Webpräsenz zu veröffentlichen.V Schlussbestimmungen
            \end{itemize}    
        \paragraph{§19  Befangenheiten}
            \begin{itemize}
                \item[(1)]    Die §§ 20 und 21 Verwaltungsverfahrensgesetzgelten für die Schlichtungskommission, alle anderenOrgane der Studierendenschaft und deren Verwaltung entsprechend. § 20 Absatz 4 Satz 4Verwaltungsverfahrensgesetz gilt nicht für öffentliche Sitzungen einesOrgans, an dem jedes Mitglied der Studierendenschaft teilnehmen darf.
                \item[(2)]    Liegt keine Befangenheit nach Absatz 1 vor, so gilt für dieSchlichtungskommission überdies § 33 Absatz 4 OrgS.
            \end{itemize}
        \myparagraph{§20 Fristen}
        Die §§ 187 bis 193 BGB sind bei der Berechnung aller in dieser aber auch in allen anderen Satzungen und Ordnungen der Studierendenschaft vorgesehenen Fristen anzuwenden.
        \myparagraph{§21 Formen}
            (gestrichen)
        \myparagraph{§22 Begriff des Organs}
        Im Sinne dieser Satzung bedeutet „Organ“ auch „Teilorgan“, sofern die jeweilige Bestimmung im Einzelfall auch auf Teilorgane anwendbar ist.

\subsection{Aufwandsentschädigungsordnung\label{appendix:4}}
    \paragraph{§1 Allgemeines}
        \begin{itemize}
            \item[(1)] Die ehrenamtlich in der Verfassten Studierendenschaft (VS) mitwirkenden Studierenden arbeiten prinzipiell unentgeltlich an der Erfüllung des gesetzlichen und satzungsgemäßen Auftrags der Verfassten Studierendenschaft mit.
            \item[(2)]  Amts- und Mandatsträger*innen erhalten für ihr Mitwirken keine Bezahlung.
            \item[(3)]  Amtsträger*innen, welche sehr zeitintensive Tätigkeiten für die VS ausführen, haben nach Maßgabe dieser Ordnung einen Anspruch auf eine Entschädigung ihres Aufwands.
        \end{itemize}
    \paragraph{§2 Anspruchsberechtigte}
        \begin{itemize}
            \item[(1)] Anspruchsberechtigt sind:
                \begin{enumerate}
                    \item   die Mitglieder des Präsidiums des Studierendenrats, 
                    \item  Personen, die in den Sitzungen des Studierendenrats die Protokollführung übernehmen,  
                    \item  die Mitglieder der Exekutiven der VS, nämlich 
                        \begin{itemize}
                            \item[a.] die beiden Vorsitzenden, 
                            \item[b.] Stellvertretende Vorsitzende, die bei Vakanz vertretunsgweise die Vorsitzposition übernehmen,
                            \item[c.] die Mitglieder der im Anhang aufgeführten Referate, 
                        \end{itemize}
                    \item  die Mitglieder des Wahlausschusses, 
                    \item   die Helfer*innen bei Wahlen, nämlich:
                        \begin{itemize}
                            \item[a.] Wahlhelfer*innen bei zentralen Wahlen und Urabstimmungen und
                            \item[b.] die Ehrenamtlichen, welche die Fachratswahlen durchführen.
                        \end{itemize}
                \end{enumerate} 
            \item[(2)]  Kommissarische Amtsinhaber*innen haben für den ersten Monat ihrer kommissarischen Amtsführung einen Anspruch auf die Hälfte der im Folgenden und im Anhang bestimmten Aufwandsentschädigung.
        \end{itemize}
    \paragraph{§3 Entschädigung des Präsidiums  [Sitzungsleitung des Studierendenrats]}
        \begin{itemize}
            \item[(1)]  Die Mitglieder des Präsidiums können pro vorbereiteter und durchgeführter Sitzung eine Aufwandsentschädigung in Höhe von 360 Euro erhalten, welche den beteiligten Mitgliedern der Sitzungsleitung anteilig ausgezahlt wird.ä 
        \end{itemize}
    \paragraph{§4 Entschädigung der Protokollant*in}
        \begin{itemize}
            \item[(1)] Für die ehrenamtliche Protokollführung bei Sitzungen des Studierendenrats wird eine Aufwandsentschädigung von 30 Euro gezahlt.
            \item[(2)] Führt die Sitzungsleitung das Protokoll, so wird keine zusätzliche Aufwandsentschädigung gezahlt.
        \end{itemize}
    \paragraph{§5 Entschädigung der Vorsitzenden der VS}
        \begin{itemize}
            \item[(1)] Die beiden Vorsitzenden der Verfassten Studierendenschaft können jeweils eine monatliche Aufwandsentschädigung in Höhe von 500 Euro erhalten.
            \item[(2)] Tritt eine*r der Vorsitzenden vom Amt zurück, kann der*die stellvertretende Vorsitzende, der*die das Amt bis zur Nachwahl einer*eines neuen Vorsitzenden ausführt, eine Aufwandsentschädigung in Höhe von 500 Euro erhalten. 
        \end{itemize}
    \paragraph{§6 Entschädigung des Finanz-und Haushaltsreferats}
        \begin{itemize}
            \item[(1)] Der*die Finanz- und Haushaltsreferent*in kann eine monatliche Aufwandsentschädigung von 450 Euro erhalten, wenn das Referat mit einer Person besetzt ist.
            \item[(2)] Bei Besetzung des Referats mit zwei Personen, kann jede der beiden Personen eine monatliche Aufwandsentschädigung in Höhe von 400 Euro erhalten. 
        \end{itemize}
    \paragraph{§7 Entschädigung weiterer Referate}
        \begin{itemize}
            \item[(1)]  Die weiteren Referate der VS können jeweils eine monatliche Aufwandsentschädigung, deren Höhe durch den Anhang dieser Ordnung bestimmt wird, erhalten. 
            \item[(2)]  Die Aufwandsentschädigung wird anteilig den gewählten Referent*innen des jeweiligen Referats ausgezahlt.
        \end{itemize}
    \paragraph{§8 Entschädigung des Wahlausschusses}
        \begin{itemize}
            \item[(1)]  Die Mitglieder des Wahlausschuss können eine Aufwandsentschädigung gemäß der Anzahl und Art der durchgeführten Wahlen und Abstimmungen erhalten.
            \item[(2)] Die Aufwandsentschädigung beträgt für die Durchführung von
                \begin{enumerate}
                    \item   Fachschaftsratswahlen 50 Euro pro Studienfachschaft; 
                    \item   Zentralen Urabstimmungen 1200 Euro, bei mehreren zentralen Urabstimmungen am selben Termin für jede weitere zentrale Urabstimmung weitere 100 Euro; 
                    \item   Studierendenratswahlen 2000 Euro; bei Zusammenlegung von StuRa-Wahlen und zentralen Urabstimmung wird für jede zentrale Urabstimmung eine Aufwandsentschädigung von jeweils 100 Euro, zusätzlich zu der Aufwandsentschädigung für die StuRa-Wahlen, gezahlt.
                \end{enumerate}
            \item[(3)] Die Aufwandsentschädigung wird den an der Wahl beteiligten Mitgliedern des Wahlausschusses anteilig ausgezahlt. 
            \item[(4)] Für Fachschaftsrats- und Studierendenratswahlen sowie Urabstimmungen legt jedes einzelne Wahlausschuss-Mitglied einen Stundenzettel an, welcher Datum, Uhrzeit und eine Kurzbeschreibung der Tätigkeiten zu diesen Zeiten beinhalten. Dieser Stundenzettel dient der Berechnung der anteiligen Aufwandsentschädigung jedes Wahlausschussmitglieds. Der Wahlausschuss erstellt für jede Wahl eine Gesamtübersicht über die insgesamt aufgewendete Zeit und ihre Aufteilung auf  die einzelnen Mitglieder des Wahlausschusses. Die Gesamtübersicht ist dem Antrag auf Aufwandsentschädigung beizufügen.
        \end{itemize}
    \paragraph{§9 Entschädigung des EDV-Referats im Falle von Digitalwahlen}
        \begin{itemize}
            \item[(1)] Finden Wahlen vollständig oder teilweise im digitalen Format als Online-Wahl statt, so erhalten die beteiligten Mitglieder des EDV-Referats für ihre Unterstützung des Wahlausschusses bei der Vorbereitung, Durchführung und Nachbereitung der Wahl eine zusätzliche Aufwandsentschädigung von jeweils 250 Euro.
        \end{itemize}
    \paragraph{§10 Entschädigung von Wahlhelfer*innen}
        \begin{itemize}
            \item[(1)] Die Wahlhelfer*innen bei zentralen Urabstimmungen und Wahlen erhalten eine Aufwandsentschädigung in Höhe von 10 Euro pro Stunde.
            \item[(2)] Je Tag kann eine Aufwandsentschädigung von maximal 80 Euro ausgezahlt werden. Weitere Arbeit wird nicht entschädigt. 
        \end{itemize} 
    \paragraph{§11 Entschädigung für die Durchführung von Fachratswahlen}
        \begin{itemize}
            \item[(1)]  Die Ehrenamtlichen, welche die Fachratswahlen durchführen, können eine Aufwandsentschädigung von jeweils 50 Euro pro Fachratswahl erhalten. 
            \item[(2)]  Die Aufwandsentschädigung steht den beteiligten Ehrenamtlichen anteilig zu. 
        \end{itemize}
    \paragraph{§12 Auszahlung der Aufwandsentschädigung}
        \begin{itemize}
            \item[(1)]  Aufwandsentschädigungen werden – sofern nicht anders bestimmt – aus den zentralen Finanzmitteln der Verfassten Studierendenschaft über einen eigenen Haushaltsposten abgerechnet.
            \item[(2)] Es steht jeder Person frei, eine ihr zustehende Aufwandsentschädigung in Anspruch zu nehmen oder ganz oder teilweise auf sie zu verzichten.
            \item[(3)]  Aufwandsentschädigungen werden nur bei form- und fristgerechter Antragstellung ausgezahlt. 
            \item[(4)]  Die Auszahlung der Aufwandsentschädigung setzt voraus, dass die Berechtigten den wesentlichen Aufgaben und Verpflichtungen ihres Amtes nachgekommen sind. Zur Dokumentation der Tätigkeit werden dem Studierendenrat bzw. der Referatekonferenz Berichte vorgelegt.
            \item[(5)] Sind die Voraussetzungen nicht erfüllt, so lehnt das Finanzreferat die Auszahlung ab. 
            \item[(6)]  Die Informationen über die Auszahlungen von Aufwandsentschädigungen sind vertraulich. 
        \end{itemize}
    \paragraph{§13 Abschlussbestimmung}
        Diese Ordnung berührt in keiner Weise die Rechtsstellung, Arbeitsverhältnisse und Bezahlung der Angestellten der Verfassten Studierendenschaft.\\
        Übergangsbestimmungen\\
        Auf Refrent*innen, die vor Inkrafttreten der neuen Regelung gewählt wurden, findet bis zum Ende ihrer regulären Amtszeit die bisherige Regelung Anwendung. \\
        Auf bisherige kommissarische Referent*innen, die zum Ende des Wintersemesters bereits mehr als ein Jahr kommissarisch im Amt waren, findet ab Sommersemester 2021 die neue Regelung Anwendung. Bei Referent*innen, die bei Inkrafttreten der neuen Regelungen kommissarisch im Amt sind, finden die bisherigen Regelungen für maximal ein Jahr ab Amtsende Anwendung.
    \paragraph{Anhang zu §7 Abs. 1}
        \begin{tabular}{|l|p{7cm}|c|c|}
            \hline
            &&\multicolumn{2}{|c|}{Höhe der Aufwandsentschädigung in Euro}\\\hline
            &Referat für & insgesamt & Max. für eine Person \\\hline\hline
            Gruppe 1&EDV, Hochschulpolitische Vernetzung, Konstitution und Gremienkoordination, Soziales&250&165\\\hline
            \multirow{2}{*}{Gruppe 2}&Lehre und Lernen&165&\multirow{2}{*}{125}\\\cline{2-2}
            &QSM&125&\\\hline
            Gruppe 3 & Öffentlichkeitsarbeit, Ökologie und Nachhaltigkeit, Politische Bildung, Verkehr und Kommunales & 100 & 100\\\hline
            Gruppe 4 & Internationales, Kultur und Sport, Studierendenwerk & 75 & 75\\\hline
        \end{tabular}
\subsection{Geschäftsordnung des Stura\label{appendix:4}}
    \paragraph{§1 Geltungsbereich}
        Diese Geschäftsordnung regelt die Verfahren und Abläufe im Studierendenrat der Verfassten Studierendenschaft der Universität Heidelberg. Sie findet auf seine Ausschüsse und Kommissionen sowie weitere nachgeordnete Organe entsprechend Anwendung, sofern diese sich keine eigene Geschäftsordnung gegeben haben oder andere Regelungen zur Anwendung kommen.
    \paragraph{I. Beginn der Legislatur, Sitzungsleitung, Protokoll}
        \paragraph{§2 Konstituierende Sitzung}
            \begin{itemize}
                \item[(1)] Der Wahlausschuss lädt den Studierendenrat auf Grundlage des Wahlergebnisses und der vorliegenden ordnungsgemäßen Entsendungen zur ersten Sitzung einer neuen Legislatur ein.
                \item[(2)] Die erste Sitzung wird von den Mitgliedern des Wahlausschusses vorbereitet und bis zur Wahl einer neuen Sitzungsleitung geleitet.
                \item[(3)] Der Studierendenrat kann bis zur Wahl einer neuen Sitzungsleitung keine anderen Handlungen als die Wahl der Sitzungsleitung vornehmen. 
                \item[(4)] Wird keine Sitzungsleitung gewählt, endet die Sitzung automatisch.
                \item[(5)] Die Bestimmungen dieses Paragraphen finden für die darauffolgenden Sitzungen entsprechend Anwendung, bis eine Sitzungsleitung gewählt ist.    
            \end{itemize}
        \paragraph{§3 Wahl und Aufgaben der Sitzungsleitung}
            \begin{itemize}
                \item[(1)] Der Studierendenrat wählt zu Beginn jeder Legislaturperiode eine neue Sitzungsleitung für dieDauer der Legislatur. Spätere (Nach-)Wahlen zur Sitzungsleitung gelten für die restliche Dauer der Legislatur. 
                \item[(2)] Die Sitzungsleitung besteht aus mindestens zweiund maximal sechs Personen und soll divers besetzt sein. 
                \item[(3)] Die Sitzungsleitung bereitet die Sitzungen des Studierendenrats vor und nach. Er lädt zu den Sitzungen ein, eröffnet sie und schließt sie. Die Sitzungsleitung sorgt für einen geregelten Ablauf der Sitzungen. Sie führt ihre Arbeit unparteiisch, unbefangen und sachlich aus. %Alternativ: unparteiisch und sachorientiert aus.
                \item[(4)] Die Sitzungsleitung führt für die jeweilige Legislatur eine Übersicht über alle inhaltlichen Beschlüsse des Studierendenrats. 
                \item[(5)] Die Sitzungsleitung hat bei Präsenzsitzungen dafür Sorge zu tragen, dass die Mitglieder des Studierendenrats sich ordnungsgemäß in die Anwesenheitsliste (Mitgliederliste) eintragen und nur berechtigte Mitglieder abstimmen können. %Alternativ:(5) Die Sitzungsleitung veranlasst, dass die Anwesenheit der Mitglieder des Studierendenrats erfasst wird und nur stimmberechtigte Mitglieder abstimmen können.
            \end{itemize}
        \paragraph{§4 Protokollführung}
            \begin{itemize}
                \item[(1)] Zu Beginn jeder Sitzung benennt die Sitzungsleitung eine Person, die das Protokoll führt und gibt diese namentlich bekannt.
                \item[(2)] Ist die Sitzungsleitung mit weniger als 3 Mitgliedern besetzt, soll die protokollführende Person nicht der Sitzungsleitung angehören.
                \item[(3)] Die protokollführende Person führt das Protokoll als Verlaufsprotokoll unparteisch und nach bestem Wissen und Gewissen.
                \item[(4)] Die Mitglieder der Sitzungsleitung und die protokollführende Person tragen gemeinsam die Verantwortung für die Richtigkeit des Protokolls.
                \item[(5)] Die Mitglieder des Studierendenrats sind gehalten, das Protokoll sorgfältig zu lesen und bei Bedarf Korrekturen zu beantragen. 
            \end{itemize}
    \paragraph{II. Neue Mitglieder von Studienfachschaften, Vertretung von Mitgliedern}
        \paragraph{§5 Mitteilung neuer Vertreter*innen von Studienfachschaften}
            \begin{itemize}
                \item[(1)] Bei Studienfachschaftsvertreter*innen, die durch den Fachschaftsrat entsandt werden, leitet dieser der Sitzungsleitung das Protokoll der Entsendung und die Kontaktdaten der neuen Mitglieder zu.
                \item[(2)] Bei direkt gewählten Studienfachschaftsvertreter*innen, die nicht zusammen mit den Listenmitgliedern gewählt werden, leitet der Wahlausschuss der Sitzungsleitung das Ergebnis der Wahl und die Kontaktdaten der neuen Mitglieder zu.
                \item[(3)] Die Meldung hat bis zum Tag vor der ersten StuRa-Sitzung, an der das neue Mitglied teilnehmen soll, zu erfolgen. Erfolgt die Entsendung erst am Tag der StuRa-Sitzung, kann die Sitzungsleitung Ausnahmen zulassen.
            \end{itemize}
        \paragraph{§6 Vertretung von Mitgliedern}
            \begin{itemize}
                \item[(1)] Verhinderte stimmberechtigte Mitglieder des Studierendenrats können sich gemäß § 21 OrgS vertreten lassen. 
                \item[(2)] Die Mitglieder müssen die Sitzungsleitung bis spätestens eine Stunde vor Sitzungsbeginn über ihreVerhinderung informieren (Abmeldung)
                \item[(3)] Erfolgt die Abmeldung nicht rechtzeitig, kann die Sitzungsleitung Ausnahmen zulassen.
            \end{itemize}
    \paragraph{III. Sitzungstermine, Tagesordnung, Einberufung und Leitung der Sitzung; Ordnungsmaßnahmen}
        \paragraph{§7 Öffentlichkeit der Sitzung}
            \begin{itemize}
                \item[(1)] Der Studierendenrat und seiner Ausschüsse sowieKommissionen und nachgeordneten Organisationseinheiten tagen grundsätzlich öffentlich. Von Absatz 1 ausgenommen sind Personalangelegenheiten oder Angelegenheiten, welche die Persönlichkeitsrechte der Mitglieder betreffen.
                \item[(2)] Der Studierendenrat kann in begründeten Fällen für einzelne Punkte die Nichtöffentlichkeit beschließen.
                \item[(3)] Auf begründeten Antrag kann die Öffentlichkeit zu einzelnen Tagesordnungspunkten ganz oder teilweise ausgeschlossen werden und die Tagesordnungspunkte nichtöffentlich behandelt werden.
                \item[(4)] Nachdem ein Tagesordnungspunkt unter Ausschluss der Öffentlichkeit oder nichtöffentlich nach Abs. 2  oder 3 behandelt wurde, kann der Studierendenrat beschließen, den Tagesordnungspunktganz oder teileweise als öffentlich zu behandeln und entsprechende ins Protokoll aufzunehmen.
                \item[(5)] Über Angelegenheiten, die nichtöffentlich oder unter Ausschluss der Öffentlichkeit behandelt werden, sind alle Anwesenden zur Verschwiegenheit verpflichtet.  
            \end{itemize}
        \paragraph{§8 Einberufung von Sitzungen und Sitzungstermine}
            \begin{itemize}
                \item[(1)] Sitzungen des Studierendenrat (StuRa) finden inder Vorlesungszeit in der Regel alle zwei Wochen, mindestens jedoch einmal im Monat statt. Außerplanmäßige Sitzungen können vorgesehen werden.
                \item[(2)] Uhrzeit und Wochentag der Sitzungen sollen nachMöglichkeit gleich bleiben. 
                \item[(3)] Termine der einzelnen Sitzungen sind spätestensvier Wochen im Voraus bekannt zu geben. 
                \item[(4)] Die Sitzungsleitung (oder gemäß § 2 der Wahlausschuss) lädt zu den Sitzungen des StuRa ein.Dies geschieht grundsätzlich per E-Mail an die Mitglieder des StuRa. Für die Weitergabe der Einladung an etwaige Stellvertrer*innen ist das Mitglied selbst verantwortlich. 
                \item[(5)] Eine Sitzung beginnt am angegebenen Sitzungstermin mit der Eröffnung durch die Sitzungsleitung oder nach § 2 durch den Wahlausschusses und endet spätestens um 24 Uhr. 
                \item[(6)] Ist die Tagesordnung zum Ende der Sitzung nichtvollständig behandelt, so vertagen sich die übriggebliebenen Tagesordnungspunkte auf die nächste Sitzung. 
                \item[(7)] Sondersitzungen werden einberufen
                    \begin{itemize}
                        \item[a.] auf Beschluss der Sitzungsleitung,
                        \item[b.] auf Antrag von mindestens zehn ordentlich stimmberechtigten Mitgliedern des StuRa oder
                        \item[c.] auf Antrag von mindestens zehn Mitgliedern der Referatekonferenz. 
                    \end{itemize}
                \item[(8)] Die Einladung zur Sondersitzung muss mindestenseine Woche im Voraus auf übliche Weise erfolgen.
                \item[(9)] Wird der Antrag auf eine Sondersitzung von mindestens einem Drittel der ordentlich stimmberechtigten Mitglieder des Studierendenrats oder der Refkonf mit besonderer Dringlichkeit gestellt, so kann eine Sondersitzung auch mit einer Frist vonnur drei Tagen einberufen werden. 
            \end{itemize}
        \paragraph{§9 Alternative Sitzungsformen}
            \begin{itemize}
                \item[(1)] In besonderen Situationen kann das Präsidium (oder gemäß § 2 der Wahlausschuss) StuRa-Sitzungen als Videokonferenz durchführen. Als besondere Situation gelten insbesondere außergewöhnliche Lagen, in denen eine Präsenzsitzung nicht möglich, verhältnismäßig oder zulässig ist, insbesondere, wenn Gesetze oder gerichtliche oder behördliche Entscheidungen ein Zusammentreten vor Ort verhindern. Darüberhinaus gilt die vorlesungsfreie Zeit als besondere Situation, wenn davon ausgegangen werden kann, dass die meisten Mitglieder sich nicht vor Ort aufhalten.
                \item[(2)] Die Sitzung kann auch unter teilweiser Präsenz der Mitglieder des Gremiums und Zuschaltung einzelner Mitglieder über Telefon und / oder Video durchgeführt werden (Hybridsitzung).
                \item[(3)] Die Entscheidung über die Durchführung einer Videokonferenz oder Hybridsitzung trifft die Sitzungsleitung. Dabei muss die gewählte Form eine zu einer Präsenzsitzung im Wesentlichen vergleichbare gleichzeitige und gemeinsame Willensbildung des Gremiums ermöglichen.
                \item[(4)] Für die Durchführung der Sitzung gelten die Regelungen gemäß § 7. Zusätzlich sind mit der Einladung die Zugangsdaten zur Sitzung mitzuteilen.
                \item[(5)] Zur Abstimmung und Wahl wird ein vom EDV-Referat in Absprache mit der Sitzungsleitung ausgewähltes digitales Tool verwendet, welche den Voraussetzungen für Abstimmungen und Wahlen entspricht.
                \item[(6)] Sitzungen von Ausschüssen und Kommissionen der VS können ohne Vorliegen besonderer Situationen alsVideokonferenz oder Hybridkonferenz abgehalten werden, wenn die Mitglieder zustimmen und so eine Teilnahme aller Mitglieder und eine größere Öffentlichkeit ermöglicht wird.
            \end{itemize}
        \paragraph{§10 Tagesordnung und Anträge}
            \begin{itemize}
                \item[(1)] Die Sitzungsleitung (oder gemäß § 2 der Wahlausschuss) erarbeitet für jede Sitzung einen Vorschlag für die Tagesordnung. Diese basiert auf nicht-behandelten Tagesordnungspunkten vergangener Sitzungen, neuen Anträgen, Berichten und Kandidaturen.
                \item[(2)] Die vorläufige Tagesordnung ist mindestens dreiTage vor der Sitzung bekannt zu geben.
                \item[(3)] Anträge zur Tagesordnung müssen sechs Tage vor der Sitzung eingereicht werden. 
                \item[(4)] andidaturen können auch während der Sitzung erfolgen. Die schriftliche Kandidatur muss spätestens drei Tage später bei der Sitzungsleitung nachgereicht werden, sonst ist sie ungültig.
                \item[(5)] Die Aufnahme weiterer Tagesordnungspunkte durch die Sitzungsleitung ist im Ausnahmefall bis 48 Stunden vor Sitzungsbeginn möglich. Nach der Bekanntgabe der vorläufigen Tagesordnung gemäß Absatz 2 können Punkte jedoch nur dann in die Tagesordnung aufgenommen werden, wenn die betreffende Angelegenheit unvorhergesehen war und ihre Behandlung keinen Aufschub duldet.
                \item[(6)] Anträeg auf Aufnahme neuer Tagesordnungspunkte kann zusätzlich im StuRa zu Beginn beantragt werdenund wird mit einfacherer Mehrheit beschlossen. Diesbeinhaltet die Aufnahme und das Entfernen sowie Verschieben von Tagesordnungspunkten.
                \item[(7)] Die beschlossene Tagesordnung muss mindestens enthalten:
                    \begin{enumerate}
                        \item die Genehmigung der vorliegenden Protokolle vorausgegangener Sitzungen,
                        \item Einen Bericht der Vorsitzenden über die Tätigkeiten und Beschlüsse der Referatekonferenz,
                        \item einen Tagesordnungspunkt „Sonstiges“.
                    \end{enumerate}
                \item[(8)] Anträge müssen grundsätzlich einen Antragstitel, eine*n Antragssteller*in, einen Hinweis auf die Antragsart, einen ausformulierten Antragstext und eine Begründung beinhalten. Anträge zu Ordnungen und Satzungen müssen den alten sowie neuen Text enthalten (Synopse). Andernfalls sind Anträge von der Sitzungsleitung zwingend zurückzuweisen und abzulehnen.
                \item[(9)] Bei Finanzanträgen ist vorab das Finanzreferat zu informieren.
                \item[(10)] Bei Anträgen, die einen Bezug zum Arbeitsbereich einer oder mehrerer Referate haben, sind diese vorab in Kenntnis zu setzen.
                \item[(11)] Bei Anträgen zu Ordnungen und Satzungen muss mit der Rechtsabteilung der Universität konsultiertwerden.
                \item[(12)] Änderungsanträge zu Anträgen müssen ausformuliert eingereicht werden. Aus dem Antrag müssen Antragsteller*in und der genaue Änderungstext hervorgehen. Änderungen zu Kleinigkeiten, insbesondere redaktionelle Änderungen, können mündlich während der Sitzung erfolgen. 
            \end{itemize}
        \paragraph{§11 Ablauf der Sitzung}
            \begin{itemize}
                \item[(1)] Die Sitzungsleitung stellt fest, wann die Behandlung eines Tagesordnungspunktes, die Durchführung einer Wahlhandlung oder einer Abstimmung beginnt und endet.
                \item[(2)] Die Sitzungsleitung erteilt das Wort. Sie kann die Redezeit begrenzen. Sie kann Redner*innen zur Sacheund zur Ordnung rufen. Kommt eine Person diesem Ruf nicht nach, kann das Wort entzogen werden und die Person ggf. des Sitzungssaales bzw. der Video-/Audiokonferenz verwiesen werden.
                \item[(3)] Bei Meinungsverschiedenheiten und Zweifeln überdie Auslegung dieser Geschäftsordnung entscheidet die Sitzungsleitung. Gegen diese Entscheidung kann Widerspruch eingelegt werden. In diesem Fall entscheidet der Studierendenrat mit einfacherer Mehrheit. 
            \end{itemize}
        \paragraph{§12  Redeliste}
            \begin{itemize}
                \item[(1)] Die Sitzungsleitung führt eine Redeliste.
                \item[(2)] Die Redeliste ist zuerst nach Erstredner*innen und danach nach geschlechtlicher Selbstzuordnung zu quotieren. 
                \item[(3)] Für jeden Tagesordnungspunkt wird eine eigene Redeliste geführt. 
            \end{itemize}
        \paragraph{§13 Anträge zur Geschäftsordnung (GO-Anträge)}
            \begin{itemize}
                \item[(1)] Anträge zur Geschäftsordnung werden durch das Heben beider Arme oder durch ein mit der Sitzungsleitung vereinbartes Zeichen angezeigt. 
                \item[(2)] Anträge zur Geschäftsordnung werden unverzüglich nach Beendigung des laufenden Wortbeitrags behandelt. Sie  dürfen sich nur auf eine Sache beziehen und müssen knapp gehalten werden.
                \item[(3)] Nach Aufruf des GO-Antrags besteht die Möglichkeit einer formalen oder inhaltlichen Gegenrede.
                    \begin{enumerate}
                        \item Erfolgt keine Gegenrede, so gilt der Antrag als angenommen und muss sofort umgesetzt werden. 
                        \item Erfolgt inhaltliche Gegenrede, so darf eine Person ihre inhaltlichen Einwände gegen den Antrag vorbringen. Anschließend wird über den Antrag abgestimmt.
                        \item Erfolgt formale Gegenrede, so stimmt der Studierendenrat direkt über den GO-Antrag ab. 
                    \end{enumerate}
                \item[(4)] Anträge zur Geschäftsordnung werden sofern nicht anders vermerkt mit einer einfachen Mehrheit beschlossen.
                \item[(5)] Anträge zur Geschäftsordnung sind insbesondere:
                    \begin{enumerate}
                        \item Antrag auf Vorziehen oder Zurückstellen eines Tagesordnungspunkts;
                        \item Aufnahme eines Tagesordnungspunktes;
                        \item Antrag auf Nichtbefassung mit einem Antrag oder Tagesordnungspunkt (Beschluss mit 2/3-Mehrheit);
                        \item Antrag auf Vertagung eines Antrags oder Tagesordnungspunkts;
                        \item Antrag auf Verlängerung der Beratungszeit;
                        \item Antrag zur Begrenzung der Redezeit;
                        \item Antrag auf Schließung der Redeliste: Bei Annahme wird den Mitgliedern noch ermöglicht, sich auf die Redeliste setzen zu lassen;
                        \item Antrag auf Wiedereröffnung der Redeliste;
                        \item Antrag auf sofortigen Schluss der Debatte;
                        \item Antrag auf geheime Abstimmung (Beschluss mit absoluter Mehrheit);
                        \item Antrag auf namentliche Abstimmung mit Zugehörigkeit zu Studienfachschaft oder Liste im Protokoll vermerkt;
                        \item Antrag auf erneute Auszählung einer Abstimmung oder Wahl;
                        \item Antrag auf Ausschluss der Öffentlichkeit (Beschluss mit absoluter Mehrheit);
                        \item Antrag auf temporäre Ablösung der Sitzunsgleitung: Für entweder einen Tagesordnungspunkt oder eine gesamte Sitzung aufgrund potentieller Befangenheit oder fehlender Neutralität. Ein Mitglied aus dem Plenum übernimmt die Aufgaben der Sitzungsleitung für den weiteren Zeitraum ihrer Ablösung;
                        \item Antrag auf Ablösung der*des Protokollführende*n; Bei begründeten Zweifeln an der Objektivität oder der Fähigkeit des*der Protokollführenden, die ihm*ihr übertragenen Aufgaben korrekt auszuführen, kann diese Person durch eine andere Person abgelöst werden;
                        \item Antrag auf Unterbrechung der Sitzung;
                        \item Antrag auf Feststellung der Beschlussfähigkeit.
                    \end{enumerate}
                \item[(6)] Geheime Abstimmung (Abs. 5, Satz 10) und namentliche Abstimmung (Abs. 5, Satz 11) schließen einander aus.
                \item[(7)] Die Vertagung eines Antrags  (§ 12 Abs. 4 S. 4) ist nur zweimal möglich. Ist der Antrag trotz zweier Vertagungen nicht abschließend behandelt, so wird er von der Tagesordnung gestrichen.
                \item[(8)] Die Beratungszeit eines Antrags, gemäß § 12 Abs. 4 Satz 5, kann maximal zweimal verlängert werden. Nach der zweiten Verlängerung der Beratungszeit muss der Antrag abgestimmt oder von der Tagesordnung gestrichen werden.
                \item[(9)] Bei allen Geschäftsordnungsanträgen sind zusätzlich die beratenden Mitglieder des Studierendenrats stimmberechtigt.
            \end{itemize}
        \paragraph{aufgehoben}
        \paragraph{§14 Persönliche Erklärungen}
            \begin{itemize}
                \item[(1)] Nach Abschluss eines Tagesordnungspunktes können Mitglieder des StuRa per Wortmeldung eine persönliche Erklärung abgeben, um diese ins Protokoll aufnehmen zu lassen. Hierfür ist pro Person ein Zeitraum von drei Minuten gestattet.
                \item[(2)] Die persönliche Erklärung ist der*dem Protokollführenden anschließend schriftlich zu überreichen oder bis zur nächsten ordentlichen Sitzung nachzureichen und von der Sitzungsleitung dem Protokoll anzufügen.
            \end{itemize}
        \paragraph{§15 Feststellung der Beschlussfähigkeit}
            \begin{itemize}
                \item[(1)] Der Studierendenrat ist beschlussfähig, wenn die Voraussetzungen gemäß § 25 Abs. 1 OrgS erfüllt sind. Die Sitzungsleitung stellt dies zu Beginn der Sitzung fest.
                \item[(2)] Die Beschlussunfähigkeit kann im Verlauf der Sitzung nur auf Antrag eines stimmberechtigten Mitglieds des StuRa festgestellt werden.
                \item[(3)] Bei Feststellung mangelnder Beschlussfähigkeit,wird die Sitzung von der Sitzungsleitung umgehend beendet. Verbleibende Tagesordnungspunkte und für diese bereits angenommene GO-Anträge werden auf die Tagesordnung der nächsten Sitzung übertragen. 
                \item[(4)] Tagesordnungspunkte können nur einmal aufgrund von mangelnder Beschlussfähigkeit verschoben werden. Entsprechende Tagesordnungspunkte können in der darauffolgenden Sitzung unabhängig von den Vorgaben für Beschlussfähigkeit nach § 14 Abs. 1 behandelt werden. Zu erreichende Quoren werden auf die tatsächlichen anwesenden Mitglieder angewandt.
                \item[(5)] on Abs. 4 sind Änderungen der Organisationssatzung der VS ausgenommen.
                \item[(6)] Anträge nach § 14 Abs. 4 müssen auf der Tagesordnung der kenntlich gemacht werden.
            \end{itemize}
        \paragraph{§16 Abstimmungsregeln}
            \begin{itemize}
                \item[(1)] Bei Präsenzsitzungen wird durch das Heben der Stimmkarte abgestimmt, sofern durch GO-Antrag kein anderes Abstimmungsverfahren beschlossen wurde.
                \item[(2)] Bei digitalen Sitzungen stellt die Sitzungsleitung in Zusammenarbeit mit dem EDV-Team Möglichkeiten zur Abstimmung zur Verfügung. Hierbei muss ebenfalls die Möglichkeit zur geheimen oder namentlichen Abstimmung bestehen.
                \item[(3)] In der Regel wird mit einfacher Mehrheit beschlossen, sofern die Organisationssatzung, die Wahlordnung oder diese Geschäftsordnung keine anderen Mehrheiten vorsieht. 
                \item[(4)] Bei Stimmengleichheit der Ja- und Nein-Stimmen gilt der Antrag als abgelehnt. 
                \item[(5)] Für die Ermittlung von Mehrheiten gilt § 45 OrgS. Für die Durchführung von Wahlen gilt die Wahlordnung
            \end{itemize}
    \paragraph{III. Anträge und ihre Behandlung}
        \paragraph{§17 Beratungen}
            \begin{itemize}
                \item[(1)] Anträge werden generell in zwei Lesungen behandelt, sofern nicht anders festgelegt. In der ersten Lesung wird der Antrag vorgestellt und beraten und nach der zweiten Lesung abgestimmt.
                \item[(2)] In einer Lesung werden behandelt:
                    \begin{enumerate}
                        \item Finanzanträge unter 500 Euro;
                        \item Inhaltliche Positionierungen und allgemeine Beschlüsse zu Verhandlungs- und Vorgehensweisen, welche zur Basis bereits bestehende Beschlüsse haben;
                    \end{enumerate}
                \item[(3)] Der Studierendenrat kann bei Anträgen, welche zwei Lesungen  benötigen, auf die zweite Lesung aufAntrag verzichten, sofern es zwingend dringliche Gründe gibt (Dringlichkeit).
                \item[(4)] Die Dringlichkeit eines Antrags wird zusammen mit der Einreichung des Antrags beantragt.
                \item[(5)] Die Dringlichkeit kann mit Begründung auch während der Sitzung noch beantragt werden.
                \item[(6)] ür den Beschluss der Dringlichkeit ist eine Mehrheit von zwei Dritteln notwendig.
                \item[(7)] Dringlichkeit ist niemals bei Änderungen oder Neufassungen der Organisationssatzung zulässig.        
            \end{itemize}
        \paragraph{gestrichen}
    \paragraph{IV. Beurkundung der Beschlüsse und ihre Anfechtung}        
        \paragraph{§18 Protokoll}
            \begin{itemize}
                \item[(1)] Während jeder Sitzung des Studierendenrats wirdein (vorläufiges) Protokoll geführt.
                \item[(2)] Das vorläufige Protokoll ist nach der Sitzung der Sitzungsleitung zu übergeben, welche es aufbereitetund fertigstellt.
                \item[(3)] Ein Protokoll enthält mindestens:
                    \begin{enumerate}
                        \item Datum, Beginn und Ende der Sitzung;
                        \item Namen der*des Protokollführenden; 
                        \item Die Anwesenheitsliste (Mitgliederliste); 
                        \item Wortlaut der vorgestellten und beschlossenen Anträge sowie ggf. das Abstimmungsergebnis über diese;
                        \item Den groben Verlauf und inhaltlichen Abriss der Wortbeiträge, insbesondere der Diskussionen; 
                        \item Persönliche Erklärungen.
                    \end{enumerate}
                \item[(4)] Für nicht-öffentliche Tagesordnungspunkte wird ein nicht-öffentliches Protokoll geführt. Die Einsicht in dieses ist den Mitgliedern vor Ort beim Studierendenrat möglich. 
                \item[(5)] Das öffentliche Protokoll wird als noch nicht bestätigte Fassung den Mitgliedern innerhalb einer Woche nach Ende der Sitzung per Mail verfügbar gemacht und auf der Webpräsenz des Studierendenratsveröffentlicht. Bis zur nächsten Sitzung können Mitglieder der Sitzungsleitung Änderungen und Verbesserungsvorschläge unterbreiten, die diese aufgreifen kann und eine neue Fassung erstellen kann.
                \item[(6)] Werden zu Beginn keine Einwände gegen das Protokoll erhoben, so gilt es als angenommen.
                \item[(7)] Zu Beginn der Sitzung können gegen noch nicht bestätigte Protokolle Einsprüche erhoben werden. Wird diesen zugestimmt, wird das Protokoll von der Sitzungsleitung bis zur nächsten Sitzung korrigiert und in der neuen Fassung erneut zu Abstimmung gestellt.
                \item[(8)] Bereits korrigierte Protokolle können nach demselben Verfahren solange erneut korrigiert werden, bis sie bestätigt werden.
                \item[(9)] Nach Bestätigung des Protokolls wird das Datum der Bestätigung im Protokoll vermerkt und eine endgültige Fassung auf der Website hochgeladen.
            \end{itemize}
        \paragraph{§19 Anfechtung von Sitzungen}
            \begin{itemize}
                \item[(1)] Binnen vierzehn Tagen nach der Genehmigung des Protokolls einer Sitzung des Studierendenrats (StuRa) kann die Sitzung bei der Schlichtungskommission (SchliKo) angefochten werden.
                \item[(2)] Angefochten werden kann eine Sitzung des StuRa nur von einem stimmberechtigten Mitglied des StuRa und auf Grundlage eines Vorwurfs, dass eine Sitzungnicht ordnungsgemäß einberufen oder geleitet wordenist oder es Unregelmäßigkeiten bei Abstimmungen undWahlen gab.
                \item[(3)] Nach der Beratung über die Anfechtung spricht die SchliKo dem StuRa in Form eines Berichts eine Empfehlung aus, ob Beschlüsse oder Wahlen für nichtig zu befinden sind.
                \item[(4)] Der StuRa beschließt im Anschluss über die Empfehlung der SchliKo mit einfacherer Mehrheit undentscheidet ggf. Unmittelbar erneut über aufgehobene Anträge oder Wahlen. 
            \end{itemize}
    \paragraph{V. Schlussbestimmungen}+
        \paragraph{§20 Anwendung dieser Geschäftsordnung auf Ausschüsse und Kommissionen und dezentrale Organe}
            \begin{itemize}
                \item[(1)] Diese Geschäftsordnung findet auch auf Ausschüsse und Kommissionen auf zentrale Ebene der Verfassten Studierendenschaft Anwendung, sofern diese keinen eigene Geschäftsordnung haben oder Beschlüsse zu Verfahrensfragen gefasst haben. Dem steht eine langanhaltende und für jedermann erkennbare Übung gleich.
                    \begin{enumerate}
                        \item Abweichend von den Regelungen für den Studierendenrat können Fristen maximal um die Hälfte verkürzt werden und Abstimmungen ohne Stimmkarte durchgeführt werden. 
                        \item Sitzungen sind in geeigneter Weise mindestens fünf Tage vorher öffentlich anzukündigen.
                        \item Die konstituierende Sitzung eines Ausschusses bzw. einer Kommission wird durch eines ihrer Mitglieder in Absprache mit den übrigen Mitgliedern einberufen, sofern nicht ein Vorsitz bzw. eine Sitzungsleitung (beispielsweise von Amts wegen) bestimmt.
                        \item Erfolgt eine Konstituierung auch nach Aufforderung durch die Vorsitzenden der Verfassten Studierendenschaft nicht binnen eines Monats, wird die Sitzung durch die Vorsitzenden der VS einberufen und bis zur Bestimmung einer Sitzungsleitung oder einesVorsitzes von einem*einer Vorsitzenden der VS oder einer von ihnen bestimmten Person geleitet. 
                    \end{enumerate}
                \item[(2)] Diese Geschäftsordnung findet auch auf Organe der dezentralen Ebene (Gremien der Studienfachschaften)Anwendung, sofern diese keinen eigenen Regelungen in der Studienfachschaftssatzung oder einer Geschäftsordnung haben oder Beschlüsse zu Verfahrensfragen gefasst haben. Dem steht eine langanhaltende und für jedermann erkennbare Übung gleich.
                    \begin{enumerate}
                        \item Abweichend von den Regelungen für den Studierendenrat können Fristen maximal um die Hälfte verkürzt werden und Abstimmungen ohne Stimmkarte durchgeführt werden. 
                        \item Sitzungen sind in geeigneter Weise mindestens vier Tage vorher öffentlich anzukündigen. 
                        \item Die konstituierende Sitzung eines Organs auf Fachschaftsebene wird durch eines ihrer Mitglieder in Absprache mit den übrigen Mitgliedern einberufen, sofern es keine eigene Regelung gibt.
                    \end{enumerate}
            \end{itemize}
        \paragraph{§21 Abweichungen von dieser Geschäftsordnung}
            Abweichungen von den Vorschriften dieser Geschäftsordnung können im Einzelfall vom Studierendenrat mit einer Mehrheit von zwei Dritteln, mindestens aber mit der Mehrheit der ordentlich stimmberechtigten Mitglieder des Studierendenrates,beschlossen werden, sofern die Bestimmungen der OrgS oder andere rechtliche Bestimmungen dem nicht entgegenstehen.
\subsection{Wahlordnung\label{appendix:5}}