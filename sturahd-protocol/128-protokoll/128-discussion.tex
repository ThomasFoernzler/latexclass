

1 Begrüßung
}

2 Tagesordnung

2.1 Änderungen an der To
\textbf{X. Lesung:}
\ul{
	\li{Keine Änderungsanträge}
}

3 Protokolle
	\li{keine Änderungsanträge}
}

4 Berichte

4.1 VS in der Vl-freien Zeit
\textbf{X. Lesung:}
\ul{
}

Referat für politische Bildung
\textbf{X. Lesung:}
\ul{
	\li{keine Fragen}
}

Referat Lehre und Lernen
\textbf{X. Lesung:}
\ul{
	\li{keine Fragen}
}

Außenreferat
\textbf{X. Lesung:}
\ul{
	\li{keine Fragen}
}

EDV Referat
\textbf{X. Lesung:}
\ul{
	\li{keine Fragen}
}

Referat für hochschulpolitische Vernetzung
\textbf{X. Lesung:}
\ul{
	\li{Es gab die Rückmeldung, dass die Fragen nicht divers genug gestellt waren.}
		\noli{\ul{
		\lii{Die Rückmeldung wird weitergegeben. 
		}}}
}

Bericht des EDV Referats
\textbf{X. Lesung:}
\ul{
	\li{keine Fragen}
}

4.4 Bericht des Vorsitzes
\textbf{X. Lesung:}
\ul{
	\li{keine Fragen}
}

GO Antrag auf Aufname eines TOP's
\textbf{X. Lesung:}
\ul{
	\li{ohne Gegenrede angenommen}
}

Bericht des Verkehrsreferats
\textbf{X. Lesung:}
\ul{
	\li{keine Fragen}
}

5 Satzungen und Ordnungen

5.1 Neufassung der Studienfachschaftssatzung UFG/VA (5. Lesung)
\textbf{X. Lesung:}
\ul{
	\li{keine Fragen}
	\li{angenommen mit 61/0/1}
}

5.2 Fusion der Fachschaften Klassische Archäologie und Byzantinische Archäologie und Kunstgeschichte (5. Lesung)
\textbf{X. Lesung:}
\ul{
	\li{keine Fragen}
	\li{angenommen mit 60/0/1}
}

5.3 Satzung der neuen Fachschaft Klassische und Byzantinische Archäologie (5. Lesung)
\textbf{X. Lesung:}
\ul{
	\li{keine Fragen}
	\li{angenommen mit 60/0/1}
}

5.4 Satzungsänderungen(generelle Fragen zu allen Satzungen, ich schaue wie ich das nachher ins Protokoll mach)
\textbf{X. Lesung:}
\ul{
	\li{In der Geschäftsordnung steht, dass Ordnungsänderungen der Rechtsaufsicht vorgelegt werden müssen. Wurde das gemacht?}
		\noli{\ul{
		\lii{Ja wurde es
		}}}
	\li{In der Aufwandsentschädigungsordnung steht eine sehr hohe Erhöhung der Aufwandsentschädigung von 35e zu 460€. Warum?}
		\noli{\ul{
		\lii{Es ist sehr viel liegengeblieben. Die Sitzungsleitung ist ziemlich überlastet.
		}}}
		\noli{\ul{
		\lii{Es ist sehr günstig, wenn man bedenkt wie viel Arbeit das ist. Wenn das nicht die Sitzungleitung macht, dann müssen das die Angestellten der Universität machen, was um einiges teurer wird.
		}}}
		\noli{\ul{
		\lii{Die 465€ sind pro Sitzung und die 35€ sind pro Sitzung pro Mitglied der SL
		}}}
	\li{Es erschließt sich nicht ganz warum die Satzungen geändert wurden}
		\noli{\ul{
		\lii{Die Leute die das hinterher nutzen studieren nicht alle Jura. Für diese sind die 
		}}}
	\li{Wenn die SL des Sturas Präsidium heißt, dann sollte man das vermerken. Bei dem Vorsitz soll man den Zwang zur Quotierung in ein soll ändern.}
		\noli{\ul{
		\lii{Die Vorsitzenden sollen verschiedenes Geschlecht haben, aber bei 
		}}}
	\li{Aber warum wird das dann in ein soll geändert?}
		\noli{\ul{
		\lii{Weil es hier besser funktioniert wenn es keinen eindeutigen Zwang zu der Frauenquote gibt weil das zu Leuten führt, die den Job nur für die Quote machen.
		}}}
		\noli{\ul{
		\lii{Hier ist die Festlegung auf zwei Geschlechter auch suboptimal. Weil nonbinäre Geschlechter nicht gemeint sein müssen.
		}}}
	\li{Die Aufweichung der Frauenquote ist problematisch.}
	\li{In der Theorie sind zwei Menschen die den Vorsitz machen gut, aber in der Praxis ist dem Stura zuzutrauen hier zwischen Geschlechterrepräsentation und der besten Person zu entscheiden.}
	\li{Die Satzung ist in zweiter Lesung.}
	\li{Wir sollten uns an diesem Thema nicht zu lange aufhalten. Wer ist Antragssteller*in? }
		\noli{\ul{
		\lii{Kirsten, Harald, bei der Wahlordnung die Jusos,
		}}}
	\li{Haben alle hier die Sitzungsunterlagen gut durchgelesen? Kann man hier ein Meinungsbild machen?}
		\noli{\ul{
		\lii{Es ist die Aufgabe der Anwesenden sich diese Sachen durchzulesen.
		}}}
	\li{Es steht im Text um divers und in dieser Diskussion ist das auf Geschlecht bezogen. Im Text ist das hingegen sehr schwammig. }
		\noli{\ul{
		\lii{Im gerade diskutierten Textabschnitt kommt diese Formulierung nicht vor.
		}}}
	\li{Gibt es die Möglichkeit, drei Optionen zu haben? Also neuer Antrag mit altem Text, neuem Text und eben keine Meinung?}
		\noli{\ul{
		\lii{Ja man kann Änderungsanträge stellen.
		}}}
	\li{Wenn wir die Quotierung nur anstreben ist das ein Rückschritt. Aber es ist schwierig einen Änderungsantrag hierzu zu schreiben. Deswegen sollten wir uns mit den Referaten zusammensetzen.}
	\li{Eine Quotierung ist grundlegend gut. Aber wir müssen besser Quotierungen umsetzen durch bessere Suche nach quotierten Spitzen.}
	\li{Wir müssen über dieses Thema länger reden. Können wir dieses Thema auf die nächste Sitzung vertagen?}
	\li{Den gewählten studentischen Senatoren sollte man einen beratenden Sitz im Stura geben. Wir müssen den studentischen Senatoren die Beschlüsse aus dem Stura vorlegen und das wäre viel weniger Aufwand so.}
	\li{War hier angedacht, wenn Mitglieder des Studierendenrats ein Semester nicht kommen, als passiv eingestuft werden.}
		\noli{\ul{
		\lii{Es gab Listen die in der ganzen Legislatur nur einmal aufgetaucht sind. Wenn Listen aber nicht kommen, dann ist das für die Beschlussfähigkeit schlecht. Zusätzlich will man das aktive Mitglieder abstimmen und nicht uninformierte 
		}}}
	\li{Bei der Aufwandsentschädigungsordnung soll an einer Stelle gegendert werden, die das Verständnis erschwert & es wird zu einer "kann AE erhalten"- Regelung. Auch sollte die Aufwandsentschädigung pro Person gedeckelt werden.}
		\noli{\ul{
		\lii{Die vorherige Formulierung war ungewolt mehrdeutig. 
		}}}
	\li{Man wollte die Satzungen lesbarer machen. Ist das durch das gendern hier notwendig?}
	\li{Ist es nicht schweirig den studentischen Senator*innen Stimmrecht zu geben, weil wie ist das mit einem Sturamitglied, das auch studentische*r Senator*in ist. }
		\noli{\ul{
		\lii{Der Zweck ist, dass man schneller einreichen will 
		}}}
	\li{Zu Senatsmitgliedern in der StuRa Sitzung: Wenn sie für Repräsentation in die Sitzung kommen sollen. Repräsentieren diese den Senat?}
		\noli{\ul{
		\lii{Man sollte laut LHG das machen. Die studentischen Senator*innen müssen nicht unbedingt kommen, aber sie würden als Mitglied des Sturas automatisch Auskunft über die Sitzungsunterlagen bekommen. Das löst Verfahrensfragen. 
		}}}
	\li{Man sollte die verschiedenen Satzungen einzeln diskutieren statt alle kreuz und quer.}
		\noli{\ul{
		\lii{Bei Aufwandsentschädigung 12 (4) und 12 (5) soll gestrichen werden. Das wirkt wie ein Checks und Balances System. Aber in der Realität hat das schon öfters zu Streit zwischen den Referaten geführt.
		}}}
		\noli{\ul{
		\lii{Der Stura hat genug zu tun. Er kann sich auch nicht mit Referaten befassen die ihre Arbeit nicht erledigen. Es scheint ein bisschen bevormundent, wenn das Finanzreferat das entscheidet. Aber in der Realität müsste das gut funktionieren.
		}}}
	\li{Wenn eine Person Aufwandsentschädigung abruft, dann weiß das der Stura nicht. Das Finanzreferat aber schon. Die gefundene Variante für die Nachprüfung der Arbeit der Referate war dass die Referate regelmäßig darüber berichten. Dies ist aber in dem Stura nicht zufriedenstellend geschehen.}
	\li{Der Paragraph sollte nicht raus weil hier immernoch über Geld geredet wird. Das Geld wird ja auch von versch. Personen bezahlt.}
		\noli{\ul{
		\lii{Hier schiebt der StuRa Verantwortung ab. Das ist nicht gut.
		}}}
}

5.5 Änderung der Schlichtungsordung (2. Lesung)
\textbf{X. Lesung:}
\ul{
}

GO Antrag auf Verlängerung der Beratungsfrist der Schlichtungsordung
\textbf{X. Lesung:}
\ul{
}

5.6 Änderung der Aufwandsentschädigungsordnung (2. Lesung)
\textbf{X. Lesung:}
\ul{
}

GO Antrag auf Verlängerung der Beratungsfrist der Aufwandsentschädigungsordnung
\textbf{X. Lesung:}
\ul{
	\li{angenommen mit 37/3/5}
}

5.7 Änderung der Geschäftsordnung (2. Lesung)
\textbf{X. Lesung:}
\ul{
}

GO Antrag auf Verlängerung der Beratungsfrist der Geschäftsordnung
\textbf{X. Lesung:}
\ul{
}

5.8 Änderung der Wahlordnung (2. Lesung)
\textbf{X. Lesung:}
\ul{
	\li{angenommen mit 36/3/3}
}

GO Antrag auf Verlängerung der Beratungsfrist der Wahlordnung
\textbf{X. Lesung:}
\ul{
}

5.9 Änderung der Organisationssatzung (2. Lesung)
\textbf{X. Lesung:}
\ul{
}

GO Antrag auf Verlängerung der Beratungsfrist der Organisationssatzung
\textbf{X. Lesung:}
\ul{
	\li{Für welche Satzungen sollte man das machen?}
		\noli{\ul{
		\lii{eigentlich alle. Es gibt in allen zu viele Baustellen.
		}}}
	\li{Bei Organisationssatzung kann man das machen. Die Wahlordnung sollte man heute abstimmen. Die Schlichtungsordnung auch aber das ist nicht so dringlich wie die Wahlordnung.}
		\noli{\ul{
		\lii{Ohne gegenrede angenommen
		}}}
}

GO Antrag auf Verlängerung der Beratungsfrist Burschenschaften
\textbf{X. Lesung:}
\ul{
	\li{formale Gegenrede}
}

6 Wahlen

Kandidatur für das Referat für Lehre und Lernen
\textbf{X. Lesung:}
\ul{
	\li{gewählt mit 45/1/4}
}

Kandidatur für das EDV-Referat
\textbf{X. Lesung:}
\ul{
	\li{gewählt mit 44/1/4}
}

Kandidatur für die M-N Gesamtfakultät
\textbf{X. Lesung:}
\ul{
	\li{Ole Klarhof gewählt mit 41/2/4}
	\li{Alexander Riemer gewählt mit 42/2/3}
}

Kandidatur als Vertreter*innen in der Kommission für die Marsilius-Studien
\textbf{X. Lesung:}
\ul{
	\li{Christian Heusel gewählt mit 41/3/1}
	\li{Christoph Blattgerste gewählt mit 42/2/1}
}

6.8 Kandidaturen für die Vertretungsversammlung des Studierendenwerkes Heidelberg
\textbf{X. Lesung:}
\ul{
}

Kandidatur von Anna Pöggeler
\textbf{X. Lesung:}
\ul{
	\li{Die zu wählenden Personen sind 4 und die Kandidierenden sind 5. Ist das eine Kampfkandidatur?}
	\li{Da es eine Kampfkandidatur ist ist die Abstimmung schwierig, deswegen ist die Wahl sehr schwierig. }
	\li{gewählt sind:}
	\li{Vino: 34}
	\li{Leon: 29}
	\li{Magdalena: 29}
	\li{Anna: 28}
	\li{Stellverteter sind}
	\li{Annalena gewählt mit 27/1/5}
	\li{Julian Beier gewählt mit 29/0/2}
	\li{Anna Scherer gewählt mit 27/2/4}
	\li{Simon Kleinhanß gewählt mit 28/0/5}
	\li{Christian Heusel gewählt mit 28/1/3}
	\li{Annalena ist stellvertretende Vorsitzende der Vertretungsversammlung. Diese Kampfkandidatur hat nun die Verhandlungen erschwert weil Annalena ihren Einfluss geltend gemacht hat auf die Tagesordnung, aber jetzt de facto abgewählt wurde. }
}

Kandidatur für das Gremienreferat
\textbf{X. Lesung:}
\ul{
	\li{Was sind deine Pläne das Gremienreferat und die Fachschaften näher zusammenzubringen.}
		\noli{\ul{
		\lii{Bis jetzt keine direkte Erfahrung in Heidelberg, aber als Universitätswechsler hat er Erfahrung mit Organisation.
		}}}
	\li{Er ist Mitglied der Liste und der Partei. Ist die Kandidatur ernsthaft?}
		\noli{\ul{
		\lii{Ja
		}}}
	\li{gewählt mit 34/1/1}
}

Kandidatur für die Härtefallkommission
\textbf{X. Lesung:}
\ul{
	\li{keine Fragen}
}

Kandidatur für die QSM-Kommission
\textbf{X. Lesung:}
\ul{
	\li{keine Fragen}
}

Kandidatur für das SAL
\textbf{X. Lesung:}
\ul{
	\li{keine Fragen}
	\li{Phillip Strehlow Gewählt mit 34/0/0}
	\li{Christian Heusel Gewählt mit 31/0/3}
	\li{Henirke Arnold gewählt mit 33/0/1}
}

Antrag auf Dringlichkeit der SAL Kandidaturen
\textbf{X. Lesung:}
\ul{
	\li{ohne Gegenrede angenommen}
}

7 Diskussionen

7.1 Unvereinbarkeit mit Burschenschaften (vertagt)
\textbf{X. Lesung:}
\ul{
}

7.1.1 Änderungsantrag zu 7.1 (vertagt)
\textbf{X. Lesung:}
\ul{
}

8 Corona Sondersitzung

8.1 Wlan (vertagt)
\textbf{X. Lesung:}
\ul{
}

8.2 Freischuss Medizin
\textbf{X. Lesung:}
\ul{
	\li{Angenommen mit 29/0/2}
}

8.2.1 Änderungsantrag zu 8.2
\textbf{X. Lesung:}
\ul{
	\li{Warum wurde das geändert?}
		\noli{\ul{
		\lii{Zum besseren Verständnis zumindest wurde es so verstanden.
		}}}
	\li{In Medizin hat man pro Klausur X Versuche und somit kann diese Änderung schon Sinn ergeben.}
	\li{angenommen mit 27/0/2}
}

9 Sonstiges

9.1 StuRa-Termine für das SoSe
\textbf{X. Lesung:}
\ul{
	\li{angenommen mit 32/0/1}
}