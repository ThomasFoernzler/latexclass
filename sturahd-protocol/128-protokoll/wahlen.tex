\section{Kandidaturen und Wahlen}
Die Kandidaturtexte sind aus Datenschutzgründen nur auf der Kandidaturenseite \url{https://www.stura.uni-heidelberg.de/kandidaturen} einzusehen, welche nur aus dem Universitätsnetzwerk oder mit dem VPN der Universität besucht werden kann.

\wahl{Kandidatur für das Referat für Lehre und Lernen}{2. Lesung:}{Victoria Engels}
{
    Der Kandidaturtext findet sich auf der \kandidaturenseite.
}{
	\textbf{1. Lesung:}
	\ul{
	\li{Wie nimmst du die Arbeit des Sturas in puncto Inklusion von Menschen mit Behinderungen wahr? Und was ist dein Eindruck zu dem Gesundheitsreferat in diesem Punkt?}
		\noli{\ul{
		\lii{Es scheint beim Thema Barrierefreiheit keine klare Zuständigkeit bei den einzelnen Themen zu geben. Sie fand es sehr schwierig sich darüber einen Überblick zu verschaffen.
		}}}
	}
	\textbf{2. Lesung:}
	\ul{
	\li{Keine Fragen}
	}
}

\wahl{Kandidatur für das EDV-Referat}{2. Lesung:}{Uli Roth}
{
    Der Kandidaturtext findet sich auf der \kandidaturenseite.
}{
	\textbf{1. Lesung:}
	\ul{
	\li{Wo ist deine Kandidatur?}
		\noli{\ul{
		\lii{Sie ist online jetzt.
		}}}
	\li{In welchen Organisationen ist er?}
		\noli{\ul{
		\lii{Er ist im VVN-BDA, SDS, evtl. Mitglied der Linken,
		}}}
	\li{Gibt es einen Bereich in der EDV in dem er sich gerne stärker oder weniger involvieren würde}
		\noli{\ul{
		\lii{Er kann sich alle Aspekte des EDV-Referats vorstellen.
		}}}
	\li{Hast du generell Zeit dafür?}
		\noli{\ul{
		\lii{Ab Ende Februar hat er Zeit
		}}}
	\li{Welche Programmiersprache(n) interessieren ihn besonders?}
		\noli{\ul{
		\lii{Python, C++, und im letzten Jahr hat er sich mit Python als Webbrowser auseinandergesetzt.
		}}}
	\li{Er ist ja auch dann in der RefKonf Mitglied}
		\noli{\ul{
		\lii{Ja da hat er sich auch schon darüber informiert.
		}}}
	\li{Was ist der Plan wenn man jetzt zu zweit ist.}
		\noli{\ul{
		\lii{Da gibt es direkt keine Antwort drauf. Aber in Coronazeiten geht die EDV-Arbeit nicht wirklich aus.
		}}}
	\li{Wird bei SDS-Veranstaltungen die StuRa-Technik dann auch eingesetzt.}
		\noli{\ul{
		\lii{Wenn es einen Antrag auf Nutzung der Geräte der VS gibt, dann werden diese bereit gestellt.
		}}}
	\li{Hat er an dem Antrag über Burschenschaften mitgeschrieben?}
		\noli{\ul{
		\lii{Nein, aber selbst wenn zählt für ihn die Beschlusslage des Stura's
		}}}
}
}
\wahl{Kandidatur als Vertreter in der Kommission für die Marsilius-Studien}{2. Lesung:}{Alexander Riemer}
{
    Der Kandidaturtext findet sich auf der \kandidaturenseite.
}{
	\textbf{1. Lesung:}
	\ul{\li{Keine Fragen}}
}
\wahl{Kandidatur als Vertreter in der Kommission für die Marsilius-Studien}{2. Lesung:}{Ole Klarhof}
{
    Der Kandidaturtext findet sich auf der \kandidaturenseite.
}{
	\textbf{1. Lesung:}
	\ul{\li{Keine Fragen}}
}
\wahl{Kandidatur für die M-N Gesamtfakultät}{2. Lesung:}{Christian Heusel}
{
    Der Kandidaturtext findet sich auf der \kandidaturenseite.
}{
	\textbf{1. Lesung:}
	\ul{
	\li{Was sind deine politischen Interessen?}
		\noli{\ul{
		\lii{Ich bin in keiner politischen Organisation vertreten. Ich bin aber trotzdem in der Hochschulpolitk aktiv vertreten.
		}}}
}
}
\wahl{Kandidatur für die M-N Gesamtfakultät}{2. Lesung:}{Christoph Blattgerste}
{
    Der Kandidaturtext findet sich auf der \kandidaturenseite.
}{
	\textbf{1. Lesung:}
	\ul{
	\li{Was sind deine politischen Interessen? Wie lange wird er dieses Amt übernehmen?}
		\noli{\ul{
		\lii{Er ist in keiner Liste oder politischen Gruppe aktiv. Er ist lange genug noch für eine Amtszeit an der Universität.
		}}}
	}
}
\wahl{Gemeinsamer Wahlvorschlag StuWe-Vertretungsversammlung}{2. Lesung:}{siehe Auflistung}
{
	Der Studierendenrat wählt die nachfolgend aufgeführten Personen zu Mitgliedern der Vertretungsversammlung des Studierendenwerkes Heidelberg für die bereits laufenden Amtsperiode:\\
	Mitglieder:
	\begin{itemize}
		\item Vionjan Vijeyaranjan
		\item Annalena Wirth
		\item Leon P. Köpfle
		\item Magdalena Schwörer
		\item Anna Pöggler
	\end{itemize}
	Stellvertreter:
	\begin{enumerate}
		\item Julian Beier
		\item Anna Scherer
		\item Simon Kleinhanß
		\item Christian Heusel
	\end{enumerate}
    Der Kandidaturtext findet sich auf der \kandidaturenseite.
}{
	\textbf{1. Lesung:}
	\ul{
		\li{In welchen relevanten politischen Gruppierungen sind alle hier?}
		\noli{\ul{
		\lii{Leon Köpfle: SPD
		}}}
		\noli{\ul{
		\lii{Annalena Wirth: SPD, Verdi
		}}}
		\noli{\ul{
		\lii{Magdalena Schwörer: -
		}}}
		\noli{\ul{
		\lii{Julian Beier: 
		}}}
		\noli{\ul{
		\lii{Anna Scherer: JEF, CDU, JU
		}}}
		\noli{\ul{
		\lii{Christian Heusel: -
		}}}
	}
}
\wahl{Kandidatur für die Härtefallkommission}{2. Lesung:}{Nanina Föhr}
{
    Der Kandidaturtext findet sich auf der \kandidaturenseite.
}{
	\textbf{1. Lesung:}
	\ul{
	\li{Ich wollte eigentlich fragen, wie gut du dich vorher informiert hast, was da so auf dich zukommt und wie du von der Kommission erfahren hast?}
		\noli{\ul{
		\lii{Es wurde schon telefoniert mit der HFK und da hat man sich gut ausgetauscht.
		}}}
}
}
\wahl{Kandidatur für die QSM-Kommission}{2. Lesung}{Stefanie Fiume}
{
	Der Kandidaturtext findet sich auf der \kandidaturenseite.
}{
	\textbf{1. Lesung:}
	\ul{
	\li{keine Fragen}
}
}
\subsection{Zusammenfassung}
\begin{center}
    \begin{tabular}{|p{6cm}|m{2cm}|m{1cm}|m{1cm}|m{1cm}|}
        \hline
        Kandidatur & Gewählt & Ja & Nein & Enth\\\hline\hline
		Vicoria Engels & ausstehend & 45 & 1 & 4 \\\hline
		Uli Roth & ausstehend & 44 & 1 & 4 \\\hline
		Alexander Riemer & ausstehend & 42 & 2 & 3 \\\hline
		Ole Klarhof & ausstehend & 41 & 2 & 4 \\\hline
		Christian Heusel & ausstehend & 41 & 3 & 1 \\\hline
		Christoph Blattgerste & ausstehend & 42 & 2 & 1 \\\hline
		Wahlvorschlag & ausstehend & tba & tba & tba \\\hline
		Nanina Föhr & ausstehend & 38 & 0 & 2 \\\hline
    \end{tabular}
\end{center}