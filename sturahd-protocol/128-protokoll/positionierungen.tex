\section{Diskussionen, Inhaltliche Positionierungen}
\antrag{Unvereinbarkeit mit Burschenschaften\label{diskussion:1}}{2. Lesung}{SDS}
{
    Der StuRa erklärt seine Unvereinbarkeit mit allen Burschenschaften in Heidelberg.
    Zudem stellt die Verfasste
    Studierendenschaft ihnen keine finanziellen Mittel oder Räume zur Verfügung, auch
    dann nicht, wenn sie
    nur als Kooperationspartner auftreten. Fachschaften und Hochschulgruppen sind dazu
    angehalten, mit den
    genannten Gruppen nicht zu kooperieren. Im Fall von Corps, Landmannschaften,
    Turnerschaften,
    Damenverbindungen und sonstigen studentischen Verbindungen bedarf die etwaige
    Unterstützung oder
    Kooperation durch ein Organ der Verfassten Studierendenschaft einer Zustimmung durch
    Beschluss des
    StuRa.
}{
    Die Burschenschaften stehen in ihrem Selbstverständnis, ihrem Weltbild und ihren Praktiken der Verfassten Studierendenschaft an der Uni Heidelberg diametral gegenüber. Der entsetzliche antisemitische Übergriff im Sommer 2020, bei dem ein Korporierter mit Münzen beworfen und mit Gürteln ausgepeitscht wurde, hat uns dies letztlich besonders vor Augen geführt. Dass es bereits zu Kooperationen von aktiven Gruppen in der Verfassten Studierendenschaft und Burschenschaften gab, beweist die Vortragsreihe „Feministin und konservativ“ des RCDS Heidelberg, innerhalb welcher die homophobe Journalistin Birgit Kelle zu einem Vortrag im Haus der Alemannia eingeladen wurde. Wir halten eine dementsprechende Unvereinbarkeitsklausel dementsprechend für nötig. 
}{
    \textbf{1. Lesung:}
    \ul{
	\li{Der Antrag scheint ziemlich unmöglich. Aber dass man alle Heidelberger Burschenschaften ausschließt ist nicht gerechtfertigt. Auch ist problematisch, dass man }
		\noli{\ul{
		\lii{Eine Demokratie muss wehrhaft sein. Man will jetzt nicht jede Verbindung ausschließen. Aber bei Burschenschaften versteckt sich unter dem Deckmantel rechtsextremistisches Gedankengut, das oft auch zu Gewalt führt. Da kann man auch auf die Debatte mit dem EDV-Referat verweisen, dass man Equipment nicht an antisemitische Gruppen verleit.
		}}}
	\li{Es ist schwierig alle Burschenschaften auszuschließen.}
	\li{Die Argumentation ist nicht wirklich schlüssig, weil dieser Antrag nicht ein spezifisches Problem löst. Die volle Version des Antrags wäre auch sehr interessant. }
	\li{Gab es hier schon eine Kooperation mit Burschenschaften?}
		\noli{\ul{
		\lii{Nein gab es nicht seit 100 Jahren
		}}}
	\li{Den HSG's kann man nicht verbieten sich mit jemandem zu treffen und das ist nur eine Aufforderung. Der Vorfall der Normannia war jetzt nur das Einzige was man mitbekommen hat. Bei Burschenschaften gibt es aber sehr oft homophobe Redner:innen zum Beispiel. Dass es noch keine Kooperation mit Burschenschaften gab, hat bei früheren Anträgen zu Unvereinbarkeit auch keinen Unterschied gemacht.}
	\li{Der zweite Satz will die Mitglieder von Burschenschaften von Ämtern der VS ausschließen. Das ist nicht nur rechtswidrig, sondern auch nicht wirklich mit der Demokratie vereinbar. Deswegen hat die Sitzungsleitung diesen nicht zugelassen. Man sollte auch einmal die Blickrichtung wechseln und das beurteilen wenn man das aus Sicht von einer rechten Partei die eine linke Gruppierung ausschließen will.}
	\li{Bevor klar ist ob man etwas ausschließen sollte, sollte man schauen ob die Vereinigung verfassungsfeindlich ist.}
	\li{Dieser Antrag scheint inhaltlich unfertig.} % Wenn der Antrag gut argumentiert sollte man über ihn
	\li{Den Blickwinkel zu wechseln scheint wie Gleichmacherei. Auch Gruppen die von rechten Gruppierungen nicht gemocht werden, sind Gruppierungen für die Rechte von Homosexuellen LGBTQ Frauen etc. eintreten.}
	\li{Man könnte den Antrag sehr gut verlängern. Wenn eine Burschenschaft verfassungsfeindlich agiert, ist nicht wirklich relevant hier wenn man das Handeln der Normannia betrachtet.}
	\li{Wir sind hier Studierende und sollten anderen versuchen zu verstehen.}
}
}{
    \abstimmungsergebnis{
        %TODOTitel
    }{
        tba%Ja
    }{
        tba%Nein
    }{
        tba%Enth
    }{
        tba%Ergebnis  \ul{\noli{\ul{\lii{test}}}}
    }
}

\GOantrag{Vertagung von \autoref{diskussion:1}}{
    Der Antrag soll auf die nächste Sitzung vertagt werden.
}{
 
}{   formale Gegenrede}{\abstimmungsergebnis{\fullref{diskussion:1}}{35}{4}{4}{\ul{\lii{Angenommen}}}}
%\aenderungsantrag{\autoref{diskussion:1}\label{diskussion:1.1}}{LHG}
%{
%    Der StuRa erklärt seine Unvereinbarkeit mit allen 
%    \replace{
%        Burschenschaften in Heidelberg.
%        Zudem stellt die Verfasste
%        Studierendenschaft ihnen keine finanziellen Mittel oder Räume zur Verfügung, auch
%        dann nicht, wenn sie
%        nur als Kooperationspartner auftreten. Fachschaften und Hochschulgruppen sind dazu
%        angehalten, mit den
%        genannten Gruppen nicht zu kooperieren. Im Fall von Corps, Landmannschaften,
%        Turnerschaften,
%        Damenverbindungen und sonstigen studentischen Verbindungen bedarf die etwaige
%        Unterstützung oder
%        Kooperation durch ein Organ der Verfassten Studierendenschaft einer Zustimmung durch
%        Beschluss des
%        StuRa.
%    }{
%        studentischen Verbindungen (Burschenschaften, Corps, Landmannschaften, Turnerschaften, Damenverbindungen und sonstigen studentischen Verbindungen), bei denen gruppenbezogene Menschenfeindlichkeit (Rassismus, Antisemitismus, Homophobie u.ä.) praktiziert, geduldet oder gefördert wird, oder bei denen es substantielle personelle Überschneidungen mit Gruppen (wie z.B. der identitären Bewegung) gibt, die gruppenbezogen menschenfeindlich auftreten.\\
%        Diese Unvereinbarkeit muss im Einzelfall durch den StuRa beschlossen werden. Hierzu ist eine 2/3-Mehrheit erforderlich.\\
%        Bei Erklärung der Unvereinbarkeit wird der entsprechenden Verbindung, sowie kooperierenden Hochschulgruppen von der Verfassten Studierendenschaft keine Räume und finanzielle Mittel mehr zur Verfügung gestellt.\\
%        Fachschaften und Hochschulgruppen sind dazu angehalten mit den oben genannten Gruppen nicht zu kooperieren.
%    }
%}{
%    Änderungsantrag der LHG Heidelberg
%}{}{
%    \abstimmungsergebnis{
%        \fullref{diskussion:1.1}
%    }{
%        tba%Ja
%    }{
%        tba%Nein
%    }{
%        tba%Enth
%    }{
%        tba%Ergebnis  \ul{\noli{\ul{\lii{test}}}}
%    }
%}
\section{Beschlüsse der Sondersitzung}
    %1titel,2lesung,3antragssteller,4antragstext,5begründung,6diskussion,7abstimmung
    \antrag{Wlan \label{corona:2}}{3. Lesung, vertagt}{Antragsstellend}
    {
        Die Verfasste Studierendenschaft der Ruprecht-Karls-Universität Heidelberg fordert,
        dass alle Studierenden der Universität über eine Internetverbindung verfügen die den
        Online- Lehrbetrieb angemessen verfolgbar machen. Dazu sollen
        Studierendenwohnheime konsequent mindestens 35 MBit/s Up/Downloadspeed als feste
        Vorgabe haben und Studierende in privatem Wohnraum Zuschüsse zur Behebung
        bekommen wenn die verfügbare Geschwindigkeit 35MBit/s unterschreitet. Als
        Prüfmittel schlägt dieser Antrag den offiziellen Breitbandmesser der
        Bundesnetzagentur
        unter „https://breitbandmessung.de/“ vor.
    }{
        Die Pandemie und die daraus resultierende Online-Semester haben viele Studierende vor
        das Problem mangelnder technischer Möglichkeiten gestellt. Eines davon ist eine oft
        mangelhafte Internetverbindung die es sehr schwer macht dem alltäglichen Lehrbetrieb
        hinreichend nachzukommen. Dies betrifft vor allem Haushalte in ländlicher Gegend und
        sozial benachteiligte Studierende. Dies ist in den Augen der Verfassten
        Studierendenschaft nicht tragbar da ein jeder Mensch das Recht auf freien Zugang zur
        Bildung hat. Dass die derzeitige Pandemie dieses recht einschränket ist in unseren Augen
        nicht zutreffend, da wir die technologischen Möglichkeiten haben das zu verhindern.
        Daher ist es nun die Aufgabe der Regierung und der einzelnen Universitäten sowie ihre
        Studierendenwerke dieses Recht auf Bildung für alle in einer befriedigenden Art
        pandemiekonform umzusetzen.
    }{
        \textbf{1. Lesung:}
        \ul{
            \li{Geschwindigkeit ist schwierig weil Down- und Uploadgeschwindigkeit verschieden sind.}
                \noli{\ul{
                \lii{Stimmt deswegen Änderungsantrag für 10 Mbit/s Uploadgeschwindigkeit. 
                }}}
        }
        \textbf{2. Lesung:}
        \ul{
            \li{Der Änderungsantrag über Up und Downloadgeschwindigkeit wurde nicht geschrieben.}
            \noli{\ul{
            \lii{Ja das wurde übergangen. Das ist wegen Prüfungsstress untergegangen.
            }}}
        }
        \textbf{3. Lesung:}
        \ul{\li{Keine Fragen}}
    }{
        \abstimmungsergebnis{
            \fullref{corona:2}
        }{
            tba%Ja
        }{
            tba%Nein
        }{
            tba%Enth
        }{
            tba%Ergebnis  \ul{\noli{\ul{\lii{test}}}}
        }
    }
        \aenderungsantrag{\autoref{corona:2} \label{corona:2.1}}{}{
            Anfügung des Folgenden:\\

            "Weiterhin fordert der Studierendenrat
            \begin{itemize}
                \item WLAN-Router in einem Gemeinschaftsraum, z.B. notfalls die gemeinsamen Küche auf jeder Etage, mit der Möglichkeit dort zu arbeiten
                \item Schaffung eine Ansprechperson für Internetprobleme entweder beim Studierendenwerk oder beim jeweiligen Wohnheim (z.B. jeweiliger Hausmeister)
                \item möglicherweise weitere Räumlichkeiten zur Nutzung bereitstellen, da einige Wohnheime weit von der Altstadt bzw. dem Neuenheimer Feld entfernt sind"
            \end{itemize}
        }{
            Die betreffende Gruppe hat ihren Änderungsantrag nicht ausformuliert. Um eine schöne Positionierung zu haben, ist das noch erforderlich.
        }{
            \textbf{1. Lesung:}
            \ul{\li{Keine Fragen}}
        }{
            \abstimmungsergebnis{
                \fullref{corona:2.1}
            }{
                tba%Ja
            }{
                tba%Nein
            }{
                tba%Enth
            }{
                tba%Ergebnis  \ul{\noli{\ul{\lii{test}}}}
            }
        }
    \subsection{Freischuss für Medizin \label{corona:6} (2. Lesung)}
    Antragsstellend: 
    \myparagraph{Antragstext:}
        Die Verfasste Studierendenschaft der Ruprecht-Karls-Universität Heidelberg fordert,
        dass alle
        Studierenden der Universität für Klausuren im Zeitraum der andauernden Pandemie, je
        Klausur, einen Klausurversuch mehr erhalten und dass das Wintersemester 2020/2021
        und alle
        folgenden Semester, die aufgrund der Covid-19-Pandemie im Online-Format stattfinden,
        im
        Rahmen der Fristen der Medizinischen Prüfungsordnung nicht zu zählen.
    \myparagraph{Begründung:}
        Die Pandemie und das daraus resultierende Online-Semester, wie auch die weiteren Folgen, machen
        Studierenden und Lehrkräften zu schaffen. Schon zu Beginn, im Sommersemester 2020, wurden
        psychische Belastung, Motivationsprobleme und auch Probleme mit der Internetverbindung sofort
        zu wichtigen Themen. Und noch immer wird nicht selten von einer erschwerten Studiensituation
        gesprochen. Zwar stimmt es, dass die Durchfallquote im ersten Online-Semester nicht
        besorgniserregend höher war, als vorher angenommen wurde, aber bei diesem Argument wird nicht
        beachtet, dass viele Studierende sich gar nicht in der Lage fühlten, einige Klausuren anzutreten und
        sich entsprechend oft entscheiden mussten, sich abzumelden oder gar nicht erst anzumelden.
        Psychische Belastung war besonders für internationale Studierende schwerwiegend. Ohne die
        Möglichkeit, sich in einem fremden Land etwas aufzubauen oder Bekannte und Freunde zu treffen,
        sprachen einige von Einsamkeitsgefühlen. Doch ist dies nicht nur auf internationale Studierende
        begrenzt. Auch einheimische Studierende, besonders die Erstsemester ab diesem Wintersemester,
        sehen sich gelegentlich mit demselben Problem konfrontiert. Die Fachschaften versuchen ihren
        neusten Mitgliedern zu bieten, was sie bieten können, aber bei allen Bemühungen, ist es auch ihnen
        nicht möglich 100\% dessen zu ersetzen was den Studierendenfehlt.\\
        Das alles wirkt sich natürlich auf die Studienleistung aus. Aus einer nicht repräsentativen Umfrage
        der Fachschaft Geowissenschaften am Ende des Sommersemesters 2020 lässt sich zumindest die
        Tendenz erkennen, dass es einem Teil der Studierenden nicht möglich war, dem Online-Unterricht
        angemessen zu folgen. Auch außerhalb dieser Umfrage zeigt sich, dass eine Unsicherheit herrscht
        und die Studierenden haben Hemmungen sich für viele Kurse anzumelden. Dazukommt, dass
        Exkursionen und dergleichen wegfallen oder verschoben werden müssen. Dadurch verlängert sich
        auch noch das Studium für viele. Auch was die Klausuren selbst betrifft, besteht viel Unsicherheit.
        In einigen Kursen wird noch immer gegrübelt, in welcher Form die Prüfungsleistung denn nun
        abgenommen werden kann. Das alles sind nur ein paar der Stressfaktoren für alle Mitglieder unserer
        Universität.\\
        Somit ist es ersichtlich, dass ein Ausgleich für die erschwerten Studienbedingungen geschaffen
        werden muss. Einen solchen Ausgleich sehen wir in einem Extra-Klausurversuch je Studiengang für
        alle Studierenden. Die Verlängerung des Studiums lässt sich in einigen Fällen nicht vermeiden.
        Doch man kann den Studierenden die Angst nehmen und nicht diejenigen Bestrafen, die nichts
        desto trotz versuchen oder sogar versuchen müssen, besonders hochgesetzten Hürden zu
        überwinden.
    \myparagraph{Diskussion:}
    \ul{\li{Keine Fragen}}
    \myparagraph{Abstimmung:}
    \abstimmungsergebnis{
        \fullref{corona:6}
    }{
        29
    }{
        0
    }{
        2
    }{
        \ul{\noli{\ul{\lii{Angenommen}}}}
    }
    \aenderungsantrag{\autoref{corona:6} \label{corona:6.1}}{}{
        Neuer Text:\\
        Die Verfasste Studierendenschaft der Ruprecht-Karls-Universität Heidelberg fordert,
        dass alle Studierenden der Universität für Klausuren im Zeitraum der andauernden Pandemie, je \replace{Studiengang}{Klausur}, einen Klausurversuch mehr erhalten und dass das Wintersemester 2020/2021

        und alle
        folgenden Semester, die aufgrund der Covid-19-Pandemie im Online-Format stattfinden,
        im
        Rahmen der Fristen der Medizinischen Prüfungsordnung nicht zu zählen.
    }{
        Die betreffende Gruppe hat ihren Änderungsantrag nicht ausformuliert. Um eine schöne Positionierung zu haben, ist das noch erforderlich.
    }{
        \textbf{1. Lesung:}
        \ul{\li{Keine Fragen}}
        \textbf{2. Lesung:}
    \ul{
	\li{Warum wurde das geändert?}
		\noli{\ul{
		\lii{Zum besseren Verständnis zumindest wurde es so verstanden.
		}}}
	\li{In Medizin hat man pro Klausur X Versuche und somit kann diese Änderung schon Sinn ergeben.}
}

    }{
        \abstimmungsergebnis{
            \fullref{corona:2.1}
        }{
            27
        }{
            0
        }{
            2
        }{
            \ul{\noli{\ul{\lii{Angenommen}}}}
        }
    }
