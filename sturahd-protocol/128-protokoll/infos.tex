\section{Infos, Termine, Berichte}

\subsection{Info: VS in der vorlesungsfreien Zeit – da geht einiges!}
Liebe Sturist*innen, bald geht die vorlesungsfreie Zeit los, in der Zeit für die Erholung von der Klausurenphase, Hausarbeiten und das Aufräumen des Zimmers ist. (Okay, vielleicht schließe ich bei letzterem von mir auf euch ) Wenn ihr Lust auf ein bisschen Abwechslung habt, haben wir was für euch: Bei uns geht’s nämlich auch in der vorlesungsfreien Zeit weiter mit der hochschulpolitischen Arbeit – und ihr könnt mitmachen! Einige Referate, AKs und AGs werden euch in würziger Kürze präsentieren, was sie in der vorlesungsfreien Zeit machen wollen. Vielleicht ist auch das ein oder andere für euch dabei! Oder euch fällt jemand ein, der*die Lust haben könnte, sich zu engagieren? Dann sprecht ihn*sie gerne an!\\
Wer nun neugierig geworden ist, welche Referate, AKs und AGs wir eigentlich so haben – die Liste ist schier unendlich und für jede*n was dabei: \url{https://www.stura.uni-heidelberg.de/vs-strukturen/referate/} und \url{https://www.stura.uni-heidelberg.de/vs-strukturen/aksags/}

\subsection{Bericht des Referats für Hochschulpolitische Vernetzung}
Wie bereits mehrfach angekündigt, vom 05.-07.03.2021 findet die \textbf{Mitgliederversammlung} (MV) des \textbf{fzs} (freier zusammenschluss student*innenschaften), in dem wir Mitglied sind, statt. Die Anträge zur MV sind bereits online einzusehen auf \url{https://mv.fzs.de/web/}\\
Das Außenreferat möchte alle Interessierten zu einem \textbf{Vorbereitungstreffen} am xx.xx um uu.uu auf unseren StuRa-Konf-Server einladen: \url{https://bbb.stura.uni-heidelberg.de/b/mar-3a9-66f}. Dort wollen wir über die Anträge besprechen und schlussendlich eine Abstimmungsmatrix für die RefKonf entwerfen. Die Vorbereitung bietet eine gute Möglichkeit, weitere Aktive aus der VS (insbesondere die beiden Reffis), den fzs als Verband und dessen Arbeit näher kennenzulernen. In diesem Sinne freuen wir uns, wenn Einige von euch (mit oder ohne Voranmeldung) dafür vorbeischauen!\\
Abseits dessen ist auf Landesebene nun der von der Landesstudierendenvertretung (LaStuVe) entworfene \textbf{Studi-O-Mat} gestartet: \url{https://studiomat.lastuve-bawue.de/}\\
Dieser soll ein Informationsangebot für die Landtagswahl sein, bei dem alle zur Wahl stehenden Parteien die Möglichkeit der Stellungnahme zu allerlei hochschulpolitischen Thesen hatten. Hoffentlich werden dadurch hochschulpolitische Themen mehr in den Fokus gerückt und vielen Studierende eine Möglichkeit zur Orientierung bei der Wahl gegeben. Wir wünschen viel Spaß beim Durchklicken.\\
(Entweder habt ihr bereits oder ihr werdet dazu noch einen Bericht von uns bekommen. Sollte das nicht der Fall sein, nehmt gerne einen passenden Ausschnitt dieses Texts und bewerbt dieses Angebot innerhalb eures Dunstkreises!)

\myparagraph{Nachfragen:}
\ul{
	\li{keine Fragen}
}

\subsection{Bericht des EDV-Referats}

\myparagraph{Nachfragen:}
\ul{
	\li{keine Fragen}
}

\subsection{Bericht des Vorsitzes}

\myparagraph{Nachfragen:}
\ul{
	\li{keine Fragen}
}

\subsection{Referat für hochschulpolitische Vernetzung}
\myparagraph{Nachfragen:}
\ul{
	\li{Es gab die Rückmeldung, dass die Fragen nicht divers genug gestellt waren.}
		\noli{\ul{
		\lii{Die Rückmeldung wird weitergegeben. 
		}}}
}