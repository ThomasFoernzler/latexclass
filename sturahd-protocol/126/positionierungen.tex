\section{Diskussionen, Inhaltliche Positionierungen}

    \antrag{ heiConf für alle Studierenden öffnen \label{dis:heiConf}}{1. Lesung}{Liste GHG}{
        Der StuRa drängt die Universität und das Universitätsrechenzentrum, allen
        Studierenden die Einrichtung von Räumen in heiConf zu ermöglichen.
    }{
        Der Austausch zwischen Studierenden stellt einen essentiellen Bestandteil eines erfolgreichen Studiums dar. Da auch das Wintersemester 2020/2021 online stattfindet, sind die Studierenden gezwungen, die Prüfungsvorbereitung digital durchzuführen. Da die meisten Studierenden sich in kleineren Lerngruppen auf Prüfungen vorbereiten, ist es für sie und ihren Lernerfolg äußerst wichtig, sich gemeinsam zum Lernen digital treffen zu können. Leider haben viele keine guten Möglichkeiten, sich zu treffen. Insbesondere Konferenzen mit mehr als drei Leuten gestalten sich über gängige kostenlose Konferenzsysteme schwierig. Ein erfolgreiches Studium darf allerdings keinesfalls von den finanziellen Möglichkeiten der einzelnen Studierenden abhängen. Ein leistungsfähiges und datenschutzkonformes Konferenzsystem für alle Studierende ist dementsprechend sehr wichtig.
    }{

    }{
        \abstimmungsergebnis{
            heiConf für alle Studierenden öffnen
        }{
            tba%Ja
        }{
            tba%Nein
        }{
            tba%Enth
        }{
            tba%Ergebnis  \ul{\noli{\ul{\lii{test}}}}
        }
    }

        \GOantrag{Dringlichkeitsantrag zu zu \autoref{dis:heiConf}}{Der Studierendenrat stimmt über den Antrag 7.2 nach erster Lesung ab.}{Begründung}{Gegenrede}
        {
            \abstimmungsergebnis{
                Dringlichkeitsantrag zu 7.2
            }{
                tba%Ja
            }{
                tba%Nein
            }{
                tba%Enth
            }{
                tba%Ergebnis  \ul{\noli{\ul{\lii{test}}}}
            }
        }

    \antrag{Öffentliche Beratung über den Stand der geplanten Fuß- und Radbrücke über den Neckar}{1. Lesung}{}{
        Der StuRa berät über die geplanten Fuß- und Radbrücke über den Neckar. Das
    Verkehrsreferat stellt den bisherigen Stand vor sowie seine Sicht auf die möglichen
    Brückenentwürfe vor. Danach kann das Plenum, falls es das Thema für relevant hält,
    Rückfragen oder Feedback geben. Das Verkehrsreferat richtet seine weitere Arbeit an
    diesem Feedback aus.
    }{
        Das Umwelt- und Prognoseinstitut (UPI) schrieb das Verkehrsreferat am 13.1. an, um seine Sicht auf den Bau der Neckarbrücke darzulegen. Er findet, dass der Entwurf des Büros Mayr Lüdescher anstatt von Schlaich, Bergemann übernommen werden sollte. Ich (Michèle) fand seine verkehrsplanerischen und ökologische Argumente überzeugend, möchte mir aber vom Studierendenrat Feedback einholen, ob und wie wir in der Angelegenheit vorgehen wollen. Dafür will ich die Situation vorstellen und freue mich über Feedback.\\
        Ich will auch Dieter Teufel vom UPI selbst einladen, um das Anliegen zu erklären. Gerade habe ich noch keine Zusage.
    }{

    }{
        \abstimmungsergebnis{
            Öffentliche Beratung über den Stand der geplanten Fuß- und Radbrücke über den Neckar
        }{
            tba%Ja
        }{
            tba%Nein
        }{
            tba%Enth
        }{
            tba%Ergebnis  \ul{\noli{\ul{\lii{test}}}}
        }
    }

\section{Beschlüsse der Sondersitzung}

    \antrag{Lernräume und -orte \label{corona:1}}{1. Lesung}{Antragsstellend}
    {
        \myparagraph{mehr Transparenz}
        Wir fordern transparente, verständlichere und rechtzeitige Auskunft über den
        aktuellen
        Studienbetrieb. Zusätzlich zu den Informationen auf der Uni Website fordern wir
        Berichte über den
        aktuellen Stand der Online Lehre. Viele Studierende würden sich gerne einen Lernraum
        für die
        nächsten Wochen einrichten, aber wissen z.B. nicht ob die UB dieses Semster wieder
        öffnen kann.
        Uns ist bewusst, dass sich die Regelungen der Regierung häufig sehr schnell ändern.
        Dennoch gibt
        es bestimmt gewisse Beschränkungen, die voraussichtlich noch etwas länger bestehen
        bleiben
        müssen und die man auch einfacher über die jeweilige Fachschaft verbreiten könnte.
        \myparagraph{Lernräume}
        Wir fordern, dass in Abhängigkeit der jeweiligen Lage der Pandemie (Inzidenzabhängig,
        abhängig
        von Beschlüssen von Bund und Ländern) sowohl "stille" Lernräume als auch Lernräume in
        denen
        gesprochen werden darf, eingerichtet werden. Diese Lernräume sollen insberondere
        während der
        Klausurenphase sicherstellen, dass allen Studierenden der Universität ein
        ausreichender Zugang zu 
        Lernräumen gewährleistet ist, der aufgrund der jeweiligen Wohnsituation vieler
        Studieren der und
        der begrenzten Kapazitäten der Universitätsbibliothek nicht oder nicht in
        ausreichendem Umfang
        gegeben ist. So wird die Vorbereitung von vielen Studierenden besonders auf Prüfungen
        und
        Klausuren, aber auch auf Veranstaltungen (Seminare oder Vorlesungen) und deren
        Durchführung oft
        durch die persöhnliche wohnliche Situation eingeschränkt. Somit ist die Einrichtung
        von den
        genannten Lernräumen im Sinne der Chancengleichheit und im Interesse der Qualität der
        Lehre von
        immenser Bedeutung. Potentiell geeignete Räume, wie größere Seminarräume oder
        kleinere
        Hörsäle, oder auch Lernräume, die nur momentan geschlossen sind würden sich hierfür
        aus unserer
        Sicht eignen. Besonders wichtig ist hierbei im Hinblick auf die Gesatltung von
        potentiellen
        Lernräumen auch eine gute und stabile Internetvernindung, da nur so eine
        uneingeschränkte
        Teilnahme an der (Online-) Lehre ohne einen allzugroßen Qualitätsverlust
        gewährleistet werden
        kann.
        \myparagraph{Virtuelle Räume für Online-Lerngruppen}
        Wir fordern, dass alle Studierenden über die bereits auf der Stura-Webseite
        eingerichteten digitalen
        Lernräume ausreichend informiert werden (insofern das rechtlich vertretbar ist). Zu
        diesem Zweck
        können sowohl die Fachschaften, als auch die Fakultäten eingeschaltet werden. Für den
        Fall, dass
        die Anzahl der geschaffenen Lernräume nicht genügt oder die Studierende eine private
        Lerngruppe
        bilden wollen, sollen alle Student*Innen einen heiCONF-Account bekommen. Für die
        Benutzung
        anderer Plattformen, wie zB Webex oder Zoom, werden regelmäßig Geldbeiträge
        aufgefordert, was
        eine Hinderung und Belastung für die Student*Innen darstellt. Zusätzlich kommt es auf
        privaten
        Anbietern auch oft zu Abstürzen aufgrund der schlechten Kapazitäten. Dieses Problem
        sollte durch
        die Einrichtung einer Universitätsplattform für alle Studierende gelöst werden.
        \myparagraph{Druckerzugang}
        Der StuRa fordert, dass alle Studierende einen Zugang zu Instituts- und
        Bibliotheksdruckern
        erhalten. Viele Studierende besitzen keine eigene Möglichkeit zu drucken und sind
        daher auch im
        Online-Semester auf universitäre Angebote (zu Drucken) angewiesen. Dabei ist
        weiterhin auf eine
        pandemiegerechte Durchführung zu achten.
        Unser Vorschlag ist, dass sich hierfür an dem aktuellem Ausleihsystem orientiert wird
        und eine Art
        „Drucken to Go“ angeboten wird. Über ein Portal könnten die auszudruckenden Dokumente
        vorab
        zugesendet und in einer festgelegten Zeitspanne an der UB oder einer Zweigstelle
        abgeholt werden.
        Die Bezahlung erfolgt bei der Abholung mittels Studierendenausweises.
        Nur durch ein solches Angebot kann eine Chancengleichheit im Online-Semester
        gewährleistet
        werden.
    }{
        %Begründung
    }{
        %Diskussion
    }{
        \abstimmungsergebnis{
            Lernräume und -orte
        }{
            tba%Ja
        }{
            tba%Nein
        }{
            tba%Enth
        }{
            tba%Ergebnis  \ul{\noli{\ul{\lii{test}}}}
        }
    }
    \antrag{Bibliotheken \label{corona:2}}{1. Lesung}{Antragsstellend}
    {
        Der Studierendenrat der Ruprecht-Karls-Universität Heidelberg fordert, dass die
        Universitäts-eigenen Bibliotheken einen Dienst einrichten, um auf Anfrage Aufsätze
        und
        Bücher der Fachbibliotheken eingescannt Student*innen und Dozent*innen zur
        Verfügung zu stellen. Dabei können Wochen- oder Monatslimits (z.B. fünf Aufsätze oder
        eine Monographie bzw. Sammelband) festgelegt werden. Je nach den Erfahrungen sollte
        das Limit verringert werden, falls die Nachfrage sehr hoch ist und damit alle
        Student*innen und Dozent*innen darauf zugreifen können, oder erhöht werden, falls
        noch Kapazitäten vorhanden sind.
    }{
        Aufgrund des Lockdowns mussten die Fakultäts- und Fach-spezifischen Bibliotheken ab
        Mitte Dezember schließen, was dazu führte, dass sowohl Student*innen als auch
        Dozent*innen in ihrem Lernen, Lehren und Forschen schwer eingeschränkt sind. Dieses
        Problem soll Corona-konform gelöst werden, indem die Fachbibliotheken angefragte
        Aufsätze bzw. Bücher einscannen und per E-Mail im PDF-Format an die Anfragenden
        senden. Der Dienst soll auch dann noch weitergeführt werden, wenn die Bibliotheken
        wieder öffnen können: Viele Studierende sind nämlich wieder zu ihren Eltern gezogen,
        sodass sie nicht auf die analogen Ressourcen in Heidelberg zurückgreifen können.
        Zudem nimmt dadurch die Notwendigkeit ab, sich in die Bibliotheken zu bewegen, was
        allgemein sinnvoll ist.
    }{
        %Diskussion
    }{
        \abstimmungsergebnis{
            Bibliotheken
        }{
            tba%Ja
        }{
            tba%Nein
        }{
            tba%Enth
        }{
            tba%Ergebnis  \ul{\noli{\ul{\lii{test}}}}
        }
    }
    \antrag{Corona-Freischuss! \label{corona:3}}{1. Lesung}{Antragsstellend}
    {
        Die Verfasste Studierendenschaft der Ruprecht-Karls-Universität Heidelberg fordert,
        dass alle Studierenden der Universität für Klausuren im Zeitraum der andauernden
        Pandemie, je Studiengang, einen Klausurversuch mehr erhalten.
    }{
        Die Pandemie und das daraus resultierende Online-Semester, wie auch die weiteren
        Folgen, machen Studierenden und Lehrkräften zu schaffen. Schon zu Beginn, im
        Sommersemester 2020, wurden psychische Belastung, Motivationsprobleme und auch
        Probleme mit der Internetverbindung sofort zu wichtigen Themen. Und noch immer wird
        nicht selten von einer erschwerten Studiensituation gesprochen. Zwar stimmt es, dass die
        Durchfallquote im ersten Online-Semester nicht besorgniserregend höher war, als vorher
        angenommen wurde, aber bei diesem Argument wird nicht beachtet, dass viele
        Studierende sich gar nicht in der Lage fühlten, einige Klausuren anzutreten und sich
        entsprechend oft entscheiden mussten, sich abzumelden oder gar nicht erst anzumelden.
        Psychische Belastung war besonders für internationale Studierende schwerwiegend.
        Ohne die Möglichkeit, sich in einem fremden Land etwas aufzubauen oder Bekannte und
        Freunde zu treffen, sprachen einige von Einsamkeitsgefühlen. Doch ist dies nicht nur auf
        internationale Studierende begrenzt. Auch einheimische Studierende, besonders die
        Erstsemester ab diesem Wintersemester, sehen sich gelegentlich mit demselben Problem
        konfrontiert. Die Fachschaften versuchen ihren neusten Mitgliedern zu bieten, was sie
        bieten können, aber bei allen Bemühungen, ist es auch ihnen nicht möglich 100\% dessen
        zu ersetzen was den Studierendenfehlt. Das alles wirkt sich natürlich auf die
        Studienleistung aus. Aus einer nicht repräsentativen Umfrage der Fachschaft
        Geowissenschaften am Ende des Sommersemesters 2020 lässt sich zumindest die
        Tendenz erkennen, dass es einem Teil der Studierenden nicht möglich war, dem
        OnlineUnterricht angemessen zu folgen. Auch außerhalb dieser Umfrage zeigt sich, dass eine
        Unsicherheit herrscht und die Studierenden haben Hemmungen sich für viele Kurse
        anzumelden. Dazukommt, dass Exkursionen und dergleichen wegfallen oder verschoben
        werden müssen. Dadurch verlängert sich auch noch das Studium für viele. Auch
        was die Klausuren selbst betrifft, besteht viel Unsicherheit. In einigen Kursen wird noch
        immer gegrübelt, in welcher Form die Prüfungsleistung denn nun abgenommen werden
        kann. Das alles sind nur ein paar der Stressfaktoren für alle Mitglieder unserer
        Universität. Somit ist es ersichtlich, dass ein Ausgleich für die erschwerten
        Studienbedingungen geschaffen werden muss. Einen solchen Ausgleich sehen wir in
        einem Extra-Klausurversuch je Studiengang für alle Studierenden. Die Verlängerung des
        Studiums lässt sich in einigen Fällen nicht vermeiden. Doch man kann den Studierenden
        die Angst nehmen und nicht diejenigen Bestrafen, die nichts desto trotz versuchen oder
        sogar versuchen müssen, besonders hochgesetzten Hürden zu überwinden.
    }{
        %Diskussion
    }{
        \abstimmungsergebnis{
            Corona-Freischuss!
        }{
            tba%Ja
        }{
            tba%Nein
        }{
            tba%Enth
        }{
            tba%Ergebnis  \ul{\noli{\ul{\lii{test}}}}
        }
    }
        \aenderungsantrag{zu \autoref{corona:3}}{}{
            Dies gilt nicht für die Studierenden der Medizin und Rechtswissenschaften.
        }{
            Anträge, die die Prüfungsämter dieser Fächer betreffen, müssen von den jeweiligen Landesprüfungsämter genehmigt werden.\\
            Damit eine verweigerte Genehmigung keine Auswirkungen auf die restlichen Studierenden haben kann, werden Medizin und Jura in eigene Anträge ausgegliedert.
        }{

        }{
            \abstimmungsergebnis{
                %Titel
            }{
                tba%Ja
            }{
                tba%Nein
            }{
                tba%Enth
            }{
                tba%Ergebnis  \ul{\noli{\ul{\lii{test}}}}
            }
        }
        \aenderungsantrag{zu \autoref{corona:3}: Englisch/ internationale Studierenden/ mehr Zeit für Prüfungen}{}{
            1)Wir fordern, dass internationale Studierenden eine Möglichkeit bekommen werden, die
            Prüfungen
            auf Englisch schreiben zu dürfen, auch wenn das Fach meist die Vorlesungen, Seminaren
            usw. auf
            Deutsch abhält.\\
            2)Wir fordern, dass Prüfungen neben Deutsch auch in englischer Sprache angefertigt
            werden
            können, auch wenn die Veranstaltungssprache Deutsch ist.\\
            3)Wir fordern, dass internationale Studierende Prüfungsleistungen in Deutschland
            ablegen können,
            nachdem mit Aufenthalts- und Einreiseproblematik verbundene Ängste entsprechend
            geregelt/abgebaut wurden und die Prüfungsleistung, sofern möglich in der
            Muttersprache,
            mindestens jedoch in Englisch zu erbringen sind.\\
            4) Wir fordern das Internationale Studierenden mehr Zeit bei schriftlichen Prufüngen
            bekommen
            oder die Möglichkeit erhalten, diese Prüfungen mündlich auf Deutsch oder Englisch
            ablegen zu
            dürfen.
        }{
            Unsere Forderungen liegen darin begründet, dass die meisten Professoren die englische Sprache
            täglich benutzen und häufig in dieser Sprache auch eigene Texte publizieren. Die internationalen
            Studierenden sollen auch die Möglichkeit haben, mehr Zeit bei schriftlichen Prufüngen zu
            bekommen, da es ihnen verglichen mit muttersprachlich deutschen Studierenden schwerfallen
            kann, die Prüfung in vorgesehener Zeit abzulegen. Außerdem sollte bedacht werden, dass
            internationale Studierende häufig keine deutsche Tastatur haben, weshalb es schwieriger für sie ist,
            die Antworten schnell genug zu schreiben, was besonders bei Prüfungen, die mehrere offene Fragen
            beinhalten, zu tragen kommt. Die Studierenden sollen außerdem das Angeboten erhalten, die
            Prüfungen mündlich zu ablegen zu können, falls es bei der schriftlichen Prüfung zur einer
            Benachteiligung aufgrund zuvor genannter Gründe käme. Sollte es dazu kommen, dass die
            Studierenden nicht nach Deutschland einreisen können und deshalb keinen Zugriff auf die gefordete
            Prüfungsliteratur haben (nur Präsenznutzung bei Bibliotheken, keine Möglichkeit allgemein zur
            Bücherausleihe, keine Möglichkeit die Bücher zu bestellen und in das Heimatland zu schicken), soll
            es für die Betroffenen das Angebot geben, andere alternative Angebote als die gefordete
            Prüfungsliteratur zu bekommen.\\
            Die verschiedenen Probleme, mit denen sich internationale Studierende konfrontiert sehen,
            schränken die Leistung bei den Prüfungen nicht nur ein, sondern sind in Zeilen untragbar zum
            Bestehen eben dieser Prüfungen. Daher unsere Forderungen.
        }{

        }{
            \abstimmungsergebnis{
                Englisch/ internationale Studierenden/ mehr Zeit für Prüfungen
            }{
                tba%Ja
            }{
                tba%Nein
            }{
                tba%Enth
            }{
                tba%Ergebnis  \ul{\noli{\ul{\lii{test}}}}
            }
        }
        \aenderungsantrag{zu \autoref{corona:3}: Mehr Zeit für Jura}{}{
            Die Verfasste Studierendenschaft der Ruprecht-Karls-Universität Heidelberg fordert,
            das
            Wintersemester 2020/2021 und alle folgenden Semester, die aufgrund der Covid-19-
            Pandemie im
            Online-Format stattfinden, im Rahmen der Fristen der Juristischen Ausbildungs- und
            Prüfungsordnung (JAPrO) nicht mitzuzählen.
        }{
            Die von der Verfassten Studierendenschaft geforderte Möglichkeit, einzelne Klausuren zu
            wiederholen, lässt sich auf das System der Übungen im Jurastudium nicht übertragen. Die einzige
            praktikable Möglichkeit, die Nachteile (s.u.), die Studierenden aus der Pandemie erwachsen sind,
            im juristischen Studium sachgerecht auszugleichen, besteht in der Verlängerung der Fristen, in
            denen die Übungen und das Examen abgelegt werden müssen.\\
            Diese Verlängerung gab es bereits im Sommersemester 2020. An der schwierigen Situation hat sich
            seitdem nichts geändert.\\
            Bereits jetzt passt §29 (3a) des Landeshochschulgesetzes Baden-Württemberg die Regelstudienzeit
            einiger Studiengänge an die besonderen Umstände der Pandemie an. Dies sollte auf die Fristen in
            der JAPrO übertragen werden.\\
            Die Pandemie und das daraus resultierende Online-Semester, wie auch die weiteren Folgen, machen
            Studierenden und Lehrkräften zu schaffen. Schon zu Beginn, im Sommersemester 2020, wurden
            psychische Belastung, Motivationsprobleme und auch Probleme mit der Internetverbindung sofort
            zu wichtigen Themen. Und noch immer wird nicht selten von einer erschwerten Studiensituation
            gesprochen.\\
            Psychische Belastung war besonders für internationale Studierende schwerwiegend. Ohne die
            Möglichkeit, sich in einem fremden Land etwas aufzubauen oder Bekannte und Freunde zu treffen,
            sprachen einige von Einsamkeitsgefühlen. Doch ist dies nicht nur auf internationale Studierende
            begrenzt. Auch einheimische Studierende, besonders die Erstsemester ab diesem Wintersemester,
            sehen sich gelegentlich mit denselben Problem konfrontiert. Die Fachschaften versuchen ihren
            neusten Mitgliedern zu bieten, was sie bieten können, aber bei allen Bemühungen, ist es auch ihnen
            nicht möglich, 100\% dessen zu ersetzen, was den Studierenden fehlt.\\
            Das alles wirkt sich natürlich auf die Studienleistung aus. Aus einer nicht repräsentativen Umfrage
            der Fachschaft Geowissenschaften am Ende des Sommersemesters 2020 lässt sich zumindest die
            Tendenz erkennen, dass es einem Teil der Studierenden nicht möglich war, dem Online-Unterricht
            angemessen zu folgen. Auch außerhalb dieser Umfrage zeigt sich, dass eine Unsicherheit herrscht
            und die Studierenden Hemmungen haben sich für viele Kurse anzumelden. Auch was die
            Klausuren selbst betrifft, besteht viel Unsicherheit. Das alles sind nur ein paar der Stressfaktoren
            für alle Mitglieder unserer Universität.\\
            Somit ist es ersichtlich, dass ein Ausgleich für die erschwerten Studienbedingungen geschaffen
            werden muss. Die Verlängerung des Studiums lässt sich in einigen Fällen nicht vermeiden. Doch
            man kann den Studierenden die Angst nehmen und nicht diejenigen Bestrafen, die nichtsdestotrotz
            versuchen oder sogar versuchen müssen, besonders hochgesetzten Hürden zu überwinden.
        }{

        }{
            \abstimmungsergebnis{
                Mehr Zeit für Jura
            }{
                tba%Ja
            }{
                tba%Nein
            }{
                tba%Enth
            }{
                tba%Ergebnis  \ul{\noli{\ul{\lii{test}}}}
            }
        }
        \aenderungsantrag{zu \autoref{corona:3}: Freischuss für Medizin}{}{
            Die Verfasste Studierendenschaft der Ruprecht-Karls-Universität Heidelberg fordert,
            dass alle
            Studierenden der Universität für Klausuren im Zeitraum der andauernden Pandemie, je
            Studiengang, einen Klausurversuch mehr erhalten und dass das Wintersemester 2020/2021
            und alle
            folgenden Semester, die aufgrund der Covid-19-Pandemie im Online-Format stattfinden,
            im
            Rahmen der Fristen der Medizinischen Prüfungsordnung nicht zu zählen.
        }{
            Die Pandemie und das daraus resultierende Online-Semester, wie auch die weiteren Folgen, machen
            Studierenden und Lehrkräften zu schaffen. Schon zu Beginn, im Sommersemester 2020, wurden
            psychische Belastung, Motivationsprobleme und auch Probleme mit der Internetverbindung sofort
            zu wichtigen Themen. Und noch immer wird nicht selten von einer erschwerten Studiensituation
            gesprochen. Zwar stimmt es, dass die Durchfallquote im ersten Online-Semester nicht
            besorgniserregend höher war, als vorher angenommen wurde, aber bei diesem Argument wird nicht
            beachtet, dass viele Studierende sich gar nicht in der Lage fühlten, einige Klausuren anzutreten und
            sich entsprechend oft entscheiden mussten, sich abzumelden oder gar nicht erst anzumelden.
            Psychische Belastung war besonders für internationale Studierende schwerwiegend. Ohne die
            Möglichkeit, sich in einem fremden Land etwas aufzubauen oder Bekannte und Freunde zu treffen,
            sprachen einige von Einsamkeitsgefühlen. Doch ist dies nicht nur auf internationale Studierende
            begrenzt. Auch einheimische Studierende, besonders die Erstsemester ab diesem Wintersemester,
            sehen sich gelegentlich mit demselben Problem konfrontiert. Die Fachschaften versuchen ihren
            neusten Mitgliedern zu bieten, was sie bieten können, aber bei allen Bemühungen, ist es auch ihnen
            nicht möglich 100\% dessen zu ersetzen was den Studierendenfehlt.\\
            Das alles wirkt sich natürlich auf die Studienleistung aus. Aus einer nicht repräsentativen Umfrage
            der Fachschaft Geowissenschaften am Ende des Sommersemesters 2020 lässt sich zumindest die
            Tendenz erkennen, dass es einem Teil der Studierenden nicht möglich war, dem Online-Unterricht
            angemessen zu folgen. Auch außerhalb dieser Umfrage zeigt sich, dass eine Unsicherheit herrscht
            und die Studierenden haben Hemmungen sich für viele Kurse anzumelden. Dazukommt, dass
            Exkursionen und dergleichen wegfallen oder verschoben werden müssen. Dadurch verlängert sich
            auch noch das Studium für viele. Auch was die Klausuren selbst betrifft, besteht viel Unsicherheit.
            In einigen Kursen wird noch immer gegrübelt, in welcher Form die Prüfungsleistung denn nun
            abgenommen werden kann. Das alles sind nur ein paar der Stressfaktoren für alle Mitglieder unserer
            Universität.\\
            Somit ist es ersichtlich, dass ein Ausgleich für die erschwerten Studienbedingungen geschaffen
            werden muss. Einen solchen Ausgleich sehen wir in einem Extra-Klausurversuch je Studiengang für
            alle Studierenden. Die Verlängerung des Studiums lässt sich in einigen Fällen nicht vermeiden.
            Doch man kann den Studierenden die Angst nehmen und nicht diejenigen Bestrafen, die nichts
            desto trotz versuchen oder sogar versuchen müssen, besonders hochgesetzten Hürden zu
            überwinden.
        }{

        }{
            \abstimmungsergebnis{
                Freischuss für Medizin
            }{
                tba%Ja
            }{
                tba%Nein
            }{
                tba%Enth
            }{
                tba%Ergebnis  \ul{\noli{\ul{\lii{test}}}}
            }
        }
    \antrag{Klausurenphase \label{corona:4}}{1. Lesung}{Antragsstellend}
    {
        Der Studierendenrat fordert das Rektorat und alle Fakultäten dazu auf, die
        Studierenden rechtzeitig
        -heißt mindestens drei Wochen vor Prüfungstermin- bekannt zugeben ob und in welcher
        Form die
        Klausur stattfinden wird. Sollte das aufgrund eines dynamischen Pandemiegeschehen
        nicht möglich
        sein, müssen mindestens alle angedachten Möglichkeiten kommuniziert werden.
        Sollten die Klausuren online stattfinden fordern wir die Fakultäten und oder
        Lehrstühle dazu auf:
        \begin{itemize}
            \item Eine zuverlässige Notfallhotline einzurichten, an die sich Studierende richten können, wenn sie technische Probleme während der Klausur haben sollten.
            \item Sollte die Teilnehmerzahl eines Kurses nicht über 50 sein, muss die Möglichkeit einer mündlichen Prüfung bedacht werden.
            \item Es sollte den Teilnehmenden einer online Klausur die Möglichkeit eingerichtet werden während der Klausur Fragen zu stellen.
            \item Die Anforderungen sollten die einer Präsenzklausur nicht maßgeblich überschreiten.  
        \end{itemize}
        Zudem sollten Studierenden, die in diesem Wintersemester auf Online-Klausuren
        ausweichen
        müssen, je Klausur ein Versuch mehr eingeräumt werden.
        Für das folgende Sommersemester fordert der Studierendenrat das Rektorat und die
        Fakultäten dazu
        auf, schon bei Vorlesungsbeginn über alle Prüfungsmodalitäten zu informieren.
    }{
        Eine Umfrage des Studierendenrates zum Sommersemester 2020 zeigte, dass die Studierenden vor
        allem durch die Unsicherheit über das Stattfinden der Klausuren verärgert waren. Auch momentan
        ist es noch nicht klar in welcher Form und ob die Klausuren stattfinden sollen.\\
        Sollte sich das Infektionsgeschehen nicht verbessern, werden einige Klausuren online stattfinden
        müssen. Bei diesen sollten zumindest einige Kriterien beachtet werden, um Studierende nicht
        übermäßig zu belasten. 
    }{
        %Diskussion
    }{
        \abstimmungsergebnis{
            Klausurenphase
        }{
            tba%Ja
        }{
            tba%Nein
        }{
            tba%Enth
        }{
            tba%Ergebnis  \ul{\noli{\ul{\lii{test}}}}
        }
    }
        \aenderungsantrag{zu \autoref{corona:4}}{}{
        Der Studierendenrat fordert das Rektorat und alle Fakultäten dazu auf, die
        Studierenden rechtzeitig
        -heißt mindestens drei Wochen vor Prüfungstermin- bekannt zugeben ob und in welcher
        Form die
        Klausur stattfinden wird. Sollte das aufgrund eines dynamischen Pandemiegeschehen
        nicht möglich
        sein, müssen mindestens alle angedachten Möglichkeiten kommuniziert werden.
        \textcolor{green}{Es sollte Studierenden die Möglichkeit
        eingerichtet werden,
        bei kurzfristiger Änderung der angedachten Prüfungsform, ohne Attest von der
        Prüfungsleistung
        zurückzutreten.Der Studierendenrat fordert alle Fakultäten und Lehrstühle dazu auf, zum Schutz füt
        alle
        Beteiligten so viel wie möglich auf online Prüfungsleistungen ausgewichen werden.
        Gleichzeitig
        soll für die Studierenden, die aus technischen oder weiteren Gründen nicht in der
        Lage sind, eine
        Prüfungsleistung online abhzulegen, Räume mit Aufsichtspersonal an der Universität
        zur Verfügung
        gestellt werden.}
        Sollten die Klausuren online stattfinden fordern wir die Fakultäten und oder
        Lehrstühle dazu auf:
        \begin{itemize}
            \item Eine zuverlässige Notfallhotline einzurichten, an die sich Studierende richten können, wenn sie technische Probleme während der Klausur haben sollten.
            \item \old{Sollte die Teilnehmerzahl eines Kurses nicht über 50 sein, muss die Möglichkeit einer mündlichenPrüfung bedacht werden.}
            \item \textcolor{green}{die Möglichkeit einer mündlichen Prüfung anstatt einer anderen Prüfungsform zu bedenken, wenn die Teilnehmerzahl eines Kurses nicht über 50 liegt}
            \item \old{Rückfragemöglichkeiten für Teilnehmer*innen an einer Online-Prüfungsleistung einzurichten.}
            \item \textcolor{green}{Rückfragemöglichkeiten für Teilnehmer*innen an einer Online-Prüfungsleistung einzurichten.}
            \item die Anforderungen gegenüber \old{einer Präsenzklausur } \textcolor{green}{der regulären/ursprünglich geplanten Prüfungsform} nicht maßgeblich zu überschreiten.
            \item \textcolor{green}{die Möglichkeit zu geben, dass die Studierenden die Prüfung ohne die Notwendigkeit von Präsenz ablegen können (u.a. wegen Risikogruppen oder internationalen Studierenden, die nicht einreisen können)}
        \end{itemize}
        \textcolor{green}{Wir fordern ebenfalls dazu auf, die Dozent*innen in Bezug auf Online
        Prüfungsleistungen geschult
        oder zumindest informiert werden sollten (evtl. über ein pdf Dokument mit
        Informationen). Dabei
        sollte auf die verschiedenen Möglichkeiten bei Onlineklausuren und die gesetzlichen
        Grenzen
        eingegangen werden. Zudem sollte auf die Schwierigkeiten der Studierenden und
        mögliche
        technische Probleme Aufmerksam gemacht werden. Für das Sommersemester sollen
        unbedingt
        Schulungsangebote geschaffen werden.}
        \old{Zudem sollten Studierenden, die in diesem Wintersemester auf Online-Klausuren
        ausweichen
        müssen, je Klausur ein Versuch mehr eingeräumt werden.}
        Für das folgende Sommersemester fordert der Studierendenrat das Rektorat und die
        Fakultäten dazu
        auf, schon bei Vorlesungsbeginn über alle Prüfungsmodalitäten zu informieren.
        }{
            ine Umfrage des Studierendenrates zum Sommersemester 2020 zeigte, dass die Studierenden vor
            allem durch die Unsicherheit über das Stattfinden der Klausuren verärgert waren. Auch momentan
            ist es noch nicht klar in welcher Form und ob die Klausuren stattfinden sollen.
            Sollte sich das Infektionsgeschehen nicht verbessern, werden einige Klausuren online stattfinden
            müssen. Bei diesen sollten zumindest einige Kriterien beachtet werden, um Studierende nicht
            übermäßig zu belasten.
            \begin{itemize}
                \item Schutz der Gesundheit
                \item Probleme ausländischer Studierender
                \item Probleme wenn man nicht in der Lage ist eine Klausur Zuhause zu schreiben
            \end{itemize}
        }{

        }{
            \abstimmungsergebnis{
                %Titel
            }{
                tba%Ja
            }{
                tba%Nein
            }{
                tba%Enth
            }{
                tba%Ergebnis  \ul{\noli{\ul{\lii{test}}}}
            }
        }
    \antrag{Online-Sprechstunden \label{corona:5}}{1. Lesung}{Antragsstellend}
    {
        
    }{
        %Begründung
    }{
        %Diskussion
    }{
        \abstimmungsergebnis{
            %Titel
        }{
            tba%Ja
        }{
            tba%Nein
        }{
            tba%Enth
        }{
            tba%Ergebnis  \ul{\noli{\ul{\lii{test}}}}
        }
    }
    \antrag{Wlan \label{corona:6}}{1. Lesung}{Antragsstellend}
    {
        %Antragstext
    }{
        %Begründung
    }{
        %Diskussion
    }{
        \abstimmungsergebnis{
            Wlan
        }{
            tba%Ja
        }{
            tba%Nein
        }{
            tba%Enth
        }{
            tba%Ergebnis  \ul{\noli{\ul{\lii{test}}}}
        }
    }
        \aenderungsantrag{zu \autoref{corona:6}}{}{}{}{}{}
    \antrag{Qualität der digitalen Lehre \label{corona:7}}{1. Lesung}{Antragsstellend}
    {
        %Antragstext
    }{
        %Begründung
    }{
        %Diskussion
    }{
        \abstimmungsergebnis{
            Qualität der digitalen Lehre
        }{
            tba%Ja
        }{
            tba%Nein
        }{
            tba%Enth
        }{
            tba%Ergebnis  \ul{\noli{\ul{\lii{test}}}}
        }
    }
        \aenderungsantrag{zu \autoref{corona:7}}{}{}{}{}{}
    \antrag{Mensa-Essen \label{corona:8}}{1. Lesung}{Antragsstellend}
    {
        %Antragstext
    }{
        %Begründung
    }{
        %Diskussion
    }{
        \abstimmungsergebnis{
            Mensa-Essen
        }{
            tba%Ja
        }{
            tba%Nein
        }{
            tba%Enth
        }{
            tba%Ergebnis  \ul{\noli{\ul{\lii{test}}}}
        }
    }
        \aenderungsantrag{zu \autoref{corona:8}}{}{}{}{}{}
    \antrag{Studierende mit Kind \label{corona:9}}{1. Lesung}{Antragsstellend}
    {
        %Antragstext
    }{
        %Begründung
    }{
        %Diskussion
    }{
        \abstimmungsergebnis{
            Studierende mit Kind
        }{
            tba%Ja
        }{
            tba%Nein
        }{
            tba%Enth
        }{
            tba%Ergebnis  \ul{\noli{\ul{\lii{test}}}}
        }
    }
    \antrag{Corona und Soziales \label{corona:10}}{1. Lesung}{Antragsstellend}
    {
        %Antragstext
    }{
        %Begründung
    }{
        %Diskussion
    }{
        \abstimmungsergebnis{
            Corona und Soziales
        }{
            tba%Ja
        }{
            tba%Nein
        }{
            tba%Enth
        }{
            tba%Ergebnis  \ul{\noli{\ul{\lii{test}}}}
        }
    }