\section{Diskussionen, Inhaltliche Positionierungen}

\antrag{Lehre und Lernen in Zeiten der Pandemie}{1. Lesung}{Vorsitz}{
    Die Beschlüsse der Sondersitzung am 22.01.2021 sollen beschlossen werden
}{
    tba
}{

}
{tba}{tba}{tba}

\antrag{ heiConf für alle Studierenden öffnen }{1. Lesung}{Liste GHG}{
    Der StuRa drängt die Universität und das Universitätsrechenzentrum, allen
    Studierenden die Einrichtung von Räumen in heiConf zu ermöglichen.
}{
    Der Austausch zwischen Studierenden stellt einen essentiellen Bestandteil eines erfolgreichen Studiums dar. Da auch das Wintersemester 2020/2021 online stattfindet, sind die Studierenden gezwungen, die Prüfungsvorbereitung digital durchzuführen. Da die meisten Studierenden sich in kleineren Lerngruppen auf Prüfungen vorbereiten, ist es für sie und ihren Lernerfolg äußerst wichtig, sich gemeinsam zum Lernen digital treffen zu können. Leider haben viele keine guten Möglichkeiten, sich zu treffen. Insbesondere Konferenzen mit mehr als drei Leuten gestalten sich über gängige kostenlose Konferenzsysteme schwierig. Ein erfolgreiches Studium darf allerdings keinesfalls von den finanziellen Möglichkeiten der einzelnen Studierenden abhängen. Ein leistungsfähiges und datenschutzkonformes Konferenzsystem für alle Studierende ist dementsprechend sehr wichtig.
}{

}{}{}{}
\GOantrag{Dringlichkeitsantrag zu 7.2}{Der Studierendenrat stimmt über den Antrag 7.2 nach erster Lesung ab.}{Begründung}{Gegenrede}
{-}{-}{-}{}

\antrag{Öffentliche Beratung über den Stand der geplanten Fuß- und Radbrücke über den Neckar}{1. Lesung}{}{
    Der StuRa berät über die geplanten Fuß- und Radbrücke über den Neckar. Das
 Verkehrsreferat stellt den bisherigen Stand vor sowie seine Sicht auf die möglichen
 Brückenentwürfe vor. Danach kann das Plenum, falls es das Thema für relevant hält,
 Rückfragen oder Feedback geben. Das Verkehrsreferat richtet seine weitere Arbeit an
 diesem Feedback aus.
}{
    Das Umwelt- und Prognoseinstitut (UPI) schrieb das Verkehrsreferat am 13.1. an, um seine Sicht auf den Bau der Neckarbrücke darzulegen. Er findet, dass der Entwurf des Büros Mayr Lüdescher anstatt von Schlaich, Bergemann übernommen werden sollte. Ich (Michèle) fand seine verkehrsplanerischen und ökologische Argumente überzeugend, möchte mir aber vom Studierendenrat Feedback einholen, ob und wie wir in der Angelegenheit vorgehen wollen. Dafür will ich die Situation vorstellen und freue mich über Feedback.\\
    Ich will auch Dieter Teufel vom UPI selbst einladen, um das Anliegen zu erklären. Gerade habe ich noch keine Zusage.
}{

}{tba}{tba}{tba}