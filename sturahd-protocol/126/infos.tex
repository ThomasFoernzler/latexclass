\section{Infos, Termine, Berichte}
\subsection{Wahlen}
\begin{itemize}
    \item bis 15.12.2020: Anmeldung von Online-Wahlen
    \item 14.01.2021, 16:00: Ende des Kandidaturzeitraums
    \item 25.01.2021, 10:00 – 02.02.2021, 12:00: Online-Wahlen
\end{itemize}
\textbf{Was steht an?}
\begin{enumerate}
    \item FSR und FR Wahlen wen betrifft es?\\
    => siehe diese Tabelle: \url{https://www.stura.uni-heidelberg.de/wp-content/uploads/Wahlen_2020/Wahlen_WiSe_2020.pdf}\\
    => Link zur Bekanntmachung: \url{https://www.stura.uni-heidelberg.de/wp-content/uploads/Wahlen_2021/Bekanntgabe_Wahlen_FSR_FR_Winter_2020.pdf}
    \item Fusion der Archäologien \url{https://www.stura.uni-heidelberg.de/wp-content/uploads/Wahlen_2021/Satzungseinreicheaufforderung_Fusion_Byz-Klarch.pdf}
    \item  Satzungsüberarbeitung wir überarbeiten gerade die Wahlordnung und weitere damit zusammenhängende Satzungen, meldet euch wenn euch was auffällt
\end{enumerate}
\emph{Weitere Infos:}\\
\url{https://www.stura.uni-heidelberg.de/wahlen/}

\subsection{Bericht des Vorsitz}
\begin{itemize}
    \item {Gespräch mit Herrn Probst, Direktor der UB, am 17.12.
    \begin{itemize}
        \item Grundlinie der UB: sie möchte im Allgemeinen so schnell es geht wieder öffnen und ihre Dienste möglichst umfangreich zur Verfügung stellen; die UB bietet durch ihre Weitläufigkeit, das Hygienekonzept und des (aufgrund des Umbaus neuen) Belüftungssystems ein hohes Maß an Sicherheit; derzeit geht nur Ausleihe per Post (Studis müssen Porto zahlen) und Scans per HEDD-Dokumentenservice
        \item während des SoSe hat die UB in e-books investiert, wovon aber (naturgemäß) vor allem Natur- und Sozialwissenschaften profitieren
        \item die Direktor*innen der UBs in BaWü treffen sich alle zwei Wochen und tauschen sich über ihre Erfahrungen aus
        \item seit November gibt es das Reservierungssystem, in dem man einen 90-minütigen Slot buchen kann, innerhalb dessen man seinen Arbeitsplatz in der UB antreten muss (mind. 4 h, Möglichkeit zur Verlängerung besteht); seit November wird nur ca. die Hälfte der 340 Plätze in Anspruch genommen (im Unterschied zum Sommer, wo meist alle genutzt wurden)
        \item normalerweise werden in der UB ca. 2500 Bücher ausgeliehen, per Post werden nur ca. 250 maximal angefragt und per Post verschickt
        \item die Gruppenarbeitsräume müssen derzeit großenteils von der UB selbst genutzt werden, weil bedingt durch den Umbau Arbeitsplätze weggefallen sind
        \item eine Option, um für Studis die Möglichkeit zu schaffen, an interaktiven synchronen Veranstaltungen teilzunehmen, wäre, die UB-Terrasse zu möblieren und ca. ab Ostern zu öffnen (das gab es früher schon mal, allerdings wurde draußen auch geraucht, was ein Problem war)
        \item wir alle haben das Gespräch als sehr produktiv wahrgenommen und wollen uns Ende Januar noch einmal treffen
    \end{itemize}
    }
    \item eine*r von uns hat am Treffen des univital-Lenkungskreises (zuvor stud. Gesundheitsmanagement) teilgenommen
    \item Mail an alle Erstis wurde verschickt, ihr findet sie zum Nachlesen \href{https://www.stura.uni-heidelberg.de/2020/12/17/rundmail-drei-ratschlaege-fuer-alle-erstis/}{hier}
    \item vor den RefKonfs machen wir für neue Referent*innen nun immer eine Einführung
    \item in der RefKonf am 22.12. gab es v.a. Berichte und wir haben die bisherigen RefKonfs evaluiert
    \item wir haben an die Fachschaften, Hochschulgruppen und einige Menschen an der Uni, mit denen wir als Verfasste Studierendenschaft zusammengearbeitet haben, einen Weihnachtsbrief verschickt
    \item eine*r von uns hat sich um die Vorbereitung der Corona-Sondersitzung (s. eigener Bericht) gekümmert
\end{itemize}


\subsection{Bericht des Referats für Hochschulpolitische Vernetzung}
Der Schwerpunkt dieses Berichts liegt auf der Bundesebene und dem fzs. Nächstes Mal wir es um die Landesebene gehen.\\
Der bundesweite Dachverband der Studierendenschaften heißt fzs [https://www.fzs.de/], was für freier zusammenschluss von student*innenschaften steht. Er vertritt ca. ein Drittel aller Studierenden in Deutschland. Und er wächst: seit Oktober sind auch die Studierenden der Uni Köln dabei
\myparagraph{\textbf{Allgemeine News:}}
\begin{itemize}
    \item{Der fzs nutzt ab sofort die Plattform wechange (das ist quasi Messenger und Cloud zugleich); wir als Mitglied können sie nun auch nutzen. Link: https://portal.fzs.de/}
    \item{Der fzs hat eine Telegram-Gruppe auf Bundesebene ins Leben gerufen, über die wir sehr schnell und leicht andere Studi-Vertretungen in Deutschland erreichen können.}
    \item{"Solidarsemester"-Kampagne hat sich auf 5 Kernforderungen geeinigt; nun sollen die Landesstudivertretungen damit weitermachen:
        \begin{itemize}
            \item Finanzierung muss gewährleistet werden (-> Überbrückungshilfe, BAföG)
            \item Fristenfreiheit (Kann-Semester auch im WiSe)
            \item Personal: Überstunden, die von Personal geleistet wurden, sollen vergolten werden
            \item digitale Teilhabe (Datenschutz, Verwendung von opensource-Software, alle Studis müssen Zugang haben)
            \item Situation internationaler Studis (Betrag auf Konto, Beschränkung der Arbeitszeit, Immatrikulation soll anders möglich sein, wenn Studis nicht einreisen können)
        \end{itemize}
    }
    \item{fzs hat am 6.12. eine Demo zur Unterstützung der Budapester Universität für Theater- und Filmkunst (SZFE) durchgeführt (\url{https://www.fzs.de/2020/12/04/pressemitteilung-bundesstudierendenverband-solidarisiert-sich-mit-freeszfe/)}}
    \item{es hat sich ein Aktionsbündnis Belarus zur Unterstützung der dortigen Studis gegründet: \url{https://aktionsbuendnis-belarus.de/}}
    \item{Vorstand hat sich mit MdBs Oliver Kaczmarek (SPD) und Jens Brandenburg (FDP) getroffen und über Studienfinanzierung (Nothilfe, BAföG) und Digitalisierung gesprochen; auch mit German U15 und Akteur*innen der Hochrektorenkonferenz hat man sich getroffen; weitere Gespräche mit Politiker*innen und anderen Akteur*innen folgen in den nächsten Wochen; Vorstand hat an mehreren Podiumsdiskussionen teilgenommen und Vorträge gehalten und viel Weiteres gemacht}
    \item{am 15.\&16.01. gab es einen Prüfungsrecht-Workshop, auf den ich euch in der letzten Sitzung hingewiesen hab. War jemand von euch da?}
\end{itemize}
\myparagraph{\textbf{Mitgliederversammlung in Präsenz (5.9.20)}}
In diesem Herbst/Winter fand nicht nur eine, sondern zwei Mitgliederversammlungen (MV) statt.\\
Am 05.09. wurde bei einem eintägigen Treffen in Göttingen ein neuer Vorstand, der Ausschuss Student*innenschaften (AS) und zwei neue Antidiskriminierungsbeauftragte gewählt. Außerdem wurde ein Arbeitsprogramm, der Nachtragshaushalt sowie die Entlastung des alten Vorstands beschlossen.\\
\myparagraph{\textbf{Digitale Mitgliederversammlung (23.-25.10.20)}}
Bei einer digitalen Mitgliederversammlung vom 23.-25.10. fand die Behandlung verschiedener inhaltlicher Anträge sowie Anträge zur Änderung von Satzung und Ordnungen statt.\\
Mit der Annahme "Vertraulichkeit in Plena als Grundsatz festschreiben", "Jurasprech muss verständlich werden - gegen verklausulierte Satzungs- und Ordnungsdebatten", "Stärkung und Klarifizierung von Rechten Betroffener Personen" und "Verweisfehler korrigieren" werden Satzung und Antidiskrimierungsvorschrift erweitert. Die Geschäftsordnung wurde durch "Regelmäßige Pausen" und "Vorschlag der MV Sitzungsleitung" entsprechend konkretisiert. Die Finanzordnung wurde dahingegehnd geändert, dass Referent*innen (die es bisher nicht gibt) mit 450€ AE pro Monat vergütet werden und der Verband Betriebsmittelrücklagen bilden kann.\\
Erste Schritte zur Einführung eines partizipativen Budgets, einem Teil des Haushalts also, der flexibel per Mehrheitsbeschluss ausgegeben werden kann, wurden diskutiert, letztendlich aber abgelehnt.\\
Außerdem hat sich Carlotta vorgestellt und für den Vorstand kandidiert. Die Wahl erfolgte per Briefwahl, mittlerweile ist sie gewählt. Die Kandidat*innen für sämtliche Ausschüsse haben sich auch vorgestellt; hier erfolgt die Wahl ebenfalls durch Briefwahl. Der fzs hat Ausschüsse zu den Themen Finanzen, Frauen- und Genderpolitik, Hochschulfinanzierung/-struktur, Internationales, Verfasste Student*innenschaft/Politisches Mandat, Sozialpolitik, Studienreform, Politische Bildung. Sie sind unseren Arbeitskreisen ähnlich, haben aber ein von der MV verabschiedetes Arbeitsprogramm. Drei Ausschüsse waren nicht arbeitsfähig, da nicht genug Kandidaturen von Frauen vorlagen (ein Ausschuss muss aus mindestens 50\% Frauen bestehen und die Mindestanzahl an Ausschussmitgliedern ist drei -> mind. zwei davon müssen Frauen sein). Nachwahlen durch den AS sind möglich.\\
Die ersten Ausschüsse haben sich bereits oder konstituieren sich mit ihrer ersten Sitzung nun im Dezember.\\
Alle Anträge sind hier nachzuvollziehen: \url{https://mv.fzs.de/web/index.php?r=consultation%2Findex&consultationPath=65-MV}\\
Wenn ihr das Protokoll der Mitgliederversammlung einsehen möchtet, wenn ihr Interesse an einer Ergebnismatrix habt, in der alles noch einmal aufgelistet ist, dann meldet euch gern bei mir\\
Die nächste MV wird wahrscheinlich in der ersten Märzwoche stattfinden. Wenn der Bericht vom fzs eure Neugier geweckt hat und ihr schon jetzt Interesse an der Arbeit in diese habt, meldet euch gern bei mir. Vielleicht können wir die nächste MV zusammen vorbereiten…

\subsection{Bericht über das Treffen mit der Geschäftsführung des Studierendenwerks am 01. Dezember}
Am 01. Dezember fand das allsemestrige Treffen mit der Geschäftsführerin des Studierendenwerks statt. Es ging vor allem um die Angebote des Studierendenwerks in Corona-Zeiten. Folgende Thematiken wurden im Einzelnen besprochen:\\
finanzielle Hilfen: Sowohl der derzeitig zinslose KfW-Kredit als auch die wiedereingeführten Überbrückungshilfen (Zuschüsse) werden stark nachgefragt: Im November gab es ca. 30 Anträge für KfW-Kredite und ca. 380 Anträge für Überbrückungshilfen. Auf der Website des Studierendenwerks finden sich Information zu den verschiedenen Hilfsangeboten und den entsprechenden Antragsmodalitäten, ggf. wird es auch noch einmal ein Webinar dazu geben. Derweil setzt sich der Dachverband der Studierendenwerke politisch dafür ein, dass es großzügigere Hilfen, v.a. in Form von Zuschüssen, gibt.\\
\myparagraph{Mietverträge während Corona:}
Letztes Semester hat das Studierendenwerk Heidelberg Menschen, insbes. internationale Studierende, die aufgrund von Corona nicht nach Heidelberg ziehen konnten, sehr schnell und unkompliziert aus den Mietverträgen entlassen – im Gegensatz zu vielen anderen Studierendenwerken und mit großem finanziellen Einbußen. Dieses Semester ist die Situation weniger dramatisch, da Corona ja bekannt war, sodass versucht wird, in Einzelfällen Lösungen zu finden, was in den meisten Fällen dazu führt, dass die Studierenden relativ schnell aus den Verträgen kommen, teilweise kommt es aber auch vor, dass die dreimonatige Kündigungsfrist eingehalten wird.\\
\myparagraph{Mensabetrieb während Corona:}
Manche Mensen haben auf, machen haben zu (weil Nachfrage nicht da ist und es sich deshalb wirtschaftlich nicht lohnen würde). Das Gastronomieteam prüft auf meine Anregung hin, ob sie auch in der zeughaus-Mesa ein preiswertes Tellergericht anbieten könne, um auch in der Altstadt preiswerte Essensangebote zu machen und nicht nur das teurere Marstall-Essen anzubieten. Demnächst soll ein große Umfrage unter Studierenden zum Mensaessen durchgeführt werden.\\
\myparagraph{Internetsituation in den Wohnheimen:}
Ich habe verschiede Berichte über die schlechte Internetsituation in den Wohnheimen bekommen. Der Geschäftsführung des Studierendenwerks sind die großen Probleme bekannt. Die Probleme sind sehr unterschiedlich. Bei Problemen sollen sich Bewohner*innen an den*die Wohnheimssprecher*in wender, diese Person dann an Herrn Krull – bei Problemen können sie sich auch gerne an mich wenden.\\
Auf meinen Vorschlag hin prüft das Studierendenwerk, ob die derzeit nicht genutzten Mensen für Studierenden zum Lernen und v.a. für Videokonferenzen zur Verfügung gestellt werden können, die zuhause eine zu schlechte Internetverbindungen für das Online-Semester haben.\\
Nachhaltigkeit: Das Studierendenwerk hat einen 9-Punkte-Plan zum Thema Nachhaltigkeit erarbeitet und arbeitet bei der Umsetzung mit den Students for Future und dem Ökoreferat zusammen.\\
Das nächste Treffen mit der Geschäftsführung findet Anfang nächsten Semester am 13. April statt. Bei Fragen oder Problemen könnt ihr euch jederzeit an mich wenden. Entschuldigt bitte, dass der Bericht erst jetzt folgt – im Präweihnachtsstress ist das bei mir untergegangen.

\subsection{Bericht zur Neugestaltung des Campus Neuenheimer Feld im Rahmen des Masterplanverfahrens}
\url{https://sitzungen.sturahd.de/media/126/Studentischer%20Bericht%20zur%20Neugestaltung%20des%20Campus%20Neuenheimer%20Feld%20im%20Rahmen%20des%20Masterplanverfahrens-2.pdf}