\section{Satzungen}
\subsection{Neufassung der Satzung der Studienfachschaft UFG/VA}
    \label{sec:appendix1}
    \myparagraph{Präambel}
    In dem Bestreben, der Fachschaftsarbeit an der Ruprecht-Karls Universität Heidelberg
    eine dauerhafte und bestimmte Grundlage zu geben, haben sich die Studierenden der
    Fächer Geoarchäologie, Ur- und Frühgeschichte sowie Vorderasiatische Archäologie als
    Fachschaft Ur- und Frühgeschichte und Vorderasiatische Archäologie (UFG/VA) folgende
    Satzung gegeben.\newline Die Fachschaft steht für ein Studium ein, in dem sich alle Studierenden individuell
    entfalten und das eigene Recht auf Selbstbestimmung – im Rahmen der Gesetze –
    ausleben kann. In unserem Einsatz für ein solches Studium sehen wir uns als politisch
    neutral und respektieren die Religionsfreiheit unserer Studierenden. Wir fühlen uns
    in unserem Engagement – im Rahmen der Gesetze – ausschließlich durch den freien
    Willen und die unverletzliche Würde des Menschen bestärkt und verpflichtet. Damit
    sich dieser Gedanke in seiner Lebendigkeit entfalten und unermüdlich, aufrichtig und
    frei innerhalb von Universität und Studierendenschaft wirken kann, geben wir uns
    folgende Satzung und nehmen im Rahmen der Erfüllung unserer Aufgaben nach § 65 LHG
    unser – begrenztes – politisches Mandat wahr. Zudem ist die Fachschaft darum bemüht,
    für ein besseres Miteinander von Studierenden und Institut und einen besseren
    Zusammenhalt der Studierenden zu sorgen. Begründung: Dies ist von der VS als
    Kernaufgabe der Fachschaften vorgegeben und hatte in der bisherigen Arbeit unserer
    Fachschaft auch eine wichtige Bedeutung.
    \\
    \paragraph{§1 Allgemeines}
    \begin{enumerate}
        \item[(1)] {Die Studienfachschaft (im Folgenden „Fachschaft”) vertritt die Studierenden des Fachbereichs „Ur-und Frühgeschichte und Vorderasiatische Archäologie“ sowie „Geoarchäologie” und entscheidet insbesondere über fachspezifische Fragen und Anträge.}
        \item[(2)] {Die Zugehörigkeit zur Studienfachschaft ergibt sich aus der Liste in Anhang B.}
        \item[(3)] {Die Studienfachschaft stellt die studentischen Mitglieder der in ihrem Bereich        arbeitenden Gremien oder beteiligt sich zumindest an einem gemeinsamen Wahlvorschlag        für ebendiese.}
        \item[(4)] {Organe der Studienfachschaft sind die Fachschaftsvollversammlung und der        Fachschaftsrat.}
    \end{enumerate}
    \paragraph{§2 Fachschaftsvollversammlung}
    \begin{enumerate}
        \item[(1)] {Die Fachschaftsvollversammlung ist die Versammlung der Mitglieder der        Fachschaft.  Sie tagt öffentlich, soweit gesetzliche Bestimmungen nicht        entgegenstehen.}
        \item[(2)] {Rede-, antrags- und stimmberechtigt sind alle anwesenden Mitglieder der        Fachschaft.}
        \item[(3)] {Beschlüsse werden mit einfacher Mehrheit gefasst.}
        \item[(4)] {Die gefassten Beschlüsse sind bindend für den Fachschaftsrat.}
        \item[(5)] {Die Fachschaftsvollversammlung bestimmt im Einvernehmen des Fachschaftsrats bis        zu zwei Finanzverantwortliche der Fachschaft. Die Finanzverantwortlichen müssen        eingeschriebene Studierende sein. Die Amtszeit beträgt in der Regel ein Jahr.}
        \item[(6)] {Zum Ende der Amtszeit der Finanzverantwortlichen prüft der Fachschaftsrat deren        Arbeit und beantragt anschließend die Entlastung der Finanzverantwortlichen in der        Fachschaftsvollversammlung. Diese beschließt die Entlastung der        Finanzverantwortlichen mit einfacher Mehrheit.}
        \item[(7)] {Die Fachschaftsvollversammlung kann Abstimmungsempfehlungen für das StuRa-        Mitglied beschließen. Diese sind nicht bindend.}
        \item[(8)] {Die Fachschaftsvollversammlung bestimmt jeden November aus ihrer Mitte bis zu        drei Personen, welche die Anträge für die Qualitätssicherungsnachfolgemittel (QSM)        der Fachschaft vorbereiten (QSM-Kommission der Fachschaft). Näheres regelt § 5 dieser        Satzung.}
        \item[(9)] {Fachschaftsvollversammlungen müssen unverzüglich vom Fachschaftsrat einberufen        werden:
            \begin{enumerate}
                \item[9a]auf Antrag eines Drittels der Mitglieder des Fachschaftsrates oder
                \item[9b]auf schriftlichen Antrag von 1\% der Mitglieder der Fachschaft.
            \end{enumerate}
        }
        \item[(10)]{Die Einberufung einer Fachschaftsvollversammlung muss mindestens fünf Tage        vorher öffentlich und in geeigneter Weise bekannt gemacht werden.}
        \item[(11)]{Eine Fachschaftsvollversammlung ist beschlussfähig, wenn sie ordnungsgemäß        einberufen wurde, mindestens die Hälfte der Fachschaftsräte und insgesamt mindestens        2 Mitglieder der Fachschaft anwesend sind.} 
    \end{enumerate}
    \paragraph{§3 Fachschaftsrat}
    \begin{enumerate}
        \item[(1)] { Der Fachschaftsrat wird in gleicher, direkter, freier und geheimer Wahl gewählt.        Es findet Personenwahl statt.}
        \item[(2)] {Alle Mitglieder der Studienfachschaft haben das aktive und passive Wahlrecht. }
        \item[(3)] {Der Fachschaftsrat umfasst mindestens zwei und maximal acht Mitglieder. }
        \item[(4)] {Der Fachschaftsrat nimmt die Interessen der Mitglieder der Fachschaft wahr. }
        \item[(5)] {Zu den Aufgaben des Fachschaftsrats gehören:
            \begin{enumerate}
                \item[5a]Einberufung und Leitung der Fachschaftsvollversammlung.
                \item[5b]Ausführung der Beschlüsse der Fachschaftsvollversammlung.
                \item[5c]Führung de Finanzen sowie Prüfung der Arbeit der Finanzverantwortlichen sowie Beantragung der Entlastung dieser
                \item[5d]Beratung und Information der Studienfachschaftsmitglieder.
                \item[5e]Mitwirkung an der Lehrplangestaltung.
                \item[5f]Austausch und Zusammenarbeit mit den Mitgliedern des Lehrkörpers des Fachbereichs Ur- und Frühgeschichte und Vorderasiatische Archäologie.
                \item[5g]Unterstützung der QSM-Kommission der Fachschaft bei ihrer Arbeit.
            \end{enumerate}
        }
        \item[(6)] {Die Amtszeit der Mitglieder des Fachschaftsrats beträgt ein Jahr. Die Amtszeit        beginnt zum 01. April eines jeden Jahres.}
        \item[(7)] {Für das vorzeitige Ausscheiden aus dem Fachschaftsrat gilt die        Organisationssatzung des StuRa.}
        \item[(8)] {Im Falle des Ausscheidens eines Mitglieds des Fachschaftsrats rückt die Person        mit der nachfolgenden Stimmenzahl für die verbleibende Amtszeit des ausscheidenden        Mitglieds in den Fachschaftsrat nach.}
    \end{enumerate}
    \paragraph{§4 Kooperation und Stimmführung im Studierendenrat}
    \begin{enumerate}
        \item[(1)] {Der Fachschaftsrat entsendet ein Mitglied der Fachschaft in den Studierendenrat        (StuRa).}
        \item[(2)] {Der Fachschaftsrat entsendet zudem Stellvertreter*innen in den StuRa.}
        \item[(3)] {Die Amtszeit der Entsandten im StuRa beträgt ein Jahr.}
        \item[(4)] {Für das vorzeitige Ausscheiden aus dem Studierendenrat gilt die        Organisationssatzung des StuRa.}
        \item[(5)] {Das StuRa-Mitglied und dessen Stellvertreter*innen können per Beschluss mit 2/3-        Mehrheit in der Fachschaftsvollversammlung abberufen werden.}
        \item[(6)] {Das StuRa-Mitglied und dessen Stellvertreter*innen stimmen nach bestem Wissen und        Gewissen im Studierendenrat ab.}
        \item[(7)] {Das StuRa-Mitglied und dessen Stellverterer*innen orientieren sich an den        Abstimmungsempfehlungen der Fachschaftsvollversammlung.  }
        \item[(8)] {Die Fachschaft kann sich nach § 14 der Organisationssatzung der        Studierendenschaft mit anderen Fachschaften zu einer Kooperation zusammenschließen.}  
    \end{enumerate}
    \paragraph{§5 Qualitätssicherungsnachfolgemittel}
    \begin{enumerate}
        \item[(1)] {Die Fachschaftsvollversammlung bestimmt jeden November aus ihrer Mitte bis zu        drei Personen, welche die Anträge für die QSM vorbereiten. Diese bilden die        QSM-Kommission der Fachschaft.}
        \item[(2)] {Nach Bildung der QSM-Kommission wird das QSM-Referat über dessen Mitglieder        informiert.}
        \item[(3)] {Vorschläge für die Verwendung der QSM müssen bis spätestens zwei Wochen vor        Antragsfrist bei der QSM-Kommission der Fachschaft eingereicht werden.}
        \item[(4)] {Bei der Vergabe sind die Mittel auf UFG und VA getrennt, der Anzahl der        Studierenden entsprechend, zu veranschlagen. Die Mittel der Geoarchäologie werden        denen der UFG zugerechnet.}
        \item[(5)] {Per Beschluss der QSM-Kommission der Fachschaft können die Mittel auch gemeinsam        veranschlagt werden. Sollte die Kommission nur aus einer Person, oder nur Personen        einer der Fächer bestehen, so muss dieser Beschluss vom Fachschaftsrat getroffen        werden.}
        \item[(6)] {Aufgaben der QSM-Kommission der Fachschaft sind:
            \begin{enumerate}
                \item[6a]Die vorzeitige Information über den zur Verfügung stehenden Betrag für die QSM;
                \item[6b]Die Vorbereitung der Anträge für die QSM in Rücksprache mit der Fachschaft;
                \item[6c]Die Fristgerechte Einreichung der QSM-Anträge.  
            \end{enumerate}
        }
    \end{enumerate}
    Die Änderung dieser Satzung tritt zum 01. Januar 2021 in Kraft.
\subsection{Fusion der Fachschaften Klassische Archäologie und Byzantinische Archäologie und Kunstgeschichte}
    \label{sec:appendix2}
    \myparagraph{Anhang B: Liste der Studienfachschaften (Studienfachschaftslistenanhang)}
    Die  Ziffern  und  Namen  in  den  Klammern  hinter  dem  jeweiligen  Studienfachschafts-namen bezeichnen die zugeordneten Studiengänge nach der Studierendenstatistik der Zentralen Universitätsverwaltung.
    \begin{enumerate}[noitemsep]
        \item Ägyptologie (1, 15, 886) (Ägyptologie, Papyrologie) 
        \item Alte Geschichte (272, 2722, 2725, 2724) (Alte Geschichte) 
        \item American Studies (838) (American Studies) 
        \item Anglistik (8, 835, 8357, 8352, 8355, 8354, 836, 837, 83, 97, 9222, 9232, 9242) (Englische Philologie, English Studies/Anglistik)
        \item Assyriologie (821, 8217, 8215, 8214, 9147) (Assyriologie) 
        \item Byzantinische Archäologie und Kunstgeschichte (830, 8302, 8305, 8304) (Byzantinische Archäologie und Kunstgeschichte) 
        \item Biologie (26, 933, 881, 843) (Biologie, Biowissenschaften, Molecular Biosciences) 
        \item Chemie - Biochemie (32, 25) (Chemie, Biochemie)
        \item Computerlinguistik (160, 1607, 1602, 1605, 1604, 927) (Computerlinguistik, ) 
        \item Deutsch als Fremdsprache (826, 8267, 827, 8272, 828, 8282, 901, 9017, 9012, 9015, 9014, 939, 940, 950) (Deutsch als Fremdsprachenphilologie, Deutsch als Zweitsprache, Germanistik im Kulturvergleich) 
        \item Erziehung und Bildung (52, 868, 890, 920, 9202, 9205, 9204, 190) (Berufs- und Organisationsbezogene Beratungswissenschaft, Bildungswissenschaft, Pädagogik/Erziehungswissenschaft,) 
        \item Ethnologie (173, 1737, 1732, 1734) (Ethnologie)13.  Geographie (50, 502, 505, 504, 892, 9112, 9115) (Geographie, Governance of Risk and Resources) 
        \item Geowissenschaften (39, 65, 111) (Geowissenschaften) 
        \item Germanistik (67, 672, 675, 674, 929) (Germanistik, Editionswissenschaften und Textkritik) 
        \item Gerontologie \& Care (863, 864, 867, 9676) (Gerontologie, Gesundheit und Care, Gesundheit und Gesellschaft[Care], Gerontologie) 
        \item Geschichte (68, 687, 682, 685, 684, 273, 2735, 2734, 840, 842, 8422, 918, 935) (Mittlere und Neue Geschichte, Osteuropäische Geschichte, Deutsch-Französischer Master in Geschichtswissenschaften, Global History, Historische Grundwissenschaften) 
        \item Informatik (79, 879, 889) (angewandte Informatik, Informatik) 
        \item Islamwissenschaft (81, 883, 884, 8857, 8852, 8854, 930) (Iranistik, Islamic Studies/Islamwissenschaft, Nah- und Mitteloststudien) 
        \item Japanologie (85, 853, 8537, 8532, 8534) (Japanologie, Ostasienwissenschaften Schwerpunkt Japanologie) 
        \item Jura (135, 873, 874, 8732, 932) (International Law [LL.M.], öffentliches Recht, Rechtswissenschaft [inkl. Legum Magister], Unternehmensstrukturierung [LL.M.]) 
        \item Klassische Archäologie (831, 8317, 8312, 8315, 8314, 8347, 12N, 849) (Klassische Archäologie) 
        \item Klassische Philologie (70, 95, 912, 9122, 9125, 9124, 913, 9132, 9135, 9134, 951) (Klassische Philologie: Gräzistik, Klassische Philologie: Latinistik, Klassische und Moderne Literaturwissenschaft) 
        \item Kunstgeschichte (Europäische) (92, 927, 922, 924, 915) (Europäische Kunstgeschichte [inkl. BA int. Verlaufsvariante], Kunstgeschichte und Museologie) 
        \item Mathematik (105, 875, 934) (Mathematik, Scientific Computing) 
        \item Medizin Heidelberg (247, 804, 806, 869, 871, 876, 878, 887, 949, 893, 895) (Advanced Physical Methods ind Radiotherapy, Clinical Medical Physics, International Health, Interprofessionelle Gesundheitsversorgung, Kinder- und Jugendpsychatrie, Medical Biometry/Biostatistics, Medical Education, Humanmedizin, Medizinische Informatik, Scientarum Humanarum, Versorgungsforschung und Implentierungswissenschaft im Gesundheitswesen,) 
        \item Medizin Mannheim (805, 877, 938, 945, 946) (Biomedical Engineering, Health Economics, Medical Physics with distinction in Radiotherapy and Biomedical optics, Humanmedizin, Translational Medical Research) 
        \item Mittellatein/Mittelalterstudien (818, 917) (Lateinische Philologie des Mittelalters und der Neuzeit, Mittelalterstudien) 
        \item Molekulare Biotechnologie (802, 803) (Molekulare Biotechnologie) 
        \item Musikwissenschaft (114, 1147, 1142, 1145, 1144) (Musikwissenschaft) 
        \item Ostasiatische Kunstgeschichte (850, 8502, 853, 8537, 8532, 8534) (Kunstgeschichte Ostasiens, Ostasienwissenschaften Schwerpunkt Kunstgeschichte) 
        \item Pharmazie (126) (Pharmazie) 
        \item Philosophie (127, 1277, 1272, 1275, 1274, 9217) (Philosophie) 
        \item Physik (14, 128, 888) (Astronomie und Astrophysik, Physik, technische Informatik) 
        \item Politikwissenschaft (129, 1297,1292, 1295, 1294, 882, 931, 829) (Politikwissenschaft, Politikwissenschaften/Wirtschaftswissenschaften, Non-Profit Management und Governance) 
        \item Psychologie (132, 1322) (Psychologie) 
        \item Religionswissenschaft (136, 1367, 1362, 1364) (Religionswissenschaft) 
        \item Romanistik (59, 84, 137, 150, 855, 856, 896, 897, 899, 904, 9047, 9042, 9045, 9044, 905, 9057, 9052, 9055, 9054, 906, 9067, 9062, 9065, 9064, 9072, 9075, 9074, 9082, 9084, 9092, 9095, 9094, 9102, 948, 9482) (Romanische Philologie, Romanistik: Französisch, Transkulturelle Studien. Literaturen und Sprachkontakte im frankophonen Raum, Romanistik: Italienisch, Italien im Kontakt – Literatur, Künste, Sprachen, Kulturen, Romanistik: Portugiesisch, Romanistik: Spanisch, Iberoamerikanische Studien. Kontakt – Theorien und Methoden) 
        \item Semitistik (820, 8202, 8205, 8204) (Semitistik)
        \item Sinologie (145, 1452, 858, 860, 861, 916, 853, 8537, 8532, 8534) (Klassische Sinologie, Moderne Sinologie, Sinologie [Chinese Studies], Ostasienwissenschaften Schwerpunkt Sinologie) 
        \item Slavistik/Osteuropastudien (139, 146, 964, 1467, 1462, 1465, 1464, 865, 8652, 8654, 866, 8665, 8664) (Slavistik, Slavische und Osteuropäische Studien) und (8447, 8442, 8445, 8444) (Osteuropa- Ostmitteleuropastudien) 
        \item Soziologie (149, 1492) (Soziologie) 
        \item Sport (29, 295, 872, 898, 9377, 947) (Sportwissenschaft, Sportwissenschaft mit Schwerpunkt Prävention und Rehabilitation)  
        \item Südasienwissenschaften (Fachschaft am SAI) (841, 8412, 8415, 8414, 845, 846, 852, 8527, 8522, 8524, 902, 9022, 9025, 9024, 903, 9032, 9035, 9034, 926, 851, 969) (Kommunikation, Literatur und Medien in Südasiatischen Neusprachen, Neuere Sprachen und Literaturen Südasiens [Moderne Indologie], Kultur und Religionsgeschichte Südasiens [Klassische Indologie], Health and Society in South Asia, Politikwissenschaft Südasiens) 
        \item Theologie (Evangelische) (53, 161, 848, 859, 862, 925, 928, 73, 9252, 9255, 9254, 900, 854) (Christentum und Kultur, Diakoniewissenschaft, Diakonie- Führungsverantwortung in christlich-sozialer Praxis, Doctor of Philosophy PhD, Evangelische Theologie [alle Examen], Magister Theologiae, Management, Ethik und Innovation im Non-Profit-Bereich, Unternehmensführung im Wohlfahrtsbereich) 
        \item Transcultural Studies (891) (Transcultural Studies) 
        \item Ur- und Frühgeschichte/Vorderasiatische Archäologie (UFG/VA) (548, 5482, 5485, 5484, 832, 8327, 8322, 8325, 8324, 9197, 894) (Ur- und Frühgeschichte, Vorderasiatische Archäologie, Geoarchäologie) 
        \item Übersetzen und Dolmetschen (Fachschaft am IÜD) (810, 811, 812, 813, 814, 815, 817, 822, 823) (Konferenzdolmetschen [alle Sprachen], Translation Studies for Information Technologies, Übersetzungswissenschaft [alle Sprachen]  49.  Volkswirtschaftslehre (VWL) (175, 184, 880, 8802, 936) (Economics (Politische Ökonomik), Economics, Volkswirtschaftslehre,) 
        \item Zahnmedizin (185) (Zahnmedizin)
    \end{enumerate}
    \myparagraph{Anhang D: Abweichende Regelungen für Studienfachschaften (ARS)}
    Studienfachschaften können beim Studierendenrat nach dem Studien-fachschaftskonstitutionsanhang (Anhang A) vom Studienfachschaftsregelmodell (Anhang C) abweichende Regelungen beantragen. Diese werden hier aufgeführt: 
    \begin{enumerate}[noitemsep]
        \item Ägyptologie                                                                                                                                                                                                  \\
        \item Alte Geschichte                                                                                                                                                                                              \\
        \item American Studies                                                                                                                                                                                             \\
        \item Anglistik                                                                                                                                                                                                    \\
        \item Assyriologie                                                                                                                                                                                                 \\
        \item Biologie                                                                                                                                                                                                     \\
        \item Chemie und Biochemie                                                                                                                                                                                         \\
        \item Computerlinguistik                                                                                                                                                                                           \\
        \item Deutsch als Fremdsprache                                                                                                                                                                                     \\
        \item Erziehung und Bildung                                                                                                                                                                                       \\
        \item Ethnologie                                                                                                                                                                                                  \\
        \item Geographie                                                                                                                                                                                                  \\
        \item Geowissenschaften                                                                                                                                                                                           \\
        \item Germanistik                                                                                                                                                                                                 \\
        \item Gerontologie  Care                                                                                                                                                                                          \\
        \item Geschichte                                                                                                                                                                                                  \\
        \item Informatik                                                                                                                                                                                                  \\
        \item Islamwissenschaft                                                                                                                                                                                           \\
        \item Japanologie                                                                                                                                                                                                 \\
        \item Jura                                                                                                                                                                                                        \\
        \item Klassische und Byzantinische Archäologie                                                                                                                                                                    \\
        \item Klassische Philologie                                                                                                                                                                                       \\
        \item Kunstgeschichte (Europäische)                                                                                                                                                                               \\
        \item Mathematik                                                                                                                                                                                                  \\
        \item Medizin Heidelberg                                                                                                                                                                                          \\
        \item Medizin Mannheim                                                                                                                                                                                            \\
        \item Mittellatein/Mittelalterstudien                                                                                                                                                                             \\
        \item Molekulare Biotechnologie                                                                                                                                                                                   \\
        \item Musikwissenschaft                                                                                                                                                                                           \\
        \item Ostasiatische Kunstgeschichte                                                                                                                                                                               \\
        \item Pharmazie                                                                                                                                                                                                   \\
        \item Philosophie                                                                                                                                                                                                 \\
        \item Physik                                                                                                                                                                                                      \\
        \item Politikwissenschaft                                                                                                                                                                                         \\
        \item Psychologie                                                                                                                                                                                                 \\
        \item Religionswissenschaft                                                                                                                                                                                       \\
        \item Romanistik                                                                                                                                                                                                  \\
        \item Semitistik                                                                                                                                                                                                  \\
        \item Sinologie                                                                                                                                                                                                   \\
        \item Slavistik/Osteuropastudien                                                                                                                                                                                  \\
        \item Soziologie                                                                                                                                                                                                  \\
        \item Sport                                                                                                                                                                                                       \\
        \item Südasieninwissenschaften (Fachschaft am SAI)                                                                                                                                                                \\
        \item Theologie (Evangelische)                                                                                                                                                                                    \\
        \item Transcultural Studies (891)                                                                                                                                                                                 \\
        \item Ur- und Frühgeschichte/Vorderasiatische Archäologie (UFG/VA)                                                                                                                                                \\
        \item Übersetzen und Dolmetschen (Fachschaft am IÜD)                                                                                                                                                              \\
        \item Volkswirtschaftslehre (VWL)
    \end{enumerate}
   