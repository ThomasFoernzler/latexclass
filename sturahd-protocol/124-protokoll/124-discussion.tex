

1 Begrüßung

1.1 Begrüßung
\textbf{X. Lesung}
\ul{
}

2 Tagesordnung

2.1 Änderungen an der To
\textbf{X. Lesung}
\ul{
	\li{keine Änderungsanträge}
}

3 Protokolle
	\li{keine Änderungsanträge}
}

4 Berichte

4.1 Info zu Wahlen
\textbf{X. Lesung}
\ul{
}

4.2 Bericht des Vorsitz
\textbf{X. Lesung}
\ul{
	\li{keine Fragen}
}

4.3 Bericht zur "Sondersitzung Corona"
\textbf{X. Lesung}
\ul{
	\li{Worum gehts in den 8 Themenblöcken?}
		\noli{\ul{
		\lii{Am Beispiel der Bibliotheken: Die UB hat einen Buchservice bei denen man den Scan eines Buches bestellen können. Oder man könnte fordern, dass alle Bibliotheken offen haben müssen. 
		}}}
	\li{Was wird geplant}
		\noli{\ul{
		\lii{Im Moment nichts in der Planung was nicht in der TO steht
		}}}
	\li{Geht es bei der Sitzung allgemein über Verbesserung der Lehre, Muss man StuRa Mitglied sein, wie viel Arbeitsaufwand}
		\noli{\ul{
		\lii{Sitzung wird Minimum 2,5 Stunden gehen, Es kann jeder daran teilnehmen, wie in normalen 
		}}}
		\noli{\ul{
		\lii{Es geht nicht primär um Verbesserung der Lehre, sondern wenn jemand ein Problem hat, dann kann es in der Sondersitzung behandelt werden. 
		}}}
}

Meinungsbild über Stura Sondersitzung oder Behandlung innerhalb einer regulären Sitzung
\textbf{X. Lesung}
\ul{
	\li{Sondersitzung (Sondersitzung 26 reguläre Sitzung 21)}
}

Meinungsbild über Termin der Sondersitzung (19.01, 22.01, 29.01)
\textbf{X. Lesung}
\ul{
	\li{Für 22.01 (14 für 19.01, 17 für 22.01, 10 für 29.01)}
}

Meinungsbild über Termin der Sondersitzung (19.01, 22.01)
\textbf{X. Lesung}
\ul{
	\li{Für 22.01 (19 für 19.01, 22 für 22.01)}
}

4.4 Bericht des Öffentlichkeitsarbeitsreferats
\textbf{X. Lesung}
\ul{
	\li{keine Fragen}
}

4.5 Bericht zu Nextbike
\textbf{X. Lesung}
\ul{
}

Vorstellung des Lesekreises für politische Bildung
\textbf{X. Lesung}
\ul{
}

GO Antrag für das Vorziehen der Satzungen
\textbf{X. Lesung}
\ul{
	\li{Satzungen sollten vorgezogen werden, um zu vermeiden, dass Mitglieder die Sitzung früher verlassen und so nicht abstimmen könnten.}
		\noli{\ul{
		\lii{ohne Gegenrede angenommen
		}}}
}

5 Satzungen

5.1 Neufassung der Studienfachschaftssatzung UFG/VA
\textbf{X. Lesung}
\ul{
	\li{keine Fragen}
	\li{angenommen mit 48/0/2}
}

5.2 Fusion der Fachschaften Klassische Archäologie und Byzantinische Archäologie und Kunstgeschichte
\textbf{X. Lesung}
\ul{
	\li{keine Fragen}
	\li{angenommen mit 47/0/1}
}

5.3 Satzung der neuen Fachschaft Klassische und Byzantinische Archäologie 
\textbf{X. Lesung}
\ul{
	\li{keine Fragen}
	\li{angenommen mit 47/0/1}
}

6 Wahlen

6.1 Senatskommission für die Verleihung der Bezeichnung apl. Prof. (2. Lesung)
\textbf{X. Lesung}
\ul{
	\li{keine Fragen}
	\li{Huilin Guo gewählt mit  44/2/5}
	\li{Tomke Arand gewählt mit 43/2/6}
}

6.2 Finanzreferat
\textbf{X. Lesung}
\ul{
	\li{Ist Felix Mitglied in Parteien oder ähnlichen Organisationen?}
		\noli{\ul{
		\lii{Nein
		}}}
	\li{hat Felix Erfahrungen mit Fachschaftsarbeit und -strukturen?}
		\noli{\ul{
		\lii{Jein, In BWL nicht, aber in seinem Medizinstudium schon aber nicht allzu aktiv.
		}}}
	\li{Kann Felix sich vorstellen, das Referat mit jemandem jetzt in dem Referat zu machen}
		\noli{\ul{
		\lii{Ja
		}}}
	\li{Wäre er bereit sich auf die Suche nach einem weiblichen Mitglied des Finanzreferats zu machen?}
		\noli{\ul{
		\lii{Das hat er schon aber in seinem Umfeld gibt es keine die Zeit hat.
		}}}
}

6.3 Sozialreferat
\textbf{X. Lesung}
\ul{
	\li{keine Fragen}
}

7 Diskussionen
	\li{keine Anträge zu diesem Thema}
}

8 Finanzanträge

8.1 Globaler Klimastreik organisiert vom Ökoreferat und Fridays for Future Heidelberg
\textbf{X. Lesung}
\ul{
	\li{Ist es ratsam das während Corona in Präsenz zu machen? könnte es stattdessen auch Online stattfinden?}
		\noli{\ul{
		\lii{Grundsätzlich ist es immer wichtig auf den Klimawandel hinzuweißen. Onlinestreiks erreichen nicht sehr viele Menschen. Es wird ein umfassendes Klimakonzept geben. Aber wenn es unklug ist, dann wird er auch nicht in Präsenz stattfinden.
		}}}
	\li{Könnte der Antrag auch gekürzt werden oder kann die Verantaltung dann nicht mehr stattfinden?}
		\noli{\ul{
		\lii{Es wird auf der Demo dafür Spenden gesammelt. Das Geld ist so kalkuliert, dass es reicht, aber die Kosten sind deswegen höher kalkuliert. Nicht benötigtes Geld würde nicht beansprucht werden.
		}}}
}

9 Sonstiges

9.1 Beschluss der Beitragshöhe der Mitgliedschaft des StuRa auf Ebene der Fachschaft Medizin Heidelberg in der Bundesvertretung der Medizinstudierenden in Deutschland e.V. 
\textbf{X. Lesung}
\ul{
	\li{Ist die Bundesvertretung der Medizinstudierenden in Deutschland e.V. Non-Profit}
		\noli{\ul{
		\lii{Ja ein gemeinnütziger Verein
		}}}
	\li{Worum genau geht der Antrag?}
		\noli{\ul{
		\lii{Es gibt keinen festen Beitrag für die Bundesvertretung der Medizinstudierenden in Deutschland e.V.. Die Fachschaft Medizin hat aber noch finanziellen Spielraum und will 
		}}}
	\li{Warum wird das nicht innerhalb der Fachschaft beschlosssen?}
		\noli{\ul{
		\lii{Fachschaften sind Unterorganisationen des Stura's. Deswegen Beschluss auf Sturaebene nötig. Nur VS kann Mitglied werden, nicht einzelne Fachschaften.  
		}}}
	\li{Ist das aus dem allgemeinen Topf oder aus dem Medizinertopf?}
		\noli{\ul{
		\lii{Aus dem allgemeinen
		}}}
	\li{Es ist nicht überzeugend, warum das Geld der VS nicht Studierenden }
		\noli{\ul{
		\lii{Der Dachverband macht auch hier in heidelberg viele Projekte zum Beispiel fafa. Auch andere Fachschaften profitieren davon. Für die FS Medizin ist der Dachverband für die Koordination wichtig. Wissenshunger, Aufklärung Organspende, Fafa wären ohne die BvMD nicht möglich.
		}}}
	\li{Es gibt keine Einsicht in die Finanzen der Bundesvertretung der Medizinstudierenden in Deutschland e.V.}
		\noli{\ul{
		\lii{Die FS Medizin könnte Einsicht erhalten, aber es gibt Vertrauen, dass die Gelder gut gehandhabt werden.
		}}}
}

9.2 Überwindung des Einspruchs des Finanzreferats zur Finanzentscheidung der Fachschaft Medizin Heidelberg betreffend Antrag 2020.621.31
\textbf{X. Lesung}
\ul{
	\li{Wie groß war die Summe? Die Ausgabe ist größer als manche Fachschaften an sich zur Verfügung haben. Dozierende haben auch ein Gehalt, dass sie für das aufwenden. Der Nutzen für die Studierenden ist nicht direkt ersichtlich. Deswegen wird gefordert das zu halbieren.}
		\noli{\ul{
		\lii{Die Summe war 600€. Medizinische Studierende machen einen sehr großen Teil der Studierenden aus. Und deswegen scheint das gerechtfertigt.
		}}}
		\noli{\ul{
		\lii{Medizin ist ein großer Studiengang. Medizin ist ein Studiengang, der von dem Austausch mit den Dozierenden lebt
		}}}
}