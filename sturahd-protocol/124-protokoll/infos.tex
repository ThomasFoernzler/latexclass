\section{Infos, Termine, Berichte}
% TODO fehlende Berichte Vorsitz
\subsection{Wahlen}
\begin{itemize}
    \item bis 15.12.2020: Anmeldung von Online-Wahlen
    \item 14.01.2021, 16:00: Ende des Kandidaturzeitraums
    \item 25.01.2021, 10:00 – 02.02.2021, 12:00: Online-Wahlen
\end{itemize}
\textbf{Was steht an?}
\begin{enumerate}
    \item FSR und FR Wahlen wen betrifft es?\\
    => siehe diese Tabelle: \url{https://www.stura.uni-heidelberg.de/wp-content/uploads/Wahlen_2020/Wahlen_WiSe_2020.pdf}\\
    => Link zur Bekanntmachung: \url{https://www.stura.uni-heidelberg.de/wp-content/uploads/Wahlen_2021/Bekanntgabe_Wahlen_FSR_FR_Winter_2020.pdf}
    \item Fusion der Archäologien \url{https://www.stura.uni-heidelberg.de/wp-content/uploads/Wahlen_2021/Satzungseinreicheaufforderung_Fusion_Byz-Klarch.pdf}
    \item  Satzungsüberarbeitung wir überarbeiten gerade die Wahlordnung und weitere damit zusammenhängende Satzungen, meldet euch wenn euch was auffällt
\end{enumerate}
\emph{Weitere Infos:}\\
\url{https://www.stura.uni-heidelberg.de/wahlen/}

\subsection{Bericht des Vorsitz}
\begin{itemize}
    \item {Gespräch mit Herrn Probst, Direktor der UB, am 17.12.
    \begin{itemize}
        \item Grundlinie der UB: sie möchte im Allgemeinen so schnell es geht wieder öffnen und ihre Dienste möglichst umfangreich zur Verfügung stellen; die UB bietet durch ihre Weitläufigkeit, das Hygienekonzept und des (aufgrund des Umbaus neuen) Belüftungssystems ein hohes Maß an Sicherheit; derzeit geht nur Ausleihe per Post (Studis müssen Porto zahlen) und Scans per HEDD-Dokumentenservice
        \item während des SoSe hat die UB in e-books investiert, wovon aber (naturgemäß) vor allem Natur- und Sozialwissenschaften profitieren
        \item die Direktor*innen der UBs in BaWü treffen sich alle zwei Wochen und tauschen sich über ihre Erfahrungen aus
        \item seit November gibt es das Reservierungssystem, in dem man einen 90-minütigen Slot buchen kann, innerhalb dessen man seinen Arbeitsplatz in der UB antreten muss (mind. 4 h, Möglichkeit zur Verlängerung besteht); seit November wird nur ca. die Hälfte der 340 Plätze in Anspruch genommen (im Unterschied zum Sommer, wo meist alle genutzt wurden)
        \item normalerweise werden in der UB ca. 2500 Bücher ausgeliehen, per Post werden nur ca. 250 maximal angefragt und per Post verschickt
        \item die Gruppenarbeitsräume müssen derzeit großenteils von der UB selbst genutzt werden, weil bedingt durch den Umbau Arbeitsplätze weggefallen sind
        \item eine Option, um für Studis die Möglichkeit zu schaffen, an interaktiven synchronen Veranstaltungen teilzunehmen, wäre, die UB-Terrasse zu möblieren und ca. ab Ostern zu öffnen (das gab es früher schon mal, allerdings wurde draußen auch geraucht, was ein Problem war)
        \item wir alle haben das Gespräch als sehr produktiv wahrgenommen und wollen uns Ende Januar noch einmal treffen
    \end{itemize}
    }
    \item eine*r von uns hat am Treffen des univital-Lenkungskreises (zuvor stud. Gesundheitsmanagement) teilgenommen
    \item Mail an alle Erstis wurde verschickt, ihr findet sie zum Nachlesen \href{https://www.stura.uni-heidelberg.de/2020/12/17/rundmail-drei-ratschlaege-fuer-alle-erstis/}{hier}
    \item vor den RefKonfs machen wir für neue Referent*innen nun immer eine Einführung
    \item in der RefKonf am 22.12. gab es v.a. Berichte und wir haben die bisherigen RefKonfs evaluiert
    \item wir haben an die Fachschaften, Hochschulgruppen und einige Menschen an der Uni, mit denen wir als Verfasste Studierendenschaft zusammengearbeitet haben, einen Weihnachtsbrief verschickt
    \item eine*r von uns hat sich um die Vorbereitung der Corona-Sondersitzung (s. eigener Bericht) gekümmert
\end{itemize}
\subsection{Bericht zur “Sondersitzung Corona“}
\textbf{1. Was ist Sinn und Zweck der Sitzung?}
\ul{\li{Forderungen an Uni (zentral: Rektorat, dezentral: Fakultäten) richten}}
\textbf{2. Soll es eine eigene Sitzung oder Teil einer regulären Sitzung sein?}
\ul{\li{Wenn Sondersitzung: am 19.1., 22.1. oder 29.1.?}\noli{\ul{\lii{StuRa soll über beide Fragen abstimmen}}}}
\textbf{3. Wie läuft die Sitzung ab?}
\begin{itemize}
    \li{je eine Fachschaft (FS) oder Hochschulgruppe (HSG)  soll einen kurzen Antrag zu den Themen (siehe unter 4), der als Diskussionsgrundlage für die Sitzung dient, verfassen}
    \noli{(alle Themen-bezogenen Anträge zusammen bilden dann einen Antrag, der bei der Sondersitzung in die 1. Lesung geht)}
    \lii{\textbf{wir brauchen 8 FSen oder HSGen, die je ein Thema übernehmen und einen Antrag schreiben (wir geben euch dafür eine Vorlage)}}
    \li{zu jedem Thema gibt es eine Kleingruppe}
    \li {in jeder Kleingruppe soll es eine*n Moderator*in, eine*n Protokollant*in und (z.B.) zwei Leute, die Änderungsanträge ausformulieren, geben}
    \lii{\textbf{wir brauchen 8 Moderator*innen -> gibt es in eurer FS oder HSG Leute, die Spaß am Moderieren bzw. Sitzung-Leiten haben? Könnt ihr (während der StuRa-Sitzung) nachfragen, ob sie am gerade (unter 2) beschlossenen Termin Zeit und Lust haben, die Moderation einer Kleingruppe zu übernehmen?} (es wird für die Moderator*innen auch eine Vorbesprechung geben!)}
    \li{ggf. wird bei den Gruppen jemand dabei sein, der sich mit dem Thema auskennt und z.B. erklären kann, welche Services die Bibs bieten}
    \li{die Sitzung soll folgende Struktur haben:
\begin{longtable}{|p{7cm}|p{2.5cm}|p{3.5cm}|}
    \hline
    \textbf{Abeitsphase} & \textbf{Arbeitsform} & \textbf{Dauer} \\\hline
    \endfirsthead
    \hline
    \endhead
    \hline
    \hline
    \endfoot
    \hline
    \endlastfoot
    Informationen/Input zu den Themen & im Plenum & 30 min + 5 min Pause\\\hline
    1. Arbeitsphase: man wählt je nach Interesse ein Thema aus und diskutiert in der entsprechenden Kleingruppe mit Moderation mögliche Forderungen (auf Grundlage des Antrags)\newline \MVRightarrow \textbf{Fokus liegt auf der Diskussion}&in Kleingruppen (ca. 5-10 Leute)&60 min + 10 min Pause\\\hline
    2. Arbeitsphase \newline s.o. \newline \MVRightarrow \textbf{Fokus liegt auf dem Formulieren von Änderungsanträgen} & in Kleingruppen (ca. 5-10 Leute) & 60 min + 5 min Pause\\\hline
    Bericht im Plenum & im Plenum & ca. 20 min (3 min pro Kleingruppe)\\
\end{longtable}
    }
\end{itemize}
\textbf{4. Welche Themen sollen behandelt werden?}\\
(die Unterpunkte dienen als Anhaltspunkte, damit man sich etwas darunter vorstellen kann, sie müssen aber natürlich nicht so aufgenommen werden!)
\begin{enumerate}
    \item Lernräume bzw. -orte
        \begin{enumerate}
            \item[a]kann man solche in Bibs, in Hörsälen etc. schaffen?
            \item[b]Welche Gebäude (mit Hörsälen) kommen dafür in Frage?
            \item[c]wie kann man es Studis, die zu Hause schlechtes Internet haben, ermöglichen, in Räumen der Uni an synchronen Sitzungen teilzunehmen? 
        \end{enumerate}
    \item Bibliotheken
        \begin{enumerate}
            \item[a]kann auch in Bereichsbiblioteken ein Scan-Dienst mit Limit eingerichtet werden?
            \item[b]Kann man derzeit Bücher auch in Bereichsbibliotheken vorbestellen und ein paar Stunden oder Tage ausleihen? (auch wenn es sonst dort nicht geht)
        \end{enumerate}
    \item Freischuss
        \begin{enumerate}
            \item[a] für Klausuren während des WiSe 
        \end{enumerate}
    \item Klausurenphase
        \begin{enumerate}
            \item[a]Informationspolitik: Dozierende sollen Infos dazu geben, wie und wo die Prüfung statt, ob man sich anders als bei einer Prüfung in Präsenz darauf vorbereiten muss etc.
            \item[b]gibt es alternative Prüfungsformate?
        \end{enumerate}
    \item alle Dozierende sollten Fragestunden online anbieten
    \item WLAN
        \begin{enumerate}
            \item[a] WLAN in Studi-Wohnheimen muss verbessert werden
        \end{enumerate}
    \item Qualität der digitalen Lehre/Schulungen für Dozierende bzgl digitaler Lehre (sowohl in Bezug auf technische als auch in Bezug auf didaktische Skills)
    \item Mensa-Essen
        \begin{enumerate}
            \item[a] einige Studis, die nicht viel Geld haben, sind auf günstiges Mensa-Essen angewiesen -> das sollte auch ermöglicht werden
        \end{enumerate}
\end{enumerate}
\textbf{\MVRightArrow Welche Fachschaft bzw. Hochschulgruppe kann je einen kurzen Antrag zu einem der Themen formulieren? (Es wird ein Beispiel dafür geben, an dem man sich orientieren kann)}\\[1.5em]
\textbf{5. Welche Informationen muss man vorher einholen?}
\begin{itemize}
    \item[\MVRightArrow] \textbf{sprecht mit eurer Fachschaft über bestehende Probleme und mögliche Lösungen (und Forderungen) zu den Themen}
    \item[\MVRightArrow] \textbf{Orga-Team (bisher vier Leute) kümmern sich um allgemeine Infos (z.B. Rechtliches) zu den Themen}
\end{itemize}
\textbf{6. Wie geht’s danach weiter?}
\begin{itemize}
    \item in der darauf folgenden regulären StuRa-Sitzung findet die 2. Lesung statt
    \item bis dahin können evtl. von den Kleingruppen, einzelnen FSen oder HSGen etc. noch weitere Änderungsanträge erarbeitet und eingereicht werden
    \item die beschlossenen Forderungen sollen dann an Senat, Fakultäten, Rektorat (ggf. und weitere) weitergeleitet werden
    \item evtl. sollen sie sowohl über die social media-Kanäle des StuRa als auch der FSen und der HSGen weiterverbreitet werden
\end{itemize}

%\subsection{Bericht des Öffentlichkeitsarbeitsreferats}
%TODO
%\subsection{Bericht zu Nextbike}
%TODO
\GOantrag{Änderung der Tagesordnung}{Der Top 6 Satzungen soll vor den Top 5 Kandidaturen verschoben werden.}{Um möglichst viele an den Abstimmungen teilnehmen zu lassen sollten diese relativ früh eröffnet werden.}
{\ul{\li{Keine Gegenrede}}
}{-}{-}{-}{ohne Gegenrede angenommen}