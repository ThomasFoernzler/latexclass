\section{Infos, Termine, Berichte}
\subsection{Wahlen}
\begin{itemize}
    \item bis 15.12.2020: Anmeldung von Online-Wahlen
    \item 14.01.2021, 16:00: Ende des Kandidaturzeitraums
    \item 25.01.2021, 10:00 – 02.02.2021, 12:00: Online-Wahlen
\end{itemize}
\textbf{Was steht an?}
\begin{enumerate}
    \item FSR und FR Wahlen wen betrifft es?\\
    => siehe diese Tabelle: \url{https://www.stura.uni-heidelberg.de/wp-content/uploads/Wahlen_2020/Wahlen_WiSe_2020.pdf}\\
    => Link zur Bekanntmachung: \url{https://www.stura.uni-heidelberg.de/wp-content/uploads/Wahlen_2021/Bekanntgabe_Wahlen_FSR_FR_Winter_2020.pdf}
    \item Fusion der Archäologien \url{https://www.stura.uni-heidelberg.de/wp-content/uploads/Wahlen_2021/Satzungseinreicheaufforderung_Fusion_Byz-Klarch.pdf}
    \item  Satzungsüberarbeitung wir überarbeiten gerade die Wahlordnung und weitere damit zusammenhängende Satzungen, meldet euch wenn euch was auffällt
\end{enumerate}
\emph{Weitere Infos:}\\
\url{https://www.stura.uni-heidelberg.de/wahlen/}

\subsection{Bericht der Härtefallkommission}
Erfolgt mündlich.
\subsection{Bericht des Referats für Hochschulpolitische Vernetzung}
Dieser Bericht umfasst für die für uns relevanten landesweiten Geschehnisse im Zeitraum Oktober 2020 - Januar 2021, insbesondere die Arbeit der LaStuVe.\\
In dieser Zeit fanden drei Landes-Asten-Konferenzen (LAKs) statt: Am 25.10., 29.11.2020 und 10.01.2021. Dabei waren, neben anderen, wichtige Beschlüsse:
\begin{itemize}
    \item nach einer Briefwahl im November/Dezember wurde das Ergebnis am 13.12.2020 festgestellt. Das neue Präsidium besteht aus: Rachel Acosta, Marc Baltrun, Andreas Bauer, Johanna Ehlers, Konstantin Schmidt
    \item Befürwortung eines optionalen Landesweiten Semestertickets solange Preis und Konditionen eines teil- oder vollsolidarischen Tickets nicht bestimmt werden können. Auch mögliche Härtefälle wurden bereits vorgeschlagen.
    \item Unterzeichnung des Offenen Briefs für eine transparente und nachhaltige Versorgungsanstalt der Bundes und der Länder (VBL): \href{https://lastuve-bawue.de/unterzeichnung-des-offenen-briefs-fuer-eine-nachhaltige-vbl/}{https://lastuve-bawue.de/unterzeichnung-des-offenen-briefs-fuer-eine-nachhaltige-vbl/}
    \item Forderungskatalog "Klima und Umwelt" der Landesstudierendenvertretung Baden-Württemberg, der sich an die Regierung BWs richtet.
\end{itemize}
Was hat die LaStuVe sonst noch gemacht:\\
Schon im Vorfeld des Wintersemesters hat der AK Corona einen Forderungskatalog (\href{https://lastuve-bawue.de/forderungskatalog-wintersemester-2020-2021/}{https://lastuve-bawue.de/forderungskatalog-wintersemester-2020-2021/}) zum Wintersemester entworfen und verschickt. Auch während dieses Semesters hat sich die LaStuVe in Gesprächen mit Akteur*innen für Freiversuchsregelungen (\href{https://lastuve-bawue.de/brief-an-landesrektorinnenkonferenz-lrk-landesweite-umsetzung-von-freiversuchen/}{https://lastuve-bawue.de/brief-an-landesrektorinnenkonferenz-lrk-landesweite-umsetzung-von-freiversuchen/}) und flexible Rücktrittsmöglichkeiten eingesetzt, außerdem stets auf generelle wie individuelle Missstände aufmerksam zu machen.\\
Aktuelle "Großprojekte" der Zeit:
\begin{itemize}
    \item Zu den Landtagswahlen im März 2021 wurde ein Studi-O-Mat entworfen, diverse hochschulpolitischen Thesen liegen den Parteien aktuell zur Stellungnahme vor und Mitte Februar soll beides veröffentlicht werden. Es wurde ein breites Themenspektrum gewählt, die Thesen sind mitunter bewusst diskursiv formuliert.
    \item Zuletzt hat die Konstituierung der Landesstudierendenvertretung etwas an Fahrt verloren. Es gab einige strittige Punkte, die bald einer Mehrheitsentscheidung der Studierendenschaften benötigen. Etwas komplizierter verhält es sich bei der Finanzierung der LaStuVe, dessen Varianten umfassender sind als das den Studierendenschaften einzelne Sätze zur Abstimmung gegeben werden.
\end{itemize}
4. HRÄG trat zum 01.01. in Kraft.
\subsection{Bericht zum Studierendenwerk \label{bericht:4}}
\GOantrag{Ausschluss der Öffentlichkeit}{Den Ausschluss der Öffentlichkeit für TOP 4.4.}{Efolgt mündlich.}{Gegenrede}{
    \abstimmungsergebnis{
        \fullref{bericht:4}
    }{
        tba%Ja
    }{
        tba%Nein
    }{
        tba%Enth
    }{
        tba%Ergebnis  \ul{\noli{\ul{\lii{test}}}}
    }
}