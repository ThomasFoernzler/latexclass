\section{Diskussionen, Inhaltliche Positionierungen}
Keine Anträge hierzu.
\section{Beschlüsse der Sondersitzung}
    %1titel,2lesung,3antragssteller,4antragstext,5begründung,6diskussion,7abstimmung
    

    \antrag{Online-Sprechstunden \label{corona:1}}{2. Lesung}{Antragsstellend}
    {
        Der StuRa der Universität Heidelberg fordert, dass jede:r Dozent:in regelmäßige
        digitale
        Sprechstunden für Studierende anbietet. Die Plattformen und Formate der digitalen
        Sprech-stunden sind den Dozierenden grundsätzlich freigestellt. Zentral ist in jedem
        Fall
        eine klare und eindeutige Kommunikation über die Anmeldeverfahren und Plattformen
        der Sprechstunden. Wünschenswert ist dabei neben aktualisierten Institutswebseiten
        der
        Dozierenden auch, in allen Moodle-Kursen eine entsprechende Ankündigungszeile
        einzufügen.\\
        Bezüglich der Anmeldung zu Sprechstunden bieten sich verschiedene Möglichkeiten
        ((Moodle-)Umfrage über den Terminplaner des DFN, nach Vereinbarung per Mail, feste
        Uhr-zeiten mit offener Leitung); eine Selbsteintragung der Studierenden in offenen
        Dokumenten halten wir aufgrund der Missbrauchsgefahr (Löschung anderer
        Studierender) für nicht geeignet. Falls die Möglichkeit des Terminplaners angeboten
        wird, sollten die Timeslots nicht unter 15 Minuten dauern. Bei Themen, die absehbar
        nicht innerhalb der üblichen Sprechstunden besprochen werden können (bspw.
        Abschlussarbeiten), steht es den Dozierenden immer frei, individuelle Lösungen zu
        finden, sofern diese immer eindeutig kommuniziert werden.
        Die Dozierenden sollten über eine ausreichende technische Ausstattung verfügen, um
        sowohl telefonische als auch Sprechstunden in Videokonferenzformaten (z.B. HeiConf,
        Skype, Zoom) anbieten zu können. Studierende sollten zwischen beiden Angeboten frei
        wählen können.\\
        Digitale Sprechstunden sind auch in der vorlesungsfreien Zeit (je nach Bedarf in
        verringertem Umfang) anzubieten und geänderte Termine/Häufigkeiten den
        Studierenden mitzuteilen. Die Online-Sprechstunden sollten auch nach grundsätzlicher
        Öffnung der Institute und Seminare weitergeführt werden, solange
        Mobilitätseinschränkungen der Studierenden andauern.
        Generell sind wir für jegliche Form der Umsetzung von Seiten der Dozierenden offen,
        solange regelmäßige Sprechstunden angeboten, diese auf klarem Weg kommuniziert und
        Infoseiten bei Änderungen zeitnah aktualisiert werden.
    }{
        Aufgrund von Kontaktbeschränkungen und Schließungen der Universitätseinrichtungen
        seit Mitte Dezember und bereits deutlich zuvor ist der Austausch zwischen Studierenden
        und Lehrenden auf persönlichem Weg im Format der klassischen Sprechstunden nicht
        möglich. Für diesen Wegfall müssen coronakonforme Lösungen gefunden werden, die
        möglichst geringen (technischen) Aufwand bedeuten und alle Studierenden
        gleichermaßen in die Lage versetzen, bei Fragen mit ihren Lehrerenden in Kontakt zu
        treten. Deswegen sollte eine Vielfalt der Formen entsprechend der Bedürfnisse und
        Wünsche sowohl der Studierenden wie auch den Kapazitäten der Dozierenden
        ermöglicht sein. Die Online-Sprechstunden sollten auch nach grundsätzlicher Öffnung
        der Institute und Seminare weitergeführt werden. Viele Studierende sind wieder zu ihren 
        Eltern gezogen oder können aus anderen (gesundheitlichen) Gründen nicht in Heidelberg
        vor Ort sein. Zudem nimmt dadurch die Notwendigkeit ab, sich in die Institute zu
        begeben, was unter gesundheitspolitischen Gesichtspunkten allgemein sinnvoll ist.
    }{
        \textbf{1. Lesung:}
        \ul{\li{Keine Fragen}}
    }{
        \abstimmungsergebnis{
            \fullref{corona:1}
        }{
            tba%Ja
        }{
            tba%Nein
        }{
            tba%Enth
        }{
            tba%Ergebnis  \ul{\noli{\ul{\lii{test}}}}
        }
    }
    \antrag{Wlan \label{corona:2}}{2. Lesung}{Antragsstellend}
    {
        Die Verfasste Studierendenschaft der Ruprecht-Karls-Universität Heidelberg fordert,
        dass alle Studierenden der Universität über eine Internetverbindung verfügen die den
        Online- Lehrbetrieb angemessen verfolgbar machen. Dazu sollen
        Studierendenwohnheime konsequent mindestens 35 MBit/s Up/Downloadspeed als feste
        Vorgabe haben und Studierende in privatem Wohnraum Zuschüsse zur Behebung
        bekommen wenn die verfügbare Geschwindigkeit 35MBit/s unterschreitet. Als
        Prüfmittel schlägt dieser Antrag den offiziellen Breitbandmesser der
        Bundesnetzagentur
        unter „https://breitbandmessung.de/“ vor.
    }{
        Die Pandemie und die daraus resultierende Online-Semester haben viele Studierende vor
        das Problem mangelnder technischer Möglichkeiten gestellt. Eines davon ist eine oft
        mangelhafte Internetverbindung die es sehr schwer macht dem alltäglichen Lehrbetrieb
        hinreichend nachzukommen. Dies betrifft vor allem Haushalte in ländlicher Gegend und
        sozial benachteiligte Studierende. Dies ist in den Augen der Verfassten
        Studierendenschaft nicht tragbar da ein jeder Mensch das Recht auf freien Zugang zur
        Bildung hat. Dass die derzeitige Pandemie dieses recht einschränket ist in unseren Augen
        nicht zutreffend, da wir die technologischen Möglichkeiten haben das zu verhindern.
        Daher ist es nun die Aufgabe der Regierung und der einzelnen Universitäten sowie ihre
        Studierendenwerke dieses Recht auf Bildung für alle in einer befriedigenden Art
        pandemiekonform umzusetzen.
    }{
        \textbf{1. Lesung:}
        \ul{
            \li{Geschwindigkeit ist schwierig weil Down- und Uploadgeschwindigkeit verschieden sind.}
                \noli{\ul{
                \lii{Stimmt deswegen Änderungsantrag für 10 Mbit/s Uploadgeschwindigkeit. 
                }}}
        }
    }{
        \abstimmungsergebnis{
            \fullref{corona:2}
        }{
            tba%Ja
        }{
            tba%Nein
        }{
            tba%Enth
        }{
            tba%Ergebnis  \ul{\noli{\ul{\lii{test}}}}
        }
    }
        \aenderungsantrag{\autoref{corona:2} \label{corona:2.1}}{}{
            Anfügung des Folgenden:\\

            "Weiterhin fordert der Studierendenrat
            \begin{itemize}
                \item WLAN-Router in einem Gemeinschaftsraum, z.B. notfalls die gemeinsamen Küche auf jeder Etage, mit der Möglichkeit dort zu arbeiten
                \item Schaffung eine Ansprechperson für Internetprobleme entweder beim Studierendenwerk oder beim jeweiligen Wohnheim (z.B. jeweiliger Hausmeister)
                \item möglicherweise weitere Räumlichkeiten zur Nutzung bereitstellen, da einige Wohnheime weit von der Altstadt bzw. dem Neuenheimer Feld entfernt sind"
            \end{itemize}
        }{
            Die betreffende Gruppe hat ihren Änderungsantrag nicht ausformuliert. Um eine schöne Positionierung zu haben, ist das noch erforderlich.
        }{
            \textbf{1. Lesung:}
            \ul{\li{Keine Fragen}}
        }{
            \abstimmungsergebnis{
                \fullref{corona:2.1}
            }{
                tba%Ja
            }{
                tba%Nein
            }{
                tba%Enth
            }{
                tba%Ergebnis  \ul{\noli{\ul{\lii{test}}}}
            }
        }
    \antrag{Qualität der digitalen Lehre \label{corona:3}}{2. Lesung}{Antragsstellend}
    {
        Der Studierendenrat der Ruprecht-Karls-Universität Heidelberg fordert, dass die
        Auflagen der Prüfungsordnungen den Epidemie bedingten Zuständen und der damit
        einhergehenden Veränderungen in den Leistungsanforderungen an die Studierenden
        angepasst werden. Zudem muss eine Sensibilisierung in Hinsicht auf die Umstände
        einzelner Studierender stattfinden, da auch hier erhebliche Ansprüche auf technisches
        Equipment, eigenständiges Beschaffen von Materialien etc. gestellt werden.
        Sprachpraxis fehlt!!! Hinsichtlich Prüfungen anpassen, da eine gerechte Bewertung der
        Prüfungsleistungen unter den aktuellen Umständen nicht geleistet werden kann. Unter
        den Pandemie Umständen derzeit sind die normalen Maßstäbe nicht angemessen.
        Mehr Eigenarbeit vor Allem in Seminaren und Sprachkursen. Unangemessen hohe
        Anzahl an Arbeitsaufträgen in zu kurzer Zeit.
    }{
        %Begründung
    }{
        \textbf{1. Lesung:}
        \ul{\li{Keine Fragen}}
    }{
        \abstimmungsergebnis{
            Qualität der digitalen Lehre
        }{
            tba%Ja
        }{
            tba%Nein
        }{
            tba%Enth
        }{
            tba%Ergebnis  \ul{\noli{\ul{\lii{test}}}}
        }
    }
        \aenderungsantrag{\autoref{corona:3}\label{corona:3.1}}{}{
            Ersatz des Textes durch:\\

            "Wir fordern einen technischen Support, der zu angemessenen Zeiten, auch kurzfristig,
            für
            Dozierende erreichbar ist, wenn Probleme während der Veranstaltung auftreten. Ferner
            sollte dieser
            auch für Studierende erreichbar sein, falls den Dozierenden die technischen Probleme
            nicht bewusst
            sind und man sie auch nicht darauf aufmerksam machen kann.
            Auch Studierende sollen technischen Support erhalten, den sie erreichen können, wenn
            z.B. in
            Prüfungssituationen technische Probleme auftreten.
            Wir fordern, dass synchron stattfindende Veranstaltungen auch in gleicher oder
            ähnlicher Qualität
            asynchron mitverfolgbar sein müssen, um es auch Studierenden mit instabiler
            Internetverbindung zu
            ermöglichen, an der Veranstaltung teilzunehmen.
            Weiterhin soll bei synchronen Veranstaltungen darauf geachtet werden, dass die
            Veranstalungen
            nicht zu lang sind oder wahlweise Pausen eingelegt werden.
            Es muss auch gewährleistet werden, dass Studierende mit mangelnder technischer
            Ausrüstung an
            allen Veranstaltungen teilnehmen können, dazu soll die Uni den Studierenden die
            erforderliche
            Ausrüstung zur Verfügung zu stellen.
            Insbesondere im Hinblick auf ausfallende Seminare und Praktika ist vermehrt auf eine
            inhaltliche
            Schwerpunktsetzung in den Vorlesungen zu achten."
        }{
            Das laufende Semester hat gezeigt, dass trotz entsprechender Vorbereitungszeit nicht das volle
            Potenzial der Digitallehre genutzt wurde, weshalb wir uns dafür einsetzen die Qualität der Lehre zu
            verbessern und sie auch allen Studierenden zugänglich zu machen.
        }{
            \textbf{1. Lesung:}
            \ul{\li{Keine Fragen}}
        }{
            \abstimmungsergebnis{
                \fullref{corona:3.1}
            }{
                tba%Ja
            }{
                tba%Nein
            }{
                tba%Enth
            }{
                tba%Ergebnis  \ul{\noli{\ul{\lii{test}}}}
            }
        }
    \antrag{Mensa-Essen \label{corona:4}}{2. Lesung}{Antragsstellend}
    {
        Der StuRa spricht dem Studierendenwerk seinen Dank für die Einrichtung eines Corona-
        konformen
        und sicheren Mensabetriebes durch To-Go-Angebote und die Einführung eines festen
        Tagesgerichts
        in der Zeughaus-Mensa aus.
        Der StuRa fordert, dass das Studierendenwerk zusätzliche Mülltonnen für den
        entstandenen
        Verpackungsmüll bereitstellt.
    }{
        Für viele Studierende stellen die Mensen eine wichtige Möglichkeit dar, sich abwechslungsreich
        und dennoch kostengünstig zu ernähren. Das Studierendenwerk hat trotz der notwendigen Auflagen
        eine Möglichkeit gefunden, dies beizubehalten. Auch wurde positiv auf Vorschläge des
        Studierendenwerksreferenten eingegangen und in der Zeughaus-Mensa ein Tagesgericht mit festem
        Preis eingeführt, sodass auch in der Altstadt eine kostengünstige Alternative zu dem nach Gewicht
        der Mahlzeit gezahlten und somit in der Regel etwas teureren Buffet besteht.
        Auch wenn die Mensa aufgrund der Infektionslage zurzeit nicht, wie sonst üblich, als sozialer
        Treffpunkt genutzt werden kann, hat das Studierendenwerk dazu beigetragen, dass Studierende
        immerhin Zugriff auf eine warme, gesunde Mahlzeit haben, was das Leben während des
        Lockdowns erleichtert. Hierfür verdient das Studierendenwerk den Dank des StuRa.
        Problematisch ist jedoch, dass der durch die To-Go-Behältnisse unweigerlich anfallende Müll
        teilweise aufgrund mangelnder Mülleimer nicht entsorgt wird. Dies führt insbesondere im 
        Neuenheimer Feld zu starker Verschmutzung und lockt Ungeziefer an. Wir fordern das
        Surdierendenwerk daher auf, zeitnah weitere Container in der Nähe der Mensa aufzustellen.
    }{
        \textbf{1. Lesung:}
        \ul{\li{Keine Fragen}}
    }{
        \abstimmungsergebnis{
            \fullref{corona:4}
        }{
            tba%Ja
        }{
            tba%Nein
        }{
            tba%Enth
        }{
            tba%Ergebnis  \ul{\noli{\ul{\lii{test}}}}
        }
    }
        \aenderungsantrag{\autoref{corona:4}\label{corona:4.1}}{}{
            Der StuRa spricht dem Studierendenwerk seinen Dank für die Einrichtung eines Corona-
            konformen
            und sicheren Mensabetriebes durch To-Go-Angebote und die Einführung eines festen
            Tagesgerichts
            in der Zeughaus-Mensa aus.
            Der StuRa fordert, dass das Studierendenwerk zusätzliche Mülltonnen \new{in ausreichender Zahl} für den
            entstandenen
            Verpackungsmüll bereitstellt.\\
            \new{Zudem regt der StuRa an ein Mehrweg-Pfandsystem für Essensbehältnisse einzuführen.
            Dabei soll es den Studierenden möglich sein, gegen ein Pfand ein Behältnis zu erhalten, in welchem
            das Tagesmenü ausgegeben wird, das derzeit in einmal-Verpackungen verteilt wird.
            Diese Mehrwegboxen sollen mit dem Logo des Studierendenwerks oder der Mensa markiert, nach
            jeder Benutzung in der Mensa gereinigt und anschließend wieder mit dem Tagesmenü an
            Studierende ausgegeben werden.\\
            Weiterhin empfiehlt der StuRa den Verkauf von Pizza zu erschwinglichen Preisen, um
            eine größere Auswahl an Mahlzeiten zur Mitnahme anbieten zu können. Dieses Angebot kann auch
            nach Ende der Pandemie aufrechterhalten werden.}
        }{
            \begin{enumerate}
                \item {enfällt}
                \item {
                    Durch die Einführung des Mehrwegsystems wollen wir Müll vermeiden, der derzeit das
                    Neuenheimer Feld verschmutzt. Zudem kann das System auch nach Ende der Corona-Pandemie
                    weiter dazu genutzt werden, Essen an Studierende auszugeben, welche mittags in die Mensa gehen
                    und sich ein weiteres Gericht für Abends mit nach Hause nehmen wollen.
                }
                \item {
                    Zur Zeit ist die Auswahl an Gerichten in der Mensa relativ begrenzt. Das Angebot von to-go Pizza
                    würde diese Auswahl erweitern und träfe vermutlich auf große Nachfrage seitens der Studierenden.
                    Dies wird dadurch begünstigt, dass viele Studierende in Wohnheimen keinen Ofen zur Verfügung
                    haben. Der Antragsteller ist sich der ggf. hohen Anschaffungskosten eines oder mehrerer Pizzaöfen
                    bewusst. Dies sollte jedoch, aufgrund der Möglichkeit das Angebot von Pizza auch nach der
                    Pandemie weiterhin aufrechterhalten zu können, kein großes finanzielles Problem darstellen. Der
                    Verkaufspreis der Pizza sollte nach Möglichkeit kostendeckend für die Mensa sein, sich jedoch in
                    einem für Studierende erschwinglichen Rahmen bewegen.
                }
            \end{enumerate}
        }{
            \textbf{1. Lesung:}
            \ul{\li{Keine Fragen}}
        }{
            \abstimmungsergebnis{
                \fullref{corona:4.1}
            }{
                tba%Ja
            }{
                tba%Nein
            }{
                tba%Enth
            }{
                tba%Ergebnis  \ul{\noli{\ul{\lii{test}}}}
            }
        }
    
    \antrag{Corona und Soziales \label{corona:5}}{2. Lesung}{Antragsstellend}
    {
        Die Notlagenfonds sollen in dem Maße aufgestockt werden, dass alle Studierenden,
        deren
        finanzielle Situation eine Fortsetzung des Studiums unmöglich oder unzumutbar machen
        würde,
        ausreichend abgesichert werden. Studentische Angestellte der Universität sollen trotz
        Ausfall mit
        vollständigem Gehalt weiterbezahlt werden.
        Die Universität ihr Recht nutzen, bis zu 5\% der ausländischen Studierenden von den
        Studiengebühren zu befreien. Hierbei sollen finanziell bedürftige Studierende
        berücksichtigt
        werden, die aufgrund von Corona-bedingten Einschränkungen in ihrem Studium
        beeinträchtigt
        werden. Auch soll sie ihren Einfluss gegenüber Land und Wissenschaftsministerium
        nutzen, um
        sich für die Abschaffung der Studiengebühren und die Aufstockung der landesweiten
        Nothilfefonds
        einzusetzen.
    }{
        Erfolgt mündlich.
    }{
        \textbf{1. Lesung:}
        \ul{\li{Keine Fragen}}
    }{
        \abstimmungsergebnis{
            \fullref{corona:5}
        }{
            tba%Ja
        }{
            tba%Nein
        }{
            tba%Enth
        }{
            tba%Ergebnis  \ul{\noli{\ul{\lii{test}}}}
        }
    }
    \subsection{Freischuss für Medizin \label{corona:6} (2. Lesung)}
    Antragsstellend: 
    \myparagraph{Antragstext:}
        Die Verfasste Studierendenschaft der Ruprecht-Karls-Universität Heidelberg fordert,
        dass alle
        Studierenden der Universität für Klausuren im Zeitraum der andauernden Pandemie, je
        Studiengang, einen Klausurversuch mehr erhalten und dass das Wintersemester 2020/2021
        und alle
        folgenden Semester, die aufgrund der Covid-19-Pandemie im Online-Format stattfinden,
        im
        Rahmen der Fristen der Medizinischen Prüfungsordnung nicht zu zählen.
    \myparagraph{Begründung:}
        Die Pandemie und das daraus resultierende Online-Semester, wie auch die weiteren Folgen, machen
        Studierenden und Lehrkräften zu schaffen. Schon zu Beginn, im Sommersemester 2020, wurden
        psychische Belastung, Motivationsprobleme und auch Probleme mit der Internetverbindung sofort
        zu wichtigen Themen. Und noch immer wird nicht selten von einer erschwerten Studiensituation
        gesprochen. Zwar stimmt es, dass die Durchfallquote im ersten Online-Semester nicht
        besorgniserregend höher war, als vorher angenommen wurde, aber bei diesem Argument wird nicht
        beachtet, dass viele Studierende sich gar nicht in der Lage fühlten, einige Klausuren anzutreten und
        sich entsprechend oft entscheiden mussten, sich abzumelden oder gar nicht erst anzumelden.
        Psychische Belastung war besonders für internationale Studierende schwerwiegend. Ohne die
        Möglichkeit, sich in einem fremden Land etwas aufzubauen oder Bekannte und Freunde zu treffen,
        sprachen einige von Einsamkeitsgefühlen. Doch ist dies nicht nur auf internationale Studierende
        begrenzt. Auch einheimische Studierende, besonders die Erstsemester ab diesem Wintersemester,
        sehen sich gelegentlich mit demselben Problem konfrontiert. Die Fachschaften versuchen ihren
        neusten Mitgliedern zu bieten, was sie bieten können, aber bei allen Bemühungen, ist es auch ihnen
        nicht möglich 100\% dessen zu ersetzen was den Studierendenfehlt.\\
        Das alles wirkt sich natürlich auf die Studienleistung aus. Aus einer nicht repräsentativen Umfrage
        der Fachschaft Geowissenschaften am Ende des Sommersemesters 2020 lässt sich zumindest die
        Tendenz erkennen, dass es einem Teil der Studierenden nicht möglich war, dem Online-Unterricht
        angemessen zu folgen. Auch außerhalb dieser Umfrage zeigt sich, dass eine Unsicherheit herrscht
        und die Studierenden haben Hemmungen sich für viele Kurse anzumelden. Dazukommt, dass
        Exkursionen und dergleichen wegfallen oder verschoben werden müssen. Dadurch verlängert sich
        auch noch das Studium für viele. Auch was die Klausuren selbst betrifft, besteht viel Unsicherheit.
        In einigen Kursen wird noch immer gegrübelt, in welcher Form die Prüfungsleistung denn nun
        abgenommen werden kann. Das alles sind nur ein paar der Stressfaktoren für alle Mitglieder unserer
        Universität.\\
        Somit ist es ersichtlich, dass ein Ausgleich für die erschwerten Studienbedingungen geschaffen
        werden muss. Einen solchen Ausgleich sehen wir in einem Extra-Klausurversuch je Studiengang für
        alle Studierenden. Die Verlängerung des Studiums lässt sich in einigen Fällen nicht vermeiden.
        Doch man kann den Studierenden die Angst nehmen und nicht diejenigen Bestrafen, die nichts
        desto trotz versuchen oder sogar versuchen müssen, besonders hochgesetzten Hürden zu
        überwinden.
    \myparagraph{Diskussion:}
    \ul{\li{Keine Fragen}}
    \myparagraph{Abstimmung:}
    \abstimmungsergebnis{
        \fullref{corona:6}
    }{
        tba%Ja
    }{
        tba%Nein
    }{
        tba%Enth
    }{
        tba%Ergebnis  \ul{\noli{\ul{\lii{test}}}}
    }
