\documentclass[fachschaft=mathphys]{mathphys/mathphys-protocol}


\usepackage[utf8]{inputenc} 
\usepackage[ngerman]{babel}
\usepackage[T1]{fontenc}
\usepackage{lipsum}


\setlength{\parindent}{0pt}
\setlength{\parskip}{1em}

\renewcommand\thesection{TOP \arabic{section}:}
\renewcommand\thesubsection{TOP \arabic{section}.\arabic{subsection}:}
\renewcommand\contentsname{Tagesordnung}
\newenvironment{antrag}{\begin{quote}\begin{itshape}}{\end{itshape}\end{quote}}

\begin{document}
\date{\vspace{-2em}1. Januar 1900\vspace{-1em}} % Datum einfügen
\title{\vspace{-2em}Vorläufiges Protokoll der Fachschaftssitzung MathPhysInfo} % Fachschaftssitzungsname einfügen
\maketitle

\begin{tabbing}
\textbf{Sitzungsmoderation:}\quad\=Stefan Zentarra\\ % SiMo einfügen
\textbf{Protokoll:}\>Sabrina Hasselberger\\ % Protokollant einfügen
\textbf{Beginn:}\>19:00 Uhr\\
\textbf{Ende:}\>00:00 Uhr\\ % Sitzungsende einfügen
\end{tabbing}

\section{Begrüßung}
Die Sitzungsmoderation begrüßt die Anwesenden.

\section{Beschluss des Protokolls der vergangenen Sitzung}
Das vorliegende Protokoll wird angenommen.

\section{Feststellen der Tagesordnung}
Die vorliegende Tagesordnung wird verlesen und angenommen.

\section{Festlegen einer Sitzungsmoderation für die nächste Sitzung}
Zur Sitzungsmoderation für die nächste Sitzung wird Kai-Uwe Grabowski bestimmt. % Kai-Uwe ersetzen

\section{BeispielTOP} % unbedingt ersetzen!
Abstract in dem kurz gesagt wird um was es in dem TOP geht. % unbedingt ersetzen!

z.B. Änderungsantrag an die vorliegende Satzung: % unbedingt ersetzen!
\begin{antrag}
Hier steht der genau abzustimmende Antragstext. % unbedingt ersetzen!
\end{antrag}
Der Antrag wird mit einer Konsenssumme von X unter der Beteiligung von
X Mitgliedern angenommen. % unbedingt ersetzen!

Natürlich soll nicht nur der Antrag formuliert, sondern auch der Inhalt des
TOPs kurz zusammengefasst werden. Hierzu kann man sich auch das hervorragende
Beispiel im \href{https://mathphys.fsk.uni-heidelberg.de/svn/sumpf/fsr/vorlagen/beispiel-sitzungsprotokoll}{Sumpf}
anschauen.

\section{Verschiedenes}
Dinge, die sich erst kurzfristig ergeben haben, oder Mitteilungen für die kein
eigener TOP angekündigt wurde können unter diesem Tagesordnungspunkt besprochen
werden. Solange keine besondere Dringlichkeit vorliegt, sollen hier keine
Beschlüsse gefällt werden, sondern es soll nur informiert und gegebenenfalls
kurz diskutiert werden.

\lipsum\lipsum

\end{document}
